\documentclass[10pt,a4paper,fleqn]{article}
%\usepackage{a4p,ifthen,multicol,pifont,epsfig,latexsym,amsmath,amssymb}
\usepackage{ifthen,multicol,pifont,epsfig,latexsym,amsmath,amssymb,eurosym}
\usepackage[left=1cm, right=1cm, top=4cm]{geometry}

%
\begin{document}
\pagestyle{empty}

%
%
\newsavebox{\savepar}
\newenvironment{boxit}{\begin{lrbox}{\savepar}
\begin{minipage}[b]{16.0cm}}
{\end{minipage}\end{lrbox}\fbox{\usebox{\savepar}}}
\newcommand{\myblist}[1]{\begin{list}{#1}{\setlength{\topsep}{1.8mm}
\setlength{\parskip}{0mm} \setlength{\partopsep}{0mm} \setlength{\parsep}{0mm}
\setlength{\itemsep}{0mm}}}
\newcommand{\myhfill}[1]{\hfill {\it #1} = \underline{$~~~~~~~~~~~~~~~~~~~~~~~~~~~$}}
\newcommand{\myhfiyn}[0]{\hfill $\Box$~{\sc si}~~~~~~~~$\Box$~{\sc no}}
\newcommand{\myhfild}[2]
{\hfill {\it #1}= \underline{$~~~~~~~~~~$};~ {\it #2}= \underline{$~~~~~~~~~~$}}
\newcommand{\myhfilt}[3]
{\hfill {\it #1}= \underline{$~~~~~~~~~~~~~$};~ 
{\it #2}= \underline{$~~~~~~~~~~~~~$};~{\it #3}= \underline{$~~~~~~~~~~~~~$}}
%

\setlength{\unitlength}{1mm} 
\setlength{\headheight}{0mm} % footheight

\newcommand{\s}{\text{s}}
\newcommand{\km}{\text{km}}
\newcommand{\kg}{\text{kg}}


\vspace*{-3.5cm}
\Large
%\begin{boxit}
{
\begin{center}
\vspace{0.1cm}
{\bf  Fisica 1 per Chimica (Canali A-E ed P-Z)} \\
{\bf Esame scritto di Laboratorio/Statistica 10/09/2018}   \\
docenti: Francesco Santanastasio, Paolo Gauzzi
   \\
\end{center}
\noindent\rule{18cm}{0.4pt}  \\
\normalsize
%\vspace{0.1cm}
%\begin{center}
%{\underline {Corso di Laurea:}}
%\end{center}
%
\begin{tabular}{ll}
{\underline{Nome:}}  \hspace{6cm} & {\underline{Cognome:}} \\[0.35cm]
{\underline {Matricola}} & {\underline {Aula:}}  \\[0.35cm]
{\underline {Canale:}} & {\underline {Docente:}}
\end{tabular}
\vspace{0.5cm}
}
\small

La durata del compito \`e 3 ore. I cellulari devono essere spenti. Non \`e possibile consultare libri
di testo o appunti personali. \`E possibile utilizzare una
calcolatrice ed il formulario fornito insieme al compito. \\
Riportare a penna (non matita) sul presente foglio i risultati
numerici finali (con unit\`a di misura
ed incertezze di misura). Nell'elaborato riportare sia lo svolgimento
dettagliato degli esercizi (indicando tutte le formule utilizzate ed i passaggi) che i risultati numerici. 

\vspace{0.3cm}
%\end{boxit}
\noindent\rule{18cm}{0.4pt}  \\

\enlargethispage{0.2cm}
\normalsize

% IPOTESI ACCETTATA $\Box$ ~~~~~~~ IPOTESI RIGETTATA $\Box$
%ACCORDO $\Box$ ~~~~~~~ DISACCORDO $\Box$

\vskip0.30cm {\bf \underline {Esercizio 1}} \\
Uno studio sui consumi medi di carburante di automobili a benzina di diversa cilindrata
fornisce i risultati riportati in tabella.
Si assuma che l'incertezza sulle misure della distanza percorsa con un
litro di carburante sia $\sigma = 0.2$ km/$\ell$, mentre si consideri
trascurabile l'incertezza sulla cilindrata.
\begin{table}[h]
\begin{tabular}{|c|c|}
\hline
 Distanza al litro - D (km/$\ell$) & Cilindrata - C (cm$^3$) \\
\hline
13.1 & 1600 \\ 
6.7 & 2800 \\
8.2 & 2500 \\
9.6 & 2000 \\
12.4 & 1400 \\
\hline
\end{tabular}
\end{table}


\myblist{-}
\item[a)]Assumendo che la relazione tra la distanza percorsa con un
  litro di carburante e la
cilindrata sia lineare, $D=A+B\times C$ determinare il parametri $A$ e $B$ con le
loro incertezze. \\
$~$\myhfill{$A$} \\
$~$\myhfill{$B$} \\
\item[b)] Determinare la distanza attesa percorsa per litro di
  carburante, con la corrispondente incertezza, per un'autovettura di cilindrata 1900 cm$^3$.  \\
$~$\myhfill{$D$}
\item[c)] Una casa automobilistica afferma di aver ridotto
  i consumi, producendo una vettura di cilindrata 1900
  cm$^3$ che percorre 11.9 km/$\ell$.
Si pu\`o affermare che sia un miglioramento significativo (assumendo un livello
di significativit\`a dello 0.1\%)?\\
$~$ S\`I $\Box$ ~~~~~~~ NO $\Box$
\end{list}
~\\\\

%\vskip 1.0cm 
{\bf \underline {Esercizio 2}} \\
In un test a risposta multipla, ci sono 12 domande con 3 possibili
risposte, di cui solo una corretta, per ogni domanda. 
\myblist{-}
\item[a)] Calcolare la probabilit\`a di rispondere correttamente a tutte le
  domande scegliendo ogni volta la risposta a caso.\\
$~$ \myhfill{$p$}\\
\item[b)] Per superare il test bisogna rispondere correttamente ad almeno 7
  domande.
Calcolare la probablilt\`a di superare il test rispondendo a caso.\\
$~$ \myhfill{$p$} \\                                                          
\item[c)] In un test composto da 120 domande, calcolare la probabilit\`a
  di dare almeno 70 risposte giuste, sempre rispondendo a caso.\\
$~$\myhfill{$p$} \\                                                          

\end{list}
~\\\\

%\vskip 1.0cm 
{\bf \underline {Esercizio 3}} \\
Un studente vuole studiare il fenomeno di assorbimento nella
materia della radiazione proveniente da una sorgente radioattiva. 
A questo scopo, pone un contatore Geiger ad una certa distanza dalla
sorgente e misura $N_{1}=491$ conteggi in 1 minuto di misura. Ripete quindi la misura
ponendo tra il contatore e la sorgente uno schermo di piombo ottenendo
$N_{2}=111$ conteggi in 1 minuto. Infine, per misurare la
radioattivit\`a di fondo ambientale, egli toglie la sorgente ed ottiene  
$N_{3}=244$ conteggi in 5 minuti di misura.\\

\myblist{-}
\item[a)] Determinare il tasso di conteggi al minuto nei tre casi
  considerati con le corrispondenti incertezze. \\
$~$ \myhfill{$R_1$}\\
$~$ \myhfill{$R_2$} \\
$~$ \myhfill{$R_3$} \\
\item[b)]  Determinare il tasso di conteggi al minuto relativo alla
  sola sorgente radioattiva in assenza ($R_{sorg}$) ed in presenza
  ($R_{sorg+Pb}$) dello schemo di piombo e le relative incertezze. 
Determinare inoltre il rapporto $A=\frac{R_{sorg+Pb}}{R_{sorg}}$ e la
relativa incertezza. \\
$~$ \myhfill{$R_{sorg}$}\\
$~$ \myhfill{$R_{sorg+Pb}$} \\
$~$ \myhfill{$A$} \\
  
\item[c)] Sapendo che \`e la valida la relazione
  $A=e^{-\frac{x}{\lambda}}$, dove $x=(5.0\pm0.1)$~cm \`e lo spessore
  dello schermo di piombo, determinare il parametro $\lambda$ e la
  relativa incertezza.\\
$~$ \myhfill{$\lambda$} \\
\end{list}

\newpage

{\bf -1 punto} ogni 3 errori di questo tipo:\\
- unit\`a di misura non riportate o riportate incorrettamente \\
- errori di calcolo (procedimento e formule ok ma risultato numerico 
significativamente sbagliato)

\vskip0.30cm {\bf \underline {Soluzione Esercizio 1. } } {\bf (10 punti)}\\


a) {\bf (5 punti)}\\
$\Delta=N\sum_iC^2_i-(\sum_iC_i)^2=6960000$ cm$^6$ {\bf (1)}\\
$A=\frac{\sum_iC^2_i\sum_iD_i-\sum_iC_i\sum_iC_iD_i}{\Delta}=(19.20\pm
0.36)$ km/$\ell$ {\bf (2)}\\
$B=\frac{N\sum_iC_iD_i-\sum_iC_i\sum_iD_i}{\Delta}=(-0.00446\pm 0.00017)$
km /($\ell$cm$^3$) {\bf (2)}\\

b) {\bf (2 punti)} \\
$A+B\times 1900=(10.71\pm 0.48)$ km/$\ell$\\

c) {\bf (3 punti)} \\
$t=\frac{11.9-10.71}{0.48}=2.5~\sigma$ {\bf (2)}\\
Non \`e significativo. {\bf (1)}\\

\vskip0.30cm {\bf \underline {Soluzione Esercizio 2. } } {\bf (10 punti)}\\

a) {\bf (2 punti)}\\
$P(12)=B_{12,\frac{1}{3}}(12)=(\frac{1}{3})^{12}=1.88\times 10^{-6}$ {\bf (2)}\\

b) {\bf (5 punti)} \\ 
$B_{12,\frac{1}{3}}(7)=\binom{12}{7}(\frac{1}{3})^7(\frac{2}{3})^5=4.77\%$\\
$B_{12,\frac{1}{3}}(8)=\binom{12}{8}(\frac{1}{3})^8(\frac{2}{3})^4=1.49\%$\\
$B_{12,\frac{1}{3}}(9)=\binom{12}{9}(\frac{1}{3})^9(\frac{2}{3})^3=0.33\%$\\ 
$B_{12,\frac{1}{3}}(10)=\binom{12}{10}(\frac{1}{3})^{10}(\frac{2}{3})^2=4.97\times
10^{-4}$ \\
$B_{12,\frac{1}{3}}(11)=\binom{12}{11}(\frac{1}{3})^{11}\frac{2}{3}=4.52\times
10^{-5}$ {\bf (3)}\\
$P(\nu\geq 7)=B_{12,\frac{1}{3}}(7)+B_{12,\frac{1}{3}}(8)
+B_{12,\frac{1}{3}}(9) +B_{12,\frac{1}{3}}(10) +B_{12,\frac{1}{3}}(11)
+B_{12,\frac{1}{3}}(12)=6.64\%$ {\bf (2)}\\

c) {\bf (3 punti)} \\ 
Con $n$ grande la binomiale si approssima con una gaussiana:\\
$\mu=np=120\times\frac{1}{3}=40$ {\bf (1)}\\
$\sigma=\sqrt{np(1-p)}=5.16$ {\bf (1)}\\
$t=\frac{70-40}{5.16}=5.8$ \\
$P(t>5.8\sigma)<3\times 10^{-7}$ {\bf (1)}\\

\vskip0.30cm {\bf \underline {Soluzione Esercizio 3. } } {\bf (10 punti)}\\

a) {\bf (3 punti)} \\
$R_1=\frac{491\pm\sqrt{491}}{1}=(491 \pm 22)$ conteggi/min {\bf (1)}\\
$R_2=\frac{111\pm\sqrt{111}}{1}=(111 \pm 11)$ conteggi/min {\bf (1)}\\
$R_3=\frac{244\pm\sqrt{244}}{5}=(48.8 \pm 3.1)$ conteggi/min {\bf(1)}\\

b) {\bf (5 punti)} \\
$\sigma_{R_{sorg}}=\sqrt{\sigma^2_{R_{1}}+\sigma^2_{R_{3}}}=22$~conteggi/min \\
$R_{sorg}=R_1-R_3=(442\pm 22)$ conteggi/min {\bf (1.5)}\\
$\sigma_{R_{sorg+Pb}}=\sqrt{\sigma^2_{R_{2}}+\sigma^2_{R_{3}}}=11$~conteggi/min \\
$R_{sorg+Pb}=R_2-R_3=(62\pm 11)$ conteggi/min {\bf (1.5)}\\
$\frac{\sigma_A}{A}=\sqrt{ {(\frac{\sigma_{R_{sorg+Pb}}} {R_{sorg+Pb}})}^2
  +{(\frac{\sigma_{R_{sorg}}} {R_{sorg}})}^2  }=18.4\%$ {\bf (1)}\\
$\sigma_A=A \cdot \frac{\sigma_A}{A} = 0.02585$\\
$A=(0.141 \pm 0.026)=(14.1 \pm 2.6) \% $ {\bf (1)}\\

c) {\bf (2 punti)} \\
$\lambda = -\frac{x}{ln(A)}$\\
$\frac{\partial \lambda}{\partial x}=-\frac{1}{ln(A)}=0.5098$\\
$\frac{\partial \lambda}{\partial A}=-\frac{x}{(ln(A))^2 \cdot
  A}=-9.2398$~cm \\
$\sigma_{\lambda}=\sqrt{ (\frac{\partial \lambda}{\partial
    x})^2\sigma_x^2 + (\frac{\partial \lambda}{\partial A})^2
  \sigma_A^2  }=0.24$~cm {\bf (1)}\\
$\lambda=-\frac{x}{ln(A)}=(2.55 \pm 0.24)$~cm {\bf (1)}


\end{document}
