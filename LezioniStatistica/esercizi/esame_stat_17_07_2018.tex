\documentclass[10pt,a4paper,fleqn]{article}
%\usepackage{a4p,ifthen,multicol,pifont,epsfig,latexsym,amsmath,amssymb}
\usepackage{ifthen,multicol,pifont,epsfig,latexsym,amsmath,amssymb,eurosym}
\usepackage[left=1cm, right=1cm, top=4cm]{geometry}

%
\begin{document}
\pagestyle{empty}

%
%
\newsavebox{\savepar}
\newenvironment{boxit}{\begin{lrbox}{\savepar}
\begin{minipage}[b]{16.0cm}}
{\end{minipage}\end{lrbox}\fbox{\usebox{\savepar}}}
\newcommand{\myblist}[1]{\begin{list}{#1}{\setlength{\topsep}{1.8mm}
\setlength{\parskip}{0mm} \setlength{\partopsep}{0mm} \setlength{\parsep}{0mm}
\setlength{\itemsep}{0mm}}}
\newcommand{\myhfill}[1]{\hfill {\it #1} = \underline{$~~~~~~~~~~~~~~~~~~~~~~~~~~~$}}
\newcommand{\myhfiyn}[0]{\hfill $\Box$~{\sc si}~~~~~~~~$\Box$~{\sc no}}
\newcommand{\myhfild}[2]
{\hfill {\it #1}= \underline{$~~~~~~~~~~$};~ {\it #2}= \underline{$~~~~~~~~~~$}}
\newcommand{\myhfilt}[3]
{\hfill {\it #1}= \underline{$~~~~~~~~~~~~~$};~ 
{\it #2}= \underline{$~~~~~~~~~~~~~$};~{\it #3}= \underline{$~~~~~~~~~~~~~$}}
%

\setlength{\unitlength}{1mm} 
\setlength{\headheight}{0mm} % footheight

\newcommand{\s}{\text{s}}
\newcommand{\km}{\text{km}}
\newcommand{\kg}{\text{kg}}


\vspace*{-3.5cm}
\Large
%\begin{boxit}
{
\begin{center}
\vspace{0.1cm}
{\bf  Fisica 1 per Chimica (Canali A-E ed P-Z)} \\
{\bf Esame scritto di Laboratorio/Statistica 17/07/2018}   \\
docenti: Francesco Santanastasio, Paolo Gauzzi
   \\
\end{center}
\noindent\rule{18cm}{0.4pt}  \\
\normalsize
%\vspace{0.1cm}
%\begin{center}
%{\underline {Corso di Laurea:}}
%\end{center}
%
\begin{tabular}{ll}
{\underline{Nome:}}  \hspace{6cm} & {\underline{Cognome:}} \\[0.35cm]
{\underline {Matricola}} & {\underline {Aula:}}  \\[0.35cm]
{\underline {Canale:}} & {\underline {Docente:}}
\end{tabular}
\vspace{0.5cm}
}
\small

La durata del compito \`e 3 ore. I cellulari devono essere spenti. Non \`e possibile consultare libri
di testo o appunti personali. \`E possibile utilizzare una
calcolatrice ed il formulario fornito insieme al compito. \\
Riportare a penna (non matita) sul presente foglio i risultati
numerici finali (con unit\`a di misura
ed incertezze di misura). Nell'elaborato riportare sia lo svolgimento
dettagliato degli esercizi (indicando tutte le formule utilizzate ed i passaggi) che i risultati numerici. 

\vspace{0.3cm}
%\end{boxit}
\noindent\rule{18cm}{0.4pt}  \\

\enlargethispage{0.2cm}
\normalsize

% IPOTESI ACCETTATA $\Box$ ~~~~~~~ IPOTESI RIGETTATA $\Box$
%ACCORDO $\Box$ ~~~~~~~ DISACCORDO $\Box$

\vskip0.30cm {\bf \underline {Esercizio 1}} \\
Uno studente vuole misurare il coefficiente di attrito dinamico $\mu_d$ tra un
corpo ed un piano con il seguente esperimento. Il corpo viene lanciato
su un piano orizzontale con una velocit\`a iniziale $v_0$,
misurata con incertezza trascurabile tramite uno strumento
digitale. Si misura la distanza $S$ percorsa dal corpo prima di
fermarsi a seguito dell'attrito con il piano. L'incertezza su $S$ e'
pari a $\sigma_S=0.005$m, uguale per tutte le misure. Si ripetono le
misure per diversi valori di $v_0$ e si raccolgono i dati in tabella.
\begin{table}[h]
\begin{tabular}{|c|c|}
\hline
$v_0$ (m/s) & $S$ (m) \\
\hline
1 & 0.094 \\ 
2 & 0.412 \\
3 & 0.914 \\
4 & 1.623 \\
\hline
\end{tabular}
\end{table}

Dai principi della dinamica, \`e nota la relazione lineare che lega la
distanza $S$ al quadrato della velocit\`a $v_0^2$:
\begin{displaymath}
S=\frac{1}{2  \mu_d  g}  v_0^2
\end{displaymath} 
dove l'accelerazione di gravit\`a \`e pari a $g=9.81$~m/s$^2$ con
incertezza trascurabile.
\myblist{-}
\item[a)] Assumendo la relazione lineare tra $S$ e $v_0^2$, determinare la migliore stima del coefficiente di attrito
  dinamico con la sua incertezza utilizzando il metodo dei minimi quadrati. \\
$~$\myhfill{$\mu_d$}
\item[b)] Determinare la velocit\`a iniziale di lancio, con la sua
  incertezza, tale che il corpo si fermi dopo una distanza $L=1$~m~.  \\
$~$\myhfill{$v$}
\item[c)] Un secondo studente ripete l'esperimento ottenendo
  $\mu_d=(0.5205\pm 0.0015)$. Determinare se le due misure sono in
  accordo (entro 3 deviazioni standard). \\
$~$ ACCORDO $\Box$ ~~~~~~~ DISACCORDO $\Box$
\end{list}

{\it
NOTA:\\
Metodo dei minimi quadrati non pesati per una relazione lineare del tipo $y=Bx$\\
$B=\frac{\sum xy}{\sum x^2}$\\
$\sigma_B=\frac{\sigma_y}{\sqrt{\sum x^2}}$\\
}

\newpage

%\vskip 1.0cm 
{\bf \underline {Esercizio 2}} \\
Il numero di chiamate che arrivano ogni minuto al centralino
telefonico dell'ufficio reclami del mio comune \`e una variabile
casuale che segue una distribuzione poissoniana con valore atteso 
pari ad 1. 
\myblist{-}
\item[a)] Calcolare la probabilit\`a che in un determinato minuto 
non arrivi nessuna chiamata.\\
$~$ \myhfill{$p$}\\
\item[b)] Supponendo che il centralino possa gestire al massimo 2
  chiamate al minuto, calcolare la probabilit\`a di trovarlo occupato
  se provo a chiamare in un determinato momento della giornata.\\
$~$ \myhfill{$p$} \\                                                          
\item[c)] Se effettuo 3 chiamate in diverse ore della giornata, 
calcolare la probabilit\`a di trovare sempre libero.\\
$~$\myhfill{$p$} \\                                                          
\end{list}
~\\\\

{\bf \underline {Esercizio 3}} \\

Una ditta produttrice di stampanti sa che la durata di una macchina (in
migliaia di pagine) segue una distribuzione normale con $\mu=1600$ e
$\sigma=80$.
La ditta restituisce 1000 \euro ~ all'acquirente se la durata della macchina
\`e inferiore a 1400 migliaia di pagine.
Calcolare la probabilit\`a che: 
\myblist{-}
\item[a)] la ditta debba risarcire 1000 \euro $~$ \myhfill{$p$}\\
\item[b)] su 5 macchine vendute debba risarcire al massimo 1000 \euro
  \myhfill{$p$}\\
\item[c)] su 10000 macchine vendute debba risarcire pi\`u di 90000 \euro
  \myhfill{$p$}\\ 
\end{list}

\newpage

{\bf -1 punto} ogni 3 errori di questo tipo:\\
- unit\`a di misura non riportate o riportate incorrettamente \\
- errori di calcolo (procedimento e formule ok ma risultato numerico 
significativamente sbagliato)

\vskip0.30cm {\bf \underline {Soluzione Esercizio 1. } } {\bf (10 punti)}\\

a) {\bf (4 punti)}\\
$S=Y$, $v_0^2=X$, $Y=BX$, $B=\frac{1}{2 \mu_d g}$ {\bf (1)}\\
$\sum XY = 35.936 \, m^3/s^2 \\
\sum X^2 = 354 \, m^4/s^4$\\
$B=\frac{\sum XY}{\sum X^2}$\\
$\sigma_B=\frac{\sigma_Y}{\sqrt{\sum X^2}}$\\
$B=0.1015 \pm 0.0003 \frac{s^2}{m}$ {\bf (1)}\\
$\mu_d=\frac{1}{2Bg}$\\
$\sigma_{\mu_d}/\mu_d=\sigma_{B}/B=0.003=0.3\%$ {\bf (1)}\\
$\mu_d=(0.5021 \pm 0.0015)$ (adimensionale) {\bf (1)}\\

b) {\bf (3 punti)} \\ 
$v=\sqrt{2 \mu_d g L}$ {\bf (1)}\\
$\frac{\sigma_v}{v}=\frac{1}{2}\frac{\sigma_{\mu_d}}{\mu_d}=0.0015=0.15\%${\bf (1)}\\
$v=(3.139 \pm 0.005) \, m/s${\bf (1)}\\

c) {\bf (3 punti)} \\ 
$\mu_1=(0.5021 \pm 0.0015)$\\
$\mu_2=(0.5205 \pm 0.0015)$\\
$\Delta=|\mu_1-\mu_2|=0.0184${\bf (1)}\\
$\sigma_{\Delta}=\sqrt{2}\sigma_{\mu}=0.0021${\bf (1)}\\
$t=\frac{\Delta}{\sigma_{\Delta}}=8.8 >> 3 \rightarrow $ le due misure
non sono in accordo. {\bf (1)}\\


\vskip0.30cm {\bf \underline {Soluzione Esercizio 2. } } {\bf (10 punti)}\\


a) {\bf (3 punti)}\\
$n=$ numero di chiamate in un minuto, segue una distribuzione di
Poisson con
$\lambda=1$. $P_{\lambda}(n)=\frac{\lambda^ne^{-\lambda}}{n!}$ {\bf (1)}\\
$P_{\lambda}(n=0)=\frac{1^0e^{-1}}{0!}=36.8\%$ {\bf (2)}\\
b) {\bf (4 punti)} \\
Trovo libero se sono in corso zero ($n=0$) chiamate o una ($n=1$)
chiamata in quel dato minuto:\\
$p(libero)=P_{\lambda}(n=0)+P_{\lambda}(n=1)$ {\bf (1)} \\
$P_{\lambda}(n=1)=\frac{1^1e^{-1}}{1!}=36.8\%$ {\bf (1)}\\
$p(occupato)=1-p(libero)=100\%-36.8\%-36.8\%=26.4\%$ {\bf (2)}\\
%$P_{\lambda}(n=2)=\frac{1^2e^{-1}}{2!}=18.4\%$ {\bf (1)} \\
%$p(occupato)=100\%-36.8\%-36.8\%-18.4\%=8\%$ {\bf (1)} \\
%$p(occupato)=100\%-36.8\%-36.8\%=26.4\%$ {\bf (1)} \\
c) {\bf (3 punti)} \\
Successo: centralino libero $\rightarrow$ probabilit\`a $p(libero)=1-p(occupato)=73.6\%$ {\bf (1)}\\
$n$ numero di successi in $N=3$ prove indipendenti. {\bf (1)}\\
$p(sempre \, libero)=B_{N,p}(n=3)=\frac{3!}{3!(3-3)!}(0.736)^3(1-0.736)^{3-3}=0.736^3=39.9\%${\bf (1)}

\vskip0.30cm {\bf \underline {Soluzione Esercizio 3. } } {\bf (10 punti)}
\\
a) {\bf (3 punti)} \\
$1600-1400=200$; $\frac{200}{80}=2.5$ dev. standard {\bf (1)} \\
$P(t<2.5\sigma)=50\%-49.38\%=0.62\%$ {\bf (2)}\\

b) {\bf (4 punti)} \\
Binomiale con $N=5$ e $p=0.0062$ {\bf (1)} \\
$B_{5; 0.0062}(0)=(1-0.0062)^5=96.9\%$ {\bf (1)}\\
$B_{5; 0.0062}(1)=5\times 0.0062\times (1-0.0062)^4=3.0\%$ {\bf (1)}\\
$P=B_{5; 0.0062}(0)+B_{5; 0.0062}(1)=99.9\%$ {\bf (1)}\\
c) {\bf (3 punti)} \\
La binomiale con $N=10000$ si approssima con una gaussiana con: $\mu=Np=62$
e $\sigma=\sqrt{Np(1-p)}=7.85$ {\bf (1)}\\
90000 \euro ~ vuol dire almeno 90 macchine difettose, $\frac{90-62}{7.85}=3.5$ {\bf (1)}\\
$P(t>3.5\sigma)=50\%-49.98\%=2\times 10^{-4}$ {\bf (1)}\\

\end{document}
