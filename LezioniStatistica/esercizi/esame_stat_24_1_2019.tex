\documentclass[10pt,a4paper,fleqn]{article}
%\usepackage{a4p,ifthen,multicol,pifont,epsfig,latexsym,amsmath,amssymb}
\usepackage{ifthen,multicol,pifont,epsfig,latexsym,amsmath,amssymb,eurosym}
\usepackage[left=1cm, right=1cm, top=4cm]{geometry}

%
\begin{document}
\pagestyle{empty}

%
%
\newsavebox{\savepar}
\newenvironment{boxit}{\begin{lrbox}{\savepar}
\begin{minipage}[b]{16.0cm}}
{\end{minipage}\end{lrbox}\fbox{\usebox{\savepar}}}
\newcommand{\myblist}[1]{\begin{list}{#1}{\setlength{\topsep}{1.8mm}
\setlength{\parskip}{0mm} \setlength{\partopsep}{0mm} \setlength{\parsep}{0mm}
\setlength{\itemsep}{0mm}}}
\newcommand{\myhfill}[1]{\hfill {\it #1} = \underline{$~~~~~~~~~~~~~~~~~~~~~~~~~~~$}}
\newcommand{\myhfiyn}[0]{\hfill $\Box$~{\sc si}~~~~~~~~$\Box$~{\sc no}}
\newcommand{\myhfild}[2]
{\hfill {\it #1}= \underline{$~~~~~~~~~~$};~ {\it #2}= \underline{$~~~~~~~~~~$}}
\newcommand{\myhfilt}[3]
{\hfill {\it #1}= \underline{$~~~~~~~~~~~~~$};~ 
{\it #2}= \underline{$~~~~~~~~~~~~~$};~{\it #3}= \underline{$~~~~~~~~~~~~~$}}
%

\setlength{\unitlength}{1mm} 
\setlength{\headheight}{0mm} % footheight

\newcommand{\s}{\text{s}}
\newcommand{\km}{\text{km}}
\newcommand{\kg}{\text{kg}}


\vspace*{-3.5cm}
\Large
%\begin{boxit}
{
\begin{center}
\vspace{0.1cm}
{\bf  Fisica 1 per Chimica (Canali A-E ed P-Z)} \\
{\bf Esame scritto di Laboratorio/Statistica 24/1/2019}   \\
docenti: Francesco Santanastasio, Paolo Gauzzi
   \\
\end{center}
\noindent\rule{18cm}{0.4pt}  \\
\normalsize
%\vspace{0.1cm}
%\begin{center}
%{\underline {Corso di Laurea:}}
%\end{center}
%
\begin{tabular}{ll}
{\underline{Nome:}}  \hspace{6cm} & {\underline{Cognome:}} \\[0.35cm]
{\underline {Matricola}} & {\underline {Aula:}}  \\[0.35cm]
{\underline {Canale:}} & {\underline {Docente:}}
\end{tabular}
\vspace{0.5cm}
}
\small

La durata del compito \`e 3 ore. I cellulari devono essere spenti. Non \`e possibile consultare libri
di testo o appunti personali. \`E possibile utilizzare una
calcolatrice ed il formulario fornito insieme al compito. \\
Riportare a penna (non matita) sul presente foglio i risultati
numerici finali (con unit\`a di misura
ed incertezze di misura). Nell'elaborato riportare sia lo svolgimento
dettagliato degli esercizi (indicando tutte le formule utilizzate ed i passaggi) che i risultati numerici. 

\vspace{0.3cm}
%\end{boxit}
\noindent\rule{18cm}{0.4pt}  \\

\enlargethispage{0.2cm}
\normalsize

% IPOTESI ACCETTATA $\Box$ ~~~~~~~ IPOTESI RIGETTATA $\Box$
%ACCORDO $\Box$ ~~~~~~~ DISACCORDO $\Box$

\vskip0.30cm {\bf \underline {Esercizio 1}} \\

In un account di posta elettronica arrivano in media 36 e-mail al giorno.


\myblist{-}
\item[a)] Qual \`e il numero (intero) pi\`u probabile di e-mail trovate in un'ora? \\
$~$ \myhfill{$n $} \\

\item[b)] Qual \`e la probabilit\`a di trovare 3 o pi\`u e-mail in un'ora ?    \\
$~$ \myhfill{$P(n\geq 3)$} \\

\item[c)] Controllando l'account dopo 16 ore, qual \`e la probabilit\`a di
  trovare pi\`u di 38 nuove e-mail ? \\
$~$ \myhfill{$P(n> 38)$} 
\end{list}
~\\\\

%\vskip 1.0cm 
{\bf \underline {Esercizio 2}} \\

Una ditta fornisce resistenze di valore nominale dichiarato
$R_{\mu}=470\, \Omega$. Il valore delle resistenze \`e distribuito secondo
una gaussiana centrata in $R_{\mu}$ e con deviazione standard nota $\sigma=20\,\Omega$. Un
sistema di controllo esclude dalla vendita tutte le resistenze
prodotte superiori a 520 $\Omega$ ed inferiori a 420 $\Omega$. 

\myblist{-}
\item[a)] Determinare il numero $N$ di resistenze che vengono in
  media scartate dalla procedura di controllo ogni 10000 resistenze prodotte.\\
$~$ \myhfill{$N$}\\
\end{list}

Uno studente misura un campione di 25 resistenze e ne calcola la
media ottenendo $\overline{R}=490\,\Omega$. Si assuma che la deviazione
standard della gaussiana $\sigma$ fornita dalla ditta sia una stima corretta.

\myblist{-}
\item[b)] Determinare la miglior stima del valore nominale delle
  resistenze e la sua incertezza.\\
$~$ \myhfill{$R_{\mu,mis.}$} \\                                                          
\item[c)] Determinare se il valore misurato $R_{\mu,mis.}$ \`e in buon accordo con il valore $R_{\mu}$ dichiarato dalla ditta.\\
ACCORDO $\Box$ ~~~~~~~ DISACCORDO $\Box$\\
\end{list}
~\\\\

\newpage

%\vskip 1.0cm 
{\bf \underline {Esercizio 3}} \\

Un astronauta esegue un esperimento di fisica 
sulla superficie della Luna. La tabella riporta le sue misure 
di allungamento di una molla $\Delta L$ (misurato in cm) 
in funzione della massa $m$ (misurata in kg) dei pesi applicati.
L'incertezza sulla misura di $\Delta L$ \`e uguale per tutte le misure
e pari a 0.05 cm, mentre l'incertezza sulle misure di massa \`e
trascurabile. La costante elastica della molla \`e pari a $k=(11.6\pm0.1)$~N/m.
E' noto che esiste una relazione lineare tra l'allungamento e la
massa, $\Delta L=B \cdot m$, con $B=\frac{g_L}{k}$ dove $g_{L}$ \`e 
l'accelerazione di gravit\`a sulla Luna.

\begin{table}[h]
\begin{tabular}{|c|c|}
\hline
 $\Delta L$ (cm) & $m$ (kg) \\
\hline
0.4 & 0.0264 \\ 
1.2 & 0.0794 \\
1.6 & 0.105 \\
2.0 & 0.138 \\
\hline
\end{tabular}
\end{table}

\myblist{-}
\item[a)] Assumendo la relazione lineare, calcolare la miglior stima
  del parametro $B$ e la sua incertezza riportando il risultato finale
  con le unit\`a di misura del sistema internazionale.\\
$~$ \myhfill{$B$}\\
\item[b)] Determinare la miglior stima dell'accelerazione di gravit\`a
  sulla Luna e la sua incertezza.\\
$~$ \myhfill{$g_{L}$} \\                                                          
\item[c)] Determinare la miglior stima del raggio della Luna $r_L$ e
  la sua incertezza, sapendo che $g_L=\frac{M_L \cdot G}{r^2_L}$ dove
 la massa della Luna vale $M_L=7.348 \cdot 10^{22}$ kg e la costante di
 gravitazione universale vale $G=6.67 \cdot 10^{-11} \frac{Nm^2}{kg^2}$ (entrambe note con incertezze trascurabili).\\
\end{list}

{\it
NOTA:\\
Metodo dei minimi quadrati non pesati per una relazione lineare del tipo $y=Bx$\\
$B=\frac{\sum xy}{\sum x^2}$\\
$\sigma_B=\frac{\sigma_y}{\sqrt{\sum x^2}}$\\
}

~\\\\


%\vskip 1.0cm 

\newpage

{\bf -1 punto} ogni 3 errori di questo tipo:\\
- unit\`a di misura non riportate o riportate incorrettamente \\
- errori di calcolo (procedimento e formule ok ma risultato numerico 
significativamente sbagliato)

\vskip0.30cm {\bf \underline {Soluzione Esercizio 1. } } {\bf (10 punti)}\\
a) {\bf (4 punti)}\\
  $\mu =\frac{36}{24} = 1.5 $ e-mail/h. {\bf(1)} \\
  $P_{1.5}(\nu)=e^{-1.5}\frac{1.5^{\nu}}{\nu!}$. \\
  $P_{1.5}(0)= e^{-1.5} = 22.3\%$.\\
  $P_{1.5}(1)= 1.5 e^{-1.5} = 33.5\%$\\
  $P_{1.5}(2)= \frac{1.5^2 e^{-1.5}}{2} = 25.1\%$ {\bf (2)}\\
  Il valore pi\`u probabile in un'ora \`e $n = 1$ {\bf (1)}\\  
b) {\bf (2 punti)} \\ 
   $P_{1.5}(n\geq 3) = 1 - P_{1.5}(0) - P_{1.5}(1) - P_{1.5}(2) = 19.1\%$ {\bf (2)} \\
c) {\bf (4 punti)} \\ 
Il numero aspettato di e-mail in 16 ore \`e $1.5\times 16 = 24$. \\
Si pu\`o usare l'approssimazione gaussiana con $\mu = 24$ e $\sigma =
sqrt(\mu) = 4.9$ {\bf (1)}\\
$\frac{38-24}{4.9} = 2.86$ deviazioni standard. {\bf (1)}\\
$P(n\geq 38) = 50\% - P(t\geq 2.86) = 0.2\%${\bf (2)}\\

\vskip0.30cm {\bf \underline {Soluzione Esercizio 2. } } {\bf (10 punti)}\\
a) {\bf (5 punti)}\\
  $t=\frac{|520-470|}{20}=\frac{|420-470|}{20}=2.5$ {\bf(2)} \\
  $p(R>520 \,\Omega \, o \, R<420 \,\Omega)=2\cdot (50\%-Q(2.5))=1.24\%$ {\bf(1.5)} \\
  $N=10000*1.24\%=124$ {\bf(1.5)} \\
b) {\bf (2.5 punti)}\\
$R_{\mu,mis}=\overline{R}\pm \frac{\sigma}{\sqrt{25}}=(490\pm 4)\,\Omega$ {\bf(2.5)} \\
c) {\bf (2.5 punti)}\\
$t=\frac{|490-470|}{4}=5$ {\bf(1.5)} \\
Il valore misurato \`e lontano 5 deviazioni standard dal valore
atteso. Quindi i due valori non sono in accordo. {\bf(1)} \\

\vskip0.30cm {\bf \underline {Soluzione Esercizio 3. } } {\bf (10 punti)}\\
a) {\bf (4 punti)}\\
  $Y=\Delta L$, $X=m$, $Y=BX$ {\bf(1)} \\
  $B=\sum  m \Delta L / \sum m^2 = 14.83$ cm/kg {\bf(1)} \\
  $\sigma_B=\sigma_{\Delta L} / \sqrt{\sum m^2 }= 0.26$ cm/kg {\bf(1)} \\
  $B=(14.83 \pm 0.26) 10^{-2}$ m/kg {\bf(1)} \\
b) {\bf (3 punti)}\\
$g_{L}=B\cdot k$ \\
$\sigma_{g_{L}}/g_{L}=\sqrt{(\sigma_{B}/B)^2+(\sigma_{k}/k)^2}=2\%$  {\bf(1.5)} \\
$g_{L}=(1.720 \pm 0.034) $m/s$^2$ {\bf(1.5)} \\
c) {\bf (3 punti)}\\
$r_L=g_L^{-1/2}  \sqrt{M_L G} $\\
$\sigma_{r_L}/r_L=\frac{1}{2}\sigma_{g_{L}}/g_{L} =1\%$ {\bf(1.5)} \\
$r_L=(16.88\pm 0.17) 10^5$~m$=1688 \pm 17$~km {\bf(1.5)} \\

\end{document}
