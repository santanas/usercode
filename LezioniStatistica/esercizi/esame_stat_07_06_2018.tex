\documentclass[10pt,a4paper,fleqn]{article}
%\usepackage{a4p,ifthen,multicol,pifont,epsfig,latexsym,amsmath,amssymb}
\usepackage{ifthen,multicol,pifont,epsfig,latexsym,amsmath,amssymb}
\usepackage[left=1cm, right=1cm, top=4cm]{geometry}

%
\begin{document}
\pagestyle{empty}

%
%
\newsavebox{\savepar}
\newenvironment{boxit}{\begin{lrbox}{\savepar}
\begin{minipage}[b]{16.0cm}}
{\end{minipage}\end{lrbox}\fbox{\usebox{\savepar}}}
\newcommand{\myblist}[1]{\begin{list}{#1}{\setlength{\topsep}{1.8mm}
\setlength{\parskip}{0mm} \setlength{\partopsep}{0mm} \setlength{\parsep}{0mm}
\setlength{\itemsep}{0mm}}}
\newcommand{\myhfill}[1]{\hfill {\it #1} = \underline{$~~~~~~~~~~~~~~~~~~~~~~~~~~~$}}
\newcommand{\myhfiyn}[0]{\hfill $\Box$~{\sc si}~~~~~~~~$\Box$~{\sc no}}
\newcommand{\myhfild}[2]
{\hfill {\it #1}= \underline{$~~~~~~~~~~$};~ {\it #2}= \underline{$~~~~~~~~~~$}}
\newcommand{\myhfilt}[3]
{\hfill {\it #1}= \underline{$~~~~~~~~~~~~~$};~ 
{\it #2}= \underline{$~~~~~~~~~~~~~$};~{\it #3}= \underline{$~~~~~~~~~~~~~$}}
%

\setlength{\unitlength}{1mm} 
\setlength{\headheight}{0mm} % footheight

\newcommand{\s}{\text{s}}
\newcommand{\km}{\text{km}}
\newcommand{\kg}{\text{kg}}


\vspace*{-3.5cm}
\Large
%\begin{boxit}
{
\begin{center}
\vspace{0.1cm}
{\bf  Fisica 1 per Chimica (Canali A-E ed P-Z)} \\
{\bf Simulazione esame scritto di Laboratorio/Statistica 07/06/2018}   \\
docenti: Francesco Santanastasio, Paolo Gauzzi
   \\
\end{center}
\noindent\rule{13cm}{0.4pt}  \\
\normalsize
%\vspace{0.1cm}
%\begin{center}
%{\underline {Corso di Laurea:}}
%\end{center}
%
\begin{tabular}{ll}
{\underline{Nome:}}  \hspace{6cm} & {\underline{Cognome:}} \\[0.35cm]
{\underline {Matricola}} & {\underline {Aula:}}  \\[0.35cm]
{\underline {Canale:}} & {\underline {Docente:}}
\end{tabular}
\vspace{0.5cm}
}
\small

La durata del compito \`e 3 ore. I cellulari devono essere spenti. Non \`e possibile consultare libri
di testo o appunti personali. \`E possibile utilizzare una
calcolatrice ed il formulario fornito insieme al compito. \\
Riportare a penna (non matita) sul presente foglio i risultati
numerici finali (con unit\`a di misura
ed incertezze di misura). Nell'elaborato riportare sia lo svolgimento
dettagliato degli esercizi (indicando tutte le formule utilizzate ed i passaggi) che i risultati numerici. 

\vspace{0.3cm}
%\end{boxit}
\noindent\rule{13cm}{0.4pt}  \\

\enlargethispage{0.2cm}
\normalsize
{\bf \underline {Esercizio 1}} \\ ~\\
Durante le votazioni per eleggere il Presidente della Repubblica, dopo
che sono state scrutinate 1000 su 1007 schede, il candidato X ha
ottenuto 667 preferenze. 
\myblist{-}
\item[a)] Sulla base delle schede gi\`a scrutinate, stimare la probabilit\`a che ha il
  candidato X di ricevere un voto a favore. \\ ~\\
\myhfill{$p$}
\end{list}
Sulla base di questa stima, determinare:
\myblist{-}
\item[b)] la probabilit\`a che il candidato X non riceva nessun voto
  nelle ultime 7 schede\\~\\
\myhfill{$p(\mbox{0 voti su 7 schede})$}
\item[c)] la probabilit\`a che il candidato X riceva 5 o pi\`u voti
  nelle ultime 7 schede\\~\\
 \myhfill{$p(\ge \mbox{5 voti su 7 schede})$}
\end{list}

\vskip0.30cm {\bf \underline {Esercizio 2}} \\ ~\\
Il numero di guasti registrati alle diverse componenti di un certo
numero di computer della marca X, durante 5 anni, ha dato luogo
alla seguente distribuzione di dati:\\~\\
\begin{tabular}{|l|r|r|r|r|}
\hline
$n_k$ = numero di guasti in 5 anni & 0 & 1 & 2 & 3  \\
$O_k$ = occorrenza  & 255 & 100 & 26 & 8  \\
\hline
\end{tabular}~\\
\myblist{-}
\item[a)] Nell'ipotesi che il numero di guasti segua una distribuzione
  Poissoniana, calcolare la miglior stima del numero medio atteso
  ($\lambda$) di guasti in 5 anni. \\ ~\\
\myhfill{$\lambda$}
\end{list}
\myblist{-}
\item[b)] Ho acquistato 5 computer della marca X per il mio
  ufficio. Sulla base della stima ottenuta per $\lambda$, qual`\`e
  la probabilit\`a di non avere nessun guasto in 5 anni?\\~\\
\myhfill{$p(\mbox{nessun guasto in 5 anni})$}
\end{list}
\myblist{-}
\item[c)] Verificare la validit\`a dell'ipotesi che il numero di
  guasti segua una distribuzione Poissoniana utilizzando un test del
  chi-quadro con un livello di significativit\`a del 5\%. \\~\\
 \myhfill{Il chi-quadro ridotto $\tilde{\chi}^2_{mis}=\frac{\chi^2_{mis}}{\nu}${\it ($\nu$
  indica il numero di gradi di libert\`a)}}\\~\\
 \myhfill{La probabilit\`a $p_{\nu}(\tilde{\chi}^2>\tilde{\chi}^2_{mis})$}\\~\\
 IPOTESI ACCETTATA $\Box$ ~~~~~~~ IPOTESI RIGETTATA $\Box$
\end{list}


\newpage

\vskip0.30cm {\bf \underline {Esercizio 3}} \\ ~\\
In una ripetizione dell'esperimento di Joule (per la misura
dell'equivalente meccanico della caloria) si utilizza un motorino
di potenza $P=10.0 \pm 0.1 W$ (1 Watt = 1 Joule/secondo) che aziona un
mulinello inserito in un recipiente contenente un fluido di capacit\`a
termica $C=3.41 \pm 0.01 cal/K$. Tenendo il motorino in funzione per
un tempo $\Delta t$ viene registrata la variazione di temperatura
$\Delta T$ all'interno del fluido. La misura viene ripetuta 
per sei diversi valori di $\Delta t$ ed ogni volta il fluido viene
riportato alla stessa temperatura iniziale prima di azionare
nuovamente il motorino.  I dati dei sei esperimenti sono riportati
nella seguente tabella:
\\~\\
\begin{tabular}{|l|r|r|r|r|r|r|}
\hline
$\Delta t$ [s] & 10 & 20 & 30 & 40 & 50 & 60  \\
$\Delta T$ [K] & 7.1 & 13.9 & 21.3 & 28.3 & 34.9 & 41.6  \\
\hline
\end{tabular}~\\~\\
Le variazioni di temperatura sono note tutte con una incertezza di 0.5 K,
mentre l'incertezza sugli intervalli di tempo \`e trascurabile. \`E
noto inoltre che l'equivalente meccanico della caloria si calcola a
partire dalla seguente relazione $J_{EQ}=\frac{P\Delta t}{C \Delta
  T}$, ovvero esiste una relazione lineare tra la variazione di
temperatura e l'intervallo di tempo: $\Delta T=\frac{P}{C J_{EQ}}
\Delta t$
\myblist{-}
\item[a)] Verificare l'ipotesi di relazione lineare tra $\Delta T$ e $\Delta t$
  utilizzando il metodo dei minimi quadrati ed 
un test del chi-quadro con un livello di significativit\`a del 5\%.\\ ~\\
 \myhfill{Il chi-quadro ridotto $\tilde{\chi}^2_{mis}=\frac{\chi^2_{mis}}{\nu}${\it ($\nu$
  indica il numero di gradi di libert\`a)}}\\~\\
 \myhfill{La probabilit\`a $p_{\nu}(\tilde{\chi}^2>\tilde{\chi}^2_{mis})$}\\~\\
 IPOTESI ACCETTATA $\Box$ ~~~~~~~ IPOTESI RIGETTATA $\Box$
\end{list}
\myblist{-}
\item[b)] Determinare la miglior stima dell'equivalente meccanico
  della caloria $J_{EQ}$ e la sua incertezza (espresso in J/cal)\\~\\
\myhfill{$J_{EQ}$}
\end{list}
\myblist{-}
\item[c)] Stabilire se il valore misurato di $J_{EQ}$ \`e in accordo (``entro 2
  deviazioni standard'') con il valore accettato pari a 4.1855 J/cal \\~\\
 ACCORDO $\Box$ ~~~~~~~ DISACCORDO $\Box$
\end{list}

~\\~\\~\\~\\
{\it
NOTA:\\

Metodo dei minimi quadrati non pesati per una relazione lineare del tipo $y=Bx$\\
$B=\frac{\sum xy}{\sum x^2}$\\
$\sigma_B=\frac{\sigma_y}{\sqrt{\sum x^2}}$\\
Incertezze "a posteriori" sulle y (sulla base dei punti osservati):
$\sigma_y=\sqrt{\frac{1}{N-1} \sum_{i=1}^{N}(y_i-Bx_i)^2}$\\
}
\newpage

{\bf Soluzione Esercizio 1.}\\

a) \\
$p=\frac{667}{1000}=0.667$ \\

b) \\
$n=0$ , $N=7$ , $p=0.667$ \\
$p(\mbox{0 voti su 7 schede})=B_{N,p}(0)=\frac{7!}{0!(7-0)!}(0.667)^0
(1-0.667)^{7-0}=0.045\%$ \\

c) \\
$p(\ge \mbox{5 voti su 7 schede})=B_{N,p}(5) + B_{N,p}(6) + B_{N,p}(7)$ \\
$B_{N,p}(5)=\frac{7!}{5!(7-5)!}(0.667)^5
(1-0.667)^{7-5}=30.74\%$\\
$B_{N,p}(6)=\frac{7!}{6!(7-6)!}(0.667)^6
(1-0.667)^{7-6}=20.53\%$\\
$B_{N,p}(7)=\frac{7!}{7!(7-7)!}(0.667)^7
(1-0.667)^{7-7}=5.87\%$\\
$p(\ge \mbox{5 voti su 7 schede})=B_{N,p}(5) + B_{N,p}(6) +
B_{N,p}(7)\sim 57\%$ \\


{\bf Soluzione Esercizio 2.}\\

a) \\
$\lambda=\overline{n_k}=\frac{\sum_{k=1}^{Nbin} n_k O_k }{\sum_{k=1}^{Nbin} O_k } =
\frac{0\cdot 255 + 1 \cdot 100 + 2 \cdot 26 + 3 \cdot 8}{255+100+26+8}
= \frac{176}{389}=0.45$\\

b) \\
``successo'' $=$ nessun guasto in 5 anni per un certo computer $\rightarrow$ $p=P_{\lambda}(0)=\frac{0.45^{0} e^{-0.45}}{0!}=0.6376$\\
$N=5$ (numero di computer, numero di prove indipendenti)\\
$n = $ numero di successi in $N$ prove con probabilit\`a di successo pari a $p=0.6376$\\
$p(\mbox{nessun guasto in 5 anni})=B_{N,p}(n=5)=\frac{5!}{5!(5-5)!}(0.6376)^5 (1-0.6376)^{5-5}=0.105=10.5\%$\\

 c) \\
Il numero atteso di eventi in ciascun bin (ovvero per un certo
valore $n_k$) e' uguale ad $E_k=N\cdot p_k$ dove:\\
$N=\sum_k O_k=255+100+26+8=389$;\\
$p_k$ rappresenta la probabilita' di osservare un numero $n_k$ di
guasti assumendo una distribuzione Poissoniana
$P_{\lambda}(n_k)=\frac{\lambda^{n_k} e^{-\lambda}}{n_k!}$con valore atteso
$\lambda=0.45$.\\
$p_o=P_{\lambda}(0)=\frac{0.45^{0} e^{-0.45}}{0!}=0.6376$\\
$p_1=P_{\lambda}(1)=\frac{0.45^{1} e^{-0.45}}{1!}=0.2869$\\
$p_2=P_{\lambda}(2)=\frac{0.45^{2} e^{-0.45}}{2!}=0.0646$\\
$p_3=P_{\lambda}(3)=\frac{0.45^{3} e^{-0.45}}{3!}=0.0097$\\
I valori attesi sono quindi:\\
$E_0=N \cdot p_o=248$,
$E_1=N \cdot p_1=111.6$,
$E_2=N \cdot p_2=25.1$,
$E_3=N \cdot p_3=3.77$.\\


Per eseguire un test del chi-quadro \`e necessario avere almeno 5
eventi attesi in ciascun bin. Si sommano quindi i valori attesi relativi ad
$n_k=2$ ed $n_k=3$, ottenendo $E_{2+3}=E_2+E_3=28.87$\\

Si calcola il chi-quadro a partire dai dati in tabella: 
~\\~\\
\begin{tabular}{|l|r|r|}
\hline
$n_k$ & $O_k$ & $E_k$  \\
0        & 255 & 248  \\
1        & 100 & 111.6  \\
2 o 3  & 34 & 28.87  \\
\hline
\end{tabular}~\\~\\
ottenendo:\\
$\chi^2_{mis}=\sum_k \left(
  \frac{O_k-E_k}{\sqrt{E_k}}\right)^2=2.35$\\
il numero di gradi di libert\`a \`e $\nu=3-2=1$, essendo 3 i punti
sperimentali e 2 i parametri ($\lambda$ ed $N$) calcolati a partire
dai dati ed utilizzati per la stima dei valori attesi $E_k$.\\
Il chi-quadro ridotto \`e quindi
$\tilde{\chi}^2_{mis}=\frac{\chi^2_{mis}}{\nu}=\frac{2.35}{1}=2.35$.\\
La probabilit\`a associata \`e
$p_{\nu}(\tilde{\chi}^2>\tilde{\chi}^2_{mis})=12-14\%>\alpha=5\%$,
dove $\alpha$ e' il livello di significativit\`a del test.\\
L'ipotesi Poissoniana \`e quindi verificata.\\


{\bf Soluzione Esercizio 3.}\\

a) \\
Si applica il metodo dei minimi quadrati non pesati per una relazione
lineare del tipo $y=Bx$ dove $y=\Delta T$ ed $x=\Delta t$, con
incertezze gaussiane su $y$ tutte uguali ($\sigma_y=0.5$~K) 
ed incertezze su $x$ trascurabili.\\
$\sum xy = 6361$~sK \\
$\sum x^2=9100$~$s^2$\\
$B=\frac{\sum xy}{\sum x^2}=0.6990$~K/s\\
$\sigma_B=\frac{\sigma_y}{\sqrt{\sum x^2}}=0.0052$~K/s\\

Si esegue un test del chi-quadro sui risultati del fit ai dati.\\
$\chi^2_{mis}=\sum_i \left( \frac{y_i-Bx_i}{\sigma_y}
\right)^2=1.45$\\
numero di gradi di libert\`a $=\nu=6-1=5$ essendo 6 i punti
sperimentali e solo 1 il parametro ($B$) stimato dai dati per il
calcolo dei valori attesi di $y_i$. \\
Il chi-quadro ridotto \`e quindi
$\tilde{\chi}^2_{mis}=\frac{\chi^2_{mis}}{\nu}=\frac{1.45}{5}=0.29$.\\
La probabilit\`a associata \`e
$p_{\nu}(\tilde{\chi}^2>\tilde{\chi}^2_{mis})=85-96\%>\alpha=5\%$,
dove $\alpha$ e' il livello di significativit\`a del test.\\
L'ipotesi lineare \`e quindi accettata.\\

b) \\

$B=\frac{P}{C J_{EQ}}$ da cui $J_{EQ}=\frac{P}{CB}$ con:\\
$P=10.0 \pm 0.1 $~J/s\\
$C=3.41 \pm 0.01$~cal/K\\
$B=0.6990\pm 0.0052$~K/s\\

Utilizzando la formula di propagazione delle incertezze si ottiene:\\
$\frac{\delta J_{EQ}}{J_{EQ}}=\sqrt{(\delta P/P)^2 + (\delta C/C)^2 +
  (\delta B/B)^2}=1.3\%$ essendo\\
$\delta P/P=1\%$, $\delta C/C=0.3\%$, $\delta B/B=0.7\%$\\
$\delta_{J_{EQ}}=J_{EQ}\cdot 1.3\%$ e quindi la miglior stima di
$J_{EQ}$:
$J_{EQ}=(4.195 \pm 0.055)$ J/cal\\

c) \\

$t_{mis}=\frac{4.195-4.1855}{0.055}=0.17$ (il valore misurato dista
0.17 deviazioni standard dal valore accettato)\\ 
Quindi c'e' accordo (ben entro una deviazione standard) con
il valore accettato. 













 







\end{document}
