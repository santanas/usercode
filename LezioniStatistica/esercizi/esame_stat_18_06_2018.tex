\documentclass[10pt,a4paper,fleqn]{article}
%\usepackage{a4p,ifthen,multicol,pifont,epsfig,latexsym,amsmath,amssymb}
\usepackage{ifthen,multicol,pifont,epsfig,latexsym,amsmath,amssymb}
\usepackage[left=1cm, right=1cm, top=4cm]{geometry}

%
\begin{document}
\pagestyle{empty}

%
%
\newsavebox{\savepar}
\newenvironment{boxit}{\begin{lrbox}{\savepar}
\begin{minipage}[b]{16.0cm}}
{\end{minipage}\end{lrbox}\fbox{\usebox{\savepar}}}
\newcommand{\myblist}[1]{\begin{list}{#1}{\setlength{\topsep}{1.8mm}
\setlength{\parskip}{0mm} \setlength{\partopsep}{0mm} \setlength{\parsep}{0mm}
\setlength{\itemsep}{0mm}}}
\newcommand{\myhfill}[1]{\hfill {\it #1} = \underline{$~~~~~~~~~~~~~~~~~~~~~~~~~~~$}}
\newcommand{\myhfiyn}[0]{\hfill $\Box$~{\sc si}~~~~~~~~$\Box$~{\sc no}}
\newcommand{\myhfild}[2]
{\hfill {\it #1}= \underline{$~~~~~~~~~~$};~ {\it #2}= \underline{$~~~~~~~~~~$}}
\newcommand{\myhfilt}[3]
{\hfill {\it #1}= \underline{$~~~~~~~~~~~~~$};~ 
{\it #2}= \underline{$~~~~~~~~~~~~~$};~{\it #3}= \underline{$~~~~~~~~~~~~~$}}
%

\setlength{\unitlength}{1mm} 
\setlength{\headheight}{0mm} % footheight

\newcommand{\s}{\text{s}}
\newcommand{\km}{\text{km}}
\newcommand{\kg}{\text{kg}}


\vspace*{-3.5cm}
\Large
%\begin{boxit}
{
\begin{center}
\vspace{0.1cm}
{\bf  Fisica 1 per Chimica (Canali A-E ed P-Z)} \\
{\bf Esame scritto di Laboratorio/Statistica 18/06/2018}   \\
docenti: Francesco Santanastasio, Paolo Gauzzi
   \\
\end{center}
\noindent\rule{18cm}{0.4pt}  \\
\normalsize
%\vspace{0.1cm}
%\begin{center}
%{\underline {Corso di Laurea:}}
%\end{center}
%
\begin{tabular}{ll}
{\underline{Nome:}}  \hspace{6cm} & {\underline{Cognome:}} \\[0.35cm]
{\underline {Matricola}} & {\underline {Aula:}}  \\[0.35cm]
{\underline {Canale:}} & {\underline {Docente:}}
\end{tabular}
\vspace{0.5cm}
}
\small

La durata del compito \`e 3 ore. I cellulari devono essere spenti. Non \`e possibile consultare libri
di testo o appunti personali. \`E possibile utilizzare una
calcolatrice ed il formulario fornito insieme al compito. \\
Riportare a penna (non matita) sul presente foglio i risultati
numerici finali (con unit\`a di misura
ed incertezze di misura). Nell'elaborato riportare sia lo svolgimento
dettagliato degli esercizi (indicando tutte le formule utilizzate ed i passaggi) che i risultati numerici. 

\vspace{0.3cm}
%\end{boxit}
\noindent\rule{18cm}{0.4pt}  \\

\enlargethispage{0.2cm}
\normalsize

% IPOTESI ACCETTATA $\Box$ ~~~~~~~ IPOTESI RIGETTATA $\Box$
%ACCORDO $\Box$ ~~~~~~~ DISACCORDO $\Box$

\vskip0.30cm {\bf \underline {Esercizio 1}} \\
Per misurare la costante elastica $k$ di una molla se ne misura la
lunghezza $\ell$ appendendovi oggetti di massa nota $M$, ottenendo i risultati riportati in
tabella. L'incertezza sulla misura delle lunghezze \`e 0.005 m, quella sulle
masse \`e trascurabile.

\begin{table}[h]
\begin{tabular}{|c|c|}
\hline
$M$ (kg) & $\ell$ (m) \\
\hline
0.5 & 0.112 \\ 
0.8 & 0.143 \\
1.0 & 0.167 \\
1.2 & 0.192 \\
\hline
\end{tabular}
\end{table}

Il comportamento della molla \`e descritto dalla legge di Hooke:
\begin{displaymath}
Mg=k(\ell-\ell_0)
\end{displaymath} 
dove $\ell_0$ \`e la lunghezza della molla a riposo. L'accelerazione
di gravit\`a \`e pari a $g=9.81$~m/s$^2$ con incertezza trascurabile.
\myblist{-}
\item[a)] Verificare la relazione lineare tra $\ell$ ed $M$ utilizzando
  il metodo dei minini quadrati ed un test del $\chi^2$ con un livello di
  significativit\`a del 5\%. \\
$~$ \myhfill{Chi-quadro ridotto $\tilde\chi^2_{mis}$} \\
$~$ \myhfill{Probabilit\`a $P(\tilde\chi^2>\tilde\chi^2_{mis})$} \\
IPOTESI ACCETTATA $\Box$ ~~~~~~~ IPOTESI RIGETTATA $\Box$
\item[b)] Determinare la migliore stima della costante elastica della molla
  con la sua incertezza. \\
$~$\myhfill{$k$}
\item[c)] Determinare il valore aspettato della lunghezza della molla $\ell_m$,
  e la sua incertezza, quando vi si appende una massa $m=(2.1\pm 0.2)$ kg \\
$~$ \myhfill{$\ell_m$}
\end{list}

{\it Suggerimento: per il punto c) utilizzare la formula di propagazione delle
  incertezze assumendo che i parametri della relazione lineare, 
ottenuti con il metodo dei minimi quadrati, siano variabili casuali indipendenti.}\\

\newpage

%\vskip 1.0cm 
{\bf \underline {Esercizio 2}} \\
Per studiare l'eventualit\`a che un campione di roccia sia radioattivo, si
pone nelle sue vicinanze un contatore a scintillazione e si osservano 231
conteggi in 10 minuti di misura. \\
Per misurare la radioattivit\`a di fondo ambientale si toglie il campione
di roccia e si lascia il contatore in misura per 30 minuti, durante i quali
si ottengono 456 conteggi.\\

\myblist{-}
\item[a)] Determinare il tasso di conteggi al minuto con ($R_1$) e senza
  ($R_2$) il campione di roccia, con le corrispondenti incertezze. \\
$~$ \myhfill{$R_1$}\\
$~$ \myhfill{$R_2$} \\
\item[b)]   Qual \`e la migliore stima del tasso di attivit\`a del campione
  di roccia ? \myhfill{$R_{roccia}$} \\
                                                            
  
\item[c)] C'\`e evidenza (``oltre 3 deviazioni standard'') che la
  roccia sia radioattiva ?

SI $\Box$ ~~~~~~~ NO $\Box$

\end{list}
~\\\\

{\bf \underline {Esercizio 3}} \\
Un'azienda produce barattoli di passata di pomodoro. Una macchina si
occupa di riempire i barattoli di passata. La quantit\`a di
passata messa in ciascun barattolo  pu\`o essere
regolata ed \`e una variabile casuale $X$ che
segue una distribuzione gaussiana con media $\mu_X$ (modificabile
attraverso le impostazioni della macchina) e deviazione standard 
fissa $\sigma_X=25$~g. \\
 
\myblist{-}
\item[a)] Determinare il valore $\mu_{X}$ a cui deve essere impostata la macchina 
affinch\'e solo il 2\% dei barattoli riempiti contenga meno di 500 g di passata di
pomodoro.\\
$~$ \myhfill{$\mu_{X}$}
\end{list}

La macchina viene quindi impostata al valore $\mu_{X}$  trovato nel
punto a) per la produzione. I barattoli vuoti sono di metallo e la loro massa \`e una
variabile casuale $Y$ che segue una distribuzione gaussiana con media 90 g
e deviazione standard 8 g. Un ispettore pesa i barattoli pieni di
passata di pomodoro e scarta quelli la cui massa totale (barattolo$+$passata) \`e inferiore a 590 g. \\

\myblist{-}
\item[b)] Calcolare la percentuale dei barattoli che viene scartata
  dall'ispettore.\\
$~$ \myhfill{$p$}
\item[c)] L'ispettore esegue il controllo di 5 barattoli. Calcolare la
  probabilit\`a che almeno 4 barattoli superino il controllo.\\\\
$~$ \myhfill{$p$}
\end{list}

~\\
{\it Suggerimento: la massa totale di barattolo+passata \`e una
  variabile casuale Z=X+Y gaussiana.}


\newpage

{\bf -1 punto} ogni 3 errori di questo tipo:\\
- unit\`a di misura non riportate o riportate incorrettamente \\
- errori di calcolo (procedimento e formule ok ma risultato numerico 
significativamente sbagliato)

\vskip0.30cm {\bf \underline {Soluzione Esercizio 1. } } {\bf (10 punti)}\\

a) {\bf (4 punti)}\\
   $\ell=\ell_0+\frac{g}{k}M$ {\bf (1)} \\ 
   $A=\ell_0$, $B=\frac{g}{k}$ \\ 
   $A=0.05357\pm 0.0088$ m {\bf (0.5)} \\
   $B=0.1142\pm 0.0097$ ${\rm m^2/Ns^2}$ {\bf (0.5)}\\ 
   $\chi^2_{mis}=\frac{\sum_i(\ell_i-A-BM_i)^2}{\sigma^2_{\ell}}=0.32$,
   con   4 - 2 = 2 gradi di libert\`a {\bf (1)}\\
   $\tilde\chi^2_{mis}=0.16$ \\
   $P(\tilde\chi^2>\tilde\chi^2_{mis})>82\%>\alpha=5\%$.
   Ipotesi accettata. {\bf (1)}

b) {\bf (4 punti)} \\ 
  $\sigma_A = 0.0088$m , $\sigma_B=0.0097$ ${\rm m^2/Ns^2}$ {\bf (1)}\\ 
  $k=\frac{g}{B}$, $\frac{\sigma_k}{k}=\frac{\sigma_B}{B}=8.5\%$. {\bf (2)}\\
   Migliore stima: $k=(85.9\pm 7.3)$ N/m {\bf (1)}   

c) {\bf (2 punti)} \\ 
$\ell = A + B m$\\
$\sigma_{\ell}=\sqrt{ (\frac{\partial \ell}{\partial m} \sigma_m)^2
+ (\frac{\partial \ell}{\partial B} \sigma_B)^2 + (\frac{\partial
  \ell}{\partial A}
\sigma_A)^2}=\sqrt{B^2\sigma_m^2+m^2\sigma^2_B+\sigma_A^2}=0.032$~m
{\bf (1)} \\
 $\ell=(0.293\pm 0.032)$ m {\bf (1)}

\vskip0.30cm {\bf \underline {Soluzione Esercizio 2. } } {\bf (10 punti)}\\

a) {\bf (4 punti)} \\
$R_1=\frac{231\pm\sqrt{231}}{10}=(23.1\pm 1.5)$ conteggi/min {\bf (2)}\\
$R_2=\frac{456\pm\sqrt{456}}{30}=(15.2\pm 0.7)$ conteggi/min {\bf (2)}\\
b) {\bf (4 punti)} \\
$\sigma_{R_{roccia}}=\sqrt{\sigma^2_{R_{1}}+\sigma^2_{R_{2}}}=1.7$~conteggi/min
{\bf (2)} \\
$R_{roccia}=R_1-R_2=(7.9\pm 1.7)$ conteggi/min {\bf (2)}\\
c) {\bf (2 punti)} \\
$t=\frac{|R_{roccia}-0|}{\sigma_{R_{roccia}}}=4.65>3$. Quindi la
roccia \`e radioattiva. {\bf (2)}

\vskip0.30cm {\bf \underline {Soluzione Esercizio 3. } } {\bf (10 punti)} \\

a) {\bf (4 punti)} \\
$x_{min}=500$~g\\ 
$P(x<x_{min})=50\% - Q(t_{min})=2\%$ ~~~~~ (dove
$Q(t)=\int_{\mu_X}^{\mu_X+t\sigma_X} G_{\mu_X,\sigma_X} (x) dx$) {\bf (1)}\\
$Q(t_{min})=48\%$ da cui $t_{min}=2.05$ (dalle tabelle di
probabilit\`a gaussiana) {\bf (1)}\\
Essendo $t_{min}=\frac{|x_{min}-\mu_X|}{\sigma_X}$ si ottiene:\\
$\mu_X=x_{min}+t_{min}\cdot\sigma_X=551.25$~g {\bf (2)}

b) {\bf (4 punti)} \\
$Z=X+Y$ gaussiana\\
$\mu_Z=\mu_X+\mu_Y=641.25$~g {\bf (1)} \\
$\sigma_Z=\sqrt{\sigma^2_X+\sigma^2_Y}=26.25$~g {\bf (1)}\\
$z_{min}=590$~g\\
$t_{min}=\frac{|z_{min}-\mu_z|}{\sigma_z}=1.95$ {\bf (1)} \\
Probabilit\`a di scartare il barattolo
$=P(z<z_{min})=50\%-Q(t_{min})=2.6\%$ {\bf (1)}

c) {\bf (2 punti)} \\
Problema binomiale\\
Successo $=$ il barattolo supera il controllo \\
Probabilit\`a di successo $=p=100\%-2.6\%=97.4\%$ {\bf (1)} \\
Numero di prove $N=5$ (numero di barattoli)\\
$n = $ numero di successi in $N$ prove segue una distribuzione
binomiale $B_{N,p}(n)$.\\
$B_{N,p}(4)=0.1170$\\
$B_{N,p}(5)=0.8766$\\
$P(n\ge 4)=B_{N,p}(4)+B_{N,p}(5)=99.4\%$ {\bf (1)}

\end{document}
