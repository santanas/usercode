%%
%% This is file `cimsmple.tex',
%% generated with the docstrip utility.
%%
%% The original source files were:
%%
%% cimento.dtx  (with options: `sample')
%% 
%% IMPORTANT NOTICE:
%% 
%% For the copyright see the source file.
%% 
%% Any modified versions of this file must be renamed
%% with new filenames distinct from cimsmple.tex.
%% 
%% For distribution of the original source see the terms
%% for copying and modification in the file cimento.dtx.
%% 
%% This generated file may be distributed as long as the
%% original source files, as listed above, are part of the
%% same distribution. (The sources need not necessarily be
%% in the same archive or directory.)
%%%%%%%%%%%%%%%%%%%%%%%%%%%%%%%%%%%%%%%%%%%%%%%%%%
%%%%%%%%%%%%%%%%%%%%%%%%%%%%%%%%%%%%%%%%%%%%%%%%%%
%%%%%%%%%%%%%%%%%%%%%%%%%%%%%%%%%%%%%%%%%%%%%%%%%%
\ProvidesFile{cimsmple.tex}
      [1999/12/01 v1.4c Il Nuovo Cimento]
\documentclass{cimento}

%% \documentclass[rivista]{cimento} Use the option rivista for La Rivista del
%Nuovo Cimento

%%%%%%%%%%%%%
             %
               %    % If you are preparing Enrico Fermi School of
%VERY IMPORTANT  %  % Physics report, please read the bundled file
	       %    % README.varenna 
             %
%%%%%%%%%%%%


%\usepackage{graphicx}  % got figures? uncomment this
%\title{Prospects for exotica searches at ATLAS and CMS {\mdseries\ttfamily cimento} class}
\title{Prospects for Exotica Searches at ATLAS and CMS}
\author{F.~Santanastasio\from{ins:UMD}\ETC
%R.~Drake\from{ins:x}\\
%        \atque
%Mr.~M\from{ins:evil}\thanks{The bad fellow.}
}
\instlist{\inst{ins:UMD} Department of Physics, University of Maryland \\
College Park, MD, 20742, USA \\
E-mail: francesco.santanastasio@cern.ch \\
for the ATLAS and CMS collaborations}

%\PACSes{\PACSit{00.00}{By the way, which PACS is it, the 00.00? GOK.}
%\PACSit{---.---}{\ldots}}
\begin{document}

\maketitle

\begin{abstract}
This paper presents the prospects for the search of exotic physics 
beyond the Standard Model with the Large Hadron Collider at CERN. 
The results presented here are based on Montecarlo simulations of the
ATLAS and CMS detectors, assuming a scenario with 
100 pb$^{-1}$ of collected integrated luminosity and proton-proton collisions 
at $\sqrt{s} = 14$~TeV. A selection of benchmark analyses is discussed, 
including searches of new physics in the di-lepton, di-jet, and lepton-jet channel, 
and the description of techniques to identify the production of 
heavy stable charged particles using muon detectors. 
The impact on ATLAS and CMS discovery potential 
of having collisions at an energy lower than the design of the machine 
is also discussed.
\end{abstract}

\section{Introduction}
The Standard Model (SM) of fundamental interactions is a successful theory 
describing strong, weak and electromagnetic interactions of elementary 
particles. The SM has been verified with high accuracy by several experiments 
in the last decades, and no deviations from theoretical expectations 
have been observed. In spite of the perfect agreement with all experimental 
observations, the SM has its natural drawbacks and unsolved theoretical 
problems, ranging from the origin of the particle masses to the nature of the 
Dark Matter in the Universe.

There are several alternative theories to the SM which try to solve such 
open issues. In these models, new physics, in terms of new particles and 
new interactions, is expected to be visible at the TeV energy scale, and 
thus might be discovered at the Large Hadron Collider (LHC) at CERN.
Among them Supersymmetry (SUSY) is one of the plausible theories for the physics 
beyond the SM. In addition, several other theoretical approaches (generally classified 
as ``Exotica'') has been proposed, including for example theories with Extra Dimensions, 
Grand Unification Theories (GUT), and String Theory. 

%Within the experiments at LHC such ``non-SUSY'' approches, 
%and their relative experimental signatures, are generally classified 
%as ``Exotica'' to distinguish them from the standard SUSY searches. 
%In this convention, the boundary between the SUSY and Exotica 
%is not always fully consistent among different experiments, thus 
%resulting, for example, that the weirder SUSY signals 
%(as the ones expected from those SUSY models that predicts 
%heavy stable charged particles) might be included as part 
%of the Exotica signatures. 

This paper presents a brief overview of the Exotica searches at the LHC. 
At the present time the LHC is still in commissioning phase 
and the first collisions are expected in the Fall of 2009.
The results presented here are therefore based on detailed 
Montecarlo simulations of the ATLAS and CMS detectors, 
assuming a scenario with 100 pb$^{-1}$ of collected integrated luminosity 
and proton-proton collisions with an energy in the center of mass $\sqrt{s} = 14$~TeV. 
A selection of ATLAS and CMS four benchmark analyses with different 
experimental issues is discussed, that includes 
the search of new physics in the di-lepton~(Section~\ref{dilepton}), 
di-jet~(Section~\ref{dijet}), and lepton-jet~(Section~\ref{leptonjet}) channel, 
and the description of experimental techniques to identify the production of 
heavy stable charged particles using muon detectors~(Section~\ref{HSCP}).
It has been announced that the first LHC physics run 
will be taken at a lower energy ($\sqrt{s} = 10$~TeV) 
than the design of the machine, with approximately 100-300 pb$^{-1}$ 
of integrated luminosity collected. 
The impact of the reduced accelerator energy on the ATLAS and CMS 
discovery potential is discussed in Section~\ref{10TeVvs14TeV}. 
Conclusions are given in Section~\ref{Conclusion}.

\section{Di-lepton channel} \label{dilepton}

\section{Di-jet channel} \label{dijet}
 
\section{Lepton-jet channel} \label{leptonjet}

\section{Heavy stable charged particles} \label{HSCP}

\section{Discovery potential - 10 TeV vs 14 TeV} \label{10TeVvs14TeV}

\section{Conclusion} \label{Conclusion}










%\texttt{cimento}.

%\begin{figure}
%\includegraphics{foo}     % includes figure foo.eps
%\caption{Foo onteracting with bar}
%\end{figur

%This sample is not meant to provide documentation for the class;
%to obtain the class `manual' typeset the main distribution file,
%\texttt{cimento.dtx}.

%You can also use this file as a template for your own paper:
%copy it to another filename and then modify as needed.

%\section{Examples}

%\subsection{Tables}
%Tables~\ref{tab:pricesI}, \ref{tab:pricesII} and~\ref{tab:pricesIII}
%inserted at this point.

%\begin{table}
%  \caption{Again, this time with \texttt{narrowtabular}}
%  \label{tab:pricesIII}
%  \begin{narrowtabular}{2cm}{rcl}
%    \hline
%      Ice-cream      & !1500   & lire    \\
%      More ice-cream & 15000   & lire    \\
%      Crocodile      & !1500   & dollars \\
%    \hline
%      Phone call     & !.25    & dollars \\
%      X-Men          & 1.25    & dollars \\
%      Dollar         & 1?{.}!! & dollars \\
%    \hline
%  \end{narrowtabular}
%\end{table}


%\subsection{Mathematics}
%Here is a lettered array~(\ref{e.all}), with eqs.~(\ref{e.house})
%and~(\ref{e.phi}):
%\begin{eqnletter}
% \label{e.all}
% \drm x_\sy{F} & = & 1.2\cdot10^3\un{cm}, \qquad
%                     \tx{where\ } \sy{F} = \tx{Fermi}    \label{e.house}\\
% \phi_i        & = & i\pi                                \label{e.phi}
%\end{eqnletter}

%\subsection{Citations}
%We're almost done, just some citations~\cite{ref:apo}
%and we will be over~\cite{ref:pul,ref:bra}.


%\appendix

%\section{}
%Let us go then, you and I\ldots

\acknowledgments
ADD HERE

\begin{thebibliography}{0}
\bibitem{ref:apo} \BY{Boccaccio~G. \atque de~Cam\~oes~L.}
  \IN{Phys. Rev. A}{13}{1999}{12};
  \SAME{69}{999}{1666}.
\bibitem{ref:pul} \BY{Pulci~L.}
  preprint INFN 8181.
\bibitem{ref:bra} \BY{Bragg~B.}
  \TITLE{Tender comrade},
  in \TITLE{Workers Playtime},
                  edited by \NAME{Tizio A. \atque Caio B.}
                  (Unexeditor, Bologna) 1997, pp.~1-10.
\end{thebibliography}

\end{document}
\endinput
%%
