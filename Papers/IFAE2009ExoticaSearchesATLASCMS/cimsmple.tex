%%
%% This is file `cimsmple.tex',
%% generated with the docstrip utility.
%%
%% The original source files were:
%%
%% cimento.dtx  (with options: `sample')
%% 
%% IMPORTANT NOTICE:
%% 
%% For the copyright see the source file.
%% 
%% Any modified versions of this file must be renamed
%% with new filenames distinct from cimsmple.tex.
%% 
%% For distribution of the original source see the terms
%% for copying and modification in the file cimento.dtx.
%% 
%% This generated file may be distributed as long as the
%% original source files, as listed above, are part of the
%% same distribution. (The sources need not necessarily be
%% in the same archive or directory.)
%%%%%%%%%%%%%%%%%%%%%%%%%%%%%%%%%%%%%%%%%%%%%%%%%%
%%%%%%%%%%%%%%%%%%%%%%%%%%%%%%%%%%%%%%%%%%%%%%%%%%
%%%%%%%%%%%%%%%%%%%%%%%%%%%%%%%%%%%%%%%%%%%%%%%%%%
\ProvidesFile{cimsmple.tex}
      [1999/12/01 v1.4c Il Nuovo Cimento]
\documentclass{cimento}

%% \documentclass[rivista]{cimento} Use the option rivista for La Rivista del
%Nuovo Cimento

%%%%%%%%%%%%%
             %
               %    % If you are preparing Enrico Fermi School of
%VERY IMPORTANT  %  % Physics report, please read the bundled file
	       %    % README.varenna 
             %
%%%%%%%%%%%%


%\usepackage{graphicx}  % got figures? uncomment this
%\title{Prospects for exotica searches at ATLAS and CMS {\mdseries\ttfamily cimento} class}
\title{Prospects for Exotica Searches at ATLAS and CMS}
\author{F.~Santanastasio\from{ins:UMD}\ETC
%R.~Drake\from{ins:x}\\
%        \atque
%Mr.~M\from{ins:evil}\thanks{The bad fellow.}
}
\instlist{\inst{ins:UMD} Department of Physics, University of Maryland \\
College Park, MD, 20742, USA \\
E-mail: francesco.santanastasio@cern.ch \\
for the ATLAS and CMS collaborations}

%\PACSes{\PACSit{00.00}{By the way, which PACS is it, the 00.00? GOK.}
%\PACSit{---.---}{\ldots}}
\begin{document}

\maketitle

\begin{abstract}
This paper presents the prospects for the search of exotic physics 
beyond the Standard Model at the Large Hadron Collider. 
\end{abstract}

\section{Description}
This is a very short sample paper distributed with the class
\texttt{cimento}.
It is just a collection of examples about the syntax of commands
which behave in a different way from the standard \LaTeX
and/or new commands not defined in \LaTeX.


%\begin{figure}
%\includegraphics{foo}     % includes figure foo.eps
%\caption{Foo onteracting with bar}
%\end{figure}


This sample is not meant to provide documentation for the class;
to obtain the class `manual' typeset the main distribution file,
\texttt{cimento.dtx}.

You can also use this file as a template for your own paper:
copy it to another filename and then modify as needed.

\section{Examples}

\subsection{Tables}
Tables~\ref{tab:pricesI}, \ref{tab:pricesII} and~\ref{tab:pricesIII}
inserted at this point.

\begin{table}
  \caption{Prices of important items.}
  \label{tab:pricesI}
  \begin{tabular}{rcl}
    \hline
      Ice-cream      & 1500  & lire    \\
      More ice-cream & 15000 & lire    \\
      Crocodile      & 1500  & dollars \\
    \hline
      Phone call     & .25   & dollars \\
      X-Men          & 1.25  & dollars \\
      Dollar         & 1     & dollars \\
    \hline
  \end{tabular}
\end{table}

\begin{table}
  \caption{Same thing, but formatted using \texttt{!} and \texttt{?}.}
  \label{tab:pricesII}
  \begin{tabular}{rcl}
    \hline
      Ice-cream      & !1500   & lire    \\
      More ice-cream & 15000   & lire    \\
      Crocodile      & !1500   & dollars \\
    \hline
      Phone call     & !.25    & dollars \\
      X-Men          & 1.25    & dollars \\
      Dollar         & 1?{.}!! & dollars \\
    \hline
  \end{tabular}
\end{table}

\begin{table}
  \caption{Again, this time with \texttt{narrowtabular}}
  \label{tab:pricesIII}
  \begin{narrowtabular}{2cm}{rcl}
    \hline
      Ice-cream      & !1500   & lire    \\
      More ice-cream & 15000   & lire    \\
      Crocodile      & !1500   & dollars \\
    \hline
      Phone call     & !.25    & dollars \\
      X-Men          & 1.25    & dollars \\
      Dollar         & 1?{.}!! & dollars \\
    \hline
  \end{narrowtabular}
\end{table}


\subsection{Mathematics}
Here is a lettered array~(\ref{e.all}), with eqs.~(\ref{e.house})
and~(\ref{e.phi}):
\begin{eqnletter}
 \label{e.all}
 \drm x_\sy{F} & = & 1.2\cdot10^3\un{cm}, \qquad
                     \tx{where\ } \sy{F} = \tx{Fermi}    \label{e.house}\\
 \phi_i        & = & i\pi                                \label{e.phi}
\end{eqnletter}

\subsection{Citations}
We're almost done, just some citations~\cite{ref:apo}
and we will be over~\cite{ref:pul,ref:bra}.


\appendix

\section{}
Let us go then, you and I\ldots

\acknowledgments
This work was produced, supported and perpetrated by S. Frabetti under
the auspices of the Italian Physical Society.
Grazie to M. Missiroli for the valuable collaboration.

\begin{thebibliography}{0}
\bibitem{ref:apo} \BY{Boccaccio~G. \atque de~Cam\~oes~L.}
  \IN{Phys. Rev. A}{13}{1999}{12};
  \SAME{69}{999}{1666}.
\bibitem{ref:pul} \BY{Pulci~L.}
  preprint INFN 8181.
\bibitem{ref:bra} \BY{Bragg~B.}
  \TITLE{Tender comrade},
  in \TITLE{Workers Playtime},
                  edited by \NAME{Tizio A. \atque Caio B.}
                  (Unexeditor, Bologna) 1997, pp.~1-10.
\end{thebibliography}

\end{document}
\endinput
%%
%% End of file `cimsmple.tex'.
