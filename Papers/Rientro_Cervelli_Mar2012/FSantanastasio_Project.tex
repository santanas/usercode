%------------------------------------
% Dario Taraborelli
% Typesetting your academic CV in LaTeX
%
% URL: http://nitens.org/taraborelli/cvtex
% DISCLAIMER: This template is provided for free and without any guarantee 
% that it will correctly compile on your system if you have a non-standard  
% configuration.
%------------------------------------ 


% ! TEX TS-program = XeLaTeX -xdv2pdf
% ! TEX encoding = UTF-8 Unicode

\documentclass[10pt, a4paper]{article}
\usepackage{fontspec} 
\usepackage{xunicode} 
\usepackage{xltxtra}
% per le lettere accentate italiane sul Mac! :-)
%\usepackage[applemac]{inputenc} %with VIM
\usepackage[latin1]{inputenc} % with TeXShop


% DOCUMENT LAYOUT
\usepackage{geometry}
\geometry{a4paper, textwidth=5.5in, textheight=8.5in, marginparsep=7pt, marginparwidth=.6in}
\setlength\parindent{0in}

% ADDITIONAL SYMBOLS
%\usesymbols[mvs]

% FONTS
\defaultfontfeatures{Mapping=tex-text} % converts LaTeX specials (``quotes'' --- dashes etc.) to unicode
%\setromanfont [Ligatures={Common}, BoldFont={Fontin Bold}, ItalicFont={Fontin Italic}]{Fontin}
\setromanfont [Ligatures={Common}, BoldFont={Linux Libertine Bold}, ItalicFont={Linux Libertine Italic}]{Linux Libertine}
%\setsansfont [Ligatures={Common}, BoldFont={Fontin Sans Bold}, ItalicFont={Fontin Sans Italic}]{Fontin Sans}
\setmonofont[Scale=0.8]{Monaco} 
% ---- CUSTOM AMPERSAND
\newcommand{\amper}{{\fontspec[Scale=.95]{Linux Libertine Bold}\selectfont\itshape\&}}
% ---- MARGIN YEARS
%\newcommand{\years}[1]{\marginpar{\scriptsize #1}}
\newcommand{\years}[1]{\marginpar{\footnotesize #1}}

% HEADINGS
\usepackage{sectsty} 
\usepackage[normalem]{ulem} 
\sectionfont{\rmfamily\mdseries\upshape\Large}
\subsectionfont{\rmfamily\bfseries\upshape\normalsize} 
\subsubsectionfont{\rmfamily\mdseries\upshape\normalsize} 
%modifying section numbering
\def\thesubsection{\arabic{subsection}.\ } 

% PDF SETUP
% ---- FILL IN HERE THE DOC TITLE AND AUTHOR
\usepackage[dvipdfm, bookmarks, colorlinks, breaklinks, pdftitle={Francesco Santanastasio - Motivation for coming to CERN},pdfauthor={Francesco Santanastasio}]{hyperref}
%\hypersetup{linkcolor=blue,citecolor=blue,filecolor=black,urlcolor=blue} 
\hypersetup{linkcolor=cyan,citecolor=blue,filecolor=black,urlcolor=cyan} 

% Title of Bibliography
%\renewcommand\refname{References \\ \normalsize \begin{center} \quad \quad \textsc{Publications (relative to research activities)}\end{center} }

% DOCUMENT
\begin{document}
\reversemarginpar

%%\hrule
\section*{Research Project}

% EW SYMMETRY BREAKING IN SM
One century of experimental measurements and progress in theoretical physics 
led to an extremely compact and elegant theory of fundamental interactions between 
elementary particles, the Standard Model (SM). Its success in reproducing 
measurements from different experiments in energy regimes spanning 
over several orders of magnitude is astonishing. 
Strong, weak and electromagnetic interactions are all described within the 
same mathematical framework of gauge theories. Although the electromagnetic and weak 
interactions are related to the same SU(2)$_L$ x U(1)$_Y$ invariance, only 
the electromagnetic symmetry is manifest in the mass spectrum. 
The rest of the electroweak symmetry is hidden, that is, it is spontaneously broken. 
The detailed mechanism through which the breaking happens is not clear, though. 
The simplest way this could be explained theoretically is through the 
so-called Higgs mechanism of the SM. This mechanism explain, for instance, why 
elementary particles have mass. The Higgs mechanism postulates the existence 
of a new scalar particle, the Higgs boson, whose mass is not theoretically predicted by 
the SM, but that should be experimentally observable at particle colliders. 

% PHYSICS BEYOND SM
Beyond this, there are many unsatisfactory aspects in the picture depicted 
by the SM: the hierarchy problem, i.e. the big gap between the electroweak 
energy scale and the Planck scale at which quantum effects of gravity become strong, 
is seen as one of its major limitation and has been the driving force for many 
theoretical developments extending the SM. Although the panorama of 
alternative new physics models is very wide, the most appealing and 
popular ones are represented by Supersymmetry and Extra Dimensions. 

% LHC 
The Large Hadron Collider (LHC) is the largest proton-proton (pp) collider ever built. 
It is located at CERN, Geneva, and its main objective is to finally unravel the origin 
of the electroweak symmetry breaking, as well as to share lights on the possible 
new physics beyond the SM. Using the pp collision data at the center-of-mass energy
of 7 TeV ($\sqrt{s}=7$ TeV) collected in 2011, ATLAS and CMS, the 
largest experiments at the LHC, excluded the SM Higgs in the mass range 
127-600 GeV, while masses below 114 GeV were already excluded by 
previous experiments at the electron-positron LEP collider. 
Thus, the mass range 114 GeV-127 GeV is currently the only one in which 
a Standard Model Higgs boson can hide, and it is in fact also the range 
preferred by the electroweak precision tests performed at LEP. 
In this mass range, the ATLAS and CMS experiments observe an excursion of 
the observed data from the expected background, that is compatible with the existence 
of a SM Higgs boson with mass around 125 GeV. However, no claim of 
discovery is possible at the moment given the small statistical significance of the excess. 
On the other hand, no sign of new physics beyond the SM have been observed so far
from the results of the LHC experiments.

% THE 2012 RUN
In 2012, the LHC will collide protons at a center-of-mass energy of 8 TeV, delivering 
a number of collisions three times higher than in 2011. The higher energy and larger 
amount of data will allow to either confirm the "125 GeV" signal or rule out the existence 
of a SM Higgs by the end of the year. The LHC is scheduled to enter a 
long technical stop at the end of 2012 to prepare for running at its full design 
center-of-mass energy of around 14 TeV in early 2015.
If the "125" GeV signal is confirmed in 2012 with large statistical significance, 
it will be important to understand the real nature of this object 
by measuring its couplings with the SM particles. 
On the other hand, if nothing is found, it will become crucial 
to measure the cross-section of rare SM processes that, in absence of the 
Higgs boson contribution, would violate the unitarity of the diffusion amplitude, such 
as the scattering of two longitudinally polarized W bosons (WW scattering).
Those would be produced in association with two forward jets, that is a characteristic 
experimental signature of the Vector Boson Fusion (VBF) production mechanism at hadron colliders.
%The measurement of the WW scattering 


 

 
%\section*{Areas of competence}
%Software Development, IT, Particle detector physics
 
%\vspace{1cm}
\vfill{}
\hrulefill

% FILL IN THE FULL URL TO YOUR CV
\begin{center}
%{\footnotesize \href{http://www.ias.edu/spfeatures/einstein}{http://www.ias.edu/spfeatures/einstein} — Last updated: \today}
{\footnotesize Last updated: \today}
\end{center}


\end{document}
