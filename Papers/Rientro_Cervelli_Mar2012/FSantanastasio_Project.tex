%------------------------------------
% Dario Taraborelli
% Typesetting your academic CV in LaTeX
%
% URL: http://nitens.org/taraborelli/cvtex
% DISCLAIMER: This template is provided for free and without any guarantee 
% that it will correctly compile on your system if you have a non-standard  
% configuration.
%------------------------------------ 


% ! TEX TS-program = XeLaTeX -xdv2pdf
% ! TEX encoding = UTF-8 Unicode

\documentclass[10pt, a4paper]{article}
\usepackage{fontspec} 
\usepackage{xunicode} 
\usepackage{xltxtra}
% per le lettere accentate italiane sul Mac! :-)
%\usepackage[applemac]{inputenc} %with VIM
\usepackage[latin1]{inputenc} % with TeXShop


% DOCUMENT LAYOUT
\usepackage{geometry}
\geometry{a4paper, textwidth=5.5in, textheight=8.5in, marginparsep=7pt, marginparwidth=.6in}
\setlength\parindent{0in}

% ADDITIONAL SYMBOLS
%\usesymbols[mvs]

% FONTS
\defaultfontfeatures{Mapping=tex-text} % converts LaTeX specials (``quotes'' --- dashes etc.) to unicode
%\setromanfont [Ligatures={Common}, BoldFont={Fontin Bold}, ItalicFont={Fontin Italic}]{Fontin}
\setromanfont [Ligatures={Common}, BoldFont={Linux Libertine Bold}, ItalicFont={Linux Libertine Italic}]{Linux Libertine}
%\setsansfont [Ligatures={Common}, BoldFont={Fontin Sans Bold}, ItalicFont={Fontin Sans Italic}]{Fontin Sans}
\setmonofont[Scale=0.8]{Monaco} 
% ---- CUSTOM AMPERSAND
\newcommand{\amper}{{\fontspec[Scale=.95]{Linux Libertine Bold}\selectfont\itshape\&}}
% ---- MARGIN YEARS
%\newcommand{\years}[1]{\marginpar{\scriptsize #1}}
\newcommand{\years}[1]{\marginpar{\footnotesize #1}}

% HEADINGS
\usepackage{sectsty} 
\usepackage[normalem]{ulem} 
\sectionfont{\rmfamily\mdseries\upshape\Large}
\subsectionfont{\rmfamily\bfseries\upshape\normalsize} 
\subsubsectionfont{\rmfamily\mdseries\upshape\normalsize} 
%modifying section numbering
\def\thesubsection{\arabic{subsection}.\ } 

% PDF SETUP
% ---- FILL IN HERE THE DOC TITLE AND AUTHOR
\usepackage[dvipdfm, bookmarks, colorlinks, breaklinks, pdftitle={Francesco Santanastasio - Motivation for coming to CERN},pdfauthor={Francesco Santanastasio}]{hyperref}
%\hypersetup{linkcolor=blue,citecolor=blue,filecolor=black,urlcolor=blue} 
\hypersetup{linkcolor=cyan,citecolor=blue,filecolor=black,urlcolor=cyan} 

% Title of Bibliography
%\renewcommand\refname{References \\ \normalsize \begin{center} \quad \quad \textsc{Publications (relative to research activities)}\end{center} }

% Title of Bibliography
\renewcommand\refname{References \\ \normalsize \begin{center} \quad \quad \textsc{Publications (relative to research activities)}\end{center} }

% DOCUMENT
\begin{document}
\reversemarginpar

\section*{Title}
Study of the electroweak symmetry breaking mechanism and search 
for new physics beyond the Standard Model in final states containing a pair of 
vector bosons with the CMS detector at LHC

\section*{Titolo}
Studio del meccanismo di rottura spontanea di simmetria elettrodebole e
ricerca di nuova fisica oltre il Modello Standard in stati finali contenenti una coppia
di bosoni vettori presso l'esperimento CMS ad LHC

\section*{Keywords}
1. Particle physics
2. Electroweak simmetry breaking
3. Theories beyond the Standard Model
4. Collider physics
5. Calorimetry
%1. particle physics
%2. electroweak symmetry breaking
%3. physics beyond the Standard Model
%4. extra dimensions
%5. LHC
%6. CMS detector
%7. W and Z bosons
%8. jet substructure
%9. electromagnetic calorimeter

\section*{Parole chiave}
1. Fisica delle particelle
2. Rottura della simmetria elettrodebole
3. Teorie oltre il Modello Standard
4. Fisica dei collider
5. Calorimetria
%1. fisica delle particelle
%2. rottura di simmetria elettrodebole
%3. fisica oltre il Modello Standard
%4. dimensioni extra
%5. LHC
%6. rivelatore CMS
%7. bosoni W e Z
%8. struttura interna dei jet
%9. calorimetro elettromagnetico


\section*{Short CV}

Francesco Santanastasio was born in Rome on 09/02/1980. 

He graduated in May 2004 with full marks (110/110 "magna cum laude") with a thesis entitled "Calibration of an electromagnetic calorimeter using the energy flow method", advisors Prof. Egidio Longo and Dott. Riccardo Paramatti. In November of 2004 he began the PhD in Physics at the University of Rome "La Sapienza", passing the final examination in January 2008 with a thesis entitled "Search for Supersymmetry with Gauge-Mediated Breaking using high energy photons at CMS experiment", advisors Prof. Egidio Longo, Prof. Shahram Rahatlou, and Dott. Daniele del Re. 

Academic positions: 
From 09/2011: Research fellow at the Conseil Europeen pour la Recherche Nucleaire (CERN)
01/2008 - 08/2011: Post-Doctoral Research Assistant (Post-Doc) at University of Maryland

During his scientific career he published over 130 papers, about 20 internal notes of the experiments in which he participated, and he was intivited to give reports in various international and national conferences and workshops. 

He has a very good knowledge of spoken and written English. 

In his scientific career, Francesco Santanastasio was interested in electromagnetic and hadronic calorimetry, reconstruction of missing transverse energy, and searches for new physics beyond the Standard Model in proton-proton (pp) collisions at the LHC.

Between 2004 and 2007, he was involved in studies of the performance of the electromagnetic calorimeter (ECAL) of the CMS experiment. He has participated to the development of calibration techniques, and to the analysis and test of stability of ECAL high voltage system.
Between 2008 and 2010, he was involved in studies of the performance of the hadronic calorimeter (HCAL) of the CMS experiment. He has participated to the HCAL commissioning during the early period of cosmic-ray data taking, and to the development and implementation of algorithms for the identification of anomalous, beam-induced signals in the photomultiplier tubes of the hadronic forward calorimeter (HF) observed in the first collisions at LHC. He has also studied the performance of the missing transverse energy reconstructed in the event with early pp collisions. 

Since the beginning of his post-doctoral studies, Francesco Santanastasio was actively involved in various searches for new physics beyond the Standard Model in the CMS experiment at LHC. In 2010, he took a leading role in two different searches for pair production of leptoquarks in the final states with two electrons and two jets, or one electron, one neutrino, and two jets, which were both published in high impact scientific journals. These analyses were among the first ones at LHC to extend the search for new physics in an unexplored energy region compared to previous experiments at colliders. He has been supervising a PhD student from Princeton University to update this search with the 2011 data. Since September 2011 he is primarily involved in a search for new resonances that decay to a pair of jets (dijets) using the dijet mass spectrum. In particular, he has been the main developer of a novel trigger and data acquisition strategy that allowed to recover sensitivity to new physics at dijet masses below 1 TeV.

He held the following coordination roles within the CMS collaboration: 
Since 03/2012 : Coordinator of the group CMS Dataset Definition Team 
09/2008 - 09/2010 : Coordinator of the group CMS HCAL Prompt Feedback Group     
He was also member of the "Analysis Review Committee" for the scrutiny of two public CMS results within the collaboration: top cross section measurements in all hadronic decay channel and search for Randall-Sundrum gravitons decaying into a jet plus missing transverse energy final state.

In April 2010 he was invited to present the prospects for searches of new physics with early data at CMS at the international conference "Deep-Inelastic Scattering and Related Subjects". 
In March 2011 he was invited to present the first results of searches for new physics beyond the Standard Model using the full data sample collected by the CMS experiment in 2010 at the international conference "Rencontres de Moriond on EW Interactions and Unified Theories".


\section*{Selected Publications}

Pubblicazioni Selezionate

1. SANTANASTASIO F, CMS Collaboration. "Search for First Generation Scalar Leptoquarks in the evjj channel in pp collisions at sqrt(s) = 7 TeV". Phys. Lett. B 703, 246 (2011), arXiv:1105.5237 [hep-ex]

2. SANTANASTASIO F, CMS Collaboration. "Search for Pair Production of First-Generation Scalar Leptoquarks in pp Collisions at sqrt(s) = 7 TeV". Phys. Rev. Lett. 106, 201802 (2011), arXiv:1012.4031 [hep-ex]

3. SANTANASTASIO F, CMS Collaboration. "Search for Pair Production of Second-Generation Scalar Leptoquarks in pp Collisions at sqrt(s) = 7 TeV". Phys. Rev. Lett.  106, 201803 (2011), arXiv:1012.4033 [hep-ex]

4. SANTANASTASIO F, CMS Collaboration. "Search for Resonances in the Dijet Mass Spectrum from 7 TeV pp Collisions at CMS". Phys. Lett. B 704, 123 (2011), arXiv:1107.4771 [hep-ex]

5. SANTANASTASIO F, CMS HCAL Collaboration. "Study of various photomultiplier tubes with muon beams and Cherenkov light produced in electron showers". JINST 5, P06002 (2010)

6. SANTANASTASIO F, CMS Collaboration. "Identification and Filtering of Uncharacteristic Noise in the CMS Hadron Calorimeter". JINST 5, T03014 (2010)

7. SANTANASTASIO F, CMS Collaboration. "Performance of CMS Hadron Calorimeter Timing and Synchronization using Test Beam, Cosmic Ray, and LHC Beam Data". JINST 5, T03013 (2010).

8. SANTANASTASIO F, CMS Collaboration. "Performance of the CMS Hadron Calorimeter with Cosmic Ray Muons and LHC Beam Data". JINST 5, T03012 (2010)

9. SANTANASTASIO F, USCMS Collaboration and ECAL/HCAL Collaboration. "The CMS Barrel Calorimeter Response To Particle Beams From 2-Gev/C To 350-Gev/C". Eur. Phys. J.  C 60, 359 (2009), [Erratum-ibid.  C 61, 353 (2009)] 

10. SANTANASTASIO F, CMS Electromagnetic Calorimeter Group. "Intercalibration of the barrel electromagnetic calorimeter of the CMS  experiment at start-up". JINST 3, P10007 (2008)

11. SANTANASTASIO F, CMS Collaboration. "The CMS experiment at the CERN LHC". JINST 3, S08004 (2008)

12. SANTANASTASIO F,  BARTOLONI A, et al. "High voltage system for the CMS electromagnetic calorimeter". Nucl. Instrum. Meth.  A 582, 462 (2007)


Proceedings delle Conferenze

13. SANTANASTASIO F. "Exotica searches at the CMS experiment". Proceedings of the XLVIth Rencontres de Moriond 2011 Electroweak Interactions and Unified Theories, 125-132 (2011), edited by Etienne Auge, Jacques Dumarchez, and Jean Tran Thanh Van, Copyright The Gioi Publishers. Prepared for XLVIth Rencontres de Moriond 2011 Electroweak Interactions and Unified Theories, La Thuile, Aosta Valley, Italy, 13-20 March 2011

14. SANTANASTASIO F. "Searches With Early Data At CMS". PoS DIS2010, 206 (2010). Prepared for 18th International Workshop on Deep Inelastic Scattering and Related Subjects (DIS 2010), Florence, Italy, 19-23 Apr 2010

15. SANTANASTASIO F. "Prospects for Exotica Searches at ATLAS and CMS Experiments". Il Nuovo Cimento Vol.32 C, N.3-4 ncc9484 (2009). Prepared for Incontri di Fisica delle Alte Energie (IFAE 2009), Bari, Italy, Apr 2009

 
%%\hrule
\section*{Research Project}


One century of experimental measurements and progress in theoretical physics led to an extremely compact and elegant theory of fundamental interactions between elementary particles, the Standard Model (SM). Its success in reproducing measurements from different experiments in energy regimes spanning over several orders of magnitude is astonishing. Strong, weak and electromagnetic interactions are all described within the same mathematical framework of gauge theories. Although the electromagnetic and weak interactions are related to the same SU(2) x U(1) invariance, only the electromagnetic symmetry is manifest in the mass spectrum. The rest of the electroweak symmetry is hidden, that is, it is spontaneously broken. 
The detailed mechanism through which the breaking happens is not clear, though. The simplest way this could be explained theoretically is through the so-called Higgs mechanism of the SM. This mechanism explain, for instance, why elementary particles have mass. The Higgs mechanism postulates the existence of a new scalar particle, the Higgs boson, whose mass is not theoretically predicted by the SM, but that should be experimentally observable at particle colliders. 

The Large Hadron Collider (LHC) is the largest proton-proton (pp) collider ever built. It is located at CERN, Geneva, and its main objective is to finally unravel the origin of the electroweak symmetry breaking (EWSB). Using the pp collision data at the center-of-mass energy of 7 TeV collected in 2011, ATLAS and CMS, the largest experiments at the LHC, excluded the SM Higgs in the mass range 127-600 GeV, while masses below 114 GeV were already excluded by previous experiments at the electron-positron LEP collider. Thus, the mass range 114 GeV-127 GeV is currently the only one in which a Standard Model Higgs boson can hide, and it is in fact also the range preferred by the electroweak precision tests performed at LEP and SLC. In this mass range, the ATLAS and CMS experiments observe an excess of the data above the expected background, that is compatible with the existence of a SM Higgs boson with mass around 125 GeV. However, no claim of discovery is possible at the moment given the small statistical significance of the excess. 

In 2012, the LHC will collide protons at a center-of-mass energy of 8 TeV, delivering a number of collisions three times larger than in 2011. The higher energy and larger amount of data will allow to either confirm the "125 GeV" signal or rule out the existence of a SM Higgs by the end of the year. The LHC is scheduled to enter a long technical stop at the end of 2012 to prepare for running at its full design center-of-mass energy of around 14 TeV in early 2015.

A complementary approach to unveil the EWSB is the study of the WW system. The W boson acquires mass and the longitudinal polarization degree of freedom through the symmetry breaking. The scattering of two longitudinally polarized W bosons (WW scattering) carries a direct information about the EWSB mechanism, no matter whether a physical elementary Higgs particle exists or some kind of strongly interacting physics is responsible for this breaking. In absence of the Higgs boson contribution, this SM process would violate the unitarity of the scattering amplitude at a center-of-mass energy of around 1 TeV: in this scenario, interesting physics must emerge at that energy scale to restore unitarity. If the "125 GeV" signal is confirmed with high statistical significance by the 2012 data analysis, the energy dependence of the longitudinal WW scattering above the Higgs mass scale will tell us if the SM Higgs boson regularizes the WW scattering fully or only partially, as predicted in some theoretical models with composite Higgs. Therefore, the study of the WW scattering at high center-of-mass-energy is a fundamental milestone in the physics program of the LHC experiments.

The interest of the WW final state is not limited to the WW scattering: new resonances decaying to a pair of vector bosons are also foreseen by many models of physics beyond the SM such as technicolor, Sequential Standard Model, and others. The WW final state is also one of the possible decays of the Randall-Sundrum (RS) graviton. 
The RS model of the Extra Dimensions is one of the most appealing and popular models that predict new physics beyond the SM. This model solves the hierarchy problem, i.e. the big gap between the electroweak energy scale and the Planck scale at which quantum effects of gravity become strong, using a theoretical framework that includes a warped extra dimension in which gravity can propagate. The most distinctive feature of this scenario is the existence of spin-2 gravitons whose masses and couplings to the SM are set by the TeV scale. 
The gravitons would appear in experiments as widely separated resonances. Decays of the graviton to pairs of electrons, muons, or photons are the golden channels for searches of extra dimensions. However well-motivated extensions of the original RS model address the flavor structure of the SM through localization of fermions in the warped extra dimension. In this scenario, graviton production and decay with light fermion channels are highly suppressed and the decays into photons are negligible; while the production of gravitons from gluon fusion and their decay into a pair of massive gauge bosons (W and/or Z) is sizeable.
No signs of physics beyond the SM have been observed so far by the LHC experiments. The increase in the LHC center-of-mass energy from 7 TeV to 8 TeV and a sample of data three times larger than in 2011, will significantly extend in 2012 the discovery reach for many new physics models including those predicting a new resonance decaying in a pair of vector bosons.

The research program presented in this document is centered on the study of the WW final state at LHC and is motivated by the arguments discussed above that briefly summarize the current knowledge in the experimental and theoretical fields. This proposal fits well with the physics program of the CMS experiment at LHC, in which I have been working since the beginning of my graduate studies in 2004. The project is structured in various phases, accordingly with the LHC schedule for the next few years.

1) Search for New Resonances Decaying to Pairs of Vector Bosons

In the first period of the contract, I will search for new physics beyond the Standard Model by studying the decay of heavy resonances (X) in pairs of vector bosons (VV = WW, WZ, or ZZ). The analysis will focus on the semi-leptonic (lvjj and lljj) and fully hadronic (jjjj) final states using the data that will be collected by CMS in 2012 at a center-of-mass energy of 8 TeV.  These decay channels have the largest branching fraction, thus allowing to extend the sensitivity to new physics to higher values of resonance mass (i.e. lower cross section) compared to the fully leptonic channels. In addition, I already developed a solid expertise in the study of these final states during the past years in CMS, both in terms of analysis methods and reconstruction of physics objects. Ultimately, all channels could be combined to increase the sensitivity to the new physics in the entire mass range.

Using the first pp data collected by CMS in 2010, I took a leading role in the search for pair production of leptoquarks (LQ) in the LQLQbar-->eejj final state [1] and I was the contact person for the search LQLQbar-->evjj [2]. Both results have been published in high impact scientific journals. In addition, I have been supervising a PhD student from Princeton University to update both analyses with the almost 5 fb-1 of data collected in 2011. The experience gained in these analyses will be used to search for heavy X-->VV resonances in semi-leptonic final states. 
I have been member of the "Analysis Review Committee" for the scrutiny of a public CMS result within the collaboration: search for Randall-Sundrum gravitons decaying into a massive jet plus missing transverse energy final state (G-->ZZ-->vvjj) [3]. This analysis shares some of the experimental issues with the searches proposed in this physics program.  
I am also currently involved in a search for new resonances that decay to a pair  of jets originated by hadronization of quarks or gluons (dijets) which is aiming to deliver a public result in 2012. The fully hadronic X-->VV-->jjjj channel mentioned above can be considered an extension of the inclusive dijet search. At a sufficiently high resonance mass, the hadronic decay products of each energetic vector boson can be merged into a single massive jet, and the final state effectively assumes a dijet topology in the laboratory reference frame. This kinematic effect becomes important at resonance masses above 1 TeV.

The identification of an energetic vector boson (V) decaying into a pair of very collimated quarks, and thus resulting in a single massive jet, is an experimental challenge for both the semi-leptonic (lvjj and lljj) and fully hadronic (jjjj) channels of the aforementioned X-->VV searches. Achieving this goal is important to reduce the SM backgrounds arising from production of W/Z bosons in association with jets, QCD multijet events, and pairs of top-antitop quarks. In the past few years, several algorithms to resolve the substructure of a massive jet have been proposed. A recent comprehensive summary, result of the fruitful dialogue between theorists and experimentalists since 2009, can be found in Ref. [4]. 
I will focus on the study of the performance of the existing jet substructure algorithms, in order to identify the most promising ones in the contest of the proposed searches. Although powerful tools for physics analyses, the jet substructure observables are particularly sensitive to the specific Monte Carlo (MC) description in the simulation. Variations in the parton shower model, the underlying event activity, or the detector model can have a non-negligible impact in quantities such as the jet mass or the number of substructures in the jet. It will be important to compare the distributions of such observables between different generators and collision data in order to verify the agreement, and eventually tune the MC parameters to improve the description of the simulation for this kind of analyses.

2) Study of WW scattering in pp collisions at center-of-mass energy of 14 TeV 

The second period of the contract will be devoted to the study of the scattering of longitudinally polarized vector bosons which is a crucial measurement to understand the origin of the electroweak symmetry breaking mechanism. The WW scattering is infact a rare process in the SM and a large amount of data will be needed to perform this analysis. 
The work performed at point 1) will be preparatory to study the WW scattering in the aforementioned semi-leptonic and fully hadronic final states. The plan includes an accurate feasibility study with simulated events to be performed during the long shutdown of the accelerator, in preparation for the 14 TeV, high luminosity phase of the LHC in 2015. Although previous studies of the WW scattering have been performed in the past within the CMS collaboration, up-to-date studies with realistic detector and LHC conditions are still few. In particular, it is important to study new strategies for extracting the signal from the scattering of longitudinally polarized vector bosons from the irreducible background arising from the transverse polarizations. A possible strategy in this sense is the study of angular distributions of fermions from W decays, as recently suggested in Ref. [5].
The WW scattering process occurs via the Vector Boson Fusion (VBF) process with the associated production of two energetic forward jets. Forward jets coming from multiple interactions occurring at each crossing of protons bunches (pileup interactions) might overlap with events from inclusive (not VBF-specific) WW production, thus faking the WW scattering signature. With the increase of the LHC luminosity, it will be important to study the negative impact on the analysis of pileup interactions, as well as the techniques to mitigate such effect.

3) Calibration, Commissioning, and Monitoring of the Electromagnetic Calorimeter of CMS

The CMS electromagnetic calorimeter (ECAL) is a homogeneous crystal calorimeter, made up of lead tungstate crystals, aiming to reach an excellent energy resolution for the reconstruction of electrons and photons. For electrons of very high energy, such as those coming from the decays of energetic W/Z bosons discussed in points 1) and 2), the energy resolution is ultimately dominated by the detector calibration precision. 
During my graduate studies in Rome, I studied the stability of the ECAL high voltage (HV) system [6] and I worked at the original feasibility study in CMS of using the decays of neutral pions in two photons for the calibration of the ECAL crystals [7].  This technique has been used extensively in 2010-11 to calibrate at regular intervals of few months the entire ECAL.
In the first part of the contract, I will contribute to the ECAL calibration with neutral pions using the data collected in 2012, in particular improving the calibration of the forward part of the detector (endcaps). Towards the end of the long LHC shutdown (2013-14), I will also contribute to the restart of the commissioning and monitoring activities of the ECAL detector by spending periods of time at the CERN laboratory and performing data taking shifts. For all these activities I will take advantage of my past experience in the both the ECAL and HCAL detector groups in CMS.

---

My first choice for the institution where I intend to conduct this research project is the Physics Department of the Universita' degli Studi di Roma "La Sapienza", with a particular interest in joining the group involved in the CMS experiment. 
The CMS Rome group is actively involved in searches for Higgs boson and searches for new physics beyond the SM. Some of the experimental physicists at Rome currently cover responsibility roles within physics groups of the CMS collaboration whose research topics are closely connected with the subjects of this project. The CMS Rome group had also a leading role in all phases of the electromagnetic calorimeter project, from its design, building, operation and performance optimization.
Performing this research project in Rome will also benefit from the interplay with a part of the theoretical group of the Physics Department at "La Sapienza", which has been recently focusing on understanding the mechanism responsible for the breaking of the electroweak symmetry, and more in general on theories beyond the SM.
In conclusion, my research plan will integrates very well with the current interests and expertise of the CMS Rome group, giving me the opportunity to continue providing significant contributions to the CMS experiment. 


References

1. CMS Collaboration, Phys. Rev. Lett. 106, 201802 (2011), arXiv:1012.4031 [hep-ex]  
2. CMS Collaboration, Phys. Lett. B 703, 246 (2011), arXiv:1105.5237 [hep-ex] 
3. CMS Collaboration, CMS PAS EXO-11-061 (2011), http://cdsweb.cern.ch/record/1426654   
4. A. Abdesselam et al., Eur. Phys. J. C71:1661(2011), arXiv:1012.5412 [hep-ph] 
5. T. Han, D. Krohn, L-T. Wang and W. Zhu, JHEP 1003 (2010) 082, arXiv:0911.3656 [hep-ph] 
6. A. Bartoloni et al., Nucl. Instrum. Meth. A 582, 462 (2007), http://cdsweb.cern.ch/record/1027033 
7. D. del Re, S. Rahatlou, F. Santanastasio,  CMS IN-2006/050 (2006), "Study of ECAL calibration with pi-zero-->gamma gamma decays"  (document internal to the CMS collaboration)


%In this contest, the scattering of two longitudinally polarized W bosons (WW scattering) is a crucial channel to investigate the EWSB mechanism. In absence of the Higgs boson contribution, this SM process would violate the unitarity of the diffusion amplitude at a center-of-mass energy of around 1 TeV; thus we know that, in this scenario, interesting physics must emerge at that energy scale. The WW scattering carries a direct information about the EWSB mechanism, no matter whether a physical elementary Higgs particle exists or some kind of strongly interacting physics is responsible for this breaking. In fact, even if the "125 GeV" signal is confirmed with high statistical significance by the 2012 data analysis, the energy dependence of the longitudinal WW scattering above the Higgs candidate mass scale will tell us if the SM Higgs boson unitarizes the WW scattering fully or only partially, as predicted in some theoretical models with composite Higgs. Therefore, the study of the WW scattering at high center-of-mass-energy is, in any case, a fundamental milestone in the physics program of the LHC experiments.


%%%%%%%
%Beyond this, it is important to mention that there are many unsatisfactory aspects in the picture depicted by the SM: the hierarchy problem, i.e. the big gap between the electroweak energy scale and the Planck scale at which quantum effects of gravity become strong, is seen as one of its major limitation and has been the driving force for many theoretical developments extending the SM. Although the panorama of alternative new physics models is wide, one the most appealing and popular alternative is represented by the existence of Extra Dimensions. 
%In the original Randall-Sundrum (RS) model, the hierarchy problem of the SM is solved using a theoretical framework that includes a warped extra dimension in which gravity can propagate. The most distinctive feature of this scenario is the existence of spin-2 gravitons whose masses and couplings to the SM are set by the TeV scale. The gravitons would appear in experiments as widely separated resonances. Decays of the graviton to pairs of electrons, muons, or photons are traditionally the golden channels for searches of extra dimensions. 
%Well-motivated extensions of the original RS model address the flavor structure of the SM through localization of fermions in the warped extra dimension. This picture offers a unified geometric explanation of both the hierarchy and the flavor puzzles in the SM. In this scenario, graviton production and decay with light fermion channels are highly suppressed and the decay into photons are negligible. However the production of gravitons from gluon fusion and their decay into a pair of gauge bosons (W and/or Z) is sizeable. In general, new resonances decaying to a pair of vector bosons are also foreseen by other models of physics beyond the SM such as technicolor, Sequential Standard Model, and others.

%%%%%%%
%No signs of physics beyond the SM have been observed so far by the LHC experiments. The increase in the LHC center-of-mass energy from 7 TeV to 8 TeV and a sample of data three times larger than in 2011, will significantly extend in 2012 the discovery reach for many new physics models, including the ones aforementioned.
%%%%%%%

%The research project presented in this document is motivated by the current knowledge in the experimental and theoretical fields discussed above and it fits well with the physics program of the CMS experiment at LHC, in which I have been working since the beginning of my graduate studies in 2004. The project is structured in various phases, accordingly with the LHC schedule for the next few years.

%In the next months, I would like to focus on searches for physics beyond the standard 
%model, as well as to participate in development and performance studies of advanced algorithms 
%for jets and missing transverse energy reconstruction, because of their inherent interest and for 
%their importance in the physics analyses aforementioned. ?e detailed scope of this research will 
%continue to be shaped in the course of the data taking and analysis of the current run. In a longer- 
%term future, I am planning to explore other possible analyses (as Higgs searches), including those 
%for which the research group that I may join has expertise and interest. 

%From 2)
%
%Also leptons from W/Z decays can be emitted with small angular separation 
%if the momentum of the vector boson greatly exceed its mass. The experimental 
%issues are however less complex than the jet merging case discussed above. 
%For instance, existing analyses performed in CMS has already addressed the problem 
%of reconstructing the high momentum Z-->ll decays in the contest of searches 
%for heavy excited quarks decaying into a light quark plus a Z boson. These techniques 
%could be included in the proposed analyses. \\

%from 4)
%This method has the advantage of high statistics, since neutral pions are produced 
%in abundance at hadron colliders, and does not rely on information from the 
%detectors measuring tracks from charged particles. 

%The improved calibration would allow to improve the 
%sensitivity to heavy VV resonances in semi-leptonic final states, 
%although the ultimate mass resolution will receive an important contribution from 
%the energy resolution of the jets coming from the hadronic vector boson decays. 

% TRE FASI: 
% 1) WW/WZ/ZZ resonances, posso usare dati del 2012 
% (focus on fully hadronic, semi-leptonic)
% ricorda che hai fatto 
% 2) Studio dei decadimenti di W/Z boostati
% jet substructure, lepton isolation , 
% backgrounds
% 1) and 2) propedeutic to WW scattering
% 3) WW scattering feasibility study and analysis with 2015 data
% separation longitudinal/transverse polarization , study forward jets from VBF (can use 2012 data)
% robba vecchia da TDR, fare nuovi studi di fattibilita', se tutto va bene nel 2015 si fa l'analisisi..
% 4) ECAL calibration pi0--> gamma+gamma, monitoring HV system, data taking shifts
% di che la calibrazione l'ai inventata tu nella tua tesi di Phd, e che hai pubblicato anche su HV
% Ricorda expertise nei vari settori

% fitta bene con roma
% Roma gia' lavora su Higgs, canali esclusivi, WW scattering
% Roma gia' coinvolta in ricerche di fisica esotica, e jets.
% Interesse teorico in sede verso WW scattering ed extra dimensions

% frasona finale



\clearpage


%%%%%%%%%%%%%%%%%%%%%%%%%%%
%%% Service work
%%%%%%%%%%%%%%%%%%%%%%%%%%%

%\section*{Service work in Experiments and Collaborations}
%\subsection*{ATLAS Experiment}
%\noindent
%\textbf{Data Analysis: Supersymmetry Working Group} Working on data analysis, on exploring and implementing analysis strategies and on data files production\\
%\textbf{Development \& Upgrade} Working in the DAQ group, on the upgrade of the configuration DB system\\
%\textbf{Detector Operation} Shifter in the control room, at the Muon System, DAQ and Run Control desks\\
%\textbf{Software Framework} Taking part in code testing, and shifter for the build test system (RTT)\\
%\textbf{Documentation} Responsible person for a part of the documentation of the ATLAS data-format\\
%\textbf{Public Relations} Official ATLAS Guide, escorting VIP visits to the ATLAS cavern\\

%%%%%%%%%%%%%%%%%%%%%%%%%%%
%%% Publications & Talks
%%%%%%%%%%%%%%%%%%%%%%%%%%%

\begin{thebibliography}{599}

%\bibitem{Chatrchyan:2012hw}
%  S.~Chatrchyan {\it et al.}  [CMS Collaboration],
 % ``Measurement of the cross section for production of b b-bar X, decaying to
 % muons in pp collisions at sqrt(s)=7 TeV,''
 % arXiv:1203.3458 [hep-ex].

%\cite{Chatrchyan:2011ar}
\bibitem{Chatrchyan:2011ar}
 S.~Chatrchyan {\it et al.}  [CMS Collaboration], 
  ``Search for First Generation Scalar Leptoquarks in the evjj channel in pp collisions at sqrt(s) = 7 TeV'',
  Phys.\ Lett.\ B {\bf 703}, 246 (2011), arXiv:1105.5237 [hep-ex].
  %%CITATION = ARXIV:1105.5237;%%

%\cite{Khachatryan:2010mp}
\bibitem{Khachatryan:2010mp}
V.~Khachatryan {\it et al.}  [CMS Collaboration], 
``Search for Pair Production of First-Generation Scalar Leptoquarks in pp Collisions at sqrt(s) = 7 TeV''
Phys.\ Rev.\ Lett.\  {\bf 106}, 201802 (2011), arXiv:1012.4031 [hep-ex].
%\href{http://www.slac.stanford.edu/spires/find/hep/www?irn=8913501}{SPIRES entry}

%\cite{Chatrchyan:2011ns}
\bibitem{Chatrchyan:2011ns} 
 S.~Chatrchyan {\it et al.}  [CMS Collaboration],
 ``Search for Resonances in the Dijet Mass Spectrum from 7 TeV pp Collisions at CMS,''
Phys.\ Lett.\ B {\bf 704}, 123 (2011), arXiv:1107.4771 [hep-ex].
  %%CITATION = ARXIV:1107.4771;%%

%\cite{Chatrchyan:2010zz}
\bibitem{Chatrchyan:2010zz}
S.~Chatrchyan {\it et al.}  [CMS HCAL Collaboration], 
``Study of various photomultiplier tubes with muon beams and Cherenkov light produced in electron showers'', JINST {\bf 5}, P06002 (2010).
%\\{}CMS-NOTE-2010-003
%\href{http://www.slac.stanford.edu/spires/find/hep/www?j=jinst\%2c5\%2cp06002}{SPIRES entry}

%\cite{Chatrchyan:2009hy}
\bibitem{Chatrchyan:2009hy}
S.~Chatrchyan {\it et al.}  [CMS Collaboration], 
``Identification and Filtering of Uncharacteristic Noise in the CMS Hadron Calorimeter'', JINST {\bf 5}, T03014 (2010), arXiv:0911.4881 [physics.ins-det]
%\\{}CMS-CFT-09-019
%\href{http://www.slac.stanford.edu/spires/find/hep/www?j=jinst\%2c5\%2ct03014}{SPIRES entry}

%\cite{Abdullin:2009zz}
\bibitem{Abdullin:2009zz}
S.~Abdullin {\it et al.}  [USCMS Collaboration and ECAL/HCAL Collaboration], 
``The CMS Barrel Calorimeter Response To Particle Beams From 2-Gev/C To 350-Gev/C'', 
Eur.\ Phys.\ J.\  C {\bf 60}, 359 (2009), [Erratum-ibid.\  C {\bf 61}, 353 (2009)].
%\\{}FERMILAB-PUB-08-661-E-PPD
%\href{http://www.slac.stanford.edu/spires/find/hep/www?j=ephja\%2cc60\%2c359}{SPIRES entry}

%\cite{Adzic:2008zza}
\bibitem{Adzic:2008zza}
P.~Adzic {\it et al.}  [CMS Electromagnetic Calorimeter Group], 
``Intercalibration of the barrel electromagnetic calorimeter of the CMS  experiment at start-up'', 
JINST {\bf 3}, P10007 (2008)
%\\{}CERN-CMS-NOTE-2008-018
%\href{http://www.slac.stanford.edu/spires/find/hep/www?j=jinst\%2c3\%2cp10007}{SPIRES entry}

%\cite{Bartoloni:2007hx}
\bibitem{Bartoloni:2007hx}
A.~Bartoloni {\it et al.}, 
``High voltage system for the CMS electromagnetic calorimeter'', 
Nucl.\ Instrum.\ Meth.\  A {\bf 582}, 462 (2007)
%\\{}CERN-CMS-NOTE-2007-009
%\href{http://www.slac.stanford.edu/spires/find/hep/www?j=nuima\%2ca582\%2c462}{SPIRES entry}



%------------------------------------------------------------------------------------------------------------------------------------------------------------
\vspace{0.1cm} \begin{center} \textsc{Conference Proceedings} \end{center} \vspace{0.05cm}
%------------------------------------------------------------------------------------------------------------------------------------------------------------

%Proceedings of the XLVIth Rencontres de Moriond
%2011 Electroweak Interactions and Unified Theories
%La Thuile, Aosta Valley, Italy � March 13-20, 2011
%edited by Etienne Aug�, Jacques Dumarchez, and Jean Tr�n Thanh V�n
%� Th� Gioi Publishers, 2011

%Exotica searches at the CMS experiment 
%F. Santanastasio 
%Pages 125-132

\bibitem{MoriondEW2011}
F.~Santanastasio, 
``Exotica searches at the CMS experiment'', 
Proceedings of the XLVIth Rencontres de Moriond 
  2011 Electroweak Interactions and Unified Theories, 125-132 (2011), edited by Etienne Auge, Jacques Dumarchez, and Jean Tran Thanh Van $\textcopyright$ The Gioi Publishers
\\{}{\it Prepared for XLVIth Rencontres de Moriond 2011 Electroweak Interactions and Unified Theories, La Thuile, Aosta Valley, Italy, 13-20 March 2011}

\bibitem{Santanastasio:2010zz}
F.~Santanastasio
``Searches With Early Data At CMS'', 
PoS {\bf DIS2010}, 206 (2010)
%\href{http://www.slac.stanford.edu/spires/find/hep/www?j=posci\%2cdis2010\%2c206}{SPIRES entry}
\\{}{\it Prepared for 18th International Workshop on Deep Inelastic Scattering and Related Subjects (DIS 2010), Florence, Italy, 19-23 Apr 2010}

\bibitem{Santanastasio:IFAE2009}
F.~Santanastasio,
``Prospects for Exotica Searches at ATLAS and CMS Experiments'',
Il Nuovo Cimento Vol.32 C, N.3-4 ncc9484 (2009)
\\{}{\it Prepared for Incontri di Fisica delle Alte Energie (IFAE 2009), Bari, Italy, Apr 2009}


\clearpage

%------------------------------------------------------------------------------------------------------------------------------------------------------------
\vspace{0.1cm} \begin{center} \textsc{Theses ( \textit{Laurea} and PhD)} \end{center} \vspace{0.05cm}
%------------------------------------------------------------------------------------------------------------------------------------------------------------

\bibitem{Santanastasio:DOTTORATO}
F.~Santanastasio, 
``Search for Supersymmetry with Gauge-Mediated Breaking using high energy photons at CMS experiment'', PhD thesis at \textit{Sapienza Universit\`a di Roma} (2007)
\\{}{\it http://www.roma1.infn.it/cms/tesiPHD/santanastasio.pdf}

\bibitem{Santanastasio:LAUREA}
F.~Santanastasio,
``Calibrazione di un calorimetro elettromagnetico tramite il flusso totale di energia'',
  \textit{Laurea} thesis at \textit{Sapienza Universit\`a di Roma} (2004)
\\{}{\it http://www.roma1.infn.it/cms/tesi/santanastasio.pdf }


%------------------------------------------------------------------------------------------------------------------------------------------------------------
\vspace{0.1cm} \begin{center} \textsc{Other Publications and Pre-Prints of the CMS Collaboration} \end{center} \vspace{0.05cm}
%------------------------------------------------------------------------------------------------------------------------------------------------------------

%\cite{Chatrchyan:2012hw}
\bibitem{Chatrchyan:2012hw}
  S.~Chatrchyan {\it et al.}  [CMS Collaboration],
  ``Measurement of the cross section for production of b b-bar X, decaying to
  muons in pp collisions at sqrt(s)=7 TeV,''
  arXiv:1203.3458 [hep-ex].
  %%CITATION = ARXIV:1203.3458;%%

%\cite{Chatrchyan:2012ta}
\bibitem{Chatrchyan:2012ta}
  S.~Chatrchyan {\it et al.}  [CMS Collaboration],
  ``Search for microscopic black holes in pp collisions at sqrt(s) = 7 TeV,''
  arXiv:1202.6396 [hep-ex].
  %%CITATION = ARXIV:1202.6396;%%

%\cite{Chatrchyan:2012bf}
\bibitem{Chatrchyan:2012bf}
  S.~Chatrchyan {\it et al.}  [CMS Collaboration],
  ``Search for quark compositeness in dijet angular distributions from pp
  collisions at sqrt(s) = 7 TeV,''
  arXiv:1202.5535 [hep-ex].
  %%CITATION = ARXIV:1202.5535;%%

%\cite{Chatrchyan:2012ni}
\bibitem{Chatrchyan:2012ni}
  S.~Chatrchyan {\it et al.}  [CMS Collaboration],
  ``Jet momentum dependence of jet quenching in PbPb collisions at
  sqrt(sNN)=2.76 TeV,''
  arXiv:1202.5022 [nucl-ex].
  %%CITATION = ARXIV:1202.5022;%%

%\cite{Chatrchyan:2012dk}
\bibitem{Chatrchyan:2012dk}
  S.~Chatrchyan {\it et al.}  [CMS Collaboration],
  ``Inclusive b-jet production in pp collisions at sqrt(s)=7 TeV,''
  arXiv:1202.4617 [hep-ex].
  %%CITATION = ARXIV:1202.4617;%%

%\cite{Chatrchyan:2012ww}
\bibitem{Chatrchyan:2012ww}
  S.~Chatrchyan {\it et al.}  [CMS Collaboration],
  ``Search for the standard model Higgs boson decaying to bottom quarks in pp
  collisions at sqrt(s)=7 TeV,''
  arXiv:1202.4195 [hep-ex].
  %%CITATION = ARXIV:1202.4195;%%

%\cite{Chatrchyan:2012vp}
\bibitem{Chatrchyan:2012vp}
  S.~Chatrchyan {\it et al.}  [CMS Collaboration],
  ``Search for neutral Higgs bosons decaying to tau pairs in pp collisions at
  sqrt(s)=7 TeV,''
  arXiv:1202.4083 [hep-ex].
  %%CITATION = ARXIV:1202.4083;%%

%\cite{Chatrchyan:2012kc}
\bibitem{Chatrchyan:2012kc}
  S.~Chatrchyan {\it et al.}  [CMS Collaboration],
  ``Search for large extra dimensions in dimuon and dielectron events in pp
  collisions at sqrt(s) = 7 TeV,''
  arXiv:1202.3827 [hep-ex].
  %%CITATION = ARXIV:1202.3827;%%

%\cite{Chatrchyan:2012hr}
\bibitem{Chatrchyan:2012hr}
  S.~Chatrchyan {\it et al.}  [CMS Collaboration],
  ``Search for the standard model Higgs boson in the H to ZZ to ll tau tau
  decay channel in pp collisions at sqrt(s)=7 TeV,''
  arXiv:1202.3617 [hep-ex].
  %%CITATION = ARXIV:1202.3617;%%

%\cite{Chatrchyan:2012ft}
\bibitem{Chatrchyan:2012ft}
  S.~Chatrchyan {\it et al.}  [CMS Collaboration],
  ``Search for the standard model Higgs boson in the H to ZZ to 2l 2nu channel
  in pp collisions at sqrt(s) = 7 TeV,''
  arXiv:1202.3478 [hep-ex].
  %%CITATION = ARXIV:1202.3478;%%

%\cite{Chatrchyan:2012dg}
\bibitem{Chatrchyan:2012dg}
  S.~Chatrchyan {\it et al.}  [CMS Collaboration],
  ``Search for the standard model Higgs boson in the decay channel H to ZZ to 4
  leptons in pp collisions at sqrt(s) = 7 TeV,''
  arXiv:1202.1997 [hep-ex].
  %%CITATION = ARXIV:1202.1997;%%

%\cite{Chatrchyan:2012ty}
\bibitem{Chatrchyan:2012ty}
  S.~Chatrchyan {\it et al.}  [CMS Collaboration],
  ``Search for the standard model Higgs boson decaying to a W pair in the fully
  leptonic final state in pp collisions at sqrt(s) = 7 TeV,''
  arXiv:1202.1489 [hep-ex].
  %%CITATION = ARXIV:1202.1489;%%

%\cite{Chatrchyan:2012tx}
\bibitem{Chatrchyan:2012tx}
  S.~Chatrchyan {\it et al.}  [CMS Collaboration],
  ``Combined results of searches for the standard model Higgs boson in pp
  collisions at sqrt(s) = 7 TeV,''
  arXiv:1202.1488 [hep-ex].
  %%CITATION = ARXIV:1202.1488;%%

%\cite{Chatrchyan:2012tw}
\bibitem{Chatrchyan:2012tw}
  S.~Chatrchyan {\it et al.}  [CMS Collaboration],
  ``Search for the standard model Higgs boson decaying into two photons in pp
  collisions at sqrt(s)=7 TeV,''
  arXiv:1202.1487 [hep-ex].
  %%CITATION = ARXIV:1202.1487;%%

%\cite{Chatrchyan:2012sn}
\bibitem{Chatrchyan:2012sn}
  S.~Chatrchyan {\it et al.}  [CMS Collaboration],
  ``Search for a Higgs boson in the decay channel H to ZZ(*) to q qbar l-l+ in
  pp collisions at sqrt(s) = 7 TeV,''
  arXiv:1202.1416 [hep-ex].
  %%CITATION = ARXIV:1202.1416;%%

%\cite{Chatrchyan:2012gw}
\bibitem{Chatrchyan:2012gw}
  S.~Chatrchyan {\it et al.}  [CMS Collaboration],
  ``Measurement of the inclusive production cross sections for forward jets and
  for dijet events with one forward and one central jet in pp collisions at
  sqrt(s) = 7 TeV,''
  arXiv:1202.0704 [hep-ex].
  %%CITATION = ARXIV:1202.0704;%%

%\cite{Chatrchyan:2012np}
\bibitem{Chatrchyan:2012np}
  S.~Chatrchyan {\it et al.}  [CMS Collaboration],
  ``Suppression of non-prompt J/psi, prompt J/psi, and Y(1S) in PbPb collisions
  at sqrt(sNN) = 2.76 TeV,''
  arXiv:1201.5069 [nucl-ex].
  %%CITATION = ARXIV:1201.5069;%%

%\cite{Chatrchyan:2012wg}
\bibitem{Chatrchyan:2012wg}
  S.~Chatrchyan {\it et al.}  [CMS Collaboration],
  ``Centrality dependence of dihadron correlations and azimuthal anisotropy
  harmonics in PbPb collisions at sqrt(s[NN]) = 2.76 TeV,''
  arXiv:1201.3158 [nucl-ex].
  %%CITATION = ARXIV:1201.3158;%%

%\cite{Chatrchyan:2012vq}
\bibitem{Chatrchyan:2012vq}
  S.~Chatrchyan {\it et al.}  [CMS Collaboration],
  ``Measurement of isolated photon production in pp and PbPb collisions at
  sqrt(sNN) = 2.76 TeV,''
  arXiv:1201.3093 [nucl-ex].
  %%CITATION = ARXIV:1201.3093;%%

%\cite{Chatrchyan:2011hk}
\bibitem{Chatrchyan:2011hk}
  S.~Chatrchyan {\it et al.}  [CMS Collaboration],
  ``Measurement of the charge asymmetry in top-quark pair production in
  proton-proton collisions at sqrt(s) = 7 TeV,''
  Phys.\ Lett.\  B {\bf 709}, 28 (2012)
  [arXiv:1112.5100 [hep-ex]].
  %%CITATION = PHLTA,B709,28;%%

%\cite{Chatrchyan:2011fq}
\bibitem{Chatrchyan:2011fq}
  S.~Chatrchyan {\it et al.}  [CMS Collaboration],
  ``Search for signatures of extra dimensions in the diphoton mass spectrum at
  the Large Hadron Collider,''
  arXiv:1112.0688 [hep-ex].
  %%CITATION = ARXIV:1112.0688;%%

%\cite{Chatrchyan:2011ci}
\bibitem{Chatrchyan:2011ci}
  S.~Chatrchyan {\it et al.}  [CMS Collaboration],
  ``Exclusive photon-photon production of muon pairs in proton-proton
  collisions at sqrt(s) = 7 TeV,''
  JHEP {\bf 1201}, 052 (2012)
  [arXiv:1111.5536 [hep-ex]].
  %%CITATION = JHEPA,1201,052;%%

%\cite{Chatrchyan:2011kc}
\bibitem{Chatrchyan:2011kc}
  S.~Chatrchyan {\it et al.}  [CMS Collaboration],
  ``J/psi and psi(2S) production in pp collisions at sqrt(s) = 7 TeV,''
  JHEP {\bf 1202}, 011 (2012)
  [arXiv:1111.1557 [hep-ex]].
  %%CITATION = JHEPA,1202,011;%%

%\cite{Chatrchyan:2011qt}
\bibitem{Chatrchyan:2011qt}
  S.~Chatrchyan {\it et al.}  [CMS Collaboration],
  ``Measurement of the Production Cross Section for Pairs of Isolated Photons
  in pp collisions at sqrt(s) = 7 TeV,''
  JHEP {\bf 1201}, 133 (2012)
  [arXiv:1110.6461 [hep-ex]].
  %%CITATION = JHEPA,1201,133;%%

%\cite{Chatrchyan:2011wt}
\bibitem{Chatrchyan:2011wt}
  S.~Chatrchyan {\it et al.}  [CMS Collaboration],
  ``Measurement of the Rapidity and Transverse Momentum Distributions of Z
  Bosons in pp Collisions at sqrt(s)=7 TeV,''
  Phys.\ Rev.\  D {\bf 85}, 032002 (2012)
  [arXiv:1110.4973 [hep-ex]].
  %%CITATION = PHRVA,D85,032002;%%

%\cite{Chatrchyan:2011ne}
\bibitem{Chatrchyan:2011ne}
  S.~Chatrchyan {\it et al.}  [CMS Collaboration],
  ``Jet Production Rates in Association with W and Z Bosons in pp Collisions at
  sqrt(s) = 7 TeV,''
  JHEP {\bf 1201}, 010 (2012)
  [arXiv:1110.3226 [hep-ex]].
  %%CITATION = JHEPA,1201,010;%%

%\cite{Chatrchyan:2011ya}
\bibitem{Chatrchyan:2011ya}
  S.~Chatrchyan {\it et al.}  [CMS Collaboration],
  ``Measurement of the weak mixing angle with the Drell-Yan process in
  proton-proton collisions at the LHC,''
  Phys.\ Rev.\  D {\bf 84}, 112002 (2011)
  [arXiv:1110.2682 [hep-ex]].
  %%CITATION = PHRVA,D84,112002;%%

%\cite{Chatrchyan:2011wm}
\bibitem{Chatrchyan:2011wm}
  S.~Chatrchyan {\it et al.}  [CMS Collaboration],
  ``Measurement of energy flow at large pseudorapidities in pp collisions at
  sqrt(s) = 0.9 and 7 TeV,''
  JHEP {\bf 1111}, 148 (2011)
  [Erratum-ibid.\  {\bf 1202}, 055 (2012)]
  [arXiv:1110.0211 [hep-ex]].
  %%CITATION = JHEPA,1111,148;%%

%\cite{Chatrchyan:2011wb}
\bibitem{Chatrchyan:2011wb}
  S.~Chatrchyan {\it et al.}  [CMS Collaboration],
  ``Forward Energy Flow, Central Charged-Particle Multiplicities, and
  Pseudorapidity Gaps in W and Z Boson Events from pp Collisions at 7 TeV,''
  Eur.\ Phys.\ J.\  C {\bf 72}, 1839 (2012)
  [arXiv:1110.0181 [hep-ex]].
  %%CITATION = EPHJA,C72,1839;%%

%\cite{Chatrchyan:2011ay}
\bibitem{Chatrchyan:2011ay}
  S.~Chatrchyan {\it et al.}  [CMS Collaboration],
  ``Search for a Vector-like Quark with Charge 2/3 in t + Z Events from pp
  Collisions at sqrt(s) = 7 TeV,''
  Phys.\ Rev.\ Lett.\  {\bf 107}, 271802 (2011)
  [arXiv:1109.4985 [hep-ex]].
  %%CITATION = PRLTA,107,271802;%%

%\cite{Chatrchyan:2011zy}
\bibitem{Chatrchyan:2011zy}
  S.~Chatrchyan {\it et al.}  [CMS Collaboration],
  ``Search for Supersymmetry at the LHC in Events with Jets and Missing
  Transverse Energy,''
  Phys.\ Rev.\ Lett.\  {\bf 107}, 221804 (2011)
  [arXiv:1109.2352 [hep-ex]].
  %%CITATION = PRLTA,107,221804;%%

%\cite{Chatrchyan:2011yy}
\bibitem{Chatrchyan:2011yy}
  S.~Chatrchyan {\it et al.}  [CMS Collaboration],
  ``Measurement of the t t-bar Production Cross Section in pp Collisions at 7
  TeV in Lepton + Jets Events Using b-quark Jet Identification,''
  Phys.\ Rev.\  D {\bf 84}, 092004 (2011)
  [arXiv:1108.3773 [hep-ex]].
  %%CITATION = PHRVA,D84,092004;%%

%\cite{Chatrchyan:2011ue}
\bibitem{Chatrchyan:2011ue}
  S.~Chatrchyan {\it et al.}  [CMS Collaboration],
  ``Measurement of the Differential Cross Section for Isolated Prompt Photon
  Production in pp Collisions at 7 TeV,''
  Phys.\ Rev.\  D {\bf 84}, 052011 (2011)
  [arXiv:1108.2044 [hep-ex]].
  %%CITATION = PHRVA,D84,052011;%%

%\cite{Chatrchyan:2011cm}
\bibitem{Chatrchyan:2011cm}
  S.~Chatrchyan {\it et al.}  [CMS Collaboration],
  ``Measurement of the Drell-Yan Cross Section in pp Collisions at sqrt(s) = 7
  TeV,''
  JHEP {\bf 1110}, 007 (2011)
  [arXiv:1108.0566 [hep-ex]].
  %%CITATION = JHEPA,1110,007;%%

%\cite{Chatrchyan:2011kr}
\bibitem{Chatrchyan:2011kr}
  S.~Chatrchyan {\it et al.}  [CMS Collaboration],
  ``Search for B(s) and B to dimuon decays in pp collisions at 7 TeV,''
  Phys.\ Rev.\ Lett.\  {\bf 107}, 191802 (2011)
  [arXiv:1107.5834 [hep-ex]].
  %%CITATION = PRLTA,107,191802;%%

%\cite{Chatrchyan:2011pb}
\bibitem{Chatrchyan:2011pb}
  S.~Chatrchyan {\it et al.}  [CMS Collaboration],
  ``Dependence on pseudorapidity and centrality of charged hadron production in
  PbPb collisions at a nucleon-nucleon centre-of-mass energy of 2.76 TeV,''
  JHEP {\bf 1108}, 141 (2011)
  [arXiv:1107.4800 [nucl-ex]].
  %%CITATION = JHEPA,1108,141;%%

%\cite{CMS:2011aa}
\bibitem{CMS:2011aa}
  S.~Chatrchyan {\it et al.}  [CMS Collaboration],
  ``Measurement of the Inclusive W and Z Production Cross Sections in pp
  Collisions at sqrt(s) = 7 TeV,''
  JHEP {\bf 1110}, 132 (2011)
  [arXiv:1107.4789 [hep-ex]].
  %%CITATION = JHEPA,1110,132;%%

%\cite{Chatrchyan:2011ds}
\bibitem{Chatrchyan:2011ds}
  S.~Chatrchyan {\it et al.}  [CMS Collaboration],
  ``Determination of Jet Energy Calibration and Transverse Momentum Resolution
  in CMS,''
  JINST {\bf 6}, P11002 (2011)
  [arXiv:1107.4277 [physics.ins-det]].
  %%CITATION = JINST,6,P11002;%%

%\cite{Chatrchyan:2011cj}
\bibitem{Chatrchyan:2011cj}
  S.~Chatrchyan {\it et al.}  [CMS Collaboration],
  ``Search for Three-Jet Resonances in pp Collisions at sqrt(s) = 7 TeV,''
  Phys.\ Rev.\ Lett.\  {\bf 107}, 101801 (2011)
  [arXiv:1107.3084 [hep-ex]].
  %%CITATION = PRLTA,107,101801;%%

%\cite{Chatrchyan:2011qs}
\bibitem{Chatrchyan:2011qs}
  S.~Chatrchyan {\it et al.}  [CMS Collaboration],
  ``Search for supersymmetry in pp collisions at sqrt(s)=7 TeV in events with a
  single lepton, jets, and missing transverse momentum,''
  JHEP {\bf 1108}, 156 (2011)
  [arXiv:1107.1870 [hep-ex]].
  %%CITATION = JHEPA,1108,156;%%

%\cite{Chatrchyan:2011pg}
\bibitem{Chatrchyan:2011pg}
  S.~Chatrchyan {\it et al.}  [CMS Collaboration],
  ``A search for excited leptons in pp Collisions at sqrt(s) = 7 TeV,''
  Phys.\ Lett.\  B {\bf 704}, 143 (2011)
  [arXiv:1107.1773 [hep-ex]].
  %%CITATION = PHLTA,B704,143;%%

%\cite{Chatrchyan:2011ek}
\bibitem{Chatrchyan:2011ek}
  S.~Chatrchyan {\it et al.}  [CMS Collaboration],
  ``Inclusive search for squarks and gluinos in pp collisions at sqrt(s) = 7
  TeV,''
  Phys.\ Rev.\  D {\bf 85}, 012004 (2012)
  [arXiv:1107.1279 [hep-ex]].
  %%CITATION = PHRVA,D85,012004;%%

%\cite{Chatrchyan:2011id}
\bibitem{Chatrchyan:2011id}
  S.~Chatrchyan {\it et al.}  [CMS Collaboration],
  ``Measurement of the Underlying Event Activity at the LHC with sqrt(s)= 7 TeV
  and Comparison with sqrt(s) = 0.9 TeV,''
  JHEP {\bf 1109}, 109 (2011)
  [arXiv:1107.0330 [hep-ex]].
  %%CITATION = JHEPA,1109,109;%%

%\cite{Chatrchyan:2011tn}
\bibitem{Chatrchyan:2011tn}
  S.~Chatrchyan {\it et al.}  [CMS Collaboration],
  ``Missing transverse energy performance of the CMS detector,''
  JINST {\bf 6}, P09001 (2011)
  [arXiv:1106.5048 [physics.ins-det]].
  %%CITATION = JINST,6,P09001;%%

%\cite{Chatrchyan:2011nd}
\bibitem{Chatrchyan:2011nd}
  S.~Chatrchyan {\it et al.}  [CMS Collaboration],
  ``Search for New Physics with a Mono-Jet and Missing Transverse Energy in pp
  Collisions at sqrt(s) = 7 TeV,''
  Phys.\ Rev.\ Lett.\  {\bf 107}, 201804 (2011)
  [arXiv:1106.4775 [hep-ex]].
  %%CITATION = PRLTA,107,201804;%%

%\cite{Collaboration:2011ida}
\bibitem{Collaboration:2011ida}
  S.~Chatrchyan {\it et al.}  [CMS Collaboration],
  ``Search for New Physics with Jets and Missing Transverse Momentum in pp
  collisions at sqrt(s) = 7 TeV,''
  JHEP {\bf 1108}, 155 (2011)
  [arXiv:1106.4503 [hep-ex]].
  %%CITATION = JHEPA,1108,155;%%

%\cite{Chatrchyan:2011vh}
\bibitem{Chatrchyan:2011vh}
  S.~Chatrchyan {\it et al.}  [CMS Collaboration],
  ``Measurement of the Strange B Meson Production Cross Section with J/Psi phi
  Decays in pp Collisions at sqrt(s) = 7 TeV,''
  Phys.\ Rev.\  D {\bf 84}, 052008 (2011)
  [arXiv:1106.4048 [hep-ex]].
  %%CITATION = PHRVA,D84,052008;%%

%\cite{Chatrchyan:2011bj}
\bibitem{Chatrchyan:2011bj}
  S.~Chatrchyan {\it et al.}  [CMS Collaboration],
  ``Search for Supersymmetry in Events with b Jets and Missing Transverse
  Momentum at the LHC,''
  JHEP {\bf 1107}, 113 (2011)
  [arXiv:1106.3272 [hep-ex]].
  %%CITATION = JHEPA,1107,113;%%

%\cite{Chatrchyan:2011vp}
\bibitem{Chatrchyan:2011vp}
  S.~Chatrchyan {\it et al.}  [CMS Collaboration],
  ``Measurement of the t-channel single top quark production cross section in
  pp collisions at sqrt(s) = 7 TeV,''
  Phys.\ Rev.\ Lett.\  {\bf 107}, 091802 (2011)
  [arXiv:1106.3052 [hep-ex]].
  %%CITATION = PRLTA,107,091802;%%

%\cite{Chatrchyan:2011hr}
\bibitem{Chatrchyan:2011hr}
  S.~Chatrchyan {\it et al.}  [CMS Collaboration],
  ``Search for Light Resonances Decaying into Pairs of Muons as a Signal of New
  Physics,''
  JHEP {\bf 1107}, 098 (2011)
  [arXiv:1106.2375 [hep-ex]].
  %%CITATION = JHEPA,1107,098;%%

%\cite{Chatrchyan:2011dk}
\bibitem{Chatrchyan:2011dk}
  S.~Chatrchyan {\it et al.}  [CMS Collaboration],
  ``Search for Same-Sign Top-Quark Pair Production at sqrt(s) = 7 TeV and
  Limits on Flavour Changing Neutral Currents in the Top Sector,''
  JHEP {\bf 1108}, 005 (2011)
  [arXiv:1106.2142 [hep-ex]].
  %%CITATION = JHEPA,1108,005;%%

%\cite{Chatrchyan:2011ff}
\bibitem{Chatrchyan:2011ff}
  S.~Chatrchyan {\it et al.}  [CMS Collaboration],
  ``Search for Physics Beyond the Standard Model Using Multilepton Signatures
  in pp Collisions at sqrt(s)=7 TeV,''
  Phys.\ Lett.\  B {\bf 704}, 411 (2011)
  [arXiv:1106.0933 [hep-ex]].
  %%CITATION = PHLTA,B704,411;%%

%\cite{Chatrchyan:2011ew}
\bibitem{Chatrchyan:2011ew}
  S.~Chatrchyan {\it et al.}  [CMS Collaboration],
  ``Measurement of the Top-antitop Production Cross Section in pp Collisions at
  sqrt(s)=7 TeV using the Kinematic Properties of Events with Leptons and
  Jets,''
  Eur.\ Phys.\ J.\  C {\bf 71}, 1721 (2011)
  [arXiv:1106.0902 [hep-ex]].
  %%CITATION = EPHJA,C71,1721;%%

%\cite{Chatrchyan:2011wn}
\bibitem{Chatrchyan:2011wn}
  S.~Chatrchyan {\it et al.}  [CMS Collaboration],
  ``Measurement of the Ratio of the 3-jet to 2-jet Cross Sections in pp
  Collisions at sqrt(s) = 7 TeV,''
  Phys.\ Lett.\  B {\bf 702}, 336 (2011)
  [arXiv:1106.0647 [hep-ex]].
  %%CITATION = PHLTA,B702,336;%%

%\cite{CMS:2011ab}
\bibitem{CMS:2011ab}
  S.~Chatrchyan {\it et al.}  [CMS Collaboration],
  ``Measurement of the Inclusive Jet Cross Section in pp Collisions at sqrt(s)
  = 7 TeV,''
  Phys.\ Rev.\ Lett.\  {\bf 107}, 132001 (2011)
  [arXiv:1106.0208 [hep-ex]].
  %%CITATION = PRLTA,107,132001;%%

%\cite{Chatrchyan:2011nb}
\bibitem{Chatrchyan:2011nb}
  S.~Chatrchyan {\it et al.}  [CMS Collaboration],
  ``Measurement of the t t-bar production cross section and the top quark mass
  in the dilepton channel in pp collisions at sqrt(s) =7 TeV,''
  JHEP {\bf 1107}, 049 (2011)
  [arXiv:1105.5661 [hep-ex]].
  %%CITATION = JHEPA,1107,049;%%

%\cite{Chatrchyan:2011pe}
\bibitem{Chatrchyan:2011pe}
  S.~Chatrchyan {\it et al.}  [CMS Collaboration],
  ``Suppression of Upsilon excited states in PbPb collisions at a
  nucleon-nucleon centre-of-mass energy of 2.76 TeV,''
  Phys.\ Rev.\ Lett.\  {\bf 107}, 052302 (2011)
  [arXiv:1105.4894 [nucl-ex]].
  %%CITATION = PRLTA,107,052302;%%

%\cite{Chatrchyan:2011ah}
\bibitem{Chatrchyan:2011ah}
  S.~Chatrchyan {\it et al.}  [CMS Collaboration],
  ``Search for supersymmetry in events with a lepton, a photon, and large
  missing transverse energy in pp collisions at sqrt(s) = 7 TeV,''
  JHEP {\bf 1106}, 093 (2011)
  [arXiv:1105.3152 [hep-ex]].
  %%CITATION = JHEPA,1106,093;%%

%\cite{Chatrchyan:2011rr}
\bibitem{Chatrchyan:2011rr}
  S.~Chatrchyan {\it et al.}  [CMS Collaboration],
  ``Measurement of W-gamma and Z-gamma production in pp collisions at sqrt(s) =
  7 TeV,''
  Phys.\ Lett.\  B {\bf 701}, 535 (2011)
  [arXiv:1105.2758 [hep-ex]].
  %%CITATION = PHLTA,B701,535;%%

%\cite{Chatrchyan:2011eka}
\bibitem{Chatrchyan:2011eka}
  S.~Chatrchyan {\it et al.}  [CMS Collaboration],
  ``Long-range and short-range dihadron angular correlations in central PbPb
  collisions at a nucleon-nucleon center of mass energy of 2.76 TeV,''
  JHEP {\bf 1107}, 076 (2011)
  [arXiv:1105.2438 [nucl-ex]].
  %%CITATION = JHEPA,1107,076;%%

%\cite{Chatrchyan:2011ig}
\bibitem{Chatrchyan:2011ig}
  S.~Chatrchyan {\it et al.}  [CMS Collaboration],
  ``Measurement of the Polarization of W Bosons with Large Transverse Momenta
  in W+Jets Events at the LHC,''
  Phys.\ Rev.\ Lett.\  {\bf 107}, 021802 (2011)
  [arXiv:1104.3829 [hep-ex]].
  %%CITATION = PRLTA,107,021802;%%

%\cite{Chatrchyan:2011av}
\bibitem{Chatrchyan:2011av}
  S.~Chatrchyan {\it et al.}  [CMS Collaboration],
  ``Charged particle transverse momentum spectra in pp collisions at sqrt(s) =
  0.9 and 7 TeV,''
  JHEP {\bf 1108}, 086 (2011)
  [arXiv:1104.3547 [hep-ex]].
  %%CITATION = JHEPA,1108,086;%%

%\cite{Chatrchyan:2011wba}
\bibitem{Chatrchyan:2011wba}
  S.~Chatrchyan {\it et al.}  [CMS Collaboration],
  ``Search for new physics with same-sign isolated dilepton events with jets
  and missing transverse energy at the LHC,''
  JHEP {\bf 1106}, 077 (2011)
  [arXiv:1104.3168 [hep-ex]].
  %%CITATION = JHEPA,1106,077;%%

%\cite{Chatrchyan:2011pw}
\bibitem{Chatrchyan:2011pw}
  S.~Chatrchyan {\it et al.}  [CMS Collaboration],
  ``Measurement of the B0 production cross section in pp Collisions at sqrt(s)
  = 7 TeV,''
  Phys.\ Rev.\ Lett.\  {\bf 106}, 252001 (2011)
  [arXiv:1104.2892 [hep-ex]].
  %%CITATION = PRLTA,106,252001;%%

%\cite{Chatrchyan:2011qta}
\bibitem{Chatrchyan:2011qta}
  S.~Chatrchyan {\it et al.}  [CMS Collaboration Collaboration],
  ``Measurement of the differential dijet production cross section in
  proton-proton collisions at sqrt(s)=7 TeV,''
  Phys.\ Lett.\  B {\bf 700}, 187 (2011)
  [arXiv:1104.1693 [hep-ex]].
  %%CITATION = PHLTA,B700,187;%%

%\cite{Chatrchyan:2011nx}
\bibitem{Chatrchyan:2011nx}
  S.~Chatrchyan {\it et al.}  [CMS Collaboration],
  ``Search for Neutral MSSM Higgs Bosons Decaying to Tau Pairs in pp Collisions
  at sqrt(s)=7 TeV,''
  Phys.\ Rev.\ Lett.\  {\bf 106}, 231801 (2011)
  [arXiv:1104.1619 [hep-ex]].
%%CITATION = PRLTA,106,231801;%%

%\cite{Chatrchyan:2011nv}
\bibitem{Chatrchyan:2011nv}
  S.~Chatrchyan {\it et al.}  [CMS Collaboration],
  ``Measurement of the Inclusive Z Cross Section via Decays to Tau Pairs in pp
  Collisions at sqrt(s)=7 TeV,''
  JHEP {\bf 1108}, 117 (2011)
  [arXiv:1104.1617 [hep-ex]].
  %%CITATION = JHEPA,1108,117;%%

%\cite{Chatrchyan:2011jx}
\bibitem{Chatrchyan:2011jx}
  S.~Chatrchyan {\it et al.}  [CMS Collaboration],
  ``Search for Large Extra Dimensions in the Diphoton Final State at the Large
  Hadron Collider,''
  JHEP {\bf 1105}, 085 (2011)
  [arXiv:1103.4279 [hep-ex]].
  %%CITATION = JHEPA,1105,085;%%

%\cite{Chatrchyan:2011jz}
\bibitem{Chatrchyan:2011jz}
  S.~Chatrchyan {\it et al.}  [CMS Collaboration],
  ``Measurement of the lepton charge asymmetry in inclusive W production in pp
  collisions at sqrt(s) = 7 TeV,''
  JHEP {\bf 1104}, 050 (2011)
  [arXiv:1103.3470 [hep-ex]].
  %%CITATION = JHEPA,1104,050;%%

%\cite{Chatrchyan:2011bz}
\bibitem{Chatrchyan:2011bz}
  S.~Chatrchyan {\it et al.}  [CMS Collaboration],
  ``Search for Physics Beyond the Standard Model in Opposite-Sign Dilepton
  Events at sqrt(s) = 7 TeV,''
  JHEP {\bf 1106}, 026 (2011)
  [arXiv:1103.1348 [hep-ex]].
  %%CITATION = JHEPA,1106,026;%%

%\cite{Chatrchyan:2011wq}
\bibitem{Chatrchyan:2011wq}
  S.~Chatrchyan {\it et al.}  [CMS Collaboration],
  ``Search for Resonances in the Dilepton Mass Distribution in pp Collisions at
  sqrt(s) = 7 TeV,''
  JHEP {\bf 1105}, 093 (2011)
  [arXiv:1103.0981 [hep-ex]].
  %%CITATION = JHEPA,1105,093;%%

%\cite{Chatrchyan:2011wc}
\bibitem{Chatrchyan:2011wc}
  S.~Chatrchyan {\it et al.}  [CMS Collaboration],
  ``Search for Supersymmetry in pp Collisions at sqrt(s) = 7 TeV in Events with
  Two Photons and Missing Transverse Energy,''
  Phys.\ Rev.\ Lett.\  {\bf 106}, 211802 (2011)
  [arXiv:1103.0953 [hep-ex]].
  %%CITATION = PRLTA,106,211802;%%

%\cite{Chatrchyan:2011dx}
\bibitem{Chatrchyan:2011dx}
  S.~Chatrchyan {\it et al.}  [CMS Collaboration],
  ``Search for a W' boson decaying to a muon and a neutrino in pp collisions at
  sqrt(s) = 7 TeV,''
  Phys.\ Lett.\  B {\bf 701}, 160 (2011)
  [arXiv:1103.0030 [hep-ex]].
  %%CITATION = PHLTA,B701,160;%%

%\cite{Chatrchyan:2011ua}
\bibitem{Chatrchyan:2011ua}
  S.~Chatrchyan {\it et al.}  [CMS Collaboration],
  ``Study of Z boson production in PbPb collisions at nucleon-nucleon centre of
  mass energy = 2.76 TeV,''
  Phys.\ Rev.\ Lett.\  {\bf 106}, 212301 (2011)
  [arXiv:1102.5435 [nucl-ex]].
  %%CITATION = PRLTA,106,212301;%%

%\cite{Chatrchyan:2011tz}
\bibitem{Chatrchyan:2011tz}
  S.~Chatrchyan {\it et al.}  [CMS Collaboration],
  ``Measurement of WW Production and Search for the Higgs Boson in pp
  Collisions at sqrt(s) = 7 TeV,''
  Phys.\ Lett.\  B {\bf 699}, 25 (2011)
  [arXiv:1102.5429 [hep-ex]].
  %%CITATION = PHLTA,B699,25;%%

%\cite{Chatrchyan:2011em}
\bibitem{Chatrchyan:2011em}
  S.~Chatrchyan {\it et al.}  [CMS Collaboration],
  ``Search for a Heavy Bottom-like Quark in pp Collisions at sqrt(s) = 7 TeV,''
  Phys.\ Lett.\  B {\bf 701}, 204 (2011)
  [arXiv:1102.4746 [hep-ex]].
  %%CITATION = PHLTA,B701,204;%%

%\cite{Khachatryan:2011tm}
\bibitem{Khachatryan:2011tm}
  V.~Khachatryan {\it et al.}  [CMS Collaboration],
  ``Strange Particle Production in pp Collisions at sqrt(s) = 0.9 and 7 TeV,''
  JHEP {\bf 1105}, 064 (2011)
  [arXiv:1102.4282 [hep-ex]].
  %%CITATION = JHEPA,1105,064;%%

%\cite{Khachatryan:2011wq}
\bibitem{Khachatryan:2011wq}
  V.~Khachatryan {\it et al.}  [CMS Collaboration],
  ``Measurement of B anti-B Angular Correlations based on Secondary Vertex
  Reconstruction at sqrt(s)=7 TeV,''
  JHEP {\bf 1103}, 136 (2011)
  [arXiv:1102.3194 [hep-ex]].
  %%CITATION = JHEPA,1103,136;%%

%\cite{Khachatryan:2011as}
\bibitem{Khachatryan:2011as}
  V.~Khachatryan {\it et al.}  [CMS Collaboration],
  ``Measurement of Dijet Angular Distributions and Search for Quark
  Compositeness in pp Collisions at 7 TeV,''
  Phys.\ Rev.\ Lett.\  {\bf 106}, 201804 (2011)
  [arXiv:1102.2020 [hep-ex]].
  %%CITATION = PRLTA,106,201804;%%

%\cite{Chatrchyan:2011sx}
\bibitem{Chatrchyan:2011sx}
  S.~Chatrchyan {\it et al.}  [CMS Collaboration],
  ``Observation and studies of jet quenching in PbPb collisions at
  nucleon-nucleon center-of-mass energy = 2.76 TeV,''
  Phys.\ Rev.\  C {\bf 84}, 024906 (2011)
  [arXiv:1102.1957 [nucl-ex]].
  %%CITATION = PHRVA,C84,024906;%%

%\cite{Khachatryan:2011dx}
\bibitem{Khachatryan:2011dx}
  V.~Khachatryan {\it et al.}  [CMS Collaboration],
  ``First Measurement of Hadronic Event Shapes in pp Collisions at sqrt(s)=7
  TeV,''
  Phys.\ Lett.\  B {\bf 699}, 48 (2011)
  [arXiv:1102.0068 [hep-ex]].
  %%CITATION = PHLTA,B699,48;%%

%\cite{Khachatryan:2011zj}
\bibitem{Khachatryan:2011zj}
  V.~Khachatryan {\it et al.}  [CMS Collaboration],
  ``Dijet Azimuthal Decorrelations in pp Collisions at sqrt(s) = 7 TeV,''
  Phys.\ Rev.\ Lett.\  {\bf 106}, 122003 (2011)
  [arXiv:1101.5029 [hep-ex]].
  %%CITATION = PRLTA,106,122003;%%

%\cite{Khachatryan:2011hi}
\bibitem{Khachatryan:2011hi}
  V.~Khachatryan {\it et al.}  [CMS Collaboration],
  ``Measurement of Bose-Einstein Correlations in pp Collisions at sqrt(s)=0.9
  and 7 TeV,''
  JHEP {\bf 1105}, 029 (2011)
  [arXiv:1101.3518 [hep-ex]].
  %%CITATION = JHEPA,1105,029;%%

%\cite{Khachatryan:2011hf}
\bibitem{Khachatryan:2011hf}
  V.~Khachatryan {\it et al.}  [CMS Collaboration],
  ``Inclusive b-hadron production cross section with muons in pp collisions at
  sqrt(s) = 7 TeV,''
  JHEP {\bf 1103}, 090 (2011)
  [arXiv:1101.3512 [hep-ex]].
  %%CITATION = JHEPA,1103,090;%%

%\cite{Khachatryan:2011ts}
\bibitem{Khachatryan:2011ts}
  V.~Khachatryan {\it et al.}  [CMS Collaboration],
  ``Search for Heavy Stable Charged Particles in pp collisions at sqrt(s)=7
  TeV,''
  JHEP {\bf 1103}, 024 (2011)
  [arXiv:1101.1645 [hep-ex]].
  %%CITATION = JHEPA,1103,024;%%

%\cite{Khachatryan:2011tk}
\bibitem{Khachatryan:2011tk}
  V.~Khachatryan {\it et al.}  [CMS Collaboration],
  ``Search for Supersymmetry in pp Collisions at 7 TeV in Events with Jets and
  Missing Transverse Energy,''
  Phys.\ Lett.\  B {\bf 698}, 196 (2011)
  [arXiv:1101.1628 [hep-ex]].
  %%CITATION = PHLTA,B698,196;%%

%\cite{Khachatryan:2011mk}
\bibitem{Khachatryan:2011mk}
  V.~Khachatryan {\it et al.}  [CMS Collaboration],
  ``Measurement of the B+ Production Cross Section in pp Collisions at sqrt(s)
  = 7 TeV,''
  Phys.\ Rev.\ Lett.\  {\bf 106}, 112001 (2011)
  [arXiv:1101.0131 [hep-ex]].
  %%CITATION = PRLTA,106,112001;%%

%\cite{Khachatryan:2010fa}
\bibitem{Khachatryan:2010fa}
  V.~Khachatryan {\it et al.}  [CMS Collaboration],
  ``Search for a heavy gauge boson W' in the final state with an electron and
  large missing transverse energy in pp collisions at sqrt(s) = 7 TeV,''
  Phys.\ Lett.\  B {\bf 698}, 21 (2011)
  [arXiv:1012.5945 [hep-ex]].
  %%CITATION = PHLTA,B698,21;%%

%\cite{Khachatryan:2010zg}
\bibitem{Khachatryan:2010zg}
  V.~Khachatryan {\it et al.}  [CMS Collaboration],
  ``Measurement of the Inclusive Upsilon production cross section in pp
  collisions at sqrt(s)=7 TeV,''
  Phys.\ Rev.\  D {\bf 83}, 112004 (2011)
  [arXiv:1012.5545 [hep-ex]].
  %%CITATION = PHRVA,D83,112004;%%

%\cite{Khachatryan:2010mq}
\bibitem{Khachatryan:2010mq}
  V.~Khachatryan {\it et al.}  [CMS Collaboration],
  ``Search for Pair Production of Second-Generation Scalar Leptoquarks in pp
  Collisions at sqrt(s) = 7 TeV,''
  Phys.\ Rev.\ Lett.\  {\bf 106}, 201803 (2011)
  [arXiv:1012.4033 [hep-ex]].
  %%CITATION = PRLTA,106,201803;%%

%\cite{Khachatryan:2010wx}
\bibitem{Khachatryan:2010wx}
  V.~Khachatryan {\it et al.}  [CMS Collaboration],
  ``Search for Microscopic Black Hole Signatures at the Large Hadron
  Collider,''
  Phys.\ Lett.\  B {\bf 697}, 434 (2011)
  [arXiv:1012.3375 [hep-ex]].
  %%CITATION = PHLTA,B697,434;%%

%\cite{Khachatryan:2010xn}
\bibitem{Khachatryan:2010xn}
  V.~Khachatryan {\it et al.}  [CMS Collaboration],
  ``Measurements of Inclusive W and Z Cross Sections in pp Collisions at
  sqrt(s)=7 TeV,''
  JHEP {\bf 1101}, 080 (2011)
  [arXiv:1012.2466 [hep-ex]].
  %%CITATION = JHEPA,1101,080;%%

%\cite{Khachatryan:2010fm}
\bibitem{Khachatryan:2010fm}
  V.~Khachatryan {\it et al.}  [CMS Collaboration],
  ``Measurement of the Isolated Prompt Photon Production Cross Section in pp
  Collisions at sqrt(s) = 7 TeV,''
  Phys.\ Rev.\ Lett.\  {\bf 106}, 082001 (2011)
  [arXiv:1012.0799 [hep-ex]].
  %%CITATION = PRLTA,106,082001;%%

%\cite{Khachatryan:2010uf}
\bibitem{Khachatryan:2010uf}
  V.~Khachatryan {\it et al.}  [CMS Collaboration],
  ``Search for Stopped Gluinos in pp collisions at sqrt s = 7 TeV,''
  Phys.\ Rev.\ Lett.\  {\bf 106}, 011801 (2011)
  [arXiv:1011.5861 [hep-ex]].
  %%CITATION = PRLTA,106,011801;%%

%\cite{Khachatryan:2010nk}
\bibitem{Khachatryan:2010nk}
  V.~Khachatryan {\it et al.}  [CMS Collaboration],
  ``Charged particle multiplicities in pp interactions at sqrt(s) = 0.9, 2.36,
  and 7 TeV,''
  JHEP {\bf 1101}, 079 (2011)
  [arXiv:1011.5531 [hep-ex]].
  %%CITATION = JHEPA,1101,079;%%

%\cite{Khachatryan:2010yr}
\bibitem{Khachatryan:2010yr}
  V.~Khachatryan {\it et al.}  [CMS Collaboration],
  ``Prompt and non-prompt J/psi production in pp collisions at sqrt(s) = 7
  TeV,''
  Eur.\ Phys.\ J.\  C {\bf 71}, 1575 (2011)
  [arXiv:1011.4193 [hep-ex]].
  %%CITATION = EPHJA,C71,1575;%%

%\cite{Khachatryan:2010ez}
\bibitem{Khachatryan:2010ez}
  V.~Khachatryan {\it et al.}  [CMS Collaboration],
  ``First Measurement of the Cross Section for Top-Quark Pair Production in
  Proton-Proton Collisions at sqrt(s)=7 TeV,''
  Phys.\ Lett.\  B {\bf 695}, 424 (2011)
  [arXiv:1010.5994 [hep-ex]].
  %%CITATION = PHLTA,B695,424;%%

%\cite{Khachatryan:2010te}
\bibitem{Khachatryan:2010te}
  V.~Khachatryan {\it et al.}  [CMS Collaboration],
  ``Search for Quark Compositeness with the Dijet Centrality Ratio in pp
  Collisions at sqrt(s)=7 TeV,''
  Phys.\ Rev.\ Lett.\  {\bf 105}, 262001 (2010)
  [arXiv:1010.4439 [hep-ex]].
  %%CITATION = PRLTA,105,262001;%%

%\cite{Khachatryan:2010jd}
\bibitem{Khachatryan:2010jd}
  V.~Khachatryan {\it et al.}  [CMS Collaboration],
  ``Search for Dijet Resonances in 7 TeV pp Collisions at CMS,''
  Phys.\ Rev.\ Lett.\  {\bf 105}, 211801 (2010)
  [arXiv:1010.0203 [hep-ex]].
  %%CITATION = PRLTA,105,211801;%%

%\cite{Khachatryan:2010gv}
\bibitem{Khachatryan:2010gv}
  V.~Khachatryan {\it et al.}  [CMS Collaboration],
  ``Observation of Long-Range Near-Side Angular Correlations in Proton-Proton
  Collisions at the LHC,''
  JHEP {\bf 1009}, 091 (2010)
  [arXiv:1009.4122 [hep-ex]].
  %%CITATION = JHEPA,1009,091;%%

%\cite{Khachatryan:2010pw}
\bibitem{Khachatryan:2010pw}
  V.~Khachatryan {\it et al.}  [CMS Collaboration],
  ``CMS Tracking Performance Results from early LHC Operation,''
  Eur.\ Phys.\ J.\  C {\bf 70}, 1165 (2010)
  [arXiv:1007.1988 [physics.ins-det]].
  %%CITATION = EPHJA,C70,1165;%%

%\cite{Khachatryan:2010pv}
\bibitem{Khachatryan:2010pv}
  V.~Khachatryan {\it et al.}  [CMS Collaboration],
  ``Measurement of the Underlying Event Activity in Proton-Proton Collisions at
  0.9 TeV,''
  Eur.\ Phys.\ J.\  C {\bf 70}, 555 (2010)
  [arXiv:1006.2083 [hep-ex]].
  %%CITATION = EPHJA,C70,555;%%

%\cite{Khachatryan:2010mw}
\bibitem{Khachatryan:2010mw}
  V.~Khachatryan {\it et al.}  [CMS Collaboration],
  ``Measurement of the charge ratio of atmospheric muons with the CMS
  detector,''
  Phys.\ Lett.\  B {\bf 692}, 83 (2010)
  [arXiv:1005.5332 [hep-ex]].
  %%CITATION = PHLTA,B692,83;%%

%\cite{Khachatryan:2010us}
\bibitem{Khachatryan:2010us}
  V.~Khachatryan {\it et al.}  [CMS Collaboration],
  ``Transverse-momentum and pseudorapidity distributions of charged hadrons in
  pp collisions at sqrt(s) = 7 TeV,''
  Phys.\ Rev.\ Lett.\  {\bf 105}, 022002 (2010)
  [arXiv:1005.3299 [hep-ex]].
  %%CITATION = PRLTA,105,022002;%%

%\cite{Khachatryan:2010un}
\bibitem{Khachatryan:2010un}
  V.~Khachatryan {\it et al.}  [CMS Collaboration],
  ``Measurement of Bose-Einstein correlations with first CMS data,''
  Phys.\ Rev.\ Lett.\  {\bf 105}, 032001 (2010)
  [arXiv:1005.3294 [hep-ex]].
  %%CITATION = PRLTA,105,032001;%%

%\cite{Khachatryan:2010xs}
\bibitem{Khachatryan:2010xs}
  V.~Khachatryan {\it et al.}  [CMS Collaboration],
  ``Transverse momentum and pseudorapidity distributions of charged hadrons in
  pp collisions at sqrt(s) = 0.9 and 2.36 TeV,''
  JHEP {\bf 1002}, 041 (2010)
  [arXiv:1002.0621 [hep-ex]].
  %%CITATION = JHEPA,1002,041;%%

%\cite{:2009dv}
\bibitem{:2009dv}
  S.~Chatrchyan {\it et al.}  [CMS Collaboration],
  ``Commissioning and Performance of the CMS Pixel Tracker with Cosmic Ray
  Muons,''
  JINST {\bf 5}, T03007 (2010)
  [arXiv:0911.5434 [physics.ins-det]].
  %%CITATION = JINST,5,T03007;%%

%\cite{:2009dq}
\bibitem{:2009dq}
  S.~Chatrchyan {\it et al.}  [CMS Collaboration],
  ``Performance of the CMS Level-1 Trigger during Commissioning with Cosmic Ray
  Muons,''
  JINST {\bf 5}, T03002 (2010)
  [arXiv:0911.5422 [physics.ins-det]].
  %%CITATION = JINST,5,T03002;%%

%\cite{:2009dg}
\bibitem{:2009dg}
  S.~Chatrchyan {\it et al.}  [CMS Collaboration],
  ``Measurement of the Muon Stopping Power in Lead Tungstate,''
  JINST {\bf 5}, P03007 (2010)
  [arXiv:0911.5397 [physics.ins-det]].
  %%CITATION = JINST,5,P03007;%%

%\cite{:2009vs}
\bibitem{:2009vs}
  S.~Chatrchyan {\it et al.}  [CMS Collaboration],
  ``Commissioning and Performance of the CMS Silicon Strip Tracker with Cosmic
  Ray Muons,''
  JINST {\bf 5}, T03008 (2010)
  [arXiv:0911.4996 [physics.ins-det]].
  %%CITATION = JINST,5,T03008;%%

%\cite{:2009vq}
\bibitem{:2009vq}
  S.~Chatrchyan {\it et al.}  [CMS Collaboration],
  ``Performance of CMS Muon Reconstruction in Cosmic-Ray Events,''
  JINST {\bf 5}, T03022 (2010)
  [arXiv:0911.4994 [physics.ins-det]].
  %%CITATION = JINST,5,T03022;%%

%\cite{:2009vp}
\bibitem{:2009vp}
  S.~Chatrchyan {\it et al.}  [CMS Collaboration],
  ``Performance of the CMS Cathode Strip Chambers with Cosmic Rays,''
  JINST {\bf 5}, T03018 (2010)
  [arXiv:0911.4992 [physics.ins-det]].
  %%CITATION = JINST,5,T03018;%%

%\cite{:2009vn}
\bibitem{:2009vn}
  S.~Chatrchyan {\it et al.}  [CMS Collaboration],
  ``Performance of the CMS Hadron Calorimeter with Cosmic Ray Muons and LHC
  Beam Data,''
  JINST {\bf 5}, T03012 (2010)
  [arXiv:0911.4991 [physics.ins-det]].
  %%CITATION = JINST,5,T03012;%%

%\cite{Chatrchyan:2009im}
\bibitem{Chatrchyan:2009im}
  S.~Chatrchyan {\it et al.}  [CMS Collaboration],
  ``Fine Synchronization of the CMS Muon Drift-Tube Local Trigger using Cosmic
  Rays,''
  JINST {\bf 5}, T03004 (2010)
  [arXiv:0911.4904 [physics.ins-det]].
  %%CITATION = JINST,5,T03004;%%

%\cite{Chatrchyan:2009ih}
\bibitem{Chatrchyan:2009ih}
  S.~Chatrchyan {\it et al.}  [CMS Collaboration],
  ``Calibration of the CMS Drift Tube Chambers and Measurement of the Drift
  Velocity with Cosmic Rays,''
  JINST {\bf 5}, T03016 (2010)
  [arXiv:0911.4895 [physics.ins-det]].
  %%CITATION = JINST,5,T03016;%%

%\cite{Chatrchyan:2009ig}
\bibitem{Chatrchyan:2009ig}
  S.~Chatrchyan {\it et al.}  [CMS Collaboration],
  ``Performance of the CMS Drift-Tube Local Trigger with Cosmic Rays,''
  JINST {\bf 5}, T03003 (2010)
  [arXiv:0911.4893 [physics.ins-det]].
  %%CITATION = JINST,5,T03003;%%

%\cite{Chatrchyan:2009ic}
\bibitem{Chatrchyan:2009ic}
  S.~Chatrchyan {\it et al.}  [CMS Collaboration],
  ``Commissioning of the CMS High-Level Trigger with Cosmic Rays,''
  JINST {\bf 5}, T03005 (2010)
  [arXiv:0911.4889 [physics.ins-det]].
  %%CITATION = JINST,5,T03005;%%

%\cite{Chatrchyan:2009hw}
\bibitem{Chatrchyan:2009hw}
  S.~Chatrchyan {\it et al.}  [CMS Collaboration],
  ``Performance of CMS Hadron Calorimeter Timing and Synchronization using Test
  Beam, Cosmic Ray, and LHC Beam Data,''
  JINST {\bf 5}, T03013 (2010)
  [arXiv:0911.4877 [physics.ins-det]].
  %%CITATION = JINST,5,T03013;%%

%\cite{Chatrchyan:2009hg}
\bibitem{Chatrchyan:2009hg}
  S.~Chatrchyan {\it et al.}  [CMS Collaboration],
  ``Performance of the CMS Drift Tube Chambers with Cosmic Rays,''
  JINST {\bf 5}, T03015 (2010)
  [arXiv:0911.4855 [physics.ins-det]].
  %%CITATION = JINST,5,T03015;%%

%\cite{Chatrchyan:2009hb}
\bibitem{Chatrchyan:2009hb}
  S.~Chatrchyan {\it et al.}  [CMS Collaboration],
  ``Commissioning of the CMS Experiment and the Cosmic Run at Four Tesla,''
  JINST {\bf 5}, T03001 (2010)
  [arXiv:0911.4845 [physics.ins-det]].
  %%CITATION = JINST,5,T03001;%%

%\cite{:2009gz}
\bibitem{:2009gz}
  S.~Chatrchyan {\it et al.}  [CMS Collaboration],
  ``CMS Data Processing Workflows during an Extended Cosmic Ray Run,''
  JINST {\bf 5}, T03006 (2010)
  [arXiv:0911.4842 [physics.ins-det]].
  %%CITATION = JINST,5,T03006;%%

%\cite{:2009ft}
\bibitem{:2009ft}
  S.~Chatrchyan {\it et al.}  [CMS Collaboration],
  ``Aligning the CMS Muon Chambers with the Muon Alignment System during an
  Extended Cosmic Ray Run,''
  JINST {\bf 5}, T03019 (2010)
  [arXiv:0911.4770 [physics.ins-det]].
  %%CITATION = JINST,5,T03019;%%

%\cite{Chatrchyan:2009ks}
\bibitem{Chatrchyan:2009ks}
  S.~Chatrchyan {\it et al.}  [CMS Collaboration],
  ``Performance Study of the CMS Barrel Resistive Plate Chambers with Cosmic
  Rays,''
  JINST {\bf 5}, T03017 (2010)
  [arXiv:0911.4045 [physics.ins-det]].
  %%CITATION = JINST,5,T03017;%%

%\cite{:2009kr}
\bibitem{:2009kr}
  S.~Chatrchyan {\it et al.}  [CMS Collaboration],
  ``Time Reconstruction and Performance of the CMS Electromagnetic
  Calorimeter,''
  JINST {\bf 5}, T03011 (2010)
  [arXiv:0911.4044 [physics.ins-det]].
  %%CITATION = JINST,5,T03011;%%

%\cite{Chatrchyan:2009km}
\bibitem{Chatrchyan:2009km}
  S.~Chatrchyan {\it et al.}  [CMS Collaboration],
  ``Alignment of the CMS Muon System with Cosmic-Ray and Beam-Halo Muons,''
  JINST {\bf 5}, T03020 (2010)
  [arXiv:0911.4022 [physics.ins-det]].
  %%CITATION = JINST,5,T03020;%%

%\cite{Chatrchyan:2009si}
\bibitem{Chatrchyan:2009si}
  S.~Chatrchyan {\it et al.}  [CMS Collaboration],
  ``Precise Mapping of the Magnetic Field in the CMS Barrel Yoke using Cosmic
  Rays,''
  JINST {\bf 5}, T03021 (2010)
  [arXiv:0910.5530 [physics.ins-det]].
  %%CITATION = JINST,5,T03021;%%

%\cite{Chatrchyan:2009qm}
\bibitem{Chatrchyan:2009qm}
  S.~Chatrchyan {\it et al.}  [CMS Collaboration],
  ``Performance and Operation of the CMS Electromagnetic Calorimeter,''
  JINST {\bf 5}, T03010 (2010)
  [arXiv:0910.3423 [physics.ins-det]].
  %%CITATION = JINST,5,T03010;%%

%\cite{Chatrchyan:2009sr}
\bibitem{Chatrchyan:2009sr}
  S.~Chatrchyan {\it et al.}  [CMS Collaboration],
  ``Alignment of the CMS Silicon Tracker during Commissioning with Cosmic
  Rays,''
  JINST {\bf 5}, T03009 (2010)
  [arXiv:0910.2505 [physics.ins-det]].
  %%CITATION = JINST,5,T03009;%%

%\cite{:2008zzk}
\bibitem{:2008zzk}
  S.~Chatrchyan {\it et al.}  [CMS Collaboration],
  ``The CMS experiment at the CERN LHC,''
  JINST {\bf 3}, S08004 (2008).
  %%CITATION = JINST,3,S08004;%%


\end{thebibliography}







 
%\section*{Areas of competence}
%Software Development, IT, Particle detector physics
 
%\vspace{1cm}
\vfill{}
\hrulefill

% FILL IN THE FULL URL TO YOUR CV
\begin{center}
%{\footnotesize \href{http://www.ias.edu/spfeatures/einstein}{http://www.ias.edu/spfeatures/einstein} — Last updated: \today}
{\footnotesize Last updated: \today}
\end{center}


\end{document}
