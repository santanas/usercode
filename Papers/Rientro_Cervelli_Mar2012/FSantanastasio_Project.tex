%------------------------------------
% Dario Taraborelli
% Typesetting your academic CV in LaTeX
%
% URL: http://nitens.org/taraborelli/cvtex
% DISCLAIMER: This template is provided for free and without any guarantee 
% that it will correctly compile on your system if you have a non-standard  
% configuration.
%------------------------------------ 


% ! TEX TS-program = XeLaTeX -xdv2pdf
% ! TEX encoding = UTF-8 Unicode

\documentclass[10pt, a4paper]{article}
\usepackage{fontspec} 
\usepackage{xunicode} 
\usepackage{xltxtra}
% per le lettere accentate italiane sul Mac! :-)
%\usepackage[applemac]{inputenc} %with VIM
\usepackage[latin1]{inputenc} % with TeXShop


% DOCUMENT LAYOUT
\usepackage{geometry}
\geometry{a4paper, textwidth=5.5in, textheight=8.5in, marginparsep=7pt, marginparwidth=.6in}
\setlength\parindent{0in}

% ADDITIONAL SYMBOLS
%\usesymbols[mvs]

% FONTS
\defaultfontfeatures{Mapping=tex-text} % converts LaTeX specials (``quotes'' --- dashes etc.) to unicode
%\setromanfont [Ligatures={Common}, BoldFont={Fontin Bold}, ItalicFont={Fontin Italic}]{Fontin}
\setromanfont [Ligatures={Common}, BoldFont={Linux Libertine Bold}, ItalicFont={Linux Libertine Italic}]{Linux Libertine}
%\setsansfont [Ligatures={Common}, BoldFont={Fontin Sans Bold}, ItalicFont={Fontin Sans Italic}]{Fontin Sans}
\setmonofont[Scale=0.8]{Monaco} 
% ---- CUSTOM AMPERSAND
\newcommand{\amper}{{\fontspec[Scale=.95]{Linux Libertine Bold}\selectfont\itshape\&}}
% ---- MARGIN YEARS
%\newcommand{\years}[1]{\marginpar{\scriptsize #1}}
\newcommand{\years}[1]{\marginpar{\footnotesize #1}}

% HEADINGS
\usepackage{sectsty} 
\usepackage[normalem]{ulem} 
\sectionfont{\rmfamily\mdseries\upshape\Large}
\subsectionfont{\rmfamily\bfseries\upshape\normalsize} 
\subsubsectionfont{\rmfamily\mdseries\upshape\normalsize} 
%modifying section numbering
\def\thesubsection{\arabic{subsection}.\ } 

% PDF SETUP
% ---- FILL IN HERE THE DOC TITLE AND AUTHOR
\usepackage[dvipdfm, bookmarks, colorlinks, breaklinks, pdftitle={Francesco Santanastasio - Motivation for coming to CERN},pdfauthor={Francesco Santanastasio}]{hyperref}
%\hypersetup{linkcolor=blue,citecolor=blue,filecolor=black,urlcolor=blue} 
\hypersetup{linkcolor=cyan,citecolor=blue,filecolor=black,urlcolor=cyan} 

% Title of Bibliography
%\renewcommand\refname{References \\ \normalsize \begin{center} \quad \quad \textsc{Publications (relative to research activities)}\end{center} }

% DOCUMENT
\begin{document}
\reversemarginpar

%%\hrule
\section*{Research Project}

% EW SYMMETRY BREAKING IN SM (HIGGS)
One century of experimental measurements and progress in theoretical physics 
led to an extremely compact and elegant theory of fundamental interactions between 
elementary particles, the Standard Model (SM). Its success in reproducing 
measurements from different experiments in energy regimes spanning 
over several orders of magnitude is astonishing. 
Strong, weak and electromagnetic interactions are all described within the 
same mathematical framework of gauge theories. Although the electromagnetic and weak 
interactions are related to the same SU(2)$_L$ x U(1)$_Y$ invariance, only 
the electromagnetic symmetry is manifest in the mass spectrum. 
The rest of the electroweak symmetry is hidden, that is, it is spontaneously broken. 
The detailed mechanism through which the breaking happens is not clear, though. 
The simplest way this could be explained theoretically is through the 
so-called Higgs mechanism of the SM. This mechanism explain, for instance, why 
elementary particles have mass. The Higgs mechanism postulates the existence 
of a new scalar particle, the Higgs boson, whose mass is not theoretically predicted by 
the SM, but that should be experimentally observable at particle colliders. \\

% LHC 
The Large Hadron Collider (LHC) is the largest proton-proton (pp) collider ever built. 
It is located at CERN, Geneva, and its main objective is to finally unravel the origin 
of the electroweak symmetry breaking. Using the pp collision data at the center-of-mass energy
of 7 TeV collected in 2011, ATLAS and CMS, the 
largest experiments at the LHC, excluded the SM Higgs in the mass range 
127-600 GeV, while masses below 114 GeV were already excluded by 
previous experiments at the electron-positron LEP collider. 
Thus, the mass range 114 GeV-127 GeV is currently the only one in which 
a Standard Model Higgs boson can hide, and it is in fact also the range 
preferred by the electroweak precision tests performed at LEP. 
In this mass range, the ATLAS and CMS experiments observe an excursion of 
the observed data from the expected background, that is compatible with the existence 
of a SM Higgs boson with mass around 125 GeV. However, no claim of 
discovery is possible at the moment given the small statistical significance of the excess. 

% THE 2012 RUN
In 2012, the LHC will collide protons at a center-of-mass energy of 8 TeV, delivering 
a number of collisions three times higher than in 2011. The higher energy and larger 
amount of data will allow to either confirm the "125 GeV" signal or rule out the existence 
of a SM Higgs by the end of the year. The LHC is scheduled to enter a 
long technical stop at the end of 2012 to prepare for running at its full design 
center-of-mass energy of around 14 TeV in early 2015.

% EW SYMMETRY BREAKING IN SM (WW SCATTERING)
In this contest, the scattering of two longitudinally polarized 
W bosons (WW scattering) is a promising channel to 
investigate the electroweak symmetry breaking (EWSB) mechanism.
In fact, in absence of the Higgs boson contribution, this SM process
would violate the unitarity of the diffusion amplitude
at a center-of-mass energy of around 1 TeV; thus we know that, in this scenario, 
interesting physics must emerge at that energy scale.
Anyway, the WW scattering carries a direct information about the EWSB 
mechanism, no matter whether a physical elementary Higgs particle 
exists or some kind of strongly interacting physics is responsible for this breaking. 
In fact, even if the "125 GeV" signal is confirmed with high statistical significance by the 
2012 data analysis, the energy dependence of the longitudinal WW scattering 
above the Higgs candidate mass scale will tell us if the SM Higgs boson 
unitarizes the WW scattering fully or only partially, as predicted in some 
theoretical models with composite Higgs. Therefore, the study of the WW scattering 
at high center-of-mass-energy is, in any case, a fundamental milestone 
in the physics program of the LHC experiments. \\

% PHYSICS BEYOND SM
Beyond this, it is important to mention that there are many unsatisfactory 
aspects in the picture depicted by the SM: the hierarchy problem, i.e. the big 
gap between the electroweak energy scale and the Planck scale at which 
quantum effects of gravity become strong, is seen as one of its major limitation 
and has been the driving force for many theoretical developments extending the 
SM. Although the panorama of alternative new physics models is wide, one the 
most appealing and popular alternative is represented by the existence of Extra Dimensions. 
In the original Randall-Sundrum (RS) model, the hierarchy problem
of the SM is solved using a theoretical framework that includes a warped extra dimension
in which gravity can propagate. 
The most distinctive feature of this scenario is the existence of spin-2 gravitons
whose masses and couplings to the SM are set by the TeV scale. The gravitons 
appear in experiments as widely separated resonances. 
Decays of the graviton to pairs of electrons, muons, or photons are traditionally among 
the golden channels for searches of extra dimensions.
Well-motivated extensions of the original RS model address the flavor structure of the SM through 
localization of fermions in the warped bulk. This picture offers a unified geometric explanation 
of both the hierarchy and the flavor puzzles in the SM.
In this scenario, graviton production and decay with light fermion channels are 
highly suppressed and the decay into photons are negligible. 
However the production of gravitons from gluon fusion and 
their decay into a pair of longitudinal gauge bosons (W$_L$/Z$_L$) can be significant.
In general, new resonances decaying to a pair of vector bosons 
are also foreseen by other models of physics beyond the SM such as 
excited quarks, technicolor, etc.. \\
No signs of physics beyond the standard model have been observed so far
by the LHC experiments. The increase in the LHC energy from 7 TeV to 8 TeV 
and a sample of data three times larger than in 2011, 
will significantly extend in 2012 the discovery reach for many new physics models, 
including the ones aforementioned.\\

The research project presented in this document 
is motivated by the current knowledge in the experimental and 
theoretical fields discussed above, and it fits well with 
the physics program of the CMS experiment at LHC, in which 
I have been working since the beginning of my graduate studies in 2004.
In the past 3 years, I have been based at CERN playing a leading role in: \\ 
- various searches for new physics beyond the SM using pp collision data;\\
- commissioning, and detector performance studies of the hadronic calorimeter (HCAL); \\
- performance studies of missing transverse energy (MET) reconstruction. \\
At CMS, the HCAL is mainly employed, together 
with electromagnetic calorimeter ECAL, for the reconstruction 
of jets and the missing transverse energy in the event, 
hence playing an important role for physics analyses presented in this proposal.
The project is structured in various phases, accordingly with the 
LHC schedule for the next few years.\\

1) Search for New Resonances Decaying to Pairs of Vector Bosons\\
In the first period of the contract, I will search for new physics beyond the 
standard model by studying the decay of heavy resonances (X) in pairs of vector bosons
(VV = WW / WZ / ZZ). The analysis will focus on the semi-leptonic (lvjj and lljj) and 
fully hadronic (jjjj) final states using the data that will be collected by CMS in 2012
at a center-of-mass energy of 8 TeV. 
The motivation is twofold. First, these W/Z decay channels have the largest 
branching fraction, thus allowing to extend the sensitivity to new physics to higher values of 
resonance mass (i.e. lower cross section) compared to the 
fully leptonic channels. Second, I already developed a solid expertise in 
the study of these final states during the past years in CMS, both in terms of 
analysis methods and reconstruction of physics objects. Ultimately, all channels 
could be combined to increase the sensitivity to the new physics in the entire mass range.

Using the first pp data collected by CMS in 2010, I took a leading role 
in the search for pair production of leptoquarks (LQ) in the LQLQbar-->eejj final state [1] and 
I was the contact person for the search LQLQbar-->evjj [2]. 
Both results have been published in 
well-known scientific journals. In addition, I have been supervising 
a PhD student from Princeton University to update both analyses with 
the almost 5 fb-1 of data collected in 2011. The experience gained in these 
analyses will be used to search for heavy X-->VV resonances in semi-leptonic final states.
I have also been member of the {\it``Analysis Review Committee''} (ARC) for the scrutiny of a public CMS result within the collaboration: search for Randall-Sundrum gravitons decaying into 
a massive jet plus missing transverse energy final state (G-->ZZ-->vvjj) [3]. This analysis 
shares some of the experimental issues with the searches proposed in this physics program. 
I am also currently involved in a search for new resonances that decay to a pair 
of jets originated by hadronization of quarks or gluons (dijets) which is 
aiming to deliver a public result in early 2012. 
The fully hadronic X-->VV-->jjjj channel mentioned above 
can be considered an extension of the inclusive dijet search. 
At a sufficiently high resonance mass, the hadronic decay products of each energetic 
vector boson can be merged into a single massive jet, and the 
final state effectively assumes a dijet topology in the laboratory reference frame. 
This kinematic effect, which becomes important at resonance masses above 1 TeV, 
%is also 
%present in the leptonic W/Z decays, 
%and it 
creates challenging  experimental issues for the reconstruction 
of these boosted topologies. \\

2) Study Techniques for the Reconstruction of Boosted Vector Boson Decays\\
The identification of an energetic vector boson (V) decaying into a pair of 
very collimated quarks, and thus resulting in a single massive jet, is an experimental 
challenge for both the semi-leptonic (lvjj and lljj) and fully hadronic (jjjj) channels 
of the X-->VV searches proposed at point 1). Achieving this goal is important 
to reduce the SM backgrounds arising from production of W/Z bosons in 
association with jets, QCD multijet events, and pairs of top-antitop quarks.
In the past few years, several algorithms to resolve the substructure 
of a massive jet have been proposed. 
A recent comprehensive summary, result of the fruitful dialogue 
between theorists and experimentalists since 2009, can be found in Ref. [4]. 
I will focus on the study of the performance 
of the existing jet substructure algorithms, in order to identify the 
most promising ones in the contest of the proposed searches. 
Although powerful tools for physics analyses, 
the jet substructure observables are particularly sensitive 
to the specific Monte Carlo (MC) description in the simulation. 
Variations in the parton shower model, the underlying event 
activity, or the detector model can have a non-negligible impact 
in quantities such as the jet mass or the number of substructures in the jet. 
It will be important to compare the distributions of such observables 
between different generators and collision data in order to verify the agreement, 
and eventually tune the MC parameters to improve the description
of the simulation for this kind of analyses.\\
%
Also leptons from W/Z decays can be emitted with small angular separation 
if the momentum of the vector boson greatly exceed its mass. The experimental 
issues are however less complex than the jet merging case discussed above. 
For instance, existing analyses performed in CMS has already addressed the problem 
of reconstructing the high momentum Z-->ll decays in the contest of searches 
for heavy excited quarks decaying into a light quark plus a Z boson. These techniques 
could be included in the proposed analyses. \\


 
3) Study of WW scattering in pp collisions at center-of-mass energy of 14 TeV \\
The second part of the contract will be devoted to the study 
of the scattering of longitudinally polarized vector bosons which is 
a crucial measurement to understand the origin of the electroweak symmetry breaking mechanism. 
The work done in points 1) and 2) will be preparatory to study the WW scattering 
in the aforementioned final states.
The plan includes an accurate feasibility study with simulated events to be performed 
during the long shutdown of the accelerator, in preparation for the 14 TeV, high 
luminosity phase of the LHC in 2015. The WW scattering is infact is a rare process 
in the SM and a large amount of data will be needed to perform this analysis. 
Although previous studies of the WW scattering have been performed in the 
past within the CMS collaboration, up-to-date studies with with realistic detector and 
LHC conditions are still few. In particular, it is important to study new strategies for extracting the signal from the scattering of longitudinally polarized vector bosons from the irreducible background arising from the transverse polarizations. A possible strategy in this sense is the study of angular distributions of fermions from W decays, as recently suggested in Ref. [5]. \\
The WW scattering process occurs via the Vector Boson Fusion (VBF) process
with the associated production of two energetic forward jets. With the increase of the 
LHC luminosity, it will be important to study the impact on the analysis of multiple 
interactions occurring at each crossing of protons 
bunches (pileup interactions), as well as to find techniques to mitigate such effects.
Infact, jets coming from pileup interactions might overlap with regular 
WW events (not from VBF), thus faking the WW scattering signature. This preparatory
work will be done using both the simulation and the data collected in 2012 to understand the 
capabilitiy of the MC to model correctly these effects. \\


1. CMS Collaboration, Phys. Rev. Lett. 106, 201802 (2011), arXiv:1012.4031 [hep-ex] \\ 
2. CMS Collaboration, Phys. Lett. B 703, 246 (2011), arXiv:1105.5237 [hep-ex] \\
3. CMS Collaboration, CMS PAS EXO-11-061 (2011), http://cdsweb.cern.ch/record/1426654  \\ 
4. A. Abdesselam et al., Eur. Phys. J. C71:1661(2011), arXiv:1012.5412 [hep-ph] \\
5. T. Han, D. Krohn, L-T. Wang and W. Zhu, JHEP 1003 (2010) 082 \\
  
%In the next months, I would like to focus on searches for physics beyond the standard 
%model, as well as to participate in development and performance studies of advanced algorithms 
%for jets and missing transverse energy reconstruction, because of their inherent interest and for 
%their importance in the physics analyses aforementioned. ?e detailed scope of this research will 
%continue to be shaped in the course of the data taking and analysis of the current run. In a longer- 
%term future, I am planning to explore other possible analyses (as Higgs searches), including those 
%for which the research group that I may join has expertise and interest. 


% TRE FASI: 
% 1) WW/WZ/ZZ resonances, posso usare dati del 2012 
% (focus on fully hadronic, semi-leptonic)
% ricorda che hai fatto 
% 2) Studio dei decadimenti di W/Z boostati
% jet substructure, lepton isolation , 
% backgrounds
% 1) and 2) propedeutic to WW scattering
% 3) WW scattering feasibility study and analysis with 2015 data
% separation longitudinal/transverse polarization , study forward jets from VBF (can use 2012 data)
% robba vecchia da TDR, fare nuovi studi di fattibilita', se tutto va bene nel 2015 si fa l'analisisi..
% 4) ECAL calibration pi0--> gamma+gamma, monitoring HV system, data taking shifts
% di che la calibrazione l'ai inventata tu nella tua tesi di Phd, e che hai pubblicato anche su HV
% Ricorda expertise nei vari settori

% fitta bene con roma
% Roma gia' lavora su Higgs, canali esclusivi, WW scattering
% Roma gia' coinvolta in ricerche di fisica esotica, e jets.
% Interesse teorico in sede verso WW scattering ed extra dimensions

% frasona finale
 
%\section*{Areas of competence}
%Software Development, IT, Particle detector physics
 
%\vspace{1cm}
\vfill{}
\hrulefill

% FILL IN THE FULL URL TO YOUR CV
\begin{center}
%{\footnotesize \href{http://www.ias.edu/spfeatures/einstein}{http://www.ias.edu/spfeatures/einstein} — Last updated: \today}
{\footnotesize Last updated: \today}
\end{center}


\end{document}
