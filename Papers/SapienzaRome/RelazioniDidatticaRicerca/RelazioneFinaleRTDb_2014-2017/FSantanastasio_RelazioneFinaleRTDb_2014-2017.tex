%------------------------------------
% Dario Taraborelli
% Typesetting your academic CV in LaTeX
%
% URL: http://nitens.org/taraborelli/cvtex
% DISCLAIMER: This template is provided for free and without any guarantee 
% that it will correctly compile on your system if you have a non-standard  
% configuration.
%------------------------------------ 


% ! TEX TS-program = XeLaTeX -xdv2pdf
% ! TEX encoding = UTF-8 Unicode

\documentclass[10pt, a4paper]{article}
\usepackage{fontspec} 
\usepackage{xunicode} 
\usepackage{xltxtra}
% per le lettere accentate italiane sul Mac! :-)
%\usepackage[applemac]{inputenc} %with VIM
%FIXME%\usepackage[latin1]{inputenc} % with TeXShop
\usepackage[utf8]{inputenc} % with TeXShop

% DOCUMENT LAYOUT
\usepackage{geometry}
\geometry{a4paper, textwidth=5.5in, textheight=8.5in, marginparsep=7pt, marginparwidth=.6in}
\setlength\parindent{0in}

% ADDITIONAL SYMBOLS
%\usesymbols[mvs]

% FONTS
\defaultfontfeatures{Mapping=tex-text} % converts LaTeX specials (``quotes'' --- dashes etc.) to unicode
%\setromanfont [Ligatures={Common}, BoldFont={Fontin Bold}, ItalicFont={Fontin Italic}]{Fontin}
%\setromanfont [Ligatures={Common}, BoldFont={Linux Libertine Bold}, ItalicFont={Linux Libertine Italic}]{Linux Libertine}
%\setsansfont [Ligatures={Common}, BoldFont={Fontin Sans Bold}, ItalicFont={Fontin Sans Italic}]{Fontin Sans}
\setmonofont[Scale=0.8]{Monaco} 
% ---- CUSTOM AMPERSAND
\newcommand{\amper}{{\fontspec[Scale=.95]{Linux Libertine Bold}\selectfont\itshape\&}}
% ---- MARGIN YEARS
%\newcommand{\years}[1]{\marginpar{\scriptsize #1}}
%GOOD ONE%%\newcommand{\years}[1]{\marginpar{\footnotesize #1}}
%FIXME%\newcommand{\years}[1]{\makebox[0pt][l]{\hskip-1in{\footnotesize #1}}}
%\newcommand{\years}[1]{\makebox[0pt][l]{\hskip-1in{\footnotesize #1}}}
%\newcommand{\years}[1]{\makebox[0pt][l]{\hskip-1.2in{\footnotesize #1}}}
% ---- MARGIN YEARS
\usepackage{marginnote}
%\newcommand{\years}[1]{\marginnote{\hskip-0.2in{\scriptsize #1}}}
\newcommand{\years}[1]{\marginnote{\hskip-0.2in{\small #1}}}
\renewcommand*{\raggedrightmarginnote}{}
\setlength{\marginparsep}{7pt}
%\reversemarginpar


% HEADINGS
\usepackage{sectsty} 
\usepackage[normalem]{ulem} 
\sectionfont{\rmfamily\mdseries\upshape\Large}
\subsectionfont{\rmfamily\bfseries\upshape\normalsize} 
\subsubsectionfont{\rmfamily\mdseries\upshape\normalsize} 
%modifying section numbering
\def\thesubsection{\arabic{subsection}.\ } 

% PDF SETUP
% ---- FILL IN HERE THE DOC TITLE AND AUTHOR
%\usepackage[dvipdfm, bookmarks, colorlinks, breaklinks, pdftitle={Francesco Santanastasio - Curriculum Vitae},pdfauthor={Francesco Santanastasio}]{hyperref}
\usepackage[bookmarks, colorlinks, breaklinks, pdftitle={Francesco Santanastasio - Curriculum Vitae},pdfauthor={Francesco Santanastasio}]{hyperref}
%\hypersetup{linkcolor=blue,citecolor=blue,filecolor=black,urlcolor=blue} 
\hypersetup{linkcolor=cyan,citecolor=blue,filecolor=black,urlcolor=cyan} 

% Title of Bibliography
\renewcommand\refname{References \\ \normalsize \begin{center} \quad
    \quad \textsc{Publications quoted in this document}\end{center} }


\newcommand{\HRule}{\hrule\vspace{0.2mm}}

% DOCUMENT
\begin{document}
\reversemarginpar
{\LARGE Relazione finale dell'attività come RTDb (2014-2017) \\di Francesco Santanastasio}\\[0.5cm]
%Institute address
%begin{tabular}{ l c l }
%\emph{Institute Address}: & & \\
%University of Maryland & & \\
%Department of Physics - John S. Toll Physics Building & &\\
%College Park  & & \\
%MD  \texttt{20742-4111} & \makebox[1.2cm]{} & Tel: \texttt{+1 301 405 3401} \\
%United States of America & & Fax: \texttt{+1 301 314 9525} \\
%\end{tabular}\\[1em]

% Work address
\begin{tabular}{ l c l }
%\emph{Work Address}: & \makebox[1.cm]{} & \\
%Department of Physics, Sapienza Universit\`a di Roma & & Office Phone: \texttt{+39 06 4991 42 70} \\
%Edificio Marconi, floor 1, room 145 & & e-mail: \\
%Piazzale Aldo Moro, \texttt{2} 00185 Rome & & \href{mailto:francesco.santanastasio@cern.ch}{francesco.santanastasio@cern.ch} \\
%Italy &  &  \href{mailto:francesco.santanastasio@roma1.infn.it}{francesco.santanastasio@roma1.infn.it}  \\
Born:  9 February 1980 --- Rome, Italy & \makebox[1.cm]{}& \href{mailto:francesco.santanastasio@cern.ch}{francesco.santanastasio@cern.ch} \\
Nationality:  Italian &  \makebox[1.cm]{} &  \href{mailto:francesco.santanastasio@roma1.infn.it}{francesco.santanastasio@roma1.infn.it}  \\
\end{tabular}\\[1em]
%% Work address
%\begin{tabular}{ l c l }
%\emph{Work Address}: & & \\
%CERN (Conseil Europeen pour la Recherche Nucleaire) & \makebox[1.cm]{} & \\
%\texttt{CH-1211} Geneve  \texttt{23} & & Office Phone.: \texttt{+41 22 76 71 689}\\
%Building \texttt{40}, Room 1-\texttt{B01} & & Mobile Phone: \texttt{+41 76 22 86 127}\\ 
%Switzerland &  & email: \href{mailto:francesco.santanastasio@cern.ch}{francesco.santanastasio@cern.ch} 
%\end{tabular}\\[1em]
%\vfill
%Born:  9 February 1980 --- Rome, Italy\\
%Nationality:  Italian
%\textsc{url}: \href{http://www.ias.edu/spfeatures/einstein/}{http://www.ias.edu/spfeatures/einstein/}\\ 

This document summarizes my academic and research activities during the RTDb contract.
The complete \emph{curriculum vitae} is in attachment
(\textsc{Allegato 1}) and
also available at\\ 
\href{http://www.roma1.infn.it/~santanas/FSantanastasio_CV_ENG_SHORT.pdf}{http://www.roma1.infn.it/\~santanas/FSantanastasio\_CV\_ENG\_SHORT.pdf} 

%%\hrule
\section*{Current Position}
\vspace{-5pt}
\hrule
\vspace{10pt}
\emph{Assistant Professor (Ricercatore Tempo Determinato di Tipo B)}
from 01/03/2014 \\
Department of Physics, Sapienza Universit\`a di Roma, Rome, Italy %\\ %[1em]
%\emph{CERN Research Fellow in Experimental Particle Physics} \\
%PH Department, CERN, Geneve, Switzerland
%\emph{Post-Doctoral Research Assistant (Post-Doc) in Particle Physics} \\
%Department of Physics, University of Maryland, College Park, US
%Start of the contract: 01/03/2014 
%\\
%End of the contract: 28/02/2017 

%%\hrule
%\section*{Areas of Specialization}
%Particle Physics, Data Analysis in High Energy Physics, Physics beyond the Standard Model of Fundamental Interactions, Electromagnetic and Hadronic Calorimetry 
%\section*{Areas of competence}
%Software Development, IT, Particle detector physics

\section*{Teaching}
\vspace{-5pt}
\hrule
\vspace{10pt}
\years{AA 16-17}\textbf{Fisica I, Sapienza, Corso di Laurea in Chimica Industriale (9 CFU)} \\
\emph{Mechanics and Thermodynamics}\\ [1em]
\years{}\textbf{Teaching assistant - Fisica I, Sapienza, Corso di Laurea in Fisica (8 hours)} \\
\emph{Mechanics}\\ [1em]
\years{}\textbf{Teaching assistant - Laboratorio di Fisica Subnucleare, Sapienza, Corso di Laurea in Fisica} \\
\emph{Co-responsible for experiment on Crystal Ball calorimeter (``La Sfera'')}\\ [1em]
\years{AA 15-16}\textbf{Fisica I, Sapienza, Corso di Laurea in Chimica Industriale (9 CFU)} \\
\emph{Mechanics and Thermodynamics}\\ [1em]
\years{AA 14-15}\textbf{Fisica I, Sapienza, Corso di Laurea in Chimica Industriale (9 CFU)} \\
\emph{Mechanics and Thermodynamics} \\ [0.5em]

Results of student assessment of the Fisica 1 courses (``questionario Opis'') for AA15-16 and AA14-15 are attached to this document (\textsc{Allegati 2,3,4,5}).

\section*{Academic Responsibilities}
\vspace{-5pt}
\hrule
\vspace{10pt}
%\years{10/2014 - today}\textbf{"Referente di Con.Scienze per la Facolt\`a
%  di SMFN"} \\
\years{10/2014 - today}\textbf{Representative of ``Facolt\`a di SMFN''
  at "Conferenza Nazionale dei Presidenti e dei Direttori delle
  Strutture Universitarie di Scienze e Tecnologie (con.Scienze)"} \\
\emph{Organization of verification-of-knowledge tests for science majors required for
  registration at first year of university}\\[1em]
\years{09/2015 - today}\textbf{Member of "Commissione Didattica del
CdL in Chimica Industriale''} \\ \\[1em]
\years{2014 - today}\textbf{Member of "Commissione di Laurea in Fisica'' (8
  times)} \\[1em] 
\years{2015 - today}\textbf{``Controrelatore'' of master thesis for
  "Laurea Triennale in Fisica" (2 times)} 

\section*{Student Supervision}
\vspace{-5pt}
\hrule
\vspace{10pt}
%\years{PhD} \\ [1em]  
\subsection*{PhD (Dottorato)}
%I have been the PhD thesis co-supervisor of the following students at Sapienza:\\[1em]
\years{2015-2017} S. Gelli,\textit{``Search for new
  resonances in $V \gamma\rightarrow qq+\gamma$ final states at LHC''}, 
Thesis ongoing \\ [1em]  
\years{2014-2015} G. D'Imperio, \textit{``Search for narrow
  resonances in dijet final states at the LHC with $\sqrt{s}=13$~TeV''
} \cite{Khachatryan:2015dcf} [\href{http://www.roma1.infn.it/cms/tesiPHD/dimperio.pdf}{thesis}]%\\[1em]

\subsection*{Undergraduate (Laurea)}
\years{2016-2017}A. Tanga,\textit{``Ricerca di risonanze adroniche in
  stati finali con tre jet ad LHC''}, Thesis ongoing


\section*{Research Groups}
\vspace{-5pt}
\hrule
\vspace{10pt}%
\years{2005 - today}\textbf{Member of the CMS collaboration at the
  CERN Large Hadron Collider (LHC)} \\
\emph{CMS is one of the two general purpose particle physics detectors
operated at LHC} \\[1em] 
\years{2014 - today}\textbf{Member of the i-MCP collaboration} \\
\emph{i-MCP is an R\&D project within INFN CSN5 aimed at use of micro-channel plates for
precise timing measurement of single particles and electromagnetic showers
at collider experiments}

%\section*{Research field}
%\vspace{-5pt}
%\hrule
%\vspace{10pt}
%My research field is the experimental high energy physics. Since 2006
%I am part of the CMS collaboration, one of the 4 experiments running at
%the CERN Large Hadron Collider (LHC) in Geneva.

\section*{Abilitazione Scientifica Nazionale (ASN)}
\vspace{-5pt}
\hrule
\vspace{10pt}
\years{04/04/2017 - 04/04/2023} \textbf{Abilitazione per Professore di Seconda
Fascia}\\ 
\emph{Settore 02/A1 - Fisica Sperimentale delle Interazioni
  Fondamentali} \\[0.5em]
Quality indicators: numer of publications = 495 (51), numer of citations =
  19981 (1250), Hirsch H-index = 63 (18). The thresholds of the
  indicators for the ``settore 02/A1'' qualification are reported in parenthesis. 
  The ASN application document (\textsc{Allegato 6}) and the judgement
  of the committee (\textsc{Allegato 7}) are in attachment.

\section*{Publications}
\vspace{-5pt}
\hrule
\vspace{10pt}
\years{2017}\textbf{43 publications:} see
\href{http://inspirehep.net/search?ln=en&ln=en&p=find+a+santanastasio+and+tc+p+and+date+2017&of=hb&action_search=Search&sf=&so=d&rm=&rg=25&sc=0}{link
  to inspire}\\
{\tiny
  http://inspirehep.net/search?ln=en\&ln=en\&p=find+a+santanastasio+and+tc+p+and+date+2017\&of=hb\&action\_search=Search\&sf=\&so=d\&rm=\&rg=25\&sc=0}\\[1em]\normalsize
\years{2016}\textbf{142 publications:} see
\href{http://inspirehep.net/search?ln=en&ln=en&p=find+a+santanastasio+and+tc+p+and+date+2016&of=hb&action_search=Search&sf=&so=d&rm=&rg=25&sc=0}{link
  to inspire}\\
{\tiny
  http://inspirehep.net/search?ln=en\&ln=en\&p=find+a+santanastasio+and+tc+p+and+date+2016\&of=hb\&action\_search=Search\&sf=\&so=d\&rm=\&rg=25\&sc=0}\\[1em]\normalsize
\years{2015}\textbf{131 publications:} see
\href{http://inspirehep.net/search?ln=en&ln=en&p=find+a+santanastasio+and+tc+p+and+date+2015&of=hb&action_search=Search&sf=&so=d&rm=&rg=25&sc=0}{link
  to inspire}\\
{\tiny
  http://inspirehep.net/search?ln=en\&ln=en\&p=find+a+santanastasio+and+tc+p+and+date+2015\&of=hb\&action\_search=Search\&sf=\&so=d\&rm=\&rg=25\&sc=0}\\[1em]\normalsize
\years{2014}\textbf{109 publications:} see
\href{http://inspirehep.net/search?ln=en&ln=en&p=find+a+santanastasio+and+tc+p+and+date+2014&of=hb&action_search=Search&sf=&so=d&rm=&rg=25&sc=0}{link
  to inspire}\\
{\tiny
  http://inspirehep.net/search?ln=en\&ln=en\&p=find+a+santanastasio+and+tc+p+and+date+2014\&of=hb\&action\_search=Search\&sf=\&so=d\&rm=\&rg=25\&sc=0}\normalsize \\

Full citation report (all years since 2007):\\
Total number of publications: 570 (ISI), 591 (inspire) \\
Total number of citations: 19383 (ISI) \\
Average citations per publication: 34 (ISI) \\
h-index: 62 (ISI) 

\section*{Research Grants}
\vspace{-5pt}
\hrule
\vspace{10pt}
\years{03/2014}\textbf{Winner of ``Programma Per Giovani
  Ricercatori Rita Levi Montalcini''} \\ 
 \href{http://cervelli.cineca.it/ProgGiovRic/dm050813_683.pdf}{Risultati
   Bando 2010 del 08/2013} \\
%Research project: \emph{Search for new physics beyond the Standard Model with the CMS detector at the CERN LHC}
\emph{Three-year grant of about 220000 euros for research in experimental high-energy
  physics with the CMS detector at the CERN LHC, of which 44000 euros for research costs}


\section*{Scientific Coordination in the Period 2014-2017}
\vspace{-5pt}
\hrule
\vspace{10pt}
\years{09/2016 - today}\textbf{Coordination of the \emph{CMS Exotica
    Jets+X Working Group}} \\
I started my 2-year mandate on September 1st, 2016.
This analysis group works on searches for new physics beyond the
Standard Model in final states containing jets. The group, 
constituted by more than 50 physicists working in universities and
research institutions from all the world, performs
about 10 physics analyses in this final state. The results of these
searches are expected to be published in 2017. \\ [1em]
\years{09/2014 - 09/2016}\textbf{Coordination of the \emph{Dijet Resonance
  Team} of the CMS experiment}\\
This analysis team works on searches for new massive resonances at the TeV
scale decaying into a pair of jets using the dijet mass spectrum. It is
constituted by about 15 physicists from several institutions from
all the world. This group produced two high-impact papers using proton-proton
collisions at $\sqrt{s}=13$~TeV~\cite{Sirunyan:2016iap,Khachatryan:2015dcf} 
including the first published limits in the dijet final state on the
mass of a mediator of the interaction between a hypothetical {\it dark
  matter} particle and the Standard Model quarks.\\[1em]
%This group produced the first paper at LHC on a
%search for new physics using proton-proton collisions at $\sqrt{s}=13$~TeV~\cite{Khachatryan:2015dcf}. \\[1em]
\years{01/2013 - 01/2015}\textbf{Coordination of the \emph{CMS Exotica
    Leptons+Jets Working Group}}\\
This analysis group works on searches for new physics beyond the
Standard Model in final states containing leptons and jets. The group, 
constituted by more than 50 physicists working in universities and
research institutions from all the world, performed about 15 physics
analyses in this final state. During my convenership, the group
produced 3 publications~\cite{Khachatryan:2014ura,Khachatryan:2014dka,Khachatryan:2014gha}
and 7 preliminary results that were then published or submitted for
publication in 2015 (including~\cite{Khachatryan:2016yji,Khachatryan:2015ywa,Khachatryan:2015vaa}).

\section*{Details of Research Activity in the Period 2014-2017}
\vspace{-5pt}
\hrule
\vspace{10pt}
My research field is the experimental high energy physics. Since 2005
I am part of the CMS collaboration, one of the 4 experiments running at
the CERN Large Hadron Collider (LHC). During these years, I have been
involved in several detector activities including calibration of the
CMS electromagnetic calorimeter, commissioning of the CMS hadronic
calorimeter, performance studies of the missing transverse energy
reconstructed in the event, and I participated to several physics analyses.
In the 2014-2016 period I focused on searches for physics beyond the
Standard Model (SM) and on jet calibration studies. 
Since 2014 I'm also part of the i-MCP collaboration. This is
an R\&D project within INFN CSN5 aimed at use of micro-channel plates for
precise timing measurement of single particles and electromagnetic showers
at collider experiments. \\[1em]

{\bf Search for new resonances in the dijet final state}\\[0.5em]
I have been working on a search for new resonances with mass at the TeV
scale that decay to a pair of jets (dijet). This search is one of the most 
sensitive probes for new physics at LHC because any hypothetical new
particle that might be produced is originated from the
colliding protons and therefore it must couple to quarks and/or
gluons, thus producing jets. The analysis strategy consists in
reconstructing the invariant mass of the dijet system and searching for a
resonant peak in its data spectrum. I have been the leading author of this
search in CMS since 2012 publishing a paper on this topic with the
full dataset collected at $\sqrt{s}=8$~TeV~\cite{Khachatryan:2015sja}. After a 2-year
shutdown, the LHC has restarted proton-proton collisions in 2015 with an increased 
center-of-mass energy of 13 TeV. This energy jump has largely enhanced the
sensitivity of the dijet search to new physics. In the last two years, 
I coordinated the group
%I am coordinating the group, 
made of about 15 physicists devoted to the
analysis of the early 13 TeV data. The analysis performed on the data
collected in 2015 and 2016 has shown no sign of new particle
production, setting the most stringent limits in this final state 
for several new physics models. 
I was primary author of two high-impact papers using proton-proton
collisions at
$\sqrt{s}=13$~TeV~\cite{Sirunyan:2016iap,Khachatryan:2015dcf}.
These include the first published limits in the dijet final state on the
mass of a mediator of the interaction between a hypothetical 
{\it dark matter} particle and the SM quarks. \\[1em]
%The results of this search, thanks to their
%relevance for this scientific 
%field, have been accepted for publication by
%PRL~\cite{Khachatryan:2015dcf} in January 2016. 
%This is also the first search for new physics in proton-proton
%collisions at $\sqrt{s}=13$~TeV from LHC. 

%{\bf Search for hadronic resonances using the novel {\it data
%    scouting} technique}\\[0.5em]
{\bf The novel {\it data scouting} technique in hadronic resonance searches}\\[0.5em]
At LHC, it is important to extend the search for resonances in fully hadronic
final states in the mass region below 1 TeV in order to probe 
%hypothetical hadronic resonances with 
small couplings to quarks and gluons not yet excluded by previous colliders.
%that similar searches performed at previous
%colliders could not exclude yet. %find yet. 
The main experimental difficulties at LHC originate
from the large cross section of multijet events at low jet transverse
momentum
%dijet mass 
and the finite computing resources for processing and storing these
data. To solve this issue, I proposed in 2011 a new technique - known as 
{\it data scouting} - in the CMS experiment. This technique, by significantly reducing the event size
compared to the standard CMS data stream, enabled to relax the
trigger thresholds and record 1 KHz of fully hadronic events to extend
the search in the sub-TeV mass region. With this approach, the
analysis is performed using jets reconstructed online in the CMS
trigger computing farm. %This new method, initially developed for the dijet
%analysis, can have large impact on several new physics searches in
%fully hadronic final states.
This novel trigger strategy was fully integrated in the CMS physics program in
2012, when I was convener of the {\it``CMS Dataset Definition Team"}, 
allowing to collect data corresponding to almost 19 $\rm{fb}^{-1}$ of
integrated luminosity at $\sqrt{s}=8$~TeV; 
%in the sub-TeV dijet mass region; 
this technique was implemented also by LHCb and ATLAS
experiments at LHC in the following years.
Since 2014, I worked at a search for sub-TeV dijet resonances using these data.
No evidence for new particle production was found and the most
stringent limits to date were set on the production cross section of dijet resonances
in the mass range from 500 to 800 GeV. These results were published in
2016 by PRL~\cite{Khachatryan:2016ecr}. \\[1em]

{\bf Search for new resonances in diboson final states} \\[0.5em]
Many theories beyond the SM predict the existence of massive particles
at the TeV scale that decays into pairs of SM bosons. The bosons
coming from the massive particle can be W, Z, H or $\gamma$. 
As convener of the {\it CMS Exotica Leptons+Jets Working
Group} in 2013 and 2014, I contributed to various diboson
analyses in final states involving leptons and jets, documented in 
3 papers~\cite{Khachatryan:2016yji,Khachatryan:2015ywa,Khachatryan:2014gha}.
In particular, I was deeply involved in the search for massive resonances in the WW and ZZ 
final states using proton-proton collisions at $\sqrt{s}=8$~TeV, focusing
on the analysis of the semi-leptonic ($\ell\ell qq$ and $\ell\nu qq$)~\cite{Khachatryan:2014gha} 
and fully-hadronic ($qqqq$) decay modes~\cite{Khachatryan:2014hpa}.
These decay channels have the largest branching fraction, providing
a higher sensitivity to new physics for resonance masses above 1 TeV
compared to the fully leptonic channels.
These are quite complex analyses. For resonances with mass above 1 TeV, the momentum of the W and Z
bosons greatly exceeds their rest mass, and the quarks from their decay
are emitted with a small angular separation in the laboratory
reference frame, thus resulting into a single massive jet.  
By exploiting the substructure of a wide jet, it is possible to
discriminate between the dipolar structure of an hadronic boson decay (signal events)
and a jet created by the hadronization of a single quark/gluon (background events).
The WW and ZZ searches were among the first in CMS to include the
jet-substructure reconstruction algorithms in a physics analysis and
my work contributed to define the experimental
methodology for this kind of research in CMS. \\[1em]

{\bf Work on jets} \\[0.5em]
High momentum jets, coming from hadronization of energetic quarks and
gluons, are present in almost every collision at the LHC. The detailed
understanding of both the energy scale and resolution of the jets is of crucial importance for
many physics analyses. I worked at the jet energy
calibration using photon+jet events and I studied the energy scale 
and resolution of jets reconstructed at trigger level in the {\it data
scouting} stream. I'm also involved in studies of jet
substructure which are important for diboson analyses. Energetic W or Z
bosons decaying to a pair of collimated quarks can be reconstructed as
single massive jets in the detector. 
%Different algorithms have been developed to reconstruct the jet mass
%and identify the dipolar structure of energy deposits inside the
%jets. 
The effectiveness of algorithms to identify the jet substructure 
is studied using simulated events and it is sensitive to the details
of Monte Carlo parton showering. Since it is crucial to validate 
these methods using real data, I studied these algorithms using 
a pure sample of energetic W bosons coming from top quark 
decays in t$\bar{\rm{t}}$ semi-leptonic events. 
By comparing results in data and simulation, it is
possible to derive correction factors for the jet mass
scale/resolution and for the efficiency of the jet substructure 
requirements. These correction factors, measured with 5-10\%
uncertainty, are used in several searches for new physics in CMS. \\[1em]

{\bf Precise timing measurements at future colliders} \\[0.5em]
Future hadron colliders will provide instantaneous luminosities 
exceeding $10^{35}$ cm$^{-2}$s$^{-1}$ and the large number of 
simultaneous collisions in each interaction (pileup events) will be a
major challenge. While one collision contains the rare signatures of interest for discoveries 
or SM precision measurements, the contribution of the remaining
pileup interactions must be reduced to maintain good performance of
calorimeters in terms of energy measurements and particle
identification. For example, in the High-Luminosity phase of the LHC
(HL-LHC), more than 140 interactions per beam crossing, with 
a spread of approximately 10 cm and 300 ps
along the beam axis, are anticipated. This represent an increase of
more than a factor 5 with respect to the current LHC pileup
conditions. A possible strategy to reduce the impact of
increasing pileup, consists in complementing the transversal segmentation of
the calorimeters with an extremely high time-resolution, 
which would enable the energy deposits coming from different 
interaction vertices to be resolved in time. Micro Channel Plates
(MCPs) are good candidate detectors for this scope, thanks to their
excellent time response. I'm involved in R\&D studies on the
ionization-MCP (i-MCP) where the avalanche formation is triggered by secondary
emission of electrons directly on the MCP surface, when this is hit by
relativistic charged particles. The advantage consists in the elimination of the
photo-cathode, improving the radiation tolerance of the device.
The demonstration of the i-MCP concept has been achieved on both 
commercial device and bare MCPs tested inside a custom vacuum 
chamber designed in Rome. In some configurations, detection
efficiencies to single particles above 80\% are reached with time
resolution around 20 ps. 
Initial results have been published on NIM~\cite{Brianza:2015jia} and a second 
publication based on test beams performed in the last two years 
is expected in 2017.

%\clearpage

%{\bf Work on jets} \\ [1em]
\section*{Invited Talks at Conferences}
\vspace{-5pt}
\hrule
\vspace{10pt}
%ICHEP2014
\years{07/2016}\textbf{ICNFP2016} - International Conference on New
Frontiers in Physics, Kolymbari, Crete, \textit{``Searches for BSM
  physics in final states with jets and leptons+jets at CMS''},
Proceedings~\cite{Santanastasio:ICNFP2016} \\ [1em]
%ICHEP2014
\years{07/2014}\textbf{ICHEP2014} - International Conference on High
Energy Physics, Valencia, Spain, \textit{``Search for heavy resonances
  decaying to bosons with the ATLAS and CMS detectors''},
Proceedings~\cite{Santanastasio:ICHEP2014} 

\section*{Review Committees}
\vspace{-5pt}
\hrule
\vspace{10pt}
\noindent
\years{06/2016 - today}Referee of {\it New Journal of
  Physics}~\href{http://iopscience.iop.org/journal/1367-2630}{link to
  online journal}~(2015 impact factor = 3.570) \\  [2em] 


%\vspace{-5pt}
%\hrule
%\vspace{10pt}


%During my convenerships of analysis groups in the CMS
%experiment, I supervised the research activity of the following
%graduate students from foreign institutions:\\[1em]
%\years{2012-2015} {\bf Emine Gurpinar}, Cukurova University, Turkey\\ \textit{"Searches for heavy resonances decaying to pair of jets at CMS"}~\cite{Khachatryan:2015sja} \\ [1em]
%\years{2012-2014} {\bf Shuai Liu}, Peking University, China\\ \textit{"Searches for beyond Standard Model WW$\rightarrow \ell\nu qq$ resonances at CMS"}~\cite{Khachatryan:2014gha} \\ [1em]
%\years{2012-2014} {\bf Edmund Berry}, Princeton University, USA \\ \textit{"Searches for first-generation leptoquarks at CMS with $\sqrt{s}=7$ and  8 TeV data"}~\cite{Khachatryan:2015vaa} %\\ [1em]


%\section*{Citation report}
%\noindent

%\begin{tabular}{l l}
%Total number of publications: & ISI: 378, Inspire: 393 \\
%Total number of publications in the last 10 years: & ISI: 378, Inspire: 393 \\
%Total number of citations divided by N (6): & ISI: 8805/7$=$1257\\
%(N = number of years since the first publication)  & \\
%h factor: & ISI: 37 \\
%Normalized h factor: (each pub contributes & ISI: 40 \\
%as \#citations * 4 / (2013 - year of pub +1) ) &  \\
%\end{tabular}

%% Aug 2013 - Rientro Cervelli
%\begin{tabular}{l l}
%Total number of publications: & ISI: 267, Inspire: 270 \\
%Total number of publications in the last 10 years: & ISI: 267, Inspire: 270 \\
%Total number of citations divided by N (6): & ISI: 5311/6$=$885\\
%(N = number of years since the first publication)  & \\
%h factor: & ISI: 32 \\
%Normalized h factor: (each pub contributes & ISI: 40 \\
%as \#citations * 4 / (2013 - year of pub +1) ) &  \\
%\end{tabular}

%Exercises of classic mechanics for mathematics majors

%\section*{Physics Schools}
%\noindent
% FERMILAB 2008
%\years{12-22.08.2008}\textbf{2008 Joint CERN-Fermilab Hadron Collider Physics Summer School} \\ 
%Fermilab, Batavia, Illinois, US \\ [1em]
% LECCE 2005
%\years{09-14.06.2005}\textbf{Italo-Hellenic School of Physics 2005}  \\ 
%Martignano, Lecce, Italy \\
%{\it ``The Physics of LHC: theoretical tools and experimental challenges''}

%\section*{Languages}
%\begin{tabular}{l c l}
%\textit{Italian} (native speaker) & \makebox[4em]{} & \textit{English} (fluent)\\
%\textit{Italian} (native speaker) & \makebox[4em]{} & \textit{French} (fluent)\\
%\textit{English} (fluent) & &\textit{German} (basic)\\
%\end{tabular}

%\clearpage

%%%%%%%%%%%%%%%%%%%%%%%%%%%
%%% Service work
%%%%%%%%%%%%%%%%%%%%%%%%%%%

%\section*{Service work in Experiments and Collaborations}
%\subsection*{ATLAS Experiment}
%\noindent
%\textbf{Data Analysis: Supersymmetry Working Group} Working on data analysis, on exploring and implementing analysis strategies and on data files production\\
%\textbf{Development \& Upgrade} Working in the DAQ group, on the upgrade of the configuration DB system\\
%\textbf{Detector Operation} Shifter in the control room, at the Muon System, DAQ and Run Control desks\\
%\textbf{Software Framework} Taking part in code testing, and shifter for the build test system (RTT)\\
%\textbf{Documentation} Responsible person for a part of the documentation of the ATLAS data-format\\
%\textbf{Public Relations} Official ATLAS Guide, escorting VIP visits to the ATLAS cavern\\


%\clearpage

%%%%%%%%%%%%%%%%%%%%%%%%%%%
%%% Publications & Talks
%%%%%%%%%%%%%%%%%%%%%%%%%%%

%As of November/December 2016: ``highly cited papers'' received enough
%citations to place it in thetop 1\% of the academic field of Physics
%based on a highly cited threshold for the field and publication year.

\begin{thebibliography}{599}

\bibitem{Sirunyan:2016iap} 
  [CMS Collaboration],
  ``Search for dijet resonances in proton-proton collisions at sqrt(s) = 13 TeV and constraints on dark matter and other models,''
  Accepted for publication by Phys.\ Lett.\ B, arXiv:1611.03568
  [hep-ex]. \\
\emph{Citations: 22 (Inspire)}

\bibitem{Khachatryan:2015dcf} 
  [CMS Collaboration].
  ``Search for narrow resonances decaying to 
  dijets in proton proton
  collisions at $\sqrt{s}=13$ TeV,''
  Phys.\ Rev.\ Lett.\  116, 071801 (2016), arXiv:1512.01224 [hep-ex]. \\
\emph{Citations: 22 (ISI), 123
  (Inspire)}

\bibitem{Khachatryan:2016ecr} 
[CMS Collaboration],
  ``Search for narrow resonances in dijet final states at $\sqrt{s}=8$
  TeV with the novel CMS technique of data scouting,''
  Phys.\ Rev.\ Lett.\  117, 031802 (2016), arXiv:1604.08907 [hep-ex]. \\
\emph{Citations: 5 (ISI), 43 (Inspire)}

\bibitem{Khachatryan:2016yji} 
[CMS Collaboration],
  ``Search for massive WH resonances decaying into the $\ell \nu \mathrm{b} \overline{\mathrm{b}} $ final state at $\sqrt{s}=8$ TeV,''
  Eur.\ Phys.\ J.\ C 76, 237 (2016), arXiv:1601.06431 [hep-ex]. \\
\emph{Citations: 2 (ISI), 28 (Inspire)}

\bibitem{Khachatryan:2015ywa} 
 [CMS Collaboration],
  ``Search for narrow high-mass resonances in proton–proton collisions
  at $\sqrt{s}=8$ TeV decaying to a Z and a Higgs boson,''
  Phys.\ Lett.\ B 748, 255 (2015), arXiv:1502.04994 [hep-ex].\\
\emph{Citations: 20 (ISI), 40 (Inspire)}

\bibitem{Khachatryan:2015vaa} 
[CMS Collaboration],
  ``Search for pair production of first and second generation
  leptoquarks in proton-proton collisions at $\sqrt{s}=8$ TeV,''
  Phys.\ Rev.\ D  93, 032004 (2016), arXiv:1509.03744 [hep-ex].\\
\emph{Citations: 8 (ISI), 41 (Inspire)}

\bibitem{Khachatryan:2015sja} 
  [CMS Collaboration],
  ``Search for resonances and quantum black holes using dijet mass
  spectra in proton-proton collisions at $\sqrt{s}=8$ TeV,''
  Phys.\ Rev.\ D 91, no. 5, 052009 (2015), arXiv:1501.04198 [hep-ex].\\
\emph{Citations: 71 (ISI), 187 (Inspire)}

\bibitem{Khachatryan:2014ura} 
  [CMS Collaboration],
  ``Search for pair production of third-generation scalar leptoquarks
  and top squarks in proton–proton collisions at $\sqrt{s}=8$ TeV,''
  Phys.\ Lett.\ B 739, 229 (2014), arXiv:1408.0806 [hep-ex].\\
\emph{Citations: 44 (ISI), 68 (Inspire)}

\bibitem{Khachatryan:2014dka} 
  [CMS Collaboration],
  ``Search for heavy neutrinos and $\mathrm {W}$ bosons with
  right-handed couplings in proton-proton collisions at $\sqrt{s}=8$ TeV,''
  Eur.\ Phys.\ J.\ C 74, no. 11, 3149 (2014), arXiv:1407.3683 [hep-ex].\\
\emph{Citations: 65 (ISI), 188 (Inspire)}

\bibitem{Khachatryan:2014gha} 
  [CMS Collaboration],
  ``Search for massive resonances decaying into pairs of boosted
  bosons in semi-leptonic final states at $\sqrt{s}=8$ TeV,'''
  JHEP 1408, 174 (2014), arXiv:1405.3447 [hep-ex].\\
\emph{Citations: 43 (ISI), 189 (Inspire)}

\bibitem{Khachatryan:2014hpa}
 [CMS Collaboration],
  ``Search for massive resonances in dijet systems containing jets
  tagged as W or Z boson decays in pp collisions at $ \sqrt{s} $ = 8 TeV,''
  JHEP 1408, 173 (2014), arXiv:1405.1994 [hep-ex].\\
\emph{Citations: 49 (ISI), 205 (Inspire)}

\bibitem{Brianza:2015jia} 
[F.~Santanastasio {\it et al.}],
  ``Response of microchannel plates to single particles and to electromagnetic showers,''
Nucl. Instrum. Meth. A 797, 216 (2015), arXiv:1504.02728 [physics.ins-det].\\
\emph{Citations: 6 (ISI), 6 (Inspire)}

%------------------------------------------------------------------------------------------------
\vspace{0.1cm} \begin{center} \textsc{Conference Proceedings quoted in
    this document} \end{center} \vspace{0.05cm}
%------------------------------------------------------------------------------------------------

\bibitem{Santanastasio:ICNFP2016} 
  ``Searches for BSM physics in final states with jets and leptons+jets at CMS''
  \\{}F.~Santanastasio
   \\{} Proceedings will be published in European Physical Journal Web of Conferences
  \\{}{\it Prepared for the 5th International Conference on New
    Frontiers in Physics, Kolymbari, Crete, 6-14 July 2016} 

\bibitem{Santanastasio:ICHEP2014} 
``Search for heavy resonances decaying to bosons with the ATLAS and CMS detectors,'' 
 \\{}F.~Santanastasio 
 \\ Nucl.\ Part.\ Phys.\ Proc.\  273-275, 649 (2016)
  \\{}{\it Prepared for the XXXVII International Conference on High Energy Physics, Valencia, Spain, 2-9 July 2014}
%  ``Search for heavy resonances decaying to bosons with the ATLAS and CMS detectors''
%  \\{}F.~Santanastasio
%   \\{} Proceedings will be published in Nuclear Physics B - Proceedings Supplements

%%------------------------------------------------------------------------------------------------
%\vspace{0.1cm} \begin{center} \textsc{Preliminary results quoted in this document} \end{center} \vspace{0.05cm}
%%------------------------------------------------------------------------------------------------

%\bibitem{CMS:2015neg} 
% {}[CMS Collaboration],
%  ``Search for Resonances Decaying to Dijet Final States at $\sqrt{s} = 8$ TeV with Scouting Data,''
%  CMS-PAS-EXO-14-005 (2015). Paper in preparation.

%\bibitem{CMS:2015gla} 
%  CMS Collaboration [CMS Collaboration],
%  ``Search for massive WH resonances decaying to $\ell \nu {\rm b \bar{b}}$ final state in the boosted regime at $\sqrt{s}=8$\,TeV,''
%  CMS-PAS-EXO-14-010 (2015). Paper submitted in 2016 (arXiv:1601.06431 [hep-ex]).

%\bibitem{CMS:2014qpa} 
%  CMS Collaboration [CMS Collaboration],
%  ``Search for Pair-production of First Generation Scalar Leptoquarks in pp Collisions at sqrt s = 8 TeV,''
%  CMS-PAS-EXO-12-041 (2014). Paper submitted in 2015 (arXiv:1509.03744
%  [hep-ex]).

\end{thebibliography}

%\vspace{1cm}
\vfill{}
\hrulefill

% FILL IN THE FULL URL TO YOUR CV
\begin{center}
%{\footnotesize \href{http://www.ias.edu/spfeatures/einstein}{http://www.ias.edu/spfeatures/einstein} — Last updated: \today}
{\footnotesize Last updated: \today}
\end{center}


\end{document}
