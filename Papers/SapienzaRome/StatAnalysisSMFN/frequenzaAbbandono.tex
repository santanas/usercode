\documentclass[a4paper,11pt]{article}

\usepackage[latin1]{inputenc}
\usepackage[italian]{babel}

\title{Frequenza di abbandono di un corso di laurea triennale}
\author{Francesco Santanastasio, RTDb, Dipartimento di Fisica, Sapienza}

\begin{document}

\maketitle

\section{Definizione quantita'}

\begin{itemize}
\item Si assume un corso di laurea triennale
\item $N = $ anno in esame. \\
{\it Esempio: $N = 2015$}
\item $I_k(N) = $ numero di studenti immatricolati nell'anno $N$ al $k$-esimo anno di corso. \\
$k = $ anno di corso per il quale gli studenti si sono immatricolati.\\
{\it Esempio 1: ci sono solo 100 studenti immatricolati al primo anno di corso ($k=1$) in fisica nel 2013 $\rightarrow$ $I_1(2013) = 100 $.} \\
{\it Esempio 2: ci sono solo 5 studenti che si immatricolano direttamente al secondo anno ($k$=2) per il corso di fisica nel 2013 (sono per esempio studenti che hanno svolto gli esami del primo anno presso un'altra universita' e decidono di immatricolarsi successivamente alla Sapienza) $\rightarrow$ $I_2(2013) = 5 $}
\item $C_k(N) = $ numero totale di studenti (iscritti $+$ immatricolati) al $k$-esimo anno di corso.\\
$k = N - (\mbox{anno di immatricolazione}) + (\mbox{anno di corso al quale si sono immatricolati})$ \\
Se $k>3$ $\rightarrow$ $k \equiv F = \mbox{fuori corso}$ (sono iscritti da piu' di 3 anni al corso di studi in esame)\\
{\it Esempio: siamo nel 2015; un dato studente si e' immatricolato nel 2014 direttamente al secondo anno di corso (per esempio, proveniva da un'altra universita' o da un altro corso di studi)} \\ 
{\it $\rightarrow$ $k=2015 - 2014 + 2 = 3$, ovvero lo studente nel 2015 e' effettivamente al suo terzo anno di corso.}
\item $L(N) = $ numero di studenti laureati nell'anno $N$
\item $A_k(N) = $ numero di studenti che abbandonano il corso di studi nell'anno $N$, essendo iscritti al $k$-esimo anno di corso ({\bf questa e' la quantita' incognita da ricavare}).\\
{\it Nota: in questa formulazione, $A_3(N)$ comprende, oltre agli
  abbandoni al terzo anno di corso, anche gli abbandoni degli studenti fuori corso.}
\end{itemize} 

\section{Calcolo frequenza di abbandono}

Sono valide le seguenti relazioni che legano le quantita' definite nella sezione precedente 
relative agli anni $N$ ed $N-1$:
%
\begin{eqnarray}
C_1(N) & = & I_1(N) \\
C_2(N) & = & C_1(N-1) - A_1(N-1) + I_2(N) \\
C_3(N) & = & C_2(N-1) - A_2(N-1) + I_3(N) \\
C_F(N) & = & C_3(N-1) + C_F(N-1) - A_3(N-1) - L(N-1) ;
\end{eqnarray}
%
da cui si ricava il numero di studenti $A_k(N-1)$  che hanno abbandonato 
nell'anno $N-1$ essendo iscritti all'anno $k$-esimo di corso:
%
\begin{eqnarray}
A_1(N-1) & = & C_1(N-1) - C_2(N) + I_2(N) \\
A_2(N-1) & = & C_2(N-1) - C_3(N) + I_3(N) \\
A_3(N-1) & = & C_3(N-1) + C_F(N-1) - C_F(N) - L(N-1) 
\end{eqnarray}
%
Il calcolo di ripete in maniera ricorsiva per tutti gli anni $N$ in esame.
%

Per un dato anno $N$, possiamo dunque definire 
le frequenze di abbandono $R_k(N)$ relative al $k$-esimo anno di corso:
%
\begin{eqnarray}
R_1(N) & = & \frac{A_1(N)}{C_1(N)} \\ 
R_2(N) & = & \frac{A_2(N)}{C_2(N)} \\
R_3(N) & = & \frac{A_3(N)}{C_3(N)+C_F(N)-L(N)}
\end{eqnarray}
%
{\it Nota: in questa formulazione, $R_3(N)$ comprende, oltre agli
  abbandoni al terzo anno di corso, anche gli abbandoni degli studenti fuori corso.}

\end{document}

