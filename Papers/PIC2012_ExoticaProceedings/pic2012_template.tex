\documentclass{pic2012}
\def\runauthor{list of authors, or "first author et al"}
\def\shorttitle{short title less than 50 characters}
\begin{document}

\title{TEMPLATE FOR PIC 2012 PROCEEDINGS \\
PREPARED BY LATEX
}


\author{PIC~2012~AUTHOR}

\address{Institute of Experimental Physics SAS\\
Watsonova 47, 040 01 Ko\v{s}ice, Slovak Republic\\
E-mail: pic@saske.sk }

\maketitle

\abstracts{This is abstract for your PIC 2012 contribution. It should
be a few lines briefly describing your contribution to the
Proceedings. Please, remember that deadline for your contribution
is 1$^{\rm st}$ December 2012.
}


\section{Guidelines} Please, take from
the PIC 2012 web-site a Latex class file: 'pic2012.cls' for
preparation of your PIC 2012 contribution. This file is also part of archive, which was sent to every author, together with this template.

\section{Where to send your contribution}
 The proceedings will be first reviewed by the session convenors.
  Please, send a .pdf version to your session convenor before the
   deadline (for poster presenters with no talk, please send your
  proceeding to urban@upjs.sk). After having received the comments from
    the session convenor, the authors are required to submit their papers 
     and source files to the ePrint arXiv server.

All papers for the proceedings of this conference must be submitted to the eprint arXiv (http://arXiv.org). The web upload is particularly easy to use, from a PC, Mac or on Unix. You will receive confirmation email upon every step of the process, if you don't, it means you have still more to do. The ultimate goal is to see the full text of your paper on the arXiv.org server with an eprint number in the form "arch-ive/yymmnnn".

Please, include into Comment filed during submission into arXiv (Metadata definition step during sumbission procedure) the following:\\
{\tt Physics in Collision, Slovakia, 2012}\\
and in the Report field:\\
{\tt PSNUM XX}\\
where XX is the number of your talk on the Conference Web Scientific program page:\\
{\tt http://www.saske.sk/PIC2012/index.php?id=11}

The online conference proceeding will appear on SLAC eConf server.
In addition the regular paper book will be printed and sent to all participants.

Please, send your contribution arXiv number to our conference e-mail: \\
{\it pic@saske.sk} \\
subject: author name - session

\section{File Size}
The eprint arXiv (which will be used to submit contribution) will reject files over 650k. This problem generally occurs if the paper contains a lot of large figures. Usually large figures are the result of converting jpegs or scanned images to PS files. To avoid this problem, submit smaller graphics, with lower resolution. There is help on how to do this at the eprint arXiv Web site. In principle, you can provide a link to your Web page in the arXiv  Comments  section pointing users to better quality pictures. 

\section{Limits of PIC 2012 contributions}
The limit for the size of a PIC 2012 contribution: \\
4 pages for regular poster, \\
6 pages for oraly presented poster, \\
10 pages for talks 20-30 min's (as stated on conference Indico page)
14 pages for talks 40-45 min.

\section{Deadline for your contribution the the PIC 2012 Proceedings}
1$^{\rm st}$ December 2012

\section{Figures} 
Please prepare the figures in high resolution
(600-1200 dpi). Half-tone pictures must be sharp
enough for reproduction, otherwise they will be rejected.

\section{Text sample}
QCD is an essential ingredient of the Standard Model, and it is
well tested in  hard processes when transferred momentum is of the
order of the total collision energy (Bjorken limit: $Q^2 \sim s
\rightarrow \infty$). The cornerstones of perturbative QCD at this
kinematic regime (QCD-improved parton model):
 the Gribov--Lipatov--Altarelli--Parisi--Dokshitzer
(GLAPD) evolution equation and factorization of inclusive hard
processes provides a basis for the successful QCD-improved parton
model. The factorization theorem for inclusive hard processes
ensures that the inclusive cross section factorizes into partonic
subprocess(es) and parton distribution function(s). The GLAPD
evolution equation governs the $\log Q^2$-dependence (at $Q^2
\rightarrow \infty$) of the parton distribution functions and the
hard subprocess cross-sections at fixed scaling variable
$x=Q^2/s$.

Another kinematic domain that is very important at high-energy is
given by the (Balitsky--Fadin--Kuraev--Lipatov) BFKL limit
\cite{FKL,BL78}, or QCD Regge limit, whereby at fixed $Q^2 \gg
\Lambda_{QCD}^2$, $s \rightarrow \infty $. In the BFKL limit, the
BFKL evolution in the leading order (LO) governs $\log(1/x)$
evolution (at $x \rightarrow 0$) of inclusive processes. Note that
the BFKL evolution in the next-to-leading order (NLO)
\cite{FL,CC98,BFKLP}, unlike the LO BFKL \cite{FKL,BL78}, partly
includes GLAPD evolution with the running coupling constant of the
LO GLAPD, $\alpha_S(Q^2) = 4 \pi / \beta_0
\log(Q^2/\Lambda_{QCD}^2) $.

Photon--photon collisions, particularly $\gamma^* \gamma^*$
processes, play a special role in QCD~\cite{Budnev75}, since their
analysis is under much better control than the calculation of
lepton--hadron and hadron--hadron processes, which require the
input of non-perturbative hadronic structure functions or wave
functions.  In addition, unitarization (screening) corrections due
to multiple Pomeron exchange should be less important for the
scattering of $\gamma^*$ of high virtuality than for hadronic
collisions.

\begin{figure}[!thb]
\vspace*{7.0cm}
\begin{center}
\special{psfile=pic2012_template_fig.ps voffset=-60 vscale=40
hscale= 40 hoffset=10 angle=0}
%\centerline{\epsfxsize=2.9in\epsfbox{kim_mephi_lep.ps}}
\caption[*]{ The energy dependence of the total cross section for
highly virtual photon--photon collisions predicted by the BLM
scale-fixed NLO BFKL \cite{BFKLP2} compared with recently
finalized OPAL \cite{OPAL} and L3 \cite{L3} data from LEP2 at
CERN. The (solid) dashed curves correspond to the (N)LO BFKL
predictions for two different choices of the Regge scale: $s_0=
Q^2$  for upper curves and $s_0=4 Q^2$ for lower curves}
\end{center}
\end{figure}

The high-energy asymptotic behaviour of the $\gamma \gamma$ total
cross section in QED can be calculated~\cite{GLF} by an all-orders
resummation of the leading terms:\\
$\sigma \sim \alpha^4 s^{\omega}$, $\omega = \frac{11}{32} \pi
\alpha^2 \simeq 6 \times 10^{-5}$. However, the slowly rising
asymptotic behaviour of the QED cross section is not apparent
since large contributions come from other sources, such as the cut
of the fermion-box contribution: $\sigma \sim \alpha^2 (\log s)/s$
\cite{Budnev75} (which although subleading in energy dependence,
dominates the rising contributions by powers of the QED coupling
constant) and QCD-driven processes.


The photon--photon cross sections with LO BFKL resummation was
considered in Refs.~\cite{BL78,Bartels96,Brodsky97}.

Although the complete NLO impact factor of the virtual photon is
not known yet,  one can use the LO impact factor of Refs.~\cite{GLF,Brodsky97}, 
assuming that the main energy-dependent NLO
corrections come from the NLO BFKL subprocess rather than from the
photon impact factors \cite{BFKLP2}.


Fig.~1 compares the LO and BLM scale-fixed NLO BFKL predictions
$\sigma \sim \alpha^2 \alpha_S^2 s^{\omega}$~\cite{BFKLP,BFKLP2}
with recent CERN LEP2 data from OPAL \cite{OPAL} and L3 \cite{L3}.


\section*{Acknowledgements} We thank all DIS'03 participants
for their contributions to the DIS'03 Proceedings. This work was
supported in part by the High Energy Foundation and the World
Science Agency.

\section*{Appendix}
This is place for Appendix, if any.


\begin{thebibliography}{0}

\bibitem{FKL}   V.S.~Fadin, E.A.~Kuraev, and L.N.~Lipatov,
Phys. Lett. B {\bf60}  (1975) 50; \\
E.A.~Kuraev, L.N.~Lipatov, and V.S.~Fadin, Zh. Eksp. Teor. Fiz.
{\bf 71} (1976) 840 [Sov. Phys. JETP {\bf 44} 443 (1976)]; {\it
ibid.} {\bf 72} (1977) 377 [{\bf 45} (1977) 199].

\bibitem{BL78}
I.I.~Balitsky and L.N.~Lipatov, Yad. Fiz. {\bf28} (1978) 1597
[Sov. J. Nucl. Phys. {\bf28} (1978) 822].

\bibitem{FL}
V.S.~Fadin and L.N.~Lipatov, Phys. Lett. {\bf B429} (1998) 127.

\bibitem{CC98}  G.~Camici and M.~Ciafaloni, Phys. Lett.
{\bf B430} (1998) 349.

\bibitem{BFKLP}  S.J.~Brodsky, V.S.~Fadin, V.T.~Kim,
L.N.~Lipatov, and G.B.~Pivovarov, Pis'ma ZhETF {\bf70} (1999) 161;
[JETP Lett.\ {\bf 70} (1999) 155].


\bibitem{Budnev75}
V.M.~Budnev, I.F.~Ginzburg, G.V.~Meledin, and V.G.~Serbo,
Phys.~Rep. {\bf C15} (1975) 181.

\bibitem{GLF} V.N.~Gribov, L.N.~Lipatov, and G.V.~Frolov,
Phys.~Lett. {\bf B31} (1970) 34; Yad. Fiz. {\bf12} (1970) 994
[Sov. J. Nucl. Phys. {\bf 12} (1971) 543]; \\
H. Cheng and T.T. Wu, Phys. Rev. D {\bf1} (1970) 2775.

\bibitem{Bartels96}
J.~Bartels, A.~De~Roeck, and H.~Lotter,
Phys. Lett. {\bf B389} (1996) 742;\\
M.~Boonekamp, A.~De~Roeck, C.~Royon, and S.~Wallon, Nucl. Phys.
{\bf B555} (1999) 540.

\bibitem{Brodsky97}  S.J.~Brodsky, F.~Hautmann, and D.E.~Soper,
Phys.~Rev. {\bf D56} (1997) 6957; Phys.~Rev.~Lett. {\bf78} (1997)
803; (E) {\bf79} (1997) 3544.

\bibitem{BFKLP2}
S.J.~Brodsky, V.S.~Fadin, V.T.~Kim, L.N.~Lipatov and
G.B.~Pivovarov, Pisma ZhETF  {\bf76} (2002) 306 [JETP Lett. {\bf
76} (2002) 249].

\bibitem{OPAL}
OPAL, G. Abbiendi {\it et al.},  Eur. Phys. J. {\bf C24} (2002)
17.

\bibitem{L3} L3, P.~Achard {\it et al.},
Phys. Lett. {\bf B531} (2002) 39.

\bibitem{HERA98}
ZEUS, J. Breitweg {\it et al.},
Eur. Phys. J. {\bf C6} (1999) 239; \\
H1, C. Adloff {\it et al.}, Nucl. Phys. {\bf B538} (1999) 3.

\bibitem{D099}
D$\emptyset$, B.~Abbott {\it et al.}, Phys. Rev. Lett. {\bf 84}
(2000) 5722.



\end{thebibliography}

\end{document}
