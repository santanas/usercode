\section{Introduction}
Protons that collide at the LHC have equal and
opposite momenta. Therefore, the total vector momentum sum before and
after the collision in an event
should be zero. The hard collision happens between the partons of the
protons in the collision, and they can carry any fraction of the parent
proton. However, since the partons usually have very
little momentum in the plane transverse to the beam, the transverse
momentum can be considered as a conserved quantity to a good
approximation. 

Any transverse momentum imbalance in the detector may indicate that a
weakly interacting particle (\textit{e.g.} a neutrino) left the
detector without interacting with its material. However, instrumental or
beam related noises can cause an apparent energy imbalance in the
transverse plane. A study of the collision data collected by the CMS
experiment during the LHC commissioning phase in December 2009 has
revealed several instrumental effects that can cause an apparent energy
imbalance in the reconstruction. Since $\etmiss$ provides a powerful
tool in distinguishing rare processes with neutrinos or non-Standard
Model particles from the copious SM backgrounds, it is important to have an accurate and reliable method
of measuring this quantity. 

The $x$ and $y$ components of the raw missing transverse energy of the
event are obtained from:

\begin{align}
  \label{eq:32}
  \exmiss&=-\sum\limits^{N_{towers}}_{i=1}E_T^icos\phi_i\\
  \label{eq:33}
  \eymiss&=-\sum\limits^{N_{towers}}_{i=1}E_T^isin\phi_i
\end{align}
where the sum is taken over all calorimetric towers that are above a threshold of
$E_T^i>0.3$\,GeV, and the total electromagnetic and hadronic energy in $i^{th}$
tower is $E_T^i$. The magnitude of the missing energy is then calculated
by:
\begin{equation}
  \etmiss=\sqrt{\big\slash\hspace{-1.6ex}{E_x}^2+\big\slash\hspace{-1.6ex}{E_y}^2}
\end{equation}

The azimuthal direction of the $\etmiss$ is then given by:
\begin{equation}
  \phi_{\etmiss}=tan^{-1}\left(\frac{\big\slash\hspace{-1.6ex}{E_y}}{\big\slash\hspace{-1.6ex}{E_x}}\right)
\end{equation}

In the following we will first describe the event selection we used to
identify collision events and the data samples we used. We then describe
the features of the instrumental sources of fake $\etmiss$, and the
algorithms we used to identify them. A summary of the current status of
the beam halo identification algorithms is presented in
Sec.\ref{sc:BeamHalo}. The performance of $\etmiss$ reconstruction is
presented in the following sections, and comparisons to Monte Carlo
simulation are shown. 

