\section{Introduction}
Protons that collide at the LHC have equal and
opposite momenta. Therefore, the total vector momentum sum before and
after the collision in an event
should be zero. The hard collision happens between the partons of the
protons in the collision, and they can carry any fraction of the parent
proton. However, since the partons usually have very
little momentum in the plane transverse to the beam, the transverse
momentum can be considered as a conserved quantity to a good
approximation. 

Any transverse momentum imbalance in the detector may indicate that a
weakly interacting particle (\textit{e.g.} a neutrino) left the
detector without interacting with its material. However, instrumental or
beam related noises can cause an apparent energy imbalance in the
transverse plane. A study of the collision data collected by the CMS
experiment during the LHC commissioning phase in December 2009 has
revealed several instrumental effects that can cause an apparent energy
imbalance in the $\etmiss$ reconstruction. Since $\etmiss$ provides a powerful
tool in distinguishing rare processes with neutrinos or non-Standard
Model particles from the copious SM backgrounds, it is important to have an accurate and reliable method
of measuring this quantity~\cite{CMS:AN_2007_041},~\cite{CMS:AN_2008_089}. 

The $x$ and $y$ components of the raw Missing Transverse Energy of the
event are obtained from:

\begin{align}
  \label{eq:32}
  \exmiss&=-\sum^{N_\text{towers}}_{i=1}E_\text{T}^i\cos\phi_i\\
  \label{eq:33}
  \eymiss&=-\sum^{N_\text{towers}}_{i=1}E_\text{T}^i\sin\phi_i
\end{align}
where the sum is taken over all calorimeter towers with $E_\text{T}^i>0.3$~GeV, 
where $E_\text{T}^i$ is the total electromagnetic and hadronic transverse energy 
in the $i^{th}$ tower. The magnitude of the missing energy is then calculated
by:
\begin{equation}
  \etmiss=\sqrt{\exmiss^2+\eymiss^2}
\end{equation}

The azimuthal direction of the $\etmiss$ is then given by:
\begin{equation}
  \phi_{\etmiss}=\tan^{-1}\left(\frac{\eymiss}{\exmiss}\right)
\end{equation}

The Scalar Transverse Energy ($\sumet$), i.e. the total visible 
$E_\text{T}$ of the event in the calorimeter system, 
is defined as the scalar sum 

\begin{align}
  \label{eq:34}
  \sumet = \sum^{N_\text{towers}}_{i=1}E_\text{T}^i
\end{align}

where the sum is taken over all calorimeter towers with $E_\text{T}^i>0.3$~GeV.

The goal of this note is to study the performance of uncorrected 
calorimeter $\etmiss$ reconstruction with first collision data. 
We describe the features of the instrumental 
and beam related sources of fake $\etmiss$, and we present the algorithms we used to identify them.
We also show a comparison of MinBias and ZeroBias data with Monte Carlo simulation, for several 
$\etmiss$-related distributions.

The note is organized as follows.
The trigger selection and the definition of the data samples used are described,
respectively, in Section~\ref{sc:TriggerRequirements} and \ref{sc:DataSamples}. 
The event selection, based on reconstructed quantities, to identify collision candidate 
events and the description of the cleaning of anomalous noise observed in calorimeters 
is discussed in Section~\ref{sec:event_selection}.
A summary of the current status of the beam halo identification algorithms 
is presented in Section~\ref{sc:BeamHalo}.
The stability of $\etmiss$-related reconstructed quantities over 2009 
data-taking period is shown in Section~\ref{sc:METStab}.
A study on the calorimeter noise simulation and the comparison 
with ZeroBias data is presented in Section~\ref{sc:CaloNoise}.
Finally a comparison of several $\etmiss$-related distributions
between collision data and MinBias Monte Carlo simulation 
is shown for $\sqrt{s}=900$~GeV and $2360$~GeV, respectively, in 
Section~\ref{sc:DataVsMCMB900} and~\ref{sc:DataVsMCMB2360}.
