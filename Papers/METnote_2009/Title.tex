%==============================================================================
% title page for few authors

\begin{titlepage}

  % select one of the following and type in the proper number:

%  \cmsnote{2010/001}
%  \internalnote{2005/000}
%  \conferencereport{2005/000}
  \cmsan{2010/001}
%  \cmsdtnote{2005/000}
%  \cmspas{2005/000}
   \date{19 January 2010}

  \title{Commissioning of Uncorrected Calorimeter Missing Tranverse Energy in Zero Bias and Mininum Bias Events at $\sqrt{s}=900$~GeV and $\sqrt{s}=2360$~GeV}

  \begin{Authlist}
    A.~Author, B.~Author, C.~Author\Aref{a}
       \Instfoot{cern}{CERN, Geneva, Switzerland}
    D.~Author, E.~Author\Aref{b}, F.~Author
       \Instfoot{ieph}{Institute of Experimental Physics, Hepcity, Wonderland}
  \end{Authlist}

% if needed, use the following:
%\collaboration{Flying Saucers Investigation Group}
%\collaboration{CMS collaboration}

  \Anotfoot{a}{On leave from prison}
  \Anotfoot{b}{Now at the Moon}

  \begin{abstract}
    
    We present a study of the performance of raw calorimetric $\etmiss$ reconstruction in
    proton-proton collision events collected by the CMS experiment during the LHC
    commissioning phase in December 2009. Data samples collected at
    $\sqrt{s}=900$ and $2360$~GeV are analyzed in this note. We present
     event selection criteria to identify the collision events and remove the instrumental or beam
    related noise sources that were identified in the collision
    runs. We perform a comparison of $\etmiss$ and related distributions with
    simulation, after the $\etmiss$ in the events is corrected for noise
    effects. Additionally, $\etmiss$ resolution in data is
    compared with simulation. In general, a good agreement between data and simulation is
    observed. We observe disagreement in noise simulation in
    calorimeter, which is currently under investigation by the
    experts. We also present a study of the beam halo identification
    algorithms and their performance in the collision runs.

  \end{abstract} 

% if needed, use the following:
%\conference{Presented at {\it Physics Rumours}, Coconut Island, April 1, 2005}
%\submitted{Submitted to {\it Physics Rumours}}
%\note{Preliminary version}
  
\end{titlepage}

\setcounter{page}{2}%JPP

%==============================================================================
% title page for many authors
%
%\begin{titlepage}
%  \internalnote{2005/000}
%  \title{CMS Technical Note Template}
%
%  \begin{Authlist}
%    A.~Author\Iref{cern}, B.~Author\Iref{cern}, C.~Author\IAref{cern}{a},
%    D.~Author\IIref{cern}{ieph}, E.~Author\IIAref{cern}{ieph}{b},
%    F.~Author\Iref{ieph}
%  \end{Authlist}
%
%  \Instfoot{cern}{CERN, Geneva, Switzerland}
%  \Instfoot{ieph}{Institute of Experimental Physics, Hepcity, Wonderland}
%  \Anotfoot{a}{On leave from prison}
%  \Anotfoot{b}{Now at the Moon}
%
%  \begin{abstract}
%    This is a template of a CMS paper, written in LaTeX,
%    processed with {\it cmspaper.sty} style.
%    It is based on the {\it cernart.sty} and {\it articlet.sty} styles.
%    There are two versions of the title page.
%    The current one is designed for many authors.
%    The one on the previous page is for few authors.
%    Just delete the one which you do not need.
%  \end{abstract} 
%  
%\end{titlepage}
%
%==============================================================================
