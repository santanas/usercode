\documentclass[a4paper,11pt]{article}
%\usepackage{lineno}
\usepackage{amsfonts,amsmath,amssymb}
\usepackage[dvips]{graphicx}
\usepackage{bm}
\usepackage{multirow}
\usepackage{subfigure}  % use for side-by-side figures
%\usepackage[a4paper]{hyperref}

\def\etmiss{\big\slash\hspace{-1.6ex}{E_\text{T}}}
\def\etmissB{\big\slash\hspace{-1.6ex}{\boldsymbol{E}_\text{T}}}
\def\exmiss{\big\slash\hspace{-1.6ex}{E_{x}}}
\def\eymiss{\big\slash\hspace{-1.6ex}{E_{y}}}
\def\exymiss{\big\slash\hspace{-1.6ex}{E_{x,y}}}
\def\sumet{\sum{E_\text{T}}}
\def\sumetB{\sum{\boldsymbol{E}_\text{T}}}

\begin{document}
%\begin{linenumbers}

\section{Francesco Santanastasio - Activities in 2009}

Francesco Santanastasio is a post-doc for the University of Maryland (UMD), 
based at CERN since the beginning of his contract (December 2007).
During 2009 and early 2010, he has been involved in the 
following principal activities within the CMS experiment
(listed below in approximate chronological order from the start of the activity itself):
%
\begin{enumerate}
\item Physics Analysis - search for physics beyond SM, i.e. Leptoquarks,
\item HCAL Commissioning -  coordination of HCAL Prompt Feedback Group,
\item JetMET Analysis - $\etmiss$ commissioning with first collisions,
\item HCAL DPG Studies - studies in HF Noise Task Force.
\end{enumerate}
%
The contribution 2) is described in Section~\ref{sec:PFG},
including the future plans for 2010. The remaining activities listed above 
will instead be discussed elsewhere in the general UMD group report.
%
\subsection{Coordination of HCAL Prompt Feedback Group} \label{sec:PFG}
%
The HCAL Prompt Feedback Group (PFG) is a task-based effort aimed to investigate 
within short time scale (hours/days) anomalous problems observed in the 
detector during data taking. By its definition, this group is in closely related to 
HCAL Operation activities. In 2009, the PFG, consisting of the order of 10
graduate/undergraduate students and post-docs resident both at CERN and FNAL, 
gave an relevant contribution to the commissioning of the HCAL detector, 
by providing fast feedback analysis on the following categories of issues: 
investigation on firmware and data-format problems, studies on trigger related anomalies, 
support to data quality monitoring, run certification and shift instructions. 
Since the beginning of 2009 the coordination of the HCAL PFG was assigned 
to Francesco Santanastasio (UMD) and Sinjini Sengupta 
(University of Minnesota). 

Recently, a restructuring of several HCAL analysis groups occurred, 
aimed to improve and optimize the performance based on the 2009 experience. 
As a result of this reorganization, the HCAL PFG has been divided in 
two separate efforts: 1) a PFG for Operation support (PFG-Ops), with similar 
duties as in 2009, and 2) a PFG for longer term investigations and 
detector performance issues.

Dinko Ferencek and Francesco Santanastasio (both from UMD group) 
will coordinate the PFG-Ops group during the 2010. 
The PFG-Ops working team will be mainly constituted by the HCAL \textit{operation experts} resident at CERN, 
who provide both on-call support for HCAL shifters and prompt analysis. 
The PFG-Ops group will contribute to the smooth operation of the HCAL detector during 2010, 
by promptly addressing issues that may arise during data taking.
%
%\end{linenumbers}
\end{document}
