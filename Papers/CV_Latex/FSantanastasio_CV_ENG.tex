%------------------------------------
% Dario Taraborelli
% Typesetting your academic CV in LaTeX
%
% URL: http://nitens.org/taraborelli/cvtex
% DISCLAIMER: This template is provided for free and without any guarantee 
% that it will correctly compile on your system if you have a non-standard  
% configuration.
%------------------------------------ 


% ! TEX TS-program = XeLaTeX -xdv2pdf
% ! TEX encoding = UTF-8 Unicode

\documentclass[10pt, a4paper]{article}
\usepackage{fontspec} 
\usepackage{xunicode} 
\usepackage{xltxtra}
% per le lettere accentate italiane sul Mac! :-)
%\usepackage[applemac]{inputenc} %with VIM
\usepackage[latin1]{inputenc} % with TeXShop


% DOCUMENT LAYOUT
\usepackage{geometry}
\geometry{a4paper, textwidth=5.5in, textheight=8.5in, marginparsep=7pt, marginparwidth=.6in}
\setlength\parindent{0in}

% ADDITIONAL SYMBOLS
%\usesymbols[mvs]

% FONTS
\defaultfontfeatures{Mapping=tex-text} % converts LaTeX specials (``quotes'' --- dashes etc.) to unicode
%\setromanfont [Ligatures={Common}, BoldFont={Fontin Bold}, ItalicFont={Fontin Italic}]{Fontin}
\setromanfont [Ligatures={Common}, BoldFont={Linux Libertine Bold}, ItalicFont={Linux Libertine Italic}]{Linux Libertine}
%\setsansfont [Ligatures={Common}, BoldFont={Fontin Sans Bold}, ItalicFont={Fontin Sans Italic}]{Fontin Sans}
\setmonofont[Scale=0.8]{Monaco} 
% ---- CUSTOM AMPERSAND
\newcommand{\amper}{{\fontspec[Scale=.95]{Linux Libertine Bold}\selectfont\itshape\&}}
% ---- MARGIN YEARS
%\newcommand{\years}[1]{\marginpar{\scriptsize #1}}
\newcommand{\years}[1]{\marginpar{\footnotesize #1}}

% HEADINGS
\usepackage{sectsty} 
\usepackage[normalem]{ulem} 
\sectionfont{\rmfamily\mdseries\upshape\Large}
\subsectionfont{\rmfamily\bfseries\upshape\normalsize} 
\subsubsectionfont{\rmfamily\mdseries\upshape\normalsize} 
%modifying section numbering
\def\thesubsection{\arabic{subsection}.\ } 

% PDF SETUP
% ---- FILL IN HERE THE DOC TITLE AND AUTHOR
\usepackage[dvipdfm, bookmarks, colorlinks, breaklinks, pdftitle={Francesco Santanastasio - Curriculum Vitae},pdfauthor={Francesco Santanastasio}]{hyperref}
%\hypersetup{linkcolor=blue,citecolor=blue,filecolor=black,urlcolor=blue} 
\hypersetup{linkcolor=cyan,citecolor=blue,filecolor=black,urlcolor=cyan} 

% Title of Bibliography
\renewcommand\refname{References \\ \normalsize \begin{center} \quad \quad \textsc{Publications (relative to research activities)}\end{center} }

% DOCUMENT
\begin{document}
\reversemarginpar
{\LARGE Francesco Santanastasio}\\[1cm]
%Institute address
%begin{tabular}{ l c l }
%\emph{Institute Address}: & & \\
%University of Maryland & & \\
%Department of Physics - John S. Toll Physics Building & &\\
%College Park  & & \\
%MD  \texttt{20742-4111} & \makebox[1.2cm]{} & Tel: \texttt{+1 301 405 3401} \\
%United States of America & & Fax: \texttt{+1 301 314 9525} \\
%\end{tabular}\\[1em]

% Work address
\begin{tabular}{ l c l }
\emph{Work Address}: & & \\
CERN (Conseil Europeen pour la Recherche Nucleaire) & \makebox[1.cm]{} & \\
\texttt{CH-1211} Geneve  \texttt{23} & & Tel.: \texttt{+41 22 76 71 689}\\
Building \texttt{40}, Room 1-\texttt{B01} & & Cel: \texttt{+41 76 22 86 127}\\ 
Switzerland &  & email: \href{mailto:francesco.santanastasio@cern.ch}{francesco.santanastasio@cern.ch} 
\end{tabular}\\[1em]
%\vfill
Born:  9 February 1980 --- Roma, Italy\\
Nationality:  Italian
%\textsc{url}: \href{http://www.ias.edu/spfeatures/einstein/}{http://www.ias.edu/spfeatures/einstein/}\\ 

%%\hrule
\section*{Current Position}
\emph{CERN Research Fellow in Experimental Particle Physics} \\
PH Department, CERN, Geneve, Switzerland
%\emph{Post-Doctoral Research Assistant (Post-Doc) in Particle Physics} \\
%Department of Physics, University of Maryland, College Park, US

%%\hrule
\section*{Areas of specialization}
Particle Physics, Data Analysis in High Energy Physics, Physics beyond the Standard Model of Fundamental Interactions, Electromagnetic and Hadronic Calorimetry 
%\section*{Areas of competence}
%Software Development, IT, Particle detector physics
 
%\hrule
\section*{Career}
\noindent
%CERN Fellow
\years{Sept 2011 - today}\textbf{CERN Research Fellow in Experimental Particle Physics} \\
\textsc{Supervisor:} Dott. Maurizio Pierini (CERN) \\
\textit{CERN}, Geneve, Switzerland\\[1em]
%Post-Doc
\years{Dec 2007 - Aug 2011}\textbf{Post-Doctoral Research Assistant  (Post-Doc) in Particle Physics} \\
\textsc{Supervisor:} Prof. Sarah Eno (UMD) \\
\textit{University of Maryland}, College Park, MD, US\\
Based at \textit{CERN}, Geneve, Switzerland\\[1em]
% PhD
\years{Nov 2004 - Jan 2008}\textbf{PhD in Physics}\\ %{\small (highest honors)}\\
\textit{``Search for Supersymmetry with Gauge-Mediated Breaking using high energy photons at CMS experiment''} \cite{Santanastasio:DOTTORATO}\\
\textsc{Advisors:} Prof. Egidio Longo, Prof. Shahram Rahatlou, Dott. Daniele del Re (Sapienza) \\
\textit{Sapienza Universit\`a di Roma}, Roma, Italy\\[1em]
% Laurea
\years{Sept 1998 - May 2004}\textbf{\textit{Laurea} in Physics} {\small (highest honors)}\\
\textit{``Calibration of an electromagnetic calorimeter using the energy flow method''} \cite{Santanastasio:LAUREA}\\
\textsc{Advisors:} Prof. Egidio Longo (Sapienza), Dott. Riccardo Paramatti (INFN) \\
Mark: 110/110 \textit{``magna cum laude''}\\
\textit{Sapienza Universit\`a di Roma}, Roma, Italy
%EXAMPLE IN ENGLISH
%\years{2003-2006}\textbf{MSc (\textit{Laurea Magistrale}) in Nuclear and Subnuclear Physics} {\small (highest honours)}\\
%\textit{``Study of the ATLAS MDT Muon Chambers calibration constants with data from a testbeam''}\\
%\textsc{Advisors:} Prof. Toni Baroncelli (INFN), Prof. Filippo Ceradini (Roma Tre)\\
%Mark: 110/110 \textit{``magna cum laude''}\\
%\textit{\small expected date: August 2010}\\[1em]

\clearpage

\section*{Highlights of Research Activities}
% WHEN YOU ADD A NEW BULLET REMEMBER TO MOVE THE CLEARPAGE 
% AT THE END OF THE PAGE 
\noindent
% EXOTICA
\years{Dec 2007 - today}Actively involved in the research activities of the exotic physics group (Exotica) of the CMS experiment, looking for evidence of new physics beyond the Standard Model of fundamental interactions [see ``Talks at Conferences'']. \\ [1em]
% LEPTOQUARKS 
\years{Dec 2007 - today} Search for pair production of first generation scalar Leptoquarks ($LQ$) in the decay channels \\ $LQ \overline{LQ} \rightarrow ee qq$~\cite{Khachatryan:2010mp,EXO-10-005,EXO-08-010,AN-2010-230,AN-2008-070} and $LQ\overline{LQ} \rightarrow e\nu qq$ \cite{Chatrchyan:2011ar,AN-2010-361} with the CMS detector using the first 36 pb$^{-1}$ of LHC collisions collected in 2010. Supervising PhD student from Princeton University for the update of both analyses with 4.7 fb$^{-1}$ of data collected in 2011~\cite{AN-11-492}, extending the exclusion on the $LQ$ mass from about 400 GeV to 800 GeV. \\ [1em] 
% DIJET SEARCHES
\years{Sept 2011 - today} Search for narrow resonances decaying into a pair of jets using the 
dijet mass spectrum~\cite{AN-12-012} using the CMS detector. 
The analysis contains improvements compared to 
a previous published CMS dijet search~\cite{Chatrchyan:2011ns} and uses the entire 4.7 fb$^{-1}$  data sample collected in 2011, extending the exclusion on the resonance mass by 10\% to 30\% 
depending on the resonance type. The analysis employes a novel trigger, data acquisition, 
and analysis strategy to recover sensitivity to new physics at dijet masses below 1 TeV. \\ [1em] 
% WW/WZ/ZZ RESONANCES
\years{Dec 2011 - today} Search for heavy qW/qZ/WW/WZ/ZZ resonances in the 
W/Z-tagged dijet mass spectrum at CMS using jet substructure techniques 
to identify the hadronic decays of boosted vector bosons~\cite{AN-11-524}.  \\ [1em] 
% ARC MEMBER
\years{Jun 2011 - today} Member of the {\it``Analysis Review Committee''} for the scrutiny of two public CMS results within the collaboration: top cross section measurements in all hadronic decay channel~\cite{CMS-PAS-TOP-11-007} and search for Randall-Sundrum gravitons decaying into a jet plus missing transverse energy final state~\cite{CMS-PAS-EXO-11-061}
with 2011 collision data. \\ [1em] 
% DDT COORDINATION
\years{Mar 2012 - today} Coordination of the {\it``Dataset Definition Team"} of the CMS experiment: task force created to bring together experts from different areas (physics coordination, trigger study group, physics validation team, etc...) and acting as a main forum for the discussion of all the aspects related to the definition, maintaining and monitoring of the data 
streams to be used for physics analysis and detector calibration in 2012.  \\ [1em] 
% HCAL PFG 
\years{Sept 2008 - Sept 2010} Coordination of the {\it``Prompt Feedback Group''} of the hadronic calorimeter (HCAL) of the CMS experiment: monitoring and data analysis concerning problems
in the HCAL detector during data-taking of cosmic rays [see ``Talks in Plenary Meetings of the CMS Collaboration'' 
$\rightarrow$  presentations on behalf of the HCAL group]. \\ [1em]
%Prompt analysis during the very first LHC collisions at $\sqrt{s}=$7~TeV 
%[see ``Talks in Plenary Meetings of the CMS Collaboration`` 
%$\rightarrow$  presentation on behalf of the HCAL and Jet/MET groups] \\ [1em]
% MET
\years{Nov 2009 - Sept 2010} Commissioning of missing transverse energy (MET) 
reconstructed with the first proton-proton ({\it pp}) collisions at $\sqrt{s}=$0.9, 2.36 and 7 TeV collected by the CMS experiment \cite{JME-10-004,JME-10-002,AN-2010-219,AN-2010-029}. \\ [1em]
%HF PMT NOISE
\years{Nov 2009 - Sept 2010}Development and implementation of algorithms for the identification 
of anomalous, beam-induced signals (``noise'') in the Hadronic Forward Calorimeter (HF) of the CMS experiment, observed 
in the first {\it pp} collisions at $\sqrt{s}=$0.9, 2.36 and 7 TeV \cite{DN-2010-008}. \\ [1em]

\clearpage

% TEST BEAM HCAL 2009
\years{Jun 2009 - Jul 2009}Contribution to the test beam of the hadronic calorimeter of the CMS experiment 
(HCAL Test Beam 2009 \cite{Chatrchyan:2010zz}): commissioning and calibration 
of the {\it ``delay wire chambers''} installed along the H2 beam line (CERN, Prevessin site) 
for beam position measurements. \\ [1em]
% HCAL COMMISSIONING
\years{Jan 2008 - Jul 2008} Commissioning of the hadronic calorimeter (HCAL) of the CMS experiment: ``on-call`` support for data acquisition (DAQ) and trigger configurations of HCAL during early periods of cosmic-ray data-taking.\\ [1em]
%GMSB (TESI DOTTORATO)
\years{Dec 2006 - Dec 2007}Feasibility study of the search for Gauge Mediated Supersymmetry Breaking (GMSB) models 
in the prompt photon decay channel $pp \rightarrow \tilde{\chi}_1^0 \tilde{\chi}_1^0 + X \rightarrow \tilde{G} \tilde{G} \gamma \gamma + X$ 
\cite{Santanastasio:DOTTORATO}, with full simulation of the CMS detector. \\ [1em]
%TEST BEAM ECAL+HCAL 2006
\years{Jul 2006 - Sept 2006}Monitoring of the high voltage system of the CMS electromagnetic calorimeter (ECAL)
and data-taking shifts in the combined ECAL+HCAL test beam at CERN, Prevessin site
(H2 Test Beam 2006 \cite{Abdullin:2009zz}).\\ [1em]
%ECAL HV
\years{Mar 2006 - Nov 2006}Analysis and test of stability of ECAL high voltage system including 
development of software tools for data analysis \cite{Bartoloni:2007hx}. \\ [1em]
%%
%pi0 CALIBRATION
\years{Oct 2005 - Oct 2006}Study of the calibration of the CMS electromagnetic calorimeter
using $\pi^0 \rightarrow \gamma\gamma$ decays with full detector simulation \cite{Adzic:2008zza,DN-2007-013,IN-2006-050}.  \\[1em]
%LAUREA
\years{Jan 2003 - May 2004}Study and implementation of the energy flow technique applied to the calibration 
of the electromagnetic calorimeter of the L3 experiment at LEP (CERN) \cite{Santanastasio:LAUREA}. \\[1em]



\section*{Talks at Conferences}
\noindent
%MORIOND/EW
\years{13-20.03.2011}\textbf{Moriond/EW 2011} - Rencontres de Moriond on 
``EW Interactions and Unified Theories''\\
La Thuile, Valle D'Aosta, Italy\\
\textit{``Exotica Searches at CMS''}\\ 
Presentation in plenary session on behalf of the CMS Collaboration\\
%Talk on behalf of the CMS Collaboration\\
%Conference proceedings will be published in date and journal still to be defined\\  [1em] 
Conference proceedings \cite{MoriondEW2011} \\  [1em] 
%DIS2010
\years{19-23.04.2010}\textbf{DIS2010} - XVIII International Workshop on 
Deep-Inelastic Scattering and Related Subjects\\
Firenze, Italy\\
\textit{``Searches With Early Data At CMS''}\\ 
Presentation in parallel session on behalf of the CMS Collaboration\\
Conference proceedings \cite{Santanastasio:2010zz} \\  [1em] 
%IFAE2009
\years{15-17.04.2009}\textbf{IFAE2009} - Incontri di Fisica delle Alte Energie, VIII Edizione\\
Bari, Italy\\
\textit{``Prospects for Exotica Searches at ATLAS and CMS Experiments''}\\ 
Presentation in parallel session on behalf of the ATLAS and CMS Collaborations\\
Conference proceedings \cite{Santanastasio:IFAE2009} 

\clearpage

\section*{Talks in Plenary Meetings of the CMS Collaboration}
\noindent
%First 7TeV Collisions
\years{Mar 2010}\textbf{CMS General Weekly Meeting GWM11} - Preliminary results, plots, lessons 
from the first 7 TeV collisions - CERN, Geneve, Switzerland \\
\textit{``Report from HCAL/JetMET''}\\ 
Presentation in plenary session on behalf of the HCAL and Jet/MET groups of the CMS experiment\\ [1em] 
%CMS Italia 2010
\years{Jan 2010}\textbf{Riunione CMS Italia} - Pisa, Italy \\
\textit{``Example of prompt analysis at CERN: Jet/MET commissioning with first collision data''}\\  [1em] 
%CRAFT2009
\years{Sept 2009}\textbf{CMS Commissioning and Run Coordination meeting} - CRAFT (Cosmic Run At Four Tesla) 
2009 Data Analysis Jamboree - CERN, Geneve, Switzerland \\
\textit{``HCAL (Hadronic Calorimeter of CMS experiment) performance during CRAFT09''}\\ 
Presentation in plenary session on behalf of the HCAL group of the CMS experiment\\ [1em] 
%CRAFT2008
\years{Nov 2008}\textbf{CMS Commissioning and Run Coordination meeting} - CRAFT (Cosmic Run At Four Tesla) 
2008 Data Analysis Jamboree - CERN, Geneve, Switzerland \\
\textit{``HCAL (Hadronic Calorimeter of CMS experiment) achievements during CRAFT08''}\\ 
Presentation in plenary session on behalf of the HCAL group of the CMS experiment

\section*{Teaching}
\noindent
%Fisica Generale 1 2005-2006
\years{Oct 2005 - Feb 2006}\textbf{Sapienza Universit\`a di Roma} - Roma, Italy \\
\textit{Teaching assistant for the course of ``Fisica Generale I - meccanica classica''} \\ 
Exercises of classic mechanics for mathematics majors

\section*{Physics Schools}
\noindent
% FERMILAB 2008
\years{12-22.08.2008}\textbf{2008 Joint CERN-Fermilab Hadron Collider Physics Summer School} \\ 
Fermilab, Batavia, Illinois, US \\ [1em]
% LECCE 2005
\years{09-14.06.2005}\textbf{Italo-Hellenic School of Physics 2005}  \\ 
Martignano, Lecce, Italy \\
{\it ``The Physics of LHC: theoretical tools and experimental challenges''}

\section*{Languages}
\begin{tabular}{l c l}
\textit{Italian} (native speaker) & \makebox[4em]{} & \textit{English} (fluent)\\
%\textit{Italian} (native speaker) & \makebox[4em]{} & \textit{French} (fluent)\\
%\textit{English} (fluent) & &\textit{German} (basic)\\
\end{tabular}

\clearpage

%%%%%%%%%%%%%%%%%%%%%%%%%%%
%%% Summary of research activities
%%%%%%%%%%%%%%%%%%%%%%%%%%%
\section*{Summary of Past Research Activities}

In fall 2007, towards the end of my PhD studies in Rome, 
I decided to start a learning experience abroad in order to continue the research 
activity in particle physics at an high energy physics laboratory 
and to broaden my knowledge in this field; leaving open the possibility of a future return 
to Italy as researcher or assistant professor in the university.\\

In December 2007, I started an appointment as post-doctoral research 
assistant ({\it post-doc}) in particle physics at the University of Maryland. 
Since then I have been based at the 
CERN (Conseil Europeen pour la Recherche Nucleaire) laboratory,
working in the Compact Muon Solenoid (CMS) experiment 
at the Large Hadron Collider (LHC) and focusing primarily on:
\begin{itemize} 
\item analysis of proton-proton ({\it pp}) collision data within the 
exotic physics group (Exotica), looking for evidence 
of new physics beyond the Standard Model of Fundamental Interactions (SM); 
\item commissioning, {\it ``prompt analysis''} and detector performance studies 
of the hadronic calorimeter (HCAL);
\item performance studies of missing transverse energy (MET) reconstruction with first {\it pp} collision data at LHC. 
\end{itemize}
%
In September 2011, I started an appointment as Research Fellow in experimental
particle physics at CERN, and I decided to continue my research activities in the CMS experiment. 
In the first six months of my new contract, I have been extending 
my involvement in the Exotica group (by starting new physics analysis efforts) and, in March 2012, began to coordinate a working team of the CMS experiment devoted to the definition, maintaining, and monitoring of the data streams to be used for physics analysis and detector calibration during the 2012 data taking.\\

\begin{center} \textsc{Research activities related to physics analysis of pp collision data} \\ \end{center} 

Since the beginning of my post-doctoral appointment, I have been actively involved in the research activities of the CMS Exotica group, which is devoted to search for new physics phenomena beyond the SM. I presented the results of these analyses in international conferences on behalf of the CMS collaboration [see ``Talks at Conferences'']. \\

%LQ
I started my activities in the Exotica group in 2008 with the search for pair production of  first generation scalar {\it ``leptoquarks''} (LQ) 
in the $LQ\overline{LQ} \rightarrow ee qq$ decay channel ({\it eejj}). \\
Leptoquarks are conjectured particles
foreseen by some well-motivated theories beyond of the SM: they are coloured, have fractional electric charge, and couple to a lepton and a quark via an unknown coupling. 
%in which transitions between 
%leptonic and baryonic sectors are allowed. 
The process under study has a very characteristic signature, 
with two high transverse momentum ($p_T$) electrons and two high $p_T$ jets, and a peak in 
the electron-jet invariant mass spectrum corresponding to the LQ mass. 

The feasibility study, done in 2009 with full simulation of the CMS detector \cite{EXO-08-010,AN-2008-070},
aimed to the optimization of selection criteria to reject the SM backgrounds, and to study
techniques to estimate them directly from data. 
%This work showed that the existence of LQ with mass
%about twice higher than the current limit set by Tevatron experiments, could be excluded at CMS with about 
%100 pb$^{-1}$ of data in {\it pp} collisions at $\sqrt{s}=10$~TeV. 

The analysis has been performed with 33 pb$^{-1}$ of {\it pp} collisions at $\sqrt{s}=7$~TeV collected by the CMS experiment in 2010 
\cite{Khachatryan:2010mp,EXO-10-005,AN-2010-230}. The data was in good agreement with the SM predictions. Therefore 
a 95\% {\it ``confidence level''} lower limit was set on the mass of first generation scalar LQ at 384 GeV/$c^2$, assuming a 
branching ratio of 100\% for the decay $LQ\rightarrow eq$. This result exceeded the existing Tevatron limit on the LQ mass 
of 300 GeV/$c^2$, obtained with 1 fb$^{-1}$ of proton-antiproton collisions at $\sqrt{s}=$1.96~TeV, hence extended the search for 
leptoquarks in an unexplored mass region. The paper has been published in the 
Phys. Rev. Lett. journal. \\
%
In addition to the {\it eejj} analysis, I was the contact person of the search 
for pair production of first generation scalar leptoquarks
in the $LQ\overline{LQ} \rightarrow e\nu qq$ decay channel ({\it e$\nu$jj}) \cite{Chatrchyan:2011ar,AN-2010-361}.
The combination of the results from these two channels has been used to improve 
the sensitivity to the new physics in the space of the unknown parameters of the theory model: $M_{LQ}$ vs $\beta$, where $M_{LQ}$ is the LQ mass, and 
$\beta$ ($1-\beta$) is the branching ratio of the decay $LQ\rightarrow eq$ ($LQ\rightarrow \nu q$). 
The {\it e$\nu$jj} analysis has been published in the Phys.\ Lett.\ B journal using 36 pb$^{-1}$ of data collected in 2010. \\
%
I have been involved in this search also during 2011, by
supervising a PhD student from the Princeton University to update both the {\it eejj} and 
{\it e$\nu$jj} $LQ$ analyses using the 4.7 fb$^{-1}$ of data collected in 2011 by 
the CMS experiment~\cite{AN-11-492}. These results, which are aiming for publication in the first months of 2012 in combination with a complementary second-generation $LQ$ search, are going to significantly 
extend the new physics reach compared to 2010, thanks to the improvements in the analysis and 
the larger statistics available. \\
%
Although low mass scale LQs, accessible at current colliders, are generally not considered to be 
one of the preferred extensions of the SM, searches for LQ pair production can be regarded as a prime example for new signatures with leptons, jets and MET with a SM background dominated by weak boson and top-pair production, and therefore
represent important benchmark analyses. \\
%to compare the sensitivity to 
%new physics in general with other experiments. \\

%DIJETS
In September 2011, I started an appointment as Research Fellow 
in experimental particle physics at CERN and I joined the
CMS analysis group working on a search for new resonances 
decaying to a pair of jets in the dijet mass spectrum. \\
Proton-proton collisions can produce two or more energetic jets 
when the constituent partons are scattered with large transverse 
momenta, $p_T$.  The invariant mass spectrum of the two jets with
largest $p_T$ (dijets) is predicted to fall steeply and smoothly by quantum
chromodynamics (QCD). Many extensions of the SM predict 
the existence of new massive objects that couple to quarks (q) and/or gluons (g), 
thus resulting in resonances in the dijet mass spectrum. The analysis 
is sensitive to a wide range of new physics models, including string resonances, 
excited quarks, axigluons, new vector bosons (W', Z'), 
and Randall-Sundrum (RS) gravitons (G).\\
%
The main analysis is an update of the previous CMS published result~\cite{Chatrchyan:2011ns}, 
but performed with the entire 4.7 fb$^{-1}$ data sample 
collected in 2011. I have been focusing primarily on the improvements 
compared to the previous result. I am the main developer of the novel 
trigger, data acquisition, and analysis strategy employed in this search to 
recover sensitivity to new physics at dijet masses below 1 TeV~\cite{AN-12-012}. 
In the standard analysis, 
%that uses the regular data sample (``stream A''), 
the region below 1 TeV (dominated by a very large rate of events from QCD processes) 
is dropped due to limitations in the available jet trigger bandwidth. 
This is a natural consequence of the 
steady increase of the LHC instantaneous luminosity during 2011. 
%The same limitation is found to affect also a similar analysis from the ATLAS experiment. 
It is important to be able to explore this mass range since new 
resonances weakly-coupled to SM particles could still 
be hiding at low mass within large QCD background.\\
My proposal consists in performing the dijet search using a special sample of data 
collected with low jet $p_T$ triggers (high rate of events), but storing only 
a reduced event content (small size per event).
This allows to keep the bandwidth (rate of events $\times$ size per event)
to values acceptable by the data acquisition system.
The reduced event content consists primarily of physics objects reconstructed at 
High Level Trigger (HLT), such as HLT jets which are the main ingredient for the dijet search, 
while the raw data from the detector electronics (full event content) are not saved.
A similar conceptual design has already been used in CMS for detector calibration 
purpose, but it is the first time that this is employed for physics analysis.
This analysis is aiming to deliver a preliminary public result in the 2012.\\

%DDT
The CMS experiment is planning to employ the trigger strategy discussed above 
to design a more general {\it data scouting} tool for the online monitoring of 
those searches that are usually limited by the available trigger bandwidth.
%; for example, some Supersymmetry 
%(SUSY) scenarios with ``compressed mass spectra'' where the presence of low 
%MET and low jet $p_T$ makes challenging the online selection of the interesting events 
%using the standard approaches. \\
During 2012, this fast preview of the data would allow to change the definition of the regular data stream, in case a potential new physics signal shows up in a region not covered by the triggers defining the stream with full event content. \\
The definition of the {\it data scouting} monitoring system is part 
of the duties of the ``Dataset Definition Team'' (DDT)
that I have been asked to coordinate in March 2012. The DDT is a task force created to bring together experts from different areas (physics coordination, trigger study group, physics validation team, etc...) and acting as a main forum for the discussion of all the aspects related to the definition, maintaining and monitoring of the data streams to be used for physics analysis and detector calibration in 2012. \\

%%%%%%%%%%%%%%%%%%%%%%%%%%%%%%%%%%%
%%%%%%%%%%%%%%%%%%%%%%%%%%%%%%%%%%%
% VV resonances

% WW/WZ/ZZ RESONANCES
Profiting from the experience gained in the search for dijet resonances,
I recently joined an analysis group working on a search for 
heavy qW/qZ/WW/WZ/ZZ resonances in the W/Z-tagged dijet mass 
spectrum~\cite{AN-11-524} with data collected in 2011. 
The analysis focuses on resonances which are sufficiently 
heavy to result in vector bosons with large energy/$p_T$; the hadronic 
decay products of each vector boson are then merged into a single massive jet, 
and the final state effectively has a dijet topology.
This search can be seen as an extension of the inclusive dijet search 
discussed above, with which it shares, for instance, the basic methods
for background estimation and the statistical tools. 
I'm planning to contribute at the update of this search with 2012 data 
by studying various jet substructure techniques to identify the 
hadronic decays of boosted vector bosons.\\

%ARC
Since June 2011, I have been member of the {\it``Analysis Review Committee''} (ARC) for the scrutiny of two public CMS results within the collaboration: the top cross section measurements in all hadronic decay channel~\cite{CMS-PAS-TOP-11-007} and a search for Randall-Sundrum gravitons decaying into a massive jet plus missing transverse energy final state ($\mbox{G} \rightarrow \mbox{ZZ} \rightarrow \nu\nu jj$)~\cite{CMS-PAS-EXO-11-061}.
The ARCs take an important role in the approval process of physics analyses within 
the CMS collaboration.\\

%%%%%%%%%%%%%%%%%%%%%%%%%%%%%%%%%%%
%%%%%%%%%%%%%%%%%%%%%%%%%%%%%%%%%%%

\begin{center} \textsc{Other research activities during my post-doc at University of Maryland} \\ \end{center} 

At CMS, the hadronic calorimeter HCAL is mainly employed, together with electromagnetic calori- meter ECAL, for the reconstruction 
of jets 
%(the experimental signature of the hadronization of partons) 
and the missing transverse energy
in the event, hence playing an important role for many physics analyses. 
%feasible at an hadron collider as LHC. 
The very forward part of the HCAL is also 
used for luminosity measurement. \\

For the first six months of my appointment with the University of Maryland, I was involved in the 
HCAL commissioning, providing on-call support for data acquisition (DAQ) and trigger configurations 
during the early period of cosmic-ray data-taking by the CMS detector. 
%Thanks to this commissioning work, 
%I could learn details of the HCAL detector that were useful for the activities of data analysis described 
%in the following paragraph. 
In Summer 2009, I contributed to test beam studies of the hadronic calorimeter
(HCAL Test Beam 2009 \cite{Chatrchyan:2010zz}) by commissioning and calibrating the 
{\it ``delay wire chambers''} installed along the H2 beam line (CERN, Prevessin site) 
for beam position measurements. \\

For two years, starting from September 2008, I coordinated the 
HCAL {\it ``Prompt Feedback Group''} (PFG) of the CMS collaboration, 
composed of about 5-10 people. The PFG worked 
on data analysis related to anomalies found in the detector, including 
problems in the firmware of electronics boards, data-format and trigger issues, 
as well as the support to groups devoted to the online ({\it ``Data Quality Monitoring''}, DQM)
and offline ({\it ``Run Certification''}) control of data quality.

On various occasions, I presented to the CMS collaboration the status of the detector on behalf of the HCAL group, 
including talks in plenary meetings that followed the two main cosmic-ray data-taking periods in 2008 and 2009
[see ``Talks in Plenary Meetings of the CMS Collaboration'' $\rightarrow$ talks on behalf on the HCAL group].

At the beginning of 2010, I coordinated the HCAL PFG in preparation to the first LHC {\it pp} collisions at $\sqrt{s}=7$~TeV, 
which occurred on 30 March 2010. For this event we provided 
results in real time giving evidence of the collisions. The following day, I presented to a CMS plenary meeting the results 
of the very first detector performance analyses based on the {\it pp} collisions on behalf of the HCAL and Jet/MET groups
[see ``Talks in Plenary Meetings of the CMS Collaboration'' $\rightarrow$ talks on behalf on the HCAL and Jet/MET group]. 

In conclusion the PFG, under my coordination, provided an active contribution to both the HCAL commissioning in 2008-2009, 
and to the regular operation of the detector during the physics data-taking in 2010. \\

In addition to the research activities related to electromagnetic and hadronic calorimeters, I joined in November 2009 
the Jet/MET group of CMS, that is employed in development and performance studies of jets and MET reconstruction. 
In the first months of 2010, I played an active role in the MET commissioning, using the first {\it pp} collision data 
at $\sqrt{s}=$0.9, 2.36 \cite{JME-10-002} and 7 TeV \cite{JME-10-004}. In particular, I am the main author of the following works:
study of performance of the {\it ``uncorrected calorimeter''} MET \cite{AN-2010-029}, 
classification of events in the non-gaussian tails of the MET distribution \cite{AN-2010-219}, 
and development and implementation of algorithms for the identification of anomalous, beam-induced noise in the 
Hadronic Forward Calorimeter (HF) \cite{DN-2010-008}. The anomalous signals
observed in HF can produce large apparent MET in the event; therefore it's crucial to identify and reject them during the event reconstruction, 
since such uncharacteristic signals can worsen the precision of some physics measurements, or even simulate a fake signature 
of new physics beyond the SM. The understanding of the performance of jets and MET reconstruction is 
an important point for the physics analyses I'm currently working on. \\


\begin{center} \textsc{Research activities during my undergraduate and graduate studies} \\ \end{center} 

My interest in elementary particle physics drove me to choose this field when I was an undergraduate student in Rome 
and, since 2004, to do research as part of the CMS collaboration at the LHC 
of CERN. \\

In 2003, I started working on my undergraduate thesis at \textit{Sapienza}, Universit\`a di Roma. The work concerned the 
study of the calibration of an electromagnetic calorimeter using the energy flow method \cite{Santanastasio:LAUREA}, which allows to inter-calibrate 
calorimeter crystals by using the $\phi$ symmetry of energy deposits at a collider. \\

In October 2004, I was admitted to the graduate school in physics to work with the CMS group. 
The Rome group was heavily involved in the construction of the electromagnetic calorimeter (ECAL), 
as well as in monitoring and calibration. In my three years as a graduate student I worked on the calibration of 
the calorimeter, the stability of the ECAL high voltage (HV) system and feasibility studies for physics analysis on the search for Supersymmetry. \\

%The first year of my PhD was mostly devoted to the courses of the graduate school 
%and to learn the CMS software and analysis tools. \\

In 2006, I worked on the feasibility study of using $\pi^0 \rightarrow \gamma\gamma$ decays for the calibration 
of the ECAL crystals \cite{DN-2007-013,IN-2006-050}. This method has the advantage of high 
statistics, since $\pi^0$ are produced in abundance at hadron colliders, and does not rely on information from 
the detectors measuring tracks from charged particles, and hence could be performed {\it ``in situ''} in the early 
periods of data-taking of LHC if the alignment and calibration of the high precision tracking system are not yet understood.
The real challenge of this analysis is finding a satisfactory signal to noise ratio while maintaining high selection efficiency for such events 
in order to achieve a calibration of the entire ECAL in a short period of data-taking.

In 2011, the CMS experiment collected enough data to calibrate at regular intervals 
of few months both the central (barrel) and the forward (endcaps) region of ECAL using $\pi^0$'s.
The plans for 2012 foresee the improvements of the methods to calibrate the endcaps, to contribute at the monitoring of the crystal transparency loss due to the high radiation environment at LHC, as well as the combination of different calibration techniques that are available, in order to achieve the design precision on the ECAL calibration in the whole detector acceptance. \\

During summer of 2006, I participated in the combined test beam of the electromagnetic and 
hadronic calorimeters of the CMS experiment at the H2 area of CERN, Prevessin site 
(H2 Test Beam 2006 \cite{Abdullin:2009zz}), mainly performing data-taking shifts.
An important feature of the H2 test facility was the possibility to produce a secondary beam
of $\pi^0$'s by inserting a target along the primary charged pion beam line. 
This data \cite{Adzic:2008zza} was used to verify and improve the $\pi^0 \rightarrow \gamma\gamma$ 
reconstruction algorithm developed for the calibration studies with simulated events.
During this period, I also worked on the monitoring of the ECAL high voltage system, which is 
under the direct responsibility of the Rome group. \\
%Thanks to this activity, I was able to learn directly 
%some knowledge of the hardware part related with the operation of electromagnetic calorimeter. \\

My other activities included both development and implementation of the analysis software for the 
stability test of HV boards, and the relative analysis of data collected since 2003 \cite{Bartoloni:2007hx}. 
The stability of the HV system is very important for the operation of ECAL because it affects directly 
the energy resolution of the electromagnetic calorimeter. \\

In 2007, I worked mainly on feasibility study of the search for Supersymmetry 
with Gauge-Mediated Breaking (GMSB) in the prompt photon decay channel 
$pp \rightarrow \tilde{\chi}_1^0 \tilde{\chi}_1^0 + X \rightarrow \tilde{G} \tilde{G} \gamma \gamma + X$ 
(see PhD thesis \cite{Santanastasio:DOTTORATO}). 
The presence of two high energy photons and large missing transverse energy
in the final state due to gravitinos makes the experimental signature of such events very clear.
This feasibility study, aimed at the optimization of selection criteria to reject SM backgrounds, 
showed that GMSB models, with parameters just above the limit fixed by Tevatron experiments, 
could be an early discovery at the CMS experiment with a few tens pb$^{-1}$ of data and $\sqrt{s}=14$~TeV. This result was significantly better than the one shown by previous studies reported in the CMS collaboration.
%At the current energy of LHC, $\sqrt{s}=7$~TeV, the search for new physics in GMSB models 
%could extend beyond the limit set by previous experiments with a few hundreds pb$^{-1}$ of data. 

\clearpage


%%%%%%%%%%%%%%%%%%%%%%%%%%%
%%% Service work
%%%%%%%%%%%%%%%%%%%%%%%%%%%

%\section*{Service work in Experiments and Collaborations}
%\subsection*{ATLAS Experiment}
%\noindent
%\textbf{Data Analysis: Supersymmetry Working Group} Working on data analysis, on exploring and implementing analysis strategies and on data files production\\
%\textbf{Development \& Upgrade} Working in the DAQ group, on the upgrade of the configuration DB system\\
%\textbf{Detector Operation} Shifter in the control room, at the Muon System, DAQ and Run Control desks\\
%\textbf{Software Framework} Taking part in code testing, and shifter for the build test system (RTT)\\
%\textbf{Documentation} Responsible person for a part of the documentation of the ATLAS data-format\\
%\textbf{Public Relations} Official ATLAS Guide, escorting VIP visits to the ATLAS cavern\\

%%%%%%%%%%%%%%%%%%%%%%%%%%%
%%% Publications & Talks
%%%%%%%%%%%%%%%%%%%%%%%%%%%

\begin{thebibliography}{599}

%\cite{Chatrchyan:2011ar}
\bibitem{Chatrchyan:2011ar} 
  {\bf ``Search for First Generation Scalar Leptoquarks in the evjj channel in pp collisions at sqrt(s) = 7 TeV''}
  \\{} S.~Chatrchyan {\it et al.}  [CMS Collaboration],
  \\{} Phys.\ Lett.\ B {\bf 703}, 246 (2011), [arXiv:1105.5237 [hep-ex]].
  \\ I am the contact person and one of the two analysts (from University of Maryland group) of this CMS paper based on collision data.
  %%CITATION = ARXIV:1105.5237;%%

%\cite{Khachatryan:2010mp}
\bibitem{Khachatryan:2010mp}
{\bf ``Search for Pair Production of First-Generation Scalar Leptoquarks in pp Collisions at sqrt(s) = 7 TeV''}
  \\{}V.~Khachatryan {\it et al.}  [CMS Collaboration]
    \\{} Phys.\ Rev.\ Lett.\  {\bf 106}, 201802 (2011), [arXiv:1012.4031 [hep-ex]]
  \\I am one of the four analysts (from University of Maryland group) of this CMS paper based on collision data.
%\href{http://www.slac.stanford.edu/spires/find/hep/www?irn=8913501}{SPIRES entry}

%\cite{Chatrchyan:2011ns}
\bibitem{Chatrchyan:2011ns} 
 {\bf  ``Search for Resonances in the Dijet Mass Spectrum from 7 TeV pp Collisions at CMS,''}
 \\ S.~Chatrchyan {\it et al.}  [CMS Collaboration],
 \\ Phys.\ Lett.\ B {\bf 704}, 123 (2011), [arXiv:1107.4771 [hep-ex]].
  %%CITATION = ARXIV:1107.4771;%%

%\cite{Chatrchyan:2010zz}
\bibitem{Chatrchyan:2010zz}
{\bf ``Study of various photomultiplier tubes with muon beams and Cherenkov light produced in electron showers''}
  \\{}S.~Chatrchyan {\it et al.}  [CMS HCAL Collaboration]
  \\{}JINST {\bf 5}, P06002 (2010)
  \\ The data were collected during the HCAL Test Beam 2009. I contributed to 
  commissioning and calibration of the {\it ``delay wire chambers''} installed along the H2 
  beam line (CERN, Prevessin site) for beam position measurements.
%\\{}CMS-NOTE-2010-003
%\href{http://www.slac.stanford.edu/spires/find/hep/www?j=jinst\%2c5\%2cp06002}{SPIRES entry}

%\cite{Chatrchyan:2009hy}
\bibitem{Chatrchyan:2009hy}
{\bf ``Identification and Filtering of Uncharacteristic Noise in the CMS Hadron Calorimeter''}
  \\{}S.~Chatrchyan {\it et al.}  [CMS Collaboration]
  \\{}JINST {\bf 5}, T03014 (2010)
  [arXiv:0911.4881 [physics.ins-det]]
%\\{}CMS-CFT-09-019
%\href{http://www.slac.stanford.edu/spires/find/hep/www?j=jinst\%2c5\%2ct03014}{SPIRES entry}

%\cite{Abdullin:2009zz}
\bibitem{Abdullin:2009zz}
{\bf ``The CMS Barrel Calorimeter Response To Particle Beams From 2-Gev/C To 350-Gev/C''}
  \\{}S.~Abdullin {\it et al.}  [USCMS Collaboration and ECAL/HCAL
                  Collaboration]
  \\{}Eur.\ Phys.\ J.\  C {\bf 60}, 359 (2009)
  [Erratum-ibid.\  C {\bf 61}, 353 (2009)]
%\\{}FERMILAB-PUB-08-661-E-PPD
%\href{http://www.slac.stanford.edu/spires/find/hep/www?j=ephja\%2cc60\%2c359}{SPIRES entry}

%\cite{Adzic:2008zza}
\bibitem{Adzic:2008zza}
{\bf ``Intercalibration of the barrel electromagnetic calorimeter of the CMS  experiment at start-up''}
  \\{}P.~Adzic {\it et al.}  [CMS Electromagnetic Calorimeter Group]
  \\{}JINST {\bf 3}, P10007 (2008)
  \\  I performed a feasibility study of using $\pi^0 \rightarrow \gamma \gamma$ decays for the calibration of the ECAL crystals, with full detector simulation.
%\\{}CERN-CMS-NOTE-2008-018
%\href{http://www.slac.stanford.edu/spires/find/hep/www?j=jinst\%2c3\%2cp10007}{SPIRES entry}

%\cite{Bartoloni:2007hx}
\bibitem{Bartoloni:2007hx}
{\bf ``High voltage system for the CMS electromagnetic calorimeter''}
  \\{}A.~Bartoloni {\it et al.}
  \\{}Nucl.\ Instrum.\ Meth.\  A {\bf 582}, 462 (2007)
  \\ I performed part of the stability tests on the high voltage boards at CERN laboratory and most of the data analysis 
%\\{}CERN-CMS-NOTE-2007-009
%\href{http://www.slac.stanford.edu/spires/find/hep/www?j=nuima\%2ca582\%2c462}{SPIRES entry}

\clearpage

%------------------------------------------------------------------------------------------------------------------------------------------------------------
\vspace{0.1cm} \begin{center} \textsc{Conference Proceedings} \end{center} \vspace{0.05cm}
%------------------------------------------------------------------------------------------------------------------------------------------------------------

%Proceedings of the XLVIth Rencontres de Moriond
%2011 Electroweak Interactions and Unified Theories
%La Thuile, Aosta Valley, Italy � March 13-20, 2011
%edited by Etienne Aug�, Jacques Dumarchez, and Jean Tr�n Thanh V�n
%� Th� Gioi Publishers, 2011

%Exotica searches at the CMS experiment 
%F. Santanastasio 
%Pages 125-132

\bibitem{MoriondEW2011}
{\bf ``Exotica searches at the CMS experiment''}
  \\{}F.~Santanastasio
  \\{}Proceedings of the XLVIth Rencontres de Moriond 
  2011 Electroweak Interactions and Unified Theories, 125-132 (2011), edited by Etienne Auge, Jacques Dumarchez, and Jean Tran Thanh Van $\textcopyright$ The Gioi Publishers
\\{}{\it Prepared for XLVIth Rencontres de Moriond 2011 Electroweak Interactions and Unified Theories, La Thuile, Aosta Valley, Italy, 13-20 March 2011}

\bibitem{Santanastasio:2010zz}
{\bf ``Searches With Early Data At CMS''}
  \\{}F.~Santanastasio
  \\{}PoS {\bf DIS2010}, 206 (2010)
%\href{http://www.slac.stanford.edu/spires/find/hep/www?j=posci\%2cdis2010\%2c206}{SPIRES entry}
\\{}{\it Prepared for 18th International Workshop on Deep Inelastic Scattering and Related Subjects (DIS 2010), Florence, Italy, 19-23 Apr 2010}

\bibitem{Santanastasio:IFAE2009}
{\bf ``Prospects for Exotica Searches at ATLAS and CMS Experiments''}
  \\{}F.~Santanastasio
  \\{}Il Nuovo Cimento Vol.32 C, N.3-4 ncc9484 (2009)
\\{}{\it Prepared for Incontri di Fisica delle Alte Energie (IFAE 2009), Bari, Italy, Apr 2009}

%------------------------------------------------------------------------------------------------------------------------------------------------------------
\vspace{0.1cm} \begin{center} \textsc{Preliminary results of the CMS Collaboration (relative to research activities)} \end{center} \vspace{0.05cm}
%------------------------------------------------------------------------------------------------------------------------------------------------------------
  
%\cite{EXO-10-005}
\bibitem{EXO-10-005}
{\bf ``Search for Pair Production of First Generation Leptoquarks Using Events Containing Two Electrons and Two Jets Produced in pp Collisions at sqrt(s) = 7 TeV''}
  \\{}[CMS Collaboration]
  \\{}CMS PAS EXO-10-005 (2010), http://cdsweb.cern.ch/record/1289514/files/EXO-10-005-pas.pdf 
  \\I am co-author and one of the four analysts (from University of Maryland group) of this public CMS Physics Analysis Summary based on collision data.

%\cite{EXO-08-010}
\bibitem{EXO-08-010}
{\bf ``Search for Pair Production of First Generation Scalar Leptoquarks at the CMS Experiment''}
  \\{}[CMS Collaboration]
  \\{}CMS PAS EXO-08-010 (2009), http://cdsweb.cern.ch/record/1196076/files/EXO-08-010-pas.pdf
  \\I am co-author and one of the four analysts (from University of Maryland group) of this public CMS Physics Analysis Summary based on MC simulation.

%\cite{JME-10-004}
\bibitem{JME-10-004}
{\bf ``Missing Transverse Energy Performance in Minimum-Bias and Jet Events from Proton-Proton Collisions at sqrt(s)=7 TeV''}
  \\{}[CMS Collaboration]
  \\{}CMS PAS JME-10-004 (2010), http://cdsweb.cern.ch/record/1279142/files/JME-10-004-pas.pdf 

%\cite{JME-10-002}
\bibitem{JME-10-002}
{\bf ``Performance of Missing Transverse Energy Reconstruction in sqrt(s)=900 and 2360 GeV pp Collision Data''}
  \\{}[CMS Collaboration]
  \\{}CMS PAS JME-10-002 (2010), http://cdsweb.cern.ch/record/1247385/files/JME-10-002-pas.pdf 
  \\ I worked mostly on the section related to calorimeter MET cleaning algorithms and performances.

%\cite{CMS-PAS-TOP-11-007}
\bibitem{CMS-PAS-TOP-11-007}
{\bf ``Measurement of the ttbar production cross section in the fully hadronic decay channel in pp collisions at 7 TeV''}
  \\{}[CMS Collaboration]
  \\{}CMS PAS TOP-11-007 (2011), http://cdsweb.cern.ch/record/1371755/files/TOP-11-007-pas.pdf 
  \\I am still involved in the review of the update of this analysis, with the entire data collected in 2011, which is aiming for publication in the first months of 2012.
  
%\cite{CMS-PAS-EXO-11-061}
\bibitem{CMS-PAS-EXO-11-061}
{\bf ``Search for Randall-Sundrum Gravitons Decaying into a Jet plus Missing ET at CMS''}
  \\{}[CMS Collaboration]
  \\{}CMS PAS EXO-11-061 (2011), http://cdsweb.cern.ch/record/1426654/files/EXO-11-061-pas.pdf 

%------------------------------------------------------------------------------------------------------------------------------------------------------------
\vspace{0.1cm} \begin{center} \textsc{Internal notes of the CMS Collaboration (relative to research activities)} \end{center} \vspace{0.05cm}
%------------------------------------------------------------------------------------------------------------------------------------------------------------

%\cite{AN-12-012}
\bibitem{AN-12-012}
{\bf ``Search for Dijet Resonances in the Dijet Mass Spectrum in pp Collisions at sqrt(s)=7 TeV'}
  \\{}F.~Santanastasio {\it et al.}
  \\{}CMS AN-2012/012 (2012)
  \\ I am one of the two analysts (from a group of about 10 people from various 
  institutions including CERN) of this CMS analysis based on 4.7 fb$^{-1}$ of $pp$ collision 
  data collected in 2011. I am the main developer of the novel trigger, data acquisition, and analysis strategy employed in this search to recover sensitivity to new physics at dijet masses below 1 TeV. This analysis is currently under approval process within the CMS Collaboration, and it is aiming for a preliminary public result (CMS Physics Analysis Summary) followed by a publication in the first months of 2012. 

%\cite{AN-12-012}
\bibitem{AN-11-524}
{\bf ``Search for qW/qZ/WW/WZ/ZZ Resonances in the W/Z-tagged Dijet Mass Spectrum from 7 TeV pp Collisions at CMS''}
  \\{}F.~Santanastasio {\it et al.}
  \\{}CMS AN-2011/524 (2011)
  \\ I recently joined the analysis group involved in this CMS search which is constituted by almost 10 people from CERN, John Hopkins University, and \textit{L'Institut de Physique Nucleaire de Lyon} (INPL). I plan to contribute to the update with the 2012 data by studying various jet substructure algorithms to improve the sensitivity of the analysis to new physics. This analysis is currently under approval process within the CMS Collaboration, and it is aiming for publication in the first months of 2012, using the data collected during 2011. 

%\cite{AN-11-492}
\bibitem{AN-11-492}
{\bf ``Search for First-Generation Scalar Leptoquarks in pp Collisions at sqrt(s)=7 TeV using the CMS Detector''}
  \\{}F.~Santanastasio {\it et al.}
  \\{}CMS AN-2011/492 (2011)
  \\ I am one of the two analysts (supervising a PhD student from Princeton University) of this CMS analysis based on 4.7 fb$^{-1}$ of $pp$ collision data collected in 2011. 
This analysis is currently under approval process within the CMS Collaboration, and it is aiming for publication in the first months of 2012 in combination with a complementary second-generation leptoquark search.

%\cite{AN-2010-361}
\bibitem{AN-2010-361}
{\bf ``Search for Pair Production of First-Generation Scalar Leptoquarks Using Events Produced in pp Collisions at sqrt(s)=7 TeV Containing One Electron, Two Jets and Large Missing Transverse Energy''}
  \\{}F.~Santanastasio {\it et al.}
  \\{}CMS AN-2010/361 (2010)

%\cite{AN-2010-230}
\bibitem{AN-2010-230}
{\bf ``Search for Pair Production of First Generation Leptoquarks Using Events Containing Two Electrons and Two Jets Produced in pp Collisions at sqrt(s)=7 TeV''}
  \\{}F.~Santanastasio {\it et al.}
  \\{}CMS AN-2010/230 (2010)

%\cite{AN-2008-070}
\bibitem{AN-2008-070}
{\bf ``Search for Pair Production of First Generation Scalar Leptoquarks at the CMS Experiment''}
  \\{}F.~Santanastasio {\it et al.}
  \\{}CMS AN-2008/070 (2009)

%\cite{AN-2010-219}
\bibitem{AN-2010-219}
{\bf ``Results of a visual scan of high MET events in 7 TeV pp collision data''}
  \\{}F.~Santanastasio {\it et al.}
  \\{}CMS AN-2010/219 (2010)
  
%\cite{AN-2010-029}
\bibitem{AN-2010-029}
{\bf ``Commissioning of Uncorrected Missing Transverse Energy in Zero Bias and Minimum Bias Events at  sqrt(s)=900 GeV and  2360 GeV''}
  \\{}F.~Santanastasio {\it et al.}
  \\{}CMS AN-2010/029 (2010)

%\cite{DN-2010-008}
\bibitem{DN-2010-008}
{\bf ``Optimization and Performance of HF PMT Hit Cleaning Algorithms Developed Using pp Collision Data at sqrt(s)=0.9, 2.36 and 7 TeV''}
  \\{}F.~Santanastasio {\it et al.}
  \\{}CMS DN-2010/008 (2010)

%\cite{DN-2007-013}
\bibitem{DN-2007-013}
{\bf ``InterCalibration of the CMS Barrel Electromagnetic Calorimeter Using Neutral Pion Decays``}
   \\{}F.~Santanastasio {\it et al.}
  \\{}CMS DN-2007/013 (2007)

%\cite{IN-2006-050}
\bibitem{IN-2006-050}
{\bf ``Study of ECAL calibration with $\pi^0 \rightarrow \gamma \gamma$ decays''}
  \\{}F. ~Santanastasio, D.~del~Re, S.~Rahatlou
  \\{}CMS IN-2006/050 (2006)

%------------------------------------------------------------------------------------------------------------------------------------------------------------
\vspace{0.1cm} \begin{center} \textsc{Theses ( \textit{Laurea} and PhD)} \end{center} \vspace{0.05cm}
%------------------------------------------------------------------------------------------------------------------------------------------------------------

\bibitem{Santanastasio:DOTTORATO}
{\bf ``Search for Supersymmetry with Gauge-Mediated Breaking using high energy photons at CMS experiment''}
  \\{}F.~Santanastasio
  \\{}PhD thesis at \textit{Sapienza Universit\`a di Roma} (2007)
\\{}{\it http://www.roma1.infn.it/cms/tesiPHD/santanastasio.pdf}

\bibitem{Santanastasio:LAUREA}
{\bf ``Calibrazione di un calorimetro elettromagnetico tramite il flusso totale di energia''}
  \\{}F.~Santanastasio
  \\{}\textit{Laurea} thesis at \textit{Sapienza Universit\`a di Roma} (2004)
\\{}{\it http://www.roma1.infn.it/cms/tesi/santanastasio.pdf }


%------------------------------------------------------------------------------------------------------------------------------------------------------------
\vspace{0.1cm} \begin{center} \textsc{Other Publications and Pre-Prints of the CMS Collaboration} \end{center} \vspace{0.05cm}
%------------------------------------------------------------------------------------------------------------------------------------------------------------

%\cite{Chatrchyan:2012hw}
\bibitem{Chatrchyan:2012hw}
  S.~Chatrchyan {\it et al.}  [CMS Collaboration],
  ``Measurement of the cross section for production of b b-bar X, decaying to
  muons in pp collisions at sqrt(s)=7 TeV,''
  arXiv:1203.3458 [hep-ex].
  %%CITATION = ARXIV:1203.3458;%%

%\cite{Chatrchyan:2012ta}
\bibitem{Chatrchyan:2012ta}
  S.~Chatrchyan {\it et al.}  [CMS Collaboration],
  ``Search for microscopic black holes in pp collisions at sqrt(s) = 7 TeV,''
  arXiv:1202.6396 [hep-ex].
  %%CITATION = ARXIV:1202.6396;%%

%\cite{Chatrchyan:2012bf}
\bibitem{Chatrchyan:2012bf}
  S.~Chatrchyan {\it et al.}  [CMS Collaboration],
  ``Search for quark compositeness in dijet angular distributions from pp
  collisions at sqrt(s) = 7 TeV,''
  arXiv:1202.5535 [hep-ex].
  %%CITATION = ARXIV:1202.5535;%%

%\cite{Chatrchyan:2012ni}
\bibitem{Chatrchyan:2012ni}
  S.~Chatrchyan {\it et al.}  [CMS Collaboration],
  ``Jet momentum dependence of jet quenching in PbPb collisions at
  sqrt(sNN)=2.76 TeV,''
  arXiv:1202.5022 [nucl-ex].
  %%CITATION = ARXIV:1202.5022;%%

%\cite{Chatrchyan:2012dk}
\bibitem{Chatrchyan:2012dk}
  S.~Chatrchyan {\it et al.}  [CMS Collaboration],
  ``Inclusive b-jet production in pp collisions at sqrt(s)=7 TeV,''
  arXiv:1202.4617 [hep-ex].
  %%CITATION = ARXIV:1202.4617;%%

%\cite{Chatrchyan:2012ww}
\bibitem{Chatrchyan:2012ww}
  S.~Chatrchyan {\it et al.}  [CMS Collaboration],
  ``Search for the standard model Higgs boson decaying to bottom quarks in pp
  collisions at sqrt(s)=7 TeV,''
  arXiv:1202.4195 [hep-ex].
  %%CITATION = ARXIV:1202.4195;%%

%\cite{Chatrchyan:2012vp}
\bibitem{Chatrchyan:2012vp}
  S.~Chatrchyan {\it et al.}  [CMS Collaboration],
  ``Search for neutral Higgs bosons decaying to tau pairs in pp collisions at
  sqrt(s)=7 TeV,''
  arXiv:1202.4083 [hep-ex].
  %%CITATION = ARXIV:1202.4083;%%

%\cite{Chatrchyan:2012kc}
\bibitem{Chatrchyan:2012kc}
  S.~Chatrchyan {\it et al.}  [CMS Collaboration],
  ``Search for large extra dimensions in dimuon and dielectron events in pp
  collisions at sqrt(s) = 7 TeV,''
  arXiv:1202.3827 [hep-ex].
  %%CITATION = ARXIV:1202.3827;%%

%\cite{Chatrchyan:2012hr}
\bibitem{Chatrchyan:2012hr}
  S.~Chatrchyan {\it et al.}  [CMS Collaboration],
  ``Search for the standard model Higgs boson in the H to ZZ to ll tau tau
  decay channel in pp collisions at sqrt(s)=7 TeV,''
  arXiv:1202.3617 [hep-ex].
  %%CITATION = ARXIV:1202.3617;%%

%\cite{Chatrchyan:2012ft}
\bibitem{Chatrchyan:2012ft}
  S.~Chatrchyan {\it et al.}  [CMS Collaboration],
  ``Search for the standard model Higgs boson in the H to ZZ to 2l 2nu channel
  in pp collisions at sqrt(s) = 7 TeV,''
  arXiv:1202.3478 [hep-ex].
  %%CITATION = ARXIV:1202.3478;%%

%\cite{Chatrchyan:2012dg}
\bibitem{Chatrchyan:2012dg}
  S.~Chatrchyan {\it et al.}  [CMS Collaboration],
  ``Search for the standard model Higgs boson in the decay channel H to ZZ to 4
  leptons in pp collisions at sqrt(s) = 7 TeV,''
  arXiv:1202.1997 [hep-ex].
  %%CITATION = ARXIV:1202.1997;%%

%\cite{Chatrchyan:2012ty}
\bibitem{Chatrchyan:2012ty}
  S.~Chatrchyan {\it et al.}  [CMS Collaboration],
  ``Search for the standard model Higgs boson decaying to a W pair in the fully
  leptonic final state in pp collisions at sqrt(s) = 7 TeV,''
  arXiv:1202.1489 [hep-ex].
  %%CITATION = ARXIV:1202.1489;%%

%\cite{Chatrchyan:2012tx}
\bibitem{Chatrchyan:2012tx}
  S.~Chatrchyan {\it et al.}  [CMS Collaboration],
  ``Combined results of searches for the standard model Higgs boson in pp
  collisions at sqrt(s) = 7 TeV,''
  arXiv:1202.1488 [hep-ex].
  %%CITATION = ARXIV:1202.1488;%%

%\cite{Chatrchyan:2012tw}
\bibitem{Chatrchyan:2012tw}
  S.~Chatrchyan {\it et al.}  [CMS Collaboration],
  ``Search for the standard model Higgs boson decaying into two photons in pp
  collisions at sqrt(s)=7 TeV,''
  arXiv:1202.1487 [hep-ex].
  %%CITATION = ARXIV:1202.1487;%%

%\cite{Chatrchyan:2012sn}
\bibitem{Chatrchyan:2012sn}
  S.~Chatrchyan {\it et al.}  [CMS Collaboration],
  ``Search for a Higgs boson in the decay channel H to ZZ(*) to q qbar l-l+ in
  pp collisions at sqrt(s) = 7 TeV,''
  arXiv:1202.1416 [hep-ex].
  %%CITATION = ARXIV:1202.1416;%%

%\cite{Chatrchyan:2012gw}
\bibitem{Chatrchyan:2012gw}
  S.~Chatrchyan {\it et al.}  [CMS Collaboration],
  ``Measurement of the inclusive production cross sections for forward jets and
  for dijet events with one forward and one central jet in pp collisions at
  sqrt(s) = 7 TeV,''
  arXiv:1202.0704 [hep-ex].
  %%CITATION = ARXIV:1202.0704;%%

%\cite{Chatrchyan:2012np}
\bibitem{Chatrchyan:2012np}
  S.~Chatrchyan {\it et al.}  [CMS Collaboration],
  ``Suppression of non-prompt J/psi, prompt J/psi, and Y(1S) in PbPb collisions
  at sqrt(sNN) = 2.76 TeV,''
  arXiv:1201.5069 [nucl-ex].
  %%CITATION = ARXIV:1201.5069;%%

%\cite{Chatrchyan:2012wg}
\bibitem{Chatrchyan:2012wg}
  S.~Chatrchyan {\it et al.}  [CMS Collaboration],
  ``Centrality dependence of dihadron correlations and azimuthal anisotropy
  harmonics in PbPb collisions at sqrt(s[NN]) = 2.76 TeV,''
  arXiv:1201.3158 [nucl-ex].
  %%CITATION = ARXIV:1201.3158;%%

%\cite{Chatrchyan:2012vq}
\bibitem{Chatrchyan:2012vq}
  S.~Chatrchyan {\it et al.}  [CMS Collaboration],
  ``Measurement of isolated photon production in pp and PbPb collisions at
  sqrt(sNN) = 2.76 TeV,''
  arXiv:1201.3093 [nucl-ex].
  %%CITATION = ARXIV:1201.3093;%%

%\cite{Chatrchyan:2011hk}
\bibitem{Chatrchyan:2011hk}
  S.~Chatrchyan {\it et al.}  [CMS Collaboration],
  ``Measurement of the charge asymmetry in top-quark pair production in
  proton-proton collisions at sqrt(s) = 7 TeV,''
  Phys.\ Lett.\  B {\bf 709}, 28 (2012)
  [arXiv:1112.5100 [hep-ex]].
  %%CITATION = PHLTA,B709,28;%%

%\cite{Chatrchyan:2011fq}
\bibitem{Chatrchyan:2011fq}
  S.~Chatrchyan {\it et al.}  [CMS Collaboration],
  ``Search for signatures of extra dimensions in the diphoton mass spectrum at
  the Large Hadron Collider,''
  arXiv:1112.0688 [hep-ex].
  %%CITATION = ARXIV:1112.0688;%%

%\cite{Chatrchyan:2011ci}
\bibitem{Chatrchyan:2011ci}
  S.~Chatrchyan {\it et al.}  [CMS Collaboration],
  ``Exclusive photon-photon production of muon pairs in proton-proton
  collisions at sqrt(s) = 7 TeV,''
  JHEP {\bf 1201}, 052 (2012)
  [arXiv:1111.5536 [hep-ex]].
  %%CITATION = JHEPA,1201,052;%%

%\cite{Chatrchyan:2011kc}
\bibitem{Chatrchyan:2011kc}
  S.~Chatrchyan {\it et al.}  [CMS Collaboration],
  ``J/psi and psi(2S) production in pp collisions at sqrt(s) = 7 TeV,''
  JHEP {\bf 1202}, 011 (2012)
  [arXiv:1111.1557 [hep-ex]].
  %%CITATION = JHEPA,1202,011;%%

%\cite{Chatrchyan:2011qt}
\bibitem{Chatrchyan:2011qt}
  S.~Chatrchyan {\it et al.}  [CMS Collaboration],
  ``Measurement of the Production Cross Section for Pairs of Isolated Photons
  in pp collisions at sqrt(s) = 7 TeV,''
  JHEP {\bf 1201}, 133 (2012)
  [arXiv:1110.6461 [hep-ex]].
  %%CITATION = JHEPA,1201,133;%%

%\cite{Chatrchyan:2011wt}
\bibitem{Chatrchyan:2011wt}
  S.~Chatrchyan {\it et al.}  [CMS Collaboration],
  ``Measurement of the Rapidity and Transverse Momentum Distributions of Z
  Bosons in pp Collisions at sqrt(s)=7 TeV,''
  Phys.\ Rev.\  D {\bf 85}, 032002 (2012)
  [arXiv:1110.4973 [hep-ex]].
  %%CITATION = PHRVA,D85,032002;%%

%\cite{Chatrchyan:2011ne}
\bibitem{Chatrchyan:2011ne}
  S.~Chatrchyan {\it et al.}  [CMS Collaboration],
  ``Jet Production Rates in Association with W and Z Bosons in pp Collisions at
  sqrt(s) = 7 TeV,''
  JHEP {\bf 1201}, 010 (2012)
  [arXiv:1110.3226 [hep-ex]].
  %%CITATION = JHEPA,1201,010;%%

%\cite{Chatrchyan:2011ya}
\bibitem{Chatrchyan:2011ya}
  S.~Chatrchyan {\it et al.}  [CMS Collaboration],
  ``Measurement of the weak mixing angle with the Drell-Yan process in
  proton-proton collisions at the LHC,''
  Phys.\ Rev.\  D {\bf 84}, 112002 (2011)
  [arXiv:1110.2682 [hep-ex]].
  %%CITATION = PHRVA,D84,112002;%%

%\cite{Chatrchyan:2011wm}
\bibitem{Chatrchyan:2011wm}
  S.~Chatrchyan {\it et al.}  [CMS Collaboration],
  ``Measurement of energy flow at large pseudorapidities in pp collisions at
  sqrt(s) = 0.9 and 7 TeV,''
  JHEP {\bf 1111}, 148 (2011)
  [Erratum-ibid.\  {\bf 1202}, 055 (2012)]
  [arXiv:1110.0211 [hep-ex]].
  %%CITATION = JHEPA,1111,148;%%

%\cite{Chatrchyan:2011wb}
\bibitem{Chatrchyan:2011wb}
  S.~Chatrchyan {\it et al.}  [CMS Collaboration],
  ``Forward Energy Flow, Central Charged-Particle Multiplicities, and
  Pseudorapidity Gaps in W and Z Boson Events from pp Collisions at 7 TeV,''
  Eur.\ Phys.\ J.\  C {\bf 72}, 1839 (2012)
  [arXiv:1110.0181 [hep-ex]].
  %%CITATION = EPHJA,C72,1839;%%

%\cite{Chatrchyan:2011ay}
\bibitem{Chatrchyan:2011ay}
  S.~Chatrchyan {\it et al.}  [CMS Collaboration],
  ``Search for a Vector-like Quark with Charge 2/3 in t + Z Events from pp
  Collisions at sqrt(s) = 7 TeV,''
  Phys.\ Rev.\ Lett.\  {\bf 107}, 271802 (2011)
  [arXiv:1109.4985 [hep-ex]].
  %%CITATION = PRLTA,107,271802;%%

%\cite{Chatrchyan:2011zy}
\bibitem{Chatrchyan:2011zy}
  S.~Chatrchyan {\it et al.}  [CMS Collaboration],
  ``Search for Supersymmetry at the LHC in Events with Jets and Missing
  Transverse Energy,''
  Phys.\ Rev.\ Lett.\  {\bf 107}, 221804 (2011)
  [arXiv:1109.2352 [hep-ex]].
  %%CITATION = PRLTA,107,221804;%%

%\cite{Chatrchyan:2011yy}
\bibitem{Chatrchyan:2011yy}
  S.~Chatrchyan {\it et al.}  [CMS Collaboration],
  ``Measurement of the t t-bar Production Cross Section in pp Collisions at 7
  TeV in Lepton + Jets Events Using b-quark Jet Identification,''
  Phys.\ Rev.\  D {\bf 84}, 092004 (2011)
  [arXiv:1108.3773 [hep-ex]].
  %%CITATION = PHRVA,D84,092004;%%

%\cite{Chatrchyan:2011ue}
\bibitem{Chatrchyan:2011ue}
  S.~Chatrchyan {\it et al.}  [CMS Collaboration],
  ``Measurement of the Differential Cross Section for Isolated Prompt Photon
  Production in pp Collisions at 7 TeV,''
  Phys.\ Rev.\  D {\bf 84}, 052011 (2011)
  [arXiv:1108.2044 [hep-ex]].
  %%CITATION = PHRVA,D84,052011;%%

%\cite{Chatrchyan:2011cm}
\bibitem{Chatrchyan:2011cm}
  S.~Chatrchyan {\it et al.}  [CMS Collaboration],
  ``Measurement of the Drell-Yan Cross Section in pp Collisions at sqrt(s) = 7
  TeV,''
  JHEP {\bf 1110}, 007 (2011)
  [arXiv:1108.0566 [hep-ex]].
  %%CITATION = JHEPA,1110,007;%%

%\cite{Chatrchyan:2011kr}
\bibitem{Chatrchyan:2011kr}
  S.~Chatrchyan {\it et al.}  [CMS Collaboration],
  ``Search for B(s) and B to dimuon decays in pp collisions at 7 TeV,''
  Phys.\ Rev.\ Lett.\  {\bf 107}, 191802 (2011)
  [arXiv:1107.5834 [hep-ex]].
  %%CITATION = PRLTA,107,191802;%%

%\cite{Chatrchyan:2011pb}
\bibitem{Chatrchyan:2011pb}
  S.~Chatrchyan {\it et al.}  [CMS Collaboration],
  ``Dependence on pseudorapidity and centrality of charged hadron production in
  PbPb collisions at a nucleon-nucleon centre-of-mass energy of 2.76 TeV,''
  JHEP {\bf 1108}, 141 (2011)
  [arXiv:1107.4800 [nucl-ex]].
  %%CITATION = JHEPA,1108,141;%%

%\cite{CMS:2011aa}
\bibitem{CMS:2011aa}
  S.~Chatrchyan {\it et al.}  [CMS Collaboration],
  ``Measurement of the Inclusive W and Z Production Cross Sections in pp
  Collisions at sqrt(s) = 7 TeV,''
  JHEP {\bf 1110}, 132 (2011)
  [arXiv:1107.4789 [hep-ex]].
  %%CITATION = JHEPA,1110,132;%%

%\cite{Chatrchyan:2011ds}
\bibitem{Chatrchyan:2011ds}
  S.~Chatrchyan {\it et al.}  [CMS Collaboration],
  ``Determination of Jet Energy Calibration and Transverse Momentum Resolution
  in CMS,''
  JINST {\bf 6}, P11002 (2011)
  [arXiv:1107.4277 [physics.ins-det]].
  %%CITATION = JINST,6,P11002;%%

%\cite{Chatrchyan:2011cj}
\bibitem{Chatrchyan:2011cj}
  S.~Chatrchyan {\it et al.}  [CMS Collaboration],
  ``Search for Three-Jet Resonances in pp Collisions at sqrt(s) = 7 TeV,''
  Phys.\ Rev.\ Lett.\  {\bf 107}, 101801 (2011)
  [arXiv:1107.3084 [hep-ex]].
  %%CITATION = PRLTA,107,101801;%%

%\cite{Chatrchyan:2011qs}
\bibitem{Chatrchyan:2011qs}
  S.~Chatrchyan {\it et al.}  [CMS Collaboration],
  ``Search for supersymmetry in pp collisions at sqrt(s)=7 TeV in events with a
  single lepton, jets, and missing transverse momentum,''
  JHEP {\bf 1108}, 156 (2011)
  [arXiv:1107.1870 [hep-ex]].
  %%CITATION = JHEPA,1108,156;%%

%\cite{Chatrchyan:2011pg}
\bibitem{Chatrchyan:2011pg}
  S.~Chatrchyan {\it et al.}  [CMS Collaboration],
  ``A search for excited leptons in pp Collisions at sqrt(s) = 7 TeV,''
  Phys.\ Lett.\  B {\bf 704}, 143 (2011)
  [arXiv:1107.1773 [hep-ex]].
  %%CITATION = PHLTA,B704,143;%%

%\cite{Chatrchyan:2011ek}
\bibitem{Chatrchyan:2011ek}
  S.~Chatrchyan {\it et al.}  [CMS Collaboration],
  ``Inclusive search for squarks and gluinos in pp collisions at sqrt(s) = 7
  TeV,''
  Phys.\ Rev.\  D {\bf 85}, 012004 (2012)
  [arXiv:1107.1279 [hep-ex]].
  %%CITATION = PHRVA,D85,012004;%%

%\cite{Chatrchyan:2011id}
\bibitem{Chatrchyan:2011id}
  S.~Chatrchyan {\it et al.}  [CMS Collaboration],
  ``Measurement of the Underlying Event Activity at the LHC with sqrt(s)= 7 TeV
  and Comparison with sqrt(s) = 0.9 TeV,''
  JHEP {\bf 1109}, 109 (2011)
  [arXiv:1107.0330 [hep-ex]].
  %%CITATION = JHEPA,1109,109;%%

%\cite{Chatrchyan:2011tn}
\bibitem{Chatrchyan:2011tn}
  S.~Chatrchyan {\it et al.}  [CMS Collaboration],
  ``Missing transverse energy performance of the CMS detector,''
  JINST {\bf 6}, P09001 (2011)
  [arXiv:1106.5048 [physics.ins-det]].
  %%CITATION = JINST,6,P09001;%%

%\cite{Chatrchyan:2011nd}
\bibitem{Chatrchyan:2011nd}
  S.~Chatrchyan {\it et al.}  [CMS Collaboration],
  ``Search for New Physics with a Mono-Jet and Missing Transverse Energy in pp
  Collisions at sqrt(s) = 7 TeV,''
  Phys.\ Rev.\ Lett.\  {\bf 107}, 201804 (2011)
  [arXiv:1106.4775 [hep-ex]].
  %%CITATION = PRLTA,107,201804;%%

%\cite{Collaboration:2011ida}
\bibitem{Collaboration:2011ida}
  S.~Chatrchyan {\it et al.}  [CMS Collaboration],
  ``Search for New Physics with Jets and Missing Transverse Momentum in pp
  collisions at sqrt(s) = 7 TeV,''
  JHEP {\bf 1108}, 155 (2011)
  [arXiv:1106.4503 [hep-ex]].
  %%CITATION = JHEPA,1108,155;%%

%\cite{Chatrchyan:2011vh}
\bibitem{Chatrchyan:2011vh}
  S.~Chatrchyan {\it et al.}  [CMS Collaboration],
  ``Measurement of the Strange B Meson Production Cross Section with J/Psi phi
  Decays in pp Collisions at sqrt(s) = 7 TeV,''
  Phys.\ Rev.\  D {\bf 84}, 052008 (2011)
  [arXiv:1106.4048 [hep-ex]].
  %%CITATION = PHRVA,D84,052008;%%

%\cite{Chatrchyan:2011bj}
\bibitem{Chatrchyan:2011bj}
  S.~Chatrchyan {\it et al.}  [CMS Collaboration],
  ``Search for Supersymmetry in Events with b Jets and Missing Transverse
  Momentum at the LHC,''
  JHEP {\bf 1107}, 113 (2011)
  [arXiv:1106.3272 [hep-ex]].
  %%CITATION = JHEPA,1107,113;%%

%\cite{Chatrchyan:2011vp}
\bibitem{Chatrchyan:2011vp}
  S.~Chatrchyan {\it et al.}  [CMS Collaboration],
  ``Measurement of the t-channel single top quark production cross section in
  pp collisions at sqrt(s) = 7 TeV,''
  Phys.\ Rev.\ Lett.\  {\bf 107}, 091802 (2011)
  [arXiv:1106.3052 [hep-ex]].
  %%CITATION = PRLTA,107,091802;%%

%\cite{Chatrchyan:2011hr}
\bibitem{Chatrchyan:2011hr}
  S.~Chatrchyan {\it et al.}  [CMS Collaboration],
  ``Search for Light Resonances Decaying into Pairs of Muons as a Signal of New
  Physics,''
  JHEP {\bf 1107}, 098 (2011)
  [arXiv:1106.2375 [hep-ex]].
  %%CITATION = JHEPA,1107,098;%%

%\cite{Chatrchyan:2011dk}
\bibitem{Chatrchyan:2011dk}
  S.~Chatrchyan {\it et al.}  [CMS Collaboration],
  ``Search for Same-Sign Top-Quark Pair Production at sqrt(s) = 7 TeV and
  Limits on Flavour Changing Neutral Currents in the Top Sector,''
  JHEP {\bf 1108}, 005 (2011)
  [arXiv:1106.2142 [hep-ex]].
  %%CITATION = JHEPA,1108,005;%%

%\cite{Chatrchyan:2011ff}
\bibitem{Chatrchyan:2011ff}
  S.~Chatrchyan {\it et al.}  [CMS Collaboration],
  ``Search for Physics Beyond the Standard Model Using Multilepton Signatures
  in pp Collisions at sqrt(s)=7 TeV,''
  Phys.\ Lett.\  B {\bf 704}, 411 (2011)
  [arXiv:1106.0933 [hep-ex]].
  %%CITATION = PHLTA,B704,411;%%

%\cite{Chatrchyan:2011ew}
\bibitem{Chatrchyan:2011ew}
  S.~Chatrchyan {\it et al.}  [CMS Collaboration],
  ``Measurement of the Top-antitop Production Cross Section in pp Collisions at
  sqrt(s)=7 TeV using the Kinematic Properties of Events with Leptons and
  Jets,''
  Eur.\ Phys.\ J.\  C {\bf 71}, 1721 (2011)
  [arXiv:1106.0902 [hep-ex]].
  %%CITATION = EPHJA,C71,1721;%%

%\cite{Chatrchyan:2011wn}
\bibitem{Chatrchyan:2011wn}
  S.~Chatrchyan {\it et al.}  [CMS Collaboration],
  ``Measurement of the Ratio of the 3-jet to 2-jet Cross Sections in pp
  Collisions at sqrt(s) = 7 TeV,''
  Phys.\ Lett.\  B {\bf 702}, 336 (2011)
  [arXiv:1106.0647 [hep-ex]].
  %%CITATION = PHLTA,B702,336;%%

%\cite{CMS:2011ab}
\bibitem{CMS:2011ab}
  S.~Chatrchyan {\it et al.}  [CMS Collaboration],
  ``Measurement of the Inclusive Jet Cross Section in pp Collisions at sqrt(s)
  = 7 TeV,''
  Phys.\ Rev.\ Lett.\  {\bf 107}, 132001 (2011)
  [arXiv:1106.0208 [hep-ex]].
  %%CITATION = PRLTA,107,132001;%%

%\cite{Chatrchyan:2011nb}
\bibitem{Chatrchyan:2011nb}
  S.~Chatrchyan {\it et al.}  [CMS Collaboration],
  ``Measurement of the t t-bar production cross section and the top quark mass
  in the dilepton channel in pp collisions at sqrt(s) =7 TeV,''
  JHEP {\bf 1107}, 049 (2011)
  [arXiv:1105.5661 [hep-ex]].
  %%CITATION = JHEPA,1107,049;%%

%\cite{Chatrchyan:2011pe}
\bibitem{Chatrchyan:2011pe}
  S.~Chatrchyan {\it et al.}  [CMS Collaboration],
  ``Suppression of Upsilon excited states in PbPb collisions at a
  nucleon-nucleon centre-of-mass energy of 2.76 TeV,''
  Phys.\ Rev.\ Lett.\  {\bf 107}, 052302 (2011)
  [arXiv:1105.4894 [nucl-ex]].
  %%CITATION = PRLTA,107,052302;%%

%\cite{Chatrchyan:2011ah}
\bibitem{Chatrchyan:2011ah}
  S.~Chatrchyan {\it et al.}  [CMS Collaboration],
  ``Search for supersymmetry in events with a lepton, a photon, and large
  missing transverse energy in pp collisions at sqrt(s) = 7 TeV,''
  JHEP {\bf 1106}, 093 (2011)
  [arXiv:1105.3152 [hep-ex]].
  %%CITATION = JHEPA,1106,093;%%

%\cite{Chatrchyan:2011rr}
\bibitem{Chatrchyan:2011rr}
  S.~Chatrchyan {\it et al.}  [CMS Collaboration],
  ``Measurement of W-gamma and Z-gamma production in pp collisions at sqrt(s) =
  7 TeV,''
  Phys.\ Lett.\  B {\bf 701}, 535 (2011)
  [arXiv:1105.2758 [hep-ex]].
  %%CITATION = PHLTA,B701,535;%%

%\cite{Chatrchyan:2011eka}
\bibitem{Chatrchyan:2011eka}
  S.~Chatrchyan {\it et al.}  [CMS Collaboration],
  ``Long-range and short-range dihadron angular correlations in central PbPb
  collisions at a nucleon-nucleon center of mass energy of 2.76 TeV,''
  JHEP {\bf 1107}, 076 (2011)
  [arXiv:1105.2438 [nucl-ex]].
  %%CITATION = JHEPA,1107,076;%%

%\cite{Chatrchyan:2011ig}
\bibitem{Chatrchyan:2011ig}
  S.~Chatrchyan {\it et al.}  [CMS Collaboration],
  ``Measurement of the Polarization of W Bosons with Large Transverse Momenta
  in W+Jets Events at the LHC,''
  Phys.\ Rev.\ Lett.\  {\bf 107}, 021802 (2011)
  [arXiv:1104.3829 [hep-ex]].
  %%CITATION = PRLTA,107,021802;%%

%\cite{Chatrchyan:2011av}
\bibitem{Chatrchyan:2011av}
  S.~Chatrchyan {\it et al.}  [CMS Collaboration],
  ``Charged particle transverse momentum spectra in pp collisions at sqrt(s) =
  0.9 and 7 TeV,''
  JHEP {\bf 1108}, 086 (2011)
  [arXiv:1104.3547 [hep-ex]].
  %%CITATION = JHEPA,1108,086;%%

%\cite{Chatrchyan:2011wba}
\bibitem{Chatrchyan:2011wba}
  S.~Chatrchyan {\it et al.}  [CMS Collaboration],
  ``Search for new physics with same-sign isolated dilepton events with jets
  and missing transverse energy at the LHC,''
  JHEP {\bf 1106}, 077 (2011)
  [arXiv:1104.3168 [hep-ex]].
  %%CITATION = JHEPA,1106,077;%%

%\cite{Chatrchyan:2011pw}
\bibitem{Chatrchyan:2011pw}
  S.~Chatrchyan {\it et al.}  [CMS Collaboration],
  ``Measurement of the B0 production cross section in pp Collisions at sqrt(s)
  = 7 TeV,''
  Phys.\ Rev.\ Lett.\  {\bf 106}, 252001 (2011)
  [arXiv:1104.2892 [hep-ex]].
  %%CITATION = PRLTA,106,252001;%%

%\cite{Chatrchyan:2011qta}
\bibitem{Chatrchyan:2011qta}
  S.~Chatrchyan {\it et al.}  [CMS Collaboration Collaboration],
  ``Measurement of the differential dijet production cross section in
  proton-proton collisions at sqrt(s)=7 TeV,''
  Phys.\ Lett.\  B {\bf 700}, 187 (2011)
  [arXiv:1104.1693 [hep-ex]].
  %%CITATION = PHLTA,B700,187;%%

%\cite{Chatrchyan:2011nx}
\bibitem{Chatrchyan:2011nx}
  S.~Chatrchyan {\it et al.}  [CMS Collaboration],
  ``Search for Neutral MSSM Higgs Bosons Decaying to Tau Pairs in pp Collisions
  at sqrt(s)=7 TeV,''
  Phys.\ Rev.\ Lett.\  {\bf 106}, 231801 (2011)
  [arXiv:1104.1619 [hep-ex]].
%%CITATION = PRLTA,106,231801;%%

%\cite{Chatrchyan:2011nv}
\bibitem{Chatrchyan:2011nv}
  S.~Chatrchyan {\it et al.}  [CMS Collaboration],
  ``Measurement of the Inclusive Z Cross Section via Decays to Tau Pairs in pp
  Collisions at sqrt(s)=7 TeV,''
  JHEP {\bf 1108}, 117 (2011)
  [arXiv:1104.1617 [hep-ex]].
  %%CITATION = JHEPA,1108,117;%%

%\cite{Chatrchyan:2011jx}
\bibitem{Chatrchyan:2011jx}
  S.~Chatrchyan {\it et al.}  [CMS Collaboration],
  ``Search for Large Extra Dimensions in the Diphoton Final State at the Large
  Hadron Collider,''
  JHEP {\bf 1105}, 085 (2011)
  [arXiv:1103.4279 [hep-ex]].
  %%CITATION = JHEPA,1105,085;%%

%\cite{Chatrchyan:2011jz}
\bibitem{Chatrchyan:2011jz}
  S.~Chatrchyan {\it et al.}  [CMS Collaboration],
  ``Measurement of the lepton charge asymmetry in inclusive W production in pp
  collisions at sqrt(s) = 7 TeV,''
  JHEP {\bf 1104}, 050 (2011)
  [arXiv:1103.3470 [hep-ex]].
  %%CITATION = JHEPA,1104,050;%%

%\cite{Chatrchyan:2011bz}
\bibitem{Chatrchyan:2011bz}
  S.~Chatrchyan {\it et al.}  [CMS Collaboration],
  ``Search for Physics Beyond the Standard Model in Opposite-Sign Dilepton
  Events at sqrt(s) = 7 TeV,''
  JHEP {\bf 1106}, 026 (2011)
  [arXiv:1103.1348 [hep-ex]].
  %%CITATION = JHEPA,1106,026;%%

%\cite{Chatrchyan:2011wq}
\bibitem{Chatrchyan:2011wq}
  S.~Chatrchyan {\it et al.}  [CMS Collaboration],
  ``Search for Resonances in the Dilepton Mass Distribution in pp Collisions at
  sqrt(s) = 7 TeV,''
  JHEP {\bf 1105}, 093 (2011)
  [arXiv:1103.0981 [hep-ex]].
  %%CITATION = JHEPA,1105,093;%%

%\cite{Chatrchyan:2011wc}
\bibitem{Chatrchyan:2011wc}
  S.~Chatrchyan {\it et al.}  [CMS Collaboration],
  ``Search for Supersymmetry in pp Collisions at sqrt(s) = 7 TeV in Events with
  Two Photons and Missing Transverse Energy,''
  Phys.\ Rev.\ Lett.\  {\bf 106}, 211802 (2011)
  [arXiv:1103.0953 [hep-ex]].
  %%CITATION = PRLTA,106,211802;%%

%\cite{Chatrchyan:2011dx}
\bibitem{Chatrchyan:2011dx}
  S.~Chatrchyan {\it et al.}  [CMS Collaboration],
  ``Search for a W' boson decaying to a muon and a neutrino in pp collisions at
  sqrt(s) = 7 TeV,''
  Phys.\ Lett.\  B {\bf 701}, 160 (2011)
  [arXiv:1103.0030 [hep-ex]].
  %%CITATION = PHLTA,B701,160;%%

%\cite{Chatrchyan:2011ua}
\bibitem{Chatrchyan:2011ua}
  S.~Chatrchyan {\it et al.}  [CMS Collaboration],
  ``Study of Z boson production in PbPb collisions at nucleon-nucleon centre of
  mass energy = 2.76 TeV,''
  Phys.\ Rev.\ Lett.\  {\bf 106}, 212301 (2011)
  [arXiv:1102.5435 [nucl-ex]].
  %%CITATION = PRLTA,106,212301;%%

%\cite{Chatrchyan:2011tz}
\bibitem{Chatrchyan:2011tz}
  S.~Chatrchyan {\it et al.}  [CMS Collaboration],
  ``Measurement of WW Production and Search for the Higgs Boson in pp
  Collisions at sqrt(s) = 7 TeV,''
  Phys.\ Lett.\  B {\bf 699}, 25 (2011)
  [arXiv:1102.5429 [hep-ex]].
  %%CITATION = PHLTA,B699,25;%%

%\cite{Chatrchyan:2011em}
\bibitem{Chatrchyan:2011em}
  S.~Chatrchyan {\it et al.}  [CMS Collaboration],
  ``Search for a Heavy Bottom-like Quark in pp Collisions at sqrt(s) = 7 TeV,''
  Phys.\ Lett.\  B {\bf 701}, 204 (2011)
  [arXiv:1102.4746 [hep-ex]].
  %%CITATION = PHLTA,B701,204;%%

%\cite{Khachatryan:2011tm}
\bibitem{Khachatryan:2011tm}
  V.~Khachatryan {\it et al.}  [CMS Collaboration],
  ``Strange Particle Production in pp Collisions at sqrt(s) = 0.9 and 7 TeV,''
  JHEP {\bf 1105}, 064 (2011)
  [arXiv:1102.4282 [hep-ex]].
  %%CITATION = JHEPA,1105,064;%%

%\cite{Khachatryan:2011wq}
\bibitem{Khachatryan:2011wq}
  V.~Khachatryan {\it et al.}  [CMS Collaboration],
  ``Measurement of B anti-B Angular Correlations based on Secondary Vertex
  Reconstruction at sqrt(s)=7 TeV,''
  JHEP {\bf 1103}, 136 (2011)
  [arXiv:1102.3194 [hep-ex]].
  %%CITATION = JHEPA,1103,136;%%

%\cite{Khachatryan:2011as}
\bibitem{Khachatryan:2011as}
  V.~Khachatryan {\it et al.}  [CMS Collaboration],
  ``Measurement of Dijet Angular Distributions and Search for Quark
  Compositeness in pp Collisions at 7 TeV,''
  Phys.\ Rev.\ Lett.\  {\bf 106}, 201804 (2011)
  [arXiv:1102.2020 [hep-ex]].
  %%CITATION = PRLTA,106,201804;%%

%\cite{Chatrchyan:2011sx}
\bibitem{Chatrchyan:2011sx}
  S.~Chatrchyan {\it et al.}  [CMS Collaboration],
  ``Observation and studies of jet quenching in PbPb collisions at
  nucleon-nucleon center-of-mass energy = 2.76 TeV,''
  Phys.\ Rev.\  C {\bf 84}, 024906 (2011)
  [arXiv:1102.1957 [nucl-ex]].
  %%CITATION = PHRVA,C84,024906;%%

%\cite{Khachatryan:2011dx}
\bibitem{Khachatryan:2011dx}
  V.~Khachatryan {\it et al.}  [CMS Collaboration],
  ``First Measurement of Hadronic Event Shapes in pp Collisions at sqrt(s)=7
  TeV,''
  Phys.\ Lett.\  B {\bf 699}, 48 (2011)
  [arXiv:1102.0068 [hep-ex]].
  %%CITATION = PHLTA,B699,48;%%

%\cite{Khachatryan:2011zj}
\bibitem{Khachatryan:2011zj}
  V.~Khachatryan {\it et al.}  [CMS Collaboration],
  ``Dijet Azimuthal Decorrelations in pp Collisions at sqrt(s) = 7 TeV,''
  Phys.\ Rev.\ Lett.\  {\bf 106}, 122003 (2011)
  [arXiv:1101.5029 [hep-ex]].
  %%CITATION = PRLTA,106,122003;%%

%\cite{Khachatryan:2011hi}
\bibitem{Khachatryan:2011hi}
  V.~Khachatryan {\it et al.}  [CMS Collaboration],
  ``Measurement of Bose-Einstein Correlations in pp Collisions at sqrt(s)=0.9
  and 7 TeV,''
  JHEP {\bf 1105}, 029 (2011)
  [arXiv:1101.3518 [hep-ex]].
  %%CITATION = JHEPA,1105,029;%%

%\cite{Khachatryan:2011hf}
\bibitem{Khachatryan:2011hf}
  V.~Khachatryan {\it et al.}  [CMS Collaboration],
  ``Inclusive b-hadron production cross section with muons in pp collisions at
  sqrt(s) = 7 TeV,''
  JHEP {\bf 1103}, 090 (2011)
  [arXiv:1101.3512 [hep-ex]].
  %%CITATION = JHEPA,1103,090;%%

%\cite{Khachatryan:2011ts}
\bibitem{Khachatryan:2011ts}
  V.~Khachatryan {\it et al.}  [CMS Collaboration],
  ``Search for Heavy Stable Charged Particles in pp collisions at sqrt(s)=7
  TeV,''
  JHEP {\bf 1103}, 024 (2011)
  [arXiv:1101.1645 [hep-ex]].
  %%CITATION = JHEPA,1103,024;%%

%\cite{Khachatryan:2011tk}
\bibitem{Khachatryan:2011tk}
  V.~Khachatryan {\it et al.}  [CMS Collaboration],
  ``Search for Supersymmetry in pp Collisions at 7 TeV in Events with Jets and
  Missing Transverse Energy,''
  Phys.\ Lett.\  B {\bf 698}, 196 (2011)
  [arXiv:1101.1628 [hep-ex]].
  %%CITATION = PHLTA,B698,196;%%

%\cite{Khachatryan:2011mk}
\bibitem{Khachatryan:2011mk}
  V.~Khachatryan {\it et al.}  [CMS Collaboration],
  ``Measurement of the B+ Production Cross Section in pp Collisions at sqrt(s)
  = 7 TeV,''
  Phys.\ Rev.\ Lett.\  {\bf 106}, 112001 (2011)
  [arXiv:1101.0131 [hep-ex]].
  %%CITATION = PRLTA,106,112001;%%

%\cite{Khachatryan:2010fa}
\bibitem{Khachatryan:2010fa}
  V.~Khachatryan {\it et al.}  [CMS Collaboration],
  ``Search for a heavy gauge boson W' in the final state with an electron and
  large missing transverse energy in pp collisions at sqrt(s) = 7 TeV,''
  Phys.\ Lett.\  B {\bf 698}, 21 (2011)
  [arXiv:1012.5945 [hep-ex]].
  %%CITATION = PHLTA,B698,21;%%

%\cite{Khachatryan:2010zg}
\bibitem{Khachatryan:2010zg}
  V.~Khachatryan {\it et al.}  [CMS Collaboration],
  ``Measurement of the Inclusive Upsilon production cross section in pp
  collisions at sqrt(s)=7 TeV,''
  Phys.\ Rev.\  D {\bf 83}, 112004 (2011)
  [arXiv:1012.5545 [hep-ex]].
  %%CITATION = PHRVA,D83,112004;%%

%\cite{Khachatryan:2010mq}
\bibitem{Khachatryan:2010mq}
  V.~Khachatryan {\it et al.}  [CMS Collaboration],
  ``Search for Pair Production of Second-Generation Scalar Leptoquarks in pp
  Collisions at sqrt(s) = 7 TeV,''
  Phys.\ Rev.\ Lett.\  {\bf 106}, 201803 (2011)
  [arXiv:1012.4033 [hep-ex]].
  %%CITATION = PRLTA,106,201803;%%

%\cite{Khachatryan:2010wx}
\bibitem{Khachatryan:2010wx}
  V.~Khachatryan {\it et al.}  [CMS Collaboration],
  ``Search for Microscopic Black Hole Signatures at the Large Hadron
  Collider,''
  Phys.\ Lett.\  B {\bf 697}, 434 (2011)
  [arXiv:1012.3375 [hep-ex]].
  %%CITATION = PHLTA,B697,434;%%

%\cite{Khachatryan:2010xn}
\bibitem{Khachatryan:2010xn}
  V.~Khachatryan {\it et al.}  [CMS Collaboration],
  ``Measurements of Inclusive W and Z Cross Sections in pp Collisions at
  sqrt(s)=7 TeV,''
  JHEP {\bf 1101}, 080 (2011)
  [arXiv:1012.2466 [hep-ex]].
  %%CITATION = JHEPA,1101,080;%%

%\cite{Khachatryan:2010fm}
\bibitem{Khachatryan:2010fm}
  V.~Khachatryan {\it et al.}  [CMS Collaboration],
  ``Measurement of the Isolated Prompt Photon Production Cross Section in pp
  Collisions at sqrt(s) = 7 TeV,''
  Phys.\ Rev.\ Lett.\  {\bf 106}, 082001 (2011)
  [arXiv:1012.0799 [hep-ex]].
  %%CITATION = PRLTA,106,082001;%%

%\cite{Khachatryan:2010uf}
\bibitem{Khachatryan:2010uf}
  V.~Khachatryan {\it et al.}  [CMS Collaboration],
  ``Search for Stopped Gluinos in pp collisions at sqrt s = 7 TeV,''
  Phys.\ Rev.\ Lett.\  {\bf 106}, 011801 (2011)
  [arXiv:1011.5861 [hep-ex]].
  %%CITATION = PRLTA,106,011801;%%

%\cite{Khachatryan:2010nk}
\bibitem{Khachatryan:2010nk}
  V.~Khachatryan {\it et al.}  [CMS Collaboration],
  ``Charged particle multiplicities in pp interactions at sqrt(s) = 0.9, 2.36,
  and 7 TeV,''
  JHEP {\bf 1101}, 079 (2011)
  [arXiv:1011.5531 [hep-ex]].
  %%CITATION = JHEPA,1101,079;%%

%\cite{Khachatryan:2010yr}
\bibitem{Khachatryan:2010yr}
  V.~Khachatryan {\it et al.}  [CMS Collaboration],
  ``Prompt and non-prompt J/psi production in pp collisions at sqrt(s) = 7
  TeV,''
  Eur.\ Phys.\ J.\  C {\bf 71}, 1575 (2011)
  [arXiv:1011.4193 [hep-ex]].
  %%CITATION = EPHJA,C71,1575;%%

%\cite{Khachatryan:2010ez}
\bibitem{Khachatryan:2010ez}
  V.~Khachatryan {\it et al.}  [CMS Collaboration],
  ``First Measurement of the Cross Section for Top-Quark Pair Production in
  Proton-Proton Collisions at sqrt(s)=7 TeV,''
  Phys.\ Lett.\  B {\bf 695}, 424 (2011)
  [arXiv:1010.5994 [hep-ex]].
  %%CITATION = PHLTA,B695,424;%%

%\cite{Khachatryan:2010te}
\bibitem{Khachatryan:2010te}
  V.~Khachatryan {\it et al.}  [CMS Collaboration],
  ``Search for Quark Compositeness with the Dijet Centrality Ratio in pp
  Collisions at sqrt(s)=7 TeV,''
  Phys.\ Rev.\ Lett.\  {\bf 105}, 262001 (2010)
  [arXiv:1010.4439 [hep-ex]].
  %%CITATION = PRLTA,105,262001;%%

%\cite{Khachatryan:2010jd}
\bibitem{Khachatryan:2010jd}
  V.~Khachatryan {\it et al.}  [CMS Collaboration],
  ``Search for Dijet Resonances in 7 TeV pp Collisions at CMS,''
  Phys.\ Rev.\ Lett.\  {\bf 105}, 211801 (2010)
  [arXiv:1010.0203 [hep-ex]].
  %%CITATION = PRLTA,105,211801;%%

%\cite{Khachatryan:2010gv}
\bibitem{Khachatryan:2010gv}
  V.~Khachatryan {\it et al.}  [CMS Collaboration],
  ``Observation of Long-Range Near-Side Angular Correlations in Proton-Proton
  Collisions at the LHC,''
  JHEP {\bf 1009}, 091 (2010)
  [arXiv:1009.4122 [hep-ex]].
  %%CITATION = JHEPA,1009,091;%%

%\cite{Khachatryan:2010pw}
\bibitem{Khachatryan:2010pw}
  V.~Khachatryan {\it et al.}  [CMS Collaboration],
  ``CMS Tracking Performance Results from early LHC Operation,''
  Eur.\ Phys.\ J.\  C {\bf 70}, 1165 (2010)
  [arXiv:1007.1988 [physics.ins-det]].
  %%CITATION = EPHJA,C70,1165;%%

%\cite{Khachatryan:2010pv}
\bibitem{Khachatryan:2010pv}
  V.~Khachatryan {\it et al.}  [CMS Collaboration],
  ``Measurement of the Underlying Event Activity in Proton-Proton Collisions at
  0.9 TeV,''
  Eur.\ Phys.\ J.\  C {\bf 70}, 555 (2010)
  [arXiv:1006.2083 [hep-ex]].
  %%CITATION = EPHJA,C70,555;%%

%\cite{Khachatryan:2010mw}
\bibitem{Khachatryan:2010mw}
  V.~Khachatryan {\it et al.}  [CMS Collaboration],
  ``Measurement of the charge ratio of atmospheric muons with the CMS
  detector,''
  Phys.\ Lett.\  B {\bf 692}, 83 (2010)
  [arXiv:1005.5332 [hep-ex]].
  %%CITATION = PHLTA,B692,83;%%

%\cite{Khachatryan:2010us}
\bibitem{Khachatryan:2010us}
  V.~Khachatryan {\it et al.}  [CMS Collaboration],
  ``Transverse-momentum and pseudorapidity distributions of charged hadrons in
  pp collisions at sqrt(s) = 7 TeV,''
  Phys.\ Rev.\ Lett.\  {\bf 105}, 022002 (2010)
  [arXiv:1005.3299 [hep-ex]].
  %%CITATION = PRLTA,105,022002;%%

%\cite{Khachatryan:2010un}
\bibitem{Khachatryan:2010un}
  V.~Khachatryan {\it et al.}  [CMS Collaboration],
  ``Measurement of Bose-Einstein correlations with first CMS data,''
  Phys.\ Rev.\ Lett.\  {\bf 105}, 032001 (2010)
  [arXiv:1005.3294 [hep-ex]].
  %%CITATION = PRLTA,105,032001;%%

%\cite{Khachatryan:2010xs}
\bibitem{Khachatryan:2010xs}
  V.~Khachatryan {\it et al.}  [CMS Collaboration],
  ``Transverse momentum and pseudorapidity distributions of charged hadrons in
  pp collisions at sqrt(s) = 0.9 and 2.36 TeV,''
  JHEP {\bf 1002}, 041 (2010)
  [arXiv:1002.0621 [hep-ex]].
  %%CITATION = JHEPA,1002,041;%%

%\cite{:2009dv}
\bibitem{:2009dv}
  S.~Chatrchyan {\it et al.}  [CMS Collaboration],
  ``Commissioning and Performance of the CMS Pixel Tracker with Cosmic Ray
  Muons,''
  JINST {\bf 5}, T03007 (2010)
  [arXiv:0911.5434 [physics.ins-det]].
  %%CITATION = JINST,5,T03007;%%

%\cite{:2009dq}
\bibitem{:2009dq}
  S.~Chatrchyan {\it et al.}  [CMS Collaboration],
  ``Performance of the CMS Level-1 Trigger during Commissioning with Cosmic Ray
  Muons,''
  JINST {\bf 5}, T03002 (2010)
  [arXiv:0911.5422 [physics.ins-det]].
  %%CITATION = JINST,5,T03002;%%

%\cite{:2009dg}
\bibitem{:2009dg}
  S.~Chatrchyan {\it et al.}  [CMS Collaboration],
  ``Measurement of the Muon Stopping Power in Lead Tungstate,''
  JINST {\bf 5}, P03007 (2010)
  [arXiv:0911.5397 [physics.ins-det]].
  %%CITATION = JINST,5,P03007;%%

%\cite{:2009vs}
\bibitem{:2009vs}
  S.~Chatrchyan {\it et al.}  [CMS Collaboration],
  ``Commissioning and Performance of the CMS Silicon Strip Tracker with Cosmic
  Ray Muons,''
  JINST {\bf 5}, T03008 (2010)
  [arXiv:0911.4996 [physics.ins-det]].
  %%CITATION = JINST,5,T03008;%%

%\cite{:2009vq}
\bibitem{:2009vq}
  S.~Chatrchyan {\it et al.}  [CMS Collaboration],
  ``Performance of CMS Muon Reconstruction in Cosmic-Ray Events,''
  JINST {\bf 5}, T03022 (2010)
  [arXiv:0911.4994 [physics.ins-det]].
  %%CITATION = JINST,5,T03022;%%

%\cite{:2009vp}
\bibitem{:2009vp}
  S.~Chatrchyan {\it et al.}  [CMS Collaboration],
  ``Performance of the CMS Cathode Strip Chambers with Cosmic Rays,''
  JINST {\bf 5}, T03018 (2010)
  [arXiv:0911.4992 [physics.ins-det]].
  %%CITATION = JINST,5,T03018;%%

%\cite{:2009vn}
\bibitem{:2009vn}
  S.~Chatrchyan {\it et al.}  [CMS Collaboration],
  ``Performance of the CMS Hadron Calorimeter with Cosmic Ray Muons and LHC
  Beam Data,''
  JINST {\bf 5}, T03012 (2010)
  [arXiv:0911.4991 [physics.ins-det]].
  %%CITATION = JINST,5,T03012;%%

%\cite{Chatrchyan:2009im}
\bibitem{Chatrchyan:2009im}
  S.~Chatrchyan {\it et al.}  [CMS Collaboration],
  ``Fine Synchronization of the CMS Muon Drift-Tube Local Trigger using Cosmic
  Rays,''
  JINST {\bf 5}, T03004 (2010)
  [arXiv:0911.4904 [physics.ins-det]].
  %%CITATION = JINST,5,T03004;%%

%\cite{Chatrchyan:2009ih}
\bibitem{Chatrchyan:2009ih}
  S.~Chatrchyan {\it et al.}  [CMS Collaboration],
  ``Calibration of the CMS Drift Tube Chambers and Measurement of the Drift
  Velocity with Cosmic Rays,''
  JINST {\bf 5}, T03016 (2010)
  [arXiv:0911.4895 [physics.ins-det]].
  %%CITATION = JINST,5,T03016;%%

%\cite{Chatrchyan:2009ig}
\bibitem{Chatrchyan:2009ig}
  S.~Chatrchyan {\it et al.}  [CMS Collaboration],
  ``Performance of the CMS Drift-Tube Local Trigger with Cosmic Rays,''
  JINST {\bf 5}, T03003 (2010)
  [arXiv:0911.4893 [physics.ins-det]].
  %%CITATION = JINST,5,T03003;%%

%\cite{Chatrchyan:2009ic}
\bibitem{Chatrchyan:2009ic}
  S.~Chatrchyan {\it et al.}  [CMS Collaboration],
  ``Commissioning of the CMS High-Level Trigger with Cosmic Rays,''
  JINST {\bf 5}, T03005 (2010)
  [arXiv:0911.4889 [physics.ins-det]].
  %%CITATION = JINST,5,T03005;%%

%\cite{Chatrchyan:2009hw}
\bibitem{Chatrchyan:2009hw}
  S.~Chatrchyan {\it et al.}  [CMS Collaboration],
  ``Performance of CMS Hadron Calorimeter Timing and Synchronization using Test
  Beam, Cosmic Ray, and LHC Beam Data,''
  JINST {\bf 5}, T03013 (2010)
  [arXiv:0911.4877 [physics.ins-det]].
  %%CITATION = JINST,5,T03013;%%

%\cite{Chatrchyan:2009hg}
\bibitem{Chatrchyan:2009hg}
  S.~Chatrchyan {\it et al.}  [CMS Collaboration],
  ``Performance of the CMS Drift Tube Chambers with Cosmic Rays,''
  JINST {\bf 5}, T03015 (2010)
  [arXiv:0911.4855 [physics.ins-det]].
  %%CITATION = JINST,5,T03015;%%

%\cite{Chatrchyan:2009hb}
\bibitem{Chatrchyan:2009hb}
  S.~Chatrchyan {\it et al.}  [CMS Collaboration],
  ``Commissioning of the CMS Experiment and the Cosmic Run at Four Tesla,''
  JINST {\bf 5}, T03001 (2010)
  [arXiv:0911.4845 [physics.ins-det]].
  %%CITATION = JINST,5,T03001;%%

%\cite{:2009gz}
\bibitem{:2009gz}
  S.~Chatrchyan {\it et al.}  [CMS Collaboration],
  ``CMS Data Processing Workflows during an Extended Cosmic Ray Run,''
  JINST {\bf 5}, T03006 (2010)
  [arXiv:0911.4842 [physics.ins-det]].
  %%CITATION = JINST,5,T03006;%%

%\cite{:2009ft}
\bibitem{:2009ft}
  S.~Chatrchyan {\it et al.}  [CMS Collaboration],
  ``Aligning the CMS Muon Chambers with the Muon Alignment System during an
  Extended Cosmic Ray Run,''
  JINST {\bf 5}, T03019 (2010)
  [arXiv:0911.4770 [physics.ins-det]].
  %%CITATION = JINST,5,T03019;%%

%\cite{Chatrchyan:2009ks}
\bibitem{Chatrchyan:2009ks}
  S.~Chatrchyan {\it et al.}  [CMS Collaboration],
  ``Performance Study of the CMS Barrel Resistive Plate Chambers with Cosmic
  Rays,''
  JINST {\bf 5}, T03017 (2010)
  [arXiv:0911.4045 [physics.ins-det]].
  %%CITATION = JINST,5,T03017;%%

%\cite{:2009kr}
\bibitem{:2009kr}
  S.~Chatrchyan {\it et al.}  [CMS Collaboration],
  ``Time Reconstruction and Performance of the CMS Electromagnetic
  Calorimeter,''
  JINST {\bf 5}, T03011 (2010)
  [arXiv:0911.4044 [physics.ins-det]].
  %%CITATION = JINST,5,T03011;%%

%\cite{Chatrchyan:2009km}
\bibitem{Chatrchyan:2009km}
  S.~Chatrchyan {\it et al.}  [CMS Collaboration],
  ``Alignment of the CMS Muon System with Cosmic-Ray and Beam-Halo Muons,''
  JINST {\bf 5}, T03020 (2010)
  [arXiv:0911.4022 [physics.ins-det]].
  %%CITATION = JINST,5,T03020;%%

%\cite{Chatrchyan:2009si}
\bibitem{Chatrchyan:2009si}
  S.~Chatrchyan {\it et al.}  [CMS Collaboration],
  ``Precise Mapping of the Magnetic Field in the CMS Barrel Yoke using Cosmic
  Rays,''
  JINST {\bf 5}, T03021 (2010)
  [arXiv:0910.5530 [physics.ins-det]].
  %%CITATION = JINST,5,T03021;%%

%\cite{Chatrchyan:2009qm}
\bibitem{Chatrchyan:2009qm}
  S.~Chatrchyan {\it et al.}  [CMS Collaboration],
  ``Performance and Operation of the CMS Electromagnetic Calorimeter,''
  JINST {\bf 5}, T03010 (2010)
  [arXiv:0910.3423 [physics.ins-det]].
  %%CITATION = JINST,5,T03010;%%

%\cite{Chatrchyan:2009sr}
\bibitem{Chatrchyan:2009sr}
  S.~Chatrchyan {\it et al.}  [CMS Collaboration],
  ``Alignment of the CMS Silicon Tracker during Commissioning with Cosmic
  Rays,''
  JINST {\bf 5}, T03009 (2010)
  [arXiv:0910.2505 [physics.ins-det]].
  %%CITATION = JINST,5,T03009;%%

%\cite{:2008zzk}
\bibitem{:2008zzk}
  S.~Chatrchyan {\it et al.}  [CMS Collaboration],
  ``The CMS experiment at the CERN LHC,''
  JINST {\bf 3}, S08004 (2008).
  %%CITATION = JINST,3,S08004;%%


\end{thebibliography}

%\vspace{1cm}
\vfill{}
\hrulefill

% FILL IN THE FULL URL TO YOUR CV
\begin{center}
%{\footnotesize \href{http://www.ias.edu/spfeatures/einstein}{http://www.ias.edu/spfeatures/einstein} — Last updated: \today}
{\footnotesize Last updated: \today}
\end{center}


\end{document}
