%------------------------------------
% Dario Taraborelli
% Typesetting your academic CV in LaTeX
%
% URL: http://nitens.org/taraborelli/cvtex
% DISCLAIMER: This template is provided for free and without any guarantee 
% that it will correctly compile on your system if you have a non-standard  
% configuration.
%------------------------------------ 


% ! TEX TS-program = XeLaTeX -xdv2pdf
% ! TEX encoding = UTF-8 Unicode

\documentclass[10pt, a4paper]{article}
\usepackage{fontspec} 
\usepackage{xunicode} 
\usepackage{xltxtra}
% per le lettere accentate italiane sul Mac! :-)
%\usepackage[applemac]{inputenc} %with VIM
%FIXME%\usepackage[latin1]{inputenc} % with TeXShop
\usepackage[utf8]{inputenc} % with TeXShop

% DOCUMENT LAYOUT
\usepackage{geometry}
\geometry{a4paper, textwidth=5.5in, textheight=8.5in, marginparsep=7pt, marginparwidth=.6in}
\setlength\parindent{0in}

% ADDITIONAL SYMBOLS
%\usesymbols[mvs]

% FONTS
\defaultfontfeatures{Mapping=tex-text} % converts LaTeX specials (``quotes'' --- dashes etc.) to unicode
%\setromanfont [Ligatures={Common}, BoldFont={Fontin Bold}, ItalicFont={Fontin Italic}]{Fontin}
%\setromanfont [Ligatures={Common}, BoldFont={Linux Libertine Bold}, ItalicFont={Linux Libertine Italic}]{Linux Libertine}
%\setsansfont [Ligatures={Common}, BoldFont={Fontin Sans Bold}, ItalicFont={Fontin Sans Italic}]{Fontin Sans}
\setmonofont[Scale=0.8]{Monaco} 
% ---- CUSTOM AMPERSAND
\newcommand{\amper}{{\fontspec[Scale=.95]{Linux Libertine Bold}\selectfont\itshape\&}}
% ---- MARGIN YEARS
%\newcommand{\years}[1]{\marginpar{\scriptsize #1}}
%GOOD ONE%%\newcommand{\years}[1]{\marginpar{\footnotesize #1}}
%FIXME%\newcommand{\years}[1]{\makebox[0pt][l]{\hskip-1in{\footnotesize #1}}}
%\newcommand{\years}[1]{\makebox[0pt][l]{\hskip-1in{\footnotesize #1}}}
%\newcommand{\years}[1]{\makebox[0pt][l]{\hskip-1.2in{\footnotesize #1}}}
% ---- MARGIN YEARS
\usepackage{marginnote}
\newcommand{\years}[1]{\marginnote{\hskip-0.2in{\scriptsize #1}}}
\renewcommand*{\raggedrightmarginnote}{}
\setlength{\marginparsep}{7pt}
%\reversemarginpar


% HEADINGS
\usepackage{sectsty} 
\usepackage[normalem]{ulem} 
\sectionfont{\rmfamily\mdseries\upshape\Large}
\subsectionfont{\rmfamily\bfseries\upshape\normalsize} 
\subsubsectionfont{\rmfamily\mdseries\upshape\normalsize} 
%modifying section numbering
\def\thesubsection{\arabic{subsection}.\ } 

% PDF SETUP
% ---- FILL IN HERE THE DOC TITLE AND AUTHOR
%\usepackage[dvipdfm, bookmarks, colorlinks, breaklinks, pdftitle={Francesco Santanastasio - Curriculum Vitae},pdfauthor={Francesco Santanastasio}]{hyperref}
\usepackage[bookmarks, colorlinks, breaklinks, pdftitle={Francesco Santanastasio - Curriculum Vitae},pdfauthor={Francesco Santanastasio}]{hyperref}
%\hypersetup{linkcolor=blue,citecolor=blue,filecolor=black,urlcolor=blue} 
\hypersetup{linkcolor=cyan,citecolor=blue,filecolor=black,urlcolor=cyan} 

% Title of Bibliography
\renewcommand\refname{References \\ \normalsize \begin{center} \quad
    \quad \textsc{Publications (quoted in this document)}\end{center} }

% DOCUMENT
\begin{document}
\reversemarginpar
{\LARGE Francesco Santanastasio}\\[0.3cm]
%Institute address
%begin{tabular}{ l c l }
%\emph{Institute Address}: & & \\
%University of Maryland & & \\
%Department of Physics - John S. Toll Physics Building & &\\
%College Park  & & \\
%MD  \texttt{20742-4111} & \makebox[1.2cm]{} & Tel: \texttt{+1 301 405 3401} \\
%United States of America & & Fax: \texttt{+1 301 314 9525} \\
%\end{tabular}\\[1em]

% Work address
\begin{tabular}{ l c l }
\emph{Work Address}: & \makebox[1.cm]{}  & \\
Department of Physics, Sapienza Universit\`a di Roma & & Office Phone: \texttt{+39 06 4991 4388} \\
Edificio Marconi, floor 2, room 202-b & & e-mail: \\
Piazzale Aldo Moro, \texttt{2} 00185 Rome & & \href{mailto:francesco.santanastasio@cern.ch}{francesco.santanastasio@cern.ch} \\
Italy &  &  \href{mailto:francesco.santanastasio@roma1.infn.it}{francesco.santanastasio@roma1.infn.it} \\
\end{tabular}\\
%[1em]
%% Work address
%\begin{tabular}{ l c l }
%\emph{Work Address}: & & \\
%CERN (Conseil Europeen pour la Recherche Nucleaire) & \makebox[1.cm]{} & \\
%\texttt{CH-1211} Geneve  \texttt{23} & & Office Phone.: \texttt{+41 22 76 71 689}\\
%Building \texttt{40}, Room 1-\texttt{B01} & & Mobile Phone: \texttt{+41 76 22 86 127}\\ 
%Switzerland &  & email: \href{mailto:francesco.santanastasio@cern.ch}{francesco.santanastasio@cern.ch} 
%\end{tabular}\\[1em]
%\vfill
\begin{tabular}{ l c l }
& \makebox[1.cm]{}  & \\
Born:  9 February 1980 --- Rome, Italy & & \\
Website: \href{http://www.roma1.infn.it/~santanas/index.php}{http://www.roma1.infn.it/santanas/index.php}  & &  \\
\end{tabular}
%[1em]
%Born:  9 February 1980 --- Rome, Italy\\
%Nationality:  Italian\\
%Website: \href{http://www.roma1.infn.it/~santanas/index.php}{http://www.roma1.infn.it/santanas/index.php} \\
%\textsc{url}: \href{http://www.ias.edu/spfeatures/einstein/}{http://www.ias.edu/spfeatures/einstein/}\\ 
%%\hrule
\section*{Current Position}
\vspace{-5pt}
\hrule
\vspace{10pt}
\emph{Assistant Professor (Ricercatore Tempo Determinato di Tipo B)} \\
Department of Physics, Sapienza Universit\`a di Roma, Rome, Italy
%\emph{CERN Research Fellow in Experimental Particle Physics} \\
%PH Department, CERN, Geneve, Switzerland
%\emph{Post-Doctoral Research Assistant (Post-Doc) in Particle Physics} \\
%Department of Physics, University of Maryland, College Park, US
%\hrule
\section*{Research Groups}
\vspace{-5pt}
\hrule
\vspace{10pt}%
\years{2005 - today}\textbf{Member of the CMS collaboration at the
  CERN Large Hadron Collider (LHC)} \\
CMS is one of the two general purpose particle physics detectors
operated at LHC. \\[1em] 
\years{2014 - today}\textbf{Member of the i-MCP collaboration} \\
i-MCP is an R\&D project within INFN CSN5 aimed at use of micro-channel plates for
fast timing detection of single particles and electromagnetic showers
at collider experiments~\cite{Brianza:2015jia}.
%My research field is the experimental high-energy physics. Since 2006
%I am part of the CMS collaboration, one of the 4 experiments running at
%the CERN Large Hadron Collider (LHC) in Geneva.
%\section*{Areas of Specialization}
%\vspace{-5pt}
%\hrule
%\vspace{10pt}
%Particle Physics, Data Analysis in High Energy Physics, Physics beyond the Standard Model of Fundamental Interactions, Electromagnetic and Hadro%nic Calorimetry 
%%\section*{Areas of competence}
%%Software Development, IT, Particle detector physics
%\hrule
\section*{Employment History}
\vspace{-5pt}
\hrule
\vspace{10pt}
\noindent
%Assistant Professor Sapienza
\years{03/2014 - today}\textbf{Assistant Professor in Physics} \\
\textit{Sapienza Universit\`a di Roma}, Rome, Italy\\[1em]
%CERN Fellow
\years{09/2011 - 12/2013}\textbf{CERN Research Fellow in Experimental Particle Physics} \\
\textit{CERN}, Geneve, Switzerland\\
\textsc{Contact:} Dott. Maurizio Pierini (CERN) \\[1em]
%Post-Doc
\years{12/2007 - 08/2011}\textbf{Post-Doctoral Research Assistant  (Post-Doc) in Particle Physics} \\
\textit{University of Maryland}, College Park, MD, US\\
\textsc{Contact:} Prof. Sarah Eno (UMD) \\[1em]
%Based at \textit{CERN}, Geneve, Switzerland\\[1em]
% PhD
\years{11/2004 - 01/2008}\textbf{PhD in Physics}\\ %{\small (highest honors)}\\
\textit{``Search for Supersymmetry with Gauge-Mediated Breaking using high energy photons at CMS experiment''} \cite{Santanastasio:DOTTORATO}\\
\textsc{Advisors:} Prof. Shahram Rahatlou, Dott. Daniele del Re (Sapienza) \\
\textit{Sapienza Universit\`a di Roma}, Rome, Italy\\[1em]
% Laurea
\years{09/1998 - 05/2004}\textbf{\textit{Laurea} in Physics} {\small (highest honors)}\\
\textit{``Calibration of an electromagnetic calorimeter using the energy flow method''} \cite{Santanastasio:LAUREA}\\
\textsc{Advisors:} Dott. Riccardo Paramatti (INFN), Prof. Egidio Longo (Sapienza) \\
Mark: 110/110 \textit{``magna cum laude''}\\
\textit{Sapienza Universit\`a di Roma}, Rome, Italy
%EXAMPLE IN ENGLISH
%\years{2003-2006}\textbf{MSc (\textit{Laurea Magistrale}) in Nuclear and Subnuclear Physics} {\small (highest honours)}\\
%\textit{``Study of the ATLAS MDT Muon Chambers calibration constants with data from a testbeam''}\\
%\textsc{Advisors:} Prof. Toni Baroncelli (INFN), Prof. Filippo Ceradini (Roma Tre)\\
%Mark: 110/110 \textit{``magna cum laude''}\\
%\textit{\small expected date: August 2010}\\[1em]

\clearpage

\section*{Research Grants}
\vspace{-5pt}
\hrule
\vspace{10pt}
\years{08/2013}\textbf{Winner of Programma Per Giovani
  Ricercatori ``Rita Levi Montalcini''} \\ 
 \href{http://cervelli.cineca.it/ProgGiovRic/dm050813_683.pdf}{Risultati Bando 2010} \\
%Research project: \emph{Search for new physics beyond the Standard Model with the CMS detector at the CERN LHC}
\emph{Three-year grant of about 220000 euros for research in experimental high-energy
  physics with the CMS detector at the CERN LHC, of which 44000 euros for research costs.}

\section*{Invited Talks at Conferences}
\vspace{-5pt}
\hrule
\vspace{10pt}
\noindent

%ICHEP2014
\years{06-14/07/2016}\textbf{ICNFP2016} - International Conference on New
Frontiers in Physics, Kolymbari, Crete, \textit{``Searches for BSM
  physics in final states with jets and leptons+jets at CMS''}~\cite{Santanastasio:ICNFP2016} \\ [1em]
%ICHEP2014
\years{02-09/07/2014}\textbf{ICHEP2014} - International Conference on High Energy Physics, Valencia, Spain, \textit{``Search for heavy resonances decaying to bosons with the ATLAS and CMS detectors''}~\cite{Santanastasio:ICHEP2014} \\ [1em]  
%Workshop LHCpp 2013
\years{08-10/05/2013}\textbf{Workshop LHCpp 2013} - VI Workshop Italiano sulla Fisica p-p a LHC, Genova, Italy, \textit{``Hadronic Resonances at ATLAS and CMS''}~\cite{Santanastasio:LHCPP2013} \\ [1em]  
%INFN - Sezione di Genova \\
%Genova, Italy\\
%\textit{``Hadronic Resonances''}\\ 
%Invited talk to present a review on this topic, including results from ATLAS and CMS Collaborations\\ 
%PIC2012
\years{12-15/09/2012}\textbf{PIC2012} - XXXII Physics in Collision 2012, Strbske Pleso, Slovakia, \textit{``Exotic Phenomena Searches at Hadron Colliders''}~\cite{Santanastasio:2013iz} \\  [1em]
%Strbske Pleso, Slovakia\\
%\textit{``Exotic Phenomena Searches at Hadron Colliders''}\\ 
%Presentation in plenary session on behalf of the ATLAS and CMS Collaborations\\  
%Conference 
%MORIOND/EW
\years{13-20/03/2011}\textbf{Moriond/EW 2011} - Rencontres de Moriond on 
``EW Interactions and Unified Theories'', La Thuile, Italy, \textit{``Exotica Searches at CMS''}~\cite{MoriondEW2011} \\  [1em] 
%Presentation in plenary session on behalf of the CMS Collaboration\\
%Talk on behalf of the CMS Collaboration\\
%Conference proceedings will be published in date and journal still to be defined\\  [1em] 
%Conference 
%DIS2010
\years{19-23/04/2010}\textbf{DIS2010} - XVIII International Workshop on Deep-Inelastic Scattering and Related Subjects, Firenze, Italy, \textit{``Searches With Early Data At CMS''}~\cite{Santanastasio:2010zz} \\  [1em]  
%Presentation in parallel session on behalf of the CMS Collaboration\\
%IFAE2009
\years{15-17/04/2009}\textbf{IFAE2009} - Incontri di Fisica delle Alte Energie, VIII Edizione, Bari, Italy,\textit{``Prospects for Exotica Searches at ATLAS and CMS Experiments''}~\cite{Santanastasio:IFAE2009} 
%
%\section*{Talks in Plenary Meetings of the CMS Collaboration}
%\noindent
%First 7TeV Collisions
%\years{Mar 2010}\textbf{CMS General Weekly Meeting GWM11} - Preliminary results, plots, lessons 
%from the first 7 TeV collisions - CERN, Geneve, Switzerland \\
%\textit{``Report from HCAL/JetMET''}\\ 
%Presentation in plenary session on behalf of the HCAL and Jet/MET groups of the CMS experiment\\ [1em] 
%CMS Italia 2010
%\years{Jan 2010}\textbf{Riunione CMS Italia} - Pisa, Italy \\
%\textit{``Example of prompt analysis at CERN: Jet/MET commissioning with first collision data''}\\  [1em] 
%CRAFT2009
%\years{Sep 2009}\textbf{CMS Commissioning and Run Coordination meeting} - CRAFT (Cosmic Run At Four Tesla) 
%2009 Data Analysis Jamboree - CERN, Geneve, Switzerland \\
%\textit{``HCAL (Hadronic Calorimeter of CMS experiment) performance during CRAFT09''}\\ 
%Presentation in plenary session on behalf of the HCAL group of the CMS experiment\\ [1em] 
%CRAFT2008
%\years{Nov 2008}\textbf{CMS Commissioning and Run Coordination meeting} - CRAFT (Cosmic Run At Four Tesla) 
%2008 Data Analysis Jamboree - CERN, Geneve, Switzerland \\
%\textit{``HCAL (Hadronic Calorimeter of CMS experiment) achievements during CRAFT08''}\\ 
%Presentation in plenary session on behalf of the HCAL group of the CMS experiment

\section*{Review Committees}
\vspace{-5pt}
\hrule
\vspace{10pt}
\noindent
\years{06/2016 - today}Referee of {\it New Journal of
  Physics}~\href{http://iopscience.iop.org/journal/1367-2630}{link to
  online journal}~(2015 impact factor = 3.570) \\  [1em] 
\years{2011 - today}Member of the internal {\it``Analysis Review
  Committees''} for the scrutiny of public results of the CMS collaboration~\cite{Chatrchyan:2012rva,Chatrchyan:2013ual,CMS:2016pkl,Khachatryan:2015bma} 

\section*{Citation Report}
\vspace{-5pt}
\hrule
\vspace{10pt}
\noindent
%\years{2016}\textbf{90 publications:}
%\href{http://inspirehep.net/search?ln=en&ln=en&p=find+a+santanastasio+and+and+tc+p+and+jy+2016&of=hb&action_search=Search&sf=&so=a&rm=&rg=25&sc=0}{link
%to inspire}\\
%{\tiny
% http://inspirehep.net/search?ln=en&ln=en&p=find+a+santanastasio+and+and+tc+p+and+jy+2016&of=hb&action_search=Search&sf=&so=a&rm=&rg=25&sc=0}\\[1em]\normalsize
Last updated on 09/05/2017\\[1em]
Total number of publications:
591~\href{http://inspirehep.net/search?ln=en&ln=en&p=find+a+santanastasio+and+tc+p+and+date\%3C\%3D2017-05-09&of=hb&action_search=Search&sf=&so=a&rm=&rg=25&sc=0}{inspire
link}\\
Total number of publications from 01/01/2012 (last 5 years): 484~\href{http://inspirehep.net/search?ln=en&ln=en&p=find+a+santanastasio+and+tc+p+and+date\%3C\%3D2017-05-09+and+date\%3E\%3D2012-01-01&of=hb&action_search=Search&sf=&so=a&rm=&rg=25&sc=0}{inspire
link}\\
Total number of citations from 01/01/2007 (last 10 years): 60463~\href{http://inspirehep.net/search?ln=en&ln=en&p=find+a+santanastasio+and+tc+p+and+date\%3E\%3D2007-01-01+and+date\%3C\%3D2017-05-09&of=hcs&action_search=Search&sf=&so=a&rm=&rg=25&sc=0}{inspire
link}\\
Hirsch {\it h} index from 01/01/2007 (last 10 years): 118~\href{http://inspirehep.net/search?ln=en&ln=en&p=find+a+santanastasio+and+tc+p+and+date\%3E\%3D2007-01-01+and+date\%3C\%3D2017-05-09&of=hcs&action_search=Search&sf=&so=a&rm=&rg=25&sc=0}{inspire
link}\\
%% Aug 2013 - Rientro Cervelli
%\begin{tabular}{l l}
%Total number of publications: & ISI: 267, Inspire: 270 \\
%Total number of publications in the last 10 years: & ISI: 267, Inspire: 270 \\
%Total number of citations divided by N (6): & ISI: 5311/6$=$885\\
%(N = number of years since the first publication)  & \\
%h factor: & ISI: 32 \\
%Normalized h factor: (each pub contributes & ISI: 40 \\
%as \#citations * 4 / (2013 - year of pub +1) ) &  \\
%\end{tabular}

I'm author of 37 internal notes and 7 conference reports of the CMS experiment.

\clearpage

\section*{Teaching}
\vspace{-5pt}
\hrule
\vspace{10pt}
\years{03/2015 - today}\textbf{Corso di Fisica I, Sapienza, Corso di Laurea in Chimica Industriale} \\
\emph{Professor of mechanics and thermodynamics at chemistry majors}\\[1em]
\years{10/2005 - 02/2006}\textbf{Corso di Fisica I, Sapienza, Corso di Laurea in Matematica} \\
\emph{Teaching assistant of mechanics at mathematics majors}

\section*{Academic Responsibilities}
\vspace{-5pt}
\hrule
\vspace{10pt}
\years{10/2014 - today}\textbf{``Referente di Con.Scienze per la Facolt\`a
  di SMFN'' at Sapienza} \\
\emph{Organization of verification tests required for student
  registration at first year of University in science faculty}\\[1em]
\years{09/2015 - today}\textbf{Member of ``Commissione Didattica del
CdL in Chimica Industriale'' at Sapienza}
\emph{Coordination and rationalization of academic activities and
teaching programs in undergraduate courses for chemistry majors}

\section*{Student Supervision}
\vspace{-5pt}
\hrule
\vspace{10pt}
I have been the thesis supervisor or co-supervisor of the following
students at Sapienza:\\[1em]
\years{2016-2017}{\bf Alfonso Tanga (undergraduate)},\textit{``Search for new resonances
  in final states with jets in proton-proton collisions at $\sqrt{s}=13$~TeV''}, Thesis ongoing \\ [1em]  
\years{2015-2017}{\bf Simone Gelli (PhD)},\textit{``Search for new
  particles decaying into diboson final states in proton-proton collisions at $\sqrt{s}=13$~TeV using jet substructure techniques''}, Thesis ongoing \\ [1em]  
\years{2014-2015} {\bf Giulia D'Imperio (PhD)}, \textit{``Search for narrow resonances
  in dijet final states at the LHC with $\sqrt{s}=13$~TeV''
} \href{http://www.roma1.infn.it/cms/tesiPHD/dimperio.pdf}{http://www.roma1.infn.it/cms/tesiPHD/dimperio.pdf} ~\cite{Khachatryan:2015dcf}\\

During my convenerships of analysis groups and postdoc appointments in
the CMS experiment, I supervised the research activity of the following
students from different institutions:\\[1em]
\years{2015-2017} {\bf Federico Preiato (PhD)}, Sapienza University of
Rome, Italy\\ \textit{"Search for heavy resonances in the dijet final state and jet energy calibration"}~\cite{Sirunyan:2016iap,AN-16-344} \\ [1em]
\years{2012-2015} {\bf Emine Gurpinar (PhD)}, Cukurova University, Turkey\\ \textit{"Searches for heavy resonances decaying to pair of jets at CMS"}~\cite{Khachatryan:2015sja,CMS:2012yf} \\ [1em]
\years{2012-2014} {\bf Shuai Liu (PhD)}, Peking University, China\\ \textit{"Searches for beyond Standard Model WW$\rightarrow \ell\nu qq$ resonances at CMS"}~\cite{Khachatryan:2014gha} \\ [1em]
\years{2012-2014} {\bf Edmund Berry (PhD)}, Princeton University, USA \\ \textit{"Searches for first-generation leptoquarks at CMS with $\sqrt{s}=7$ and  8 TeV data"}~\cite{Khachatryan:2015vaa,Chatrchyan:2012vza} \\ [1em]
\years{2010-2011} {\bf Dinko Ferencek (PhD)}, University of Maryland, USA\\ \textit{"Searches for First-Generation Leptoquarks at CMS with early $\sqrt{s}=7$ TeV data"}~\cite{Chatrchyan:2011ar} \\ [1em]
\years{2008-2009} {\bf Elizabeth Twedt Lockner (PhD)}, University of Maryland, USA\\ \textit{"Feasibility study of First-Generation Leptoquark searches at CMS"}~\cite{EXO-08-010}
 
\clearpage

\section*{Scientific Coordination in the CMS experiment}
\vspace{-5pt}
\hrule
\vspace{10pt}
%%% Jets+X
\years{09/2016 - today}\textbf{Coordination of the \emph{CMS Exotica
    Jets+X Working Group}} \\
I started my 2-year mandate on September 1st, 2016.
This analysis group works on searches for new physics beyond the
Standard Model in final states containing jets. The group, 
constituted by more than 50 physicists working in universities and
research institutions from all the world, performs
about 10 physics analyses in this final state. The results of these
searches are expected to be published in 2017. \\ [1em]
%%% Dijet
\years{09/2014 - 09/2016}\textbf{Coordination of the \emph{Dijet Resonance
  Team} of the CMS experiment}\\
This analysis team works on searches for new massive resonances at the TeV
scale decaying into a pair of jets using the dijet mass spectrum. It is
constituted by almost 20 physicists from several institutions from
all the world. This group produced two high-impact papers using proton-proton
collisions at $\sqrt{s}=13$~TeV~\cite{Sirunyan:2016iap,Khachatryan:2015dcf}, 
including the first published limits in the dijet final state on the
mass of a mediator of the interaction between dark
matter and standard model particles. \\[1em]
%This group produced the first paper at LHC on a
%search for new physics using proton-proton collisions at
%$\sqrt{s}=13$~TeV~\cite{Khachatryan:2015dcf}. Using the data collected
%in 2016, the group produced a paper with the first published limits on 
%the mass of a mediator between dark matter and standard model particles~\cite{}.
%%% Leptons+Jets
\years{01/2013 - 01/2015}\textbf{Coordination of the \emph{CMS Exotica
    Leptons+Jets Working Group}}\\
This analysis group works on searches for new physics beyond the
Standard Model in final states containing leptons and jets. The group, 
constituted by more than 50 physicists working in universities and
research institutions from all the world, performed about 15 physics
analyses in this final state. During my convenership, the group
produced 3 publications~\cite{Khachatryan:2014ura,Khachatryan:2014dka,Khachatryan:2014gha}
and 7 preliminary results that were then published or submitted for
publication in 2015
(including~\cite{Khachatryan:2016yji,Khachatryan:2015ywa,Khachatryan:2015vaa}). \\[1em]
%%% DDT
\years{03/2012 - 03/2013}\textbf{Coordination of the \emph{Dataset Definition
  Team} of the CMS experiment}\\ 
Definition of the trigger requirements forming the data streams used for physics analysis and detector
calibration. This responsibility also consists in the design and
implementation of a novel strategy for {\it data parking} and {\it
  data scouting}~\cite{CMS-DP-2012-022}. LHC searches for new particles with sub-TeV masses are hindered by the
high thresholds required to limit trigger rates. The new technique of data scouting, based on online event
reconstruction and small record sizes, allows to lower trigger thresholds and
extend searches for new particles into hitherto unexplored regions.
 \\ [1em]
%%% HCAL PFG 
\years{09/2008 - 09/2010}\textbf{Coordination of the \emph{Prompt
    Feedback Group} of the hadronic calorimeter of the CMS experiment (HCAL)}\\
Monitoring and data analysis concerning problems in the detector
during cosmic-ray data-taking. The group was formed by almost 10
students and postdocs working on HCAL detector studies in early data taking periods.

\section*{Highlights of Research Activities in the CMS experiment}
\vspace{-5pt}
\hrule
\vspace{10pt}
% WHEN YOU ADD A NEW BULLET REMEMBER TO MOVE THE CLEARPAGE 
% AT THE END OF THE PAGE 
\noindent
% EXOTICA
% EXO DIJET COORDINATION
%\years{Sep 2014 - today}Coordination of the {\it``Dijet Resonance Team"} of the CMS experiment: analysis group working on searches for massive resonances decaying into a pair of jets using the dijet mass spectrum. 
%The group is constituted by more than 20 physicists from about 10 institutions from all the world.  
%The target is to coordinate the activity of students and postdocs to prepare the analysis and publish the results in Summer 2015 with 
%the first LHC proton-proton collisions at $\sqrt{s}=13$~TeV. Thanks to the increase in the collider energy, the sensitivity to new physics above the TeV scale will be extended after few weeks of data taking compared to the LHC run at $\sqrt{s}=8$~TeV. 
%\\ [1em] 
% EXO L+J COORDINATION
%\years{Jan 2013 - Jan 2015}Coordination of the {\it``Exotica Lepton+Jets Working Group"} of the CMS experiment: analysis group working on searches for new physics beyond the Standard Model in final states containing leptons and jets. The group, constituted by almost 50 physicists working in universities and research institutions from all the world, performs about 15 physics analyses in this final state. Since the beginning of my mandate, 3 publications and 3 preliminary analysis results
%have been delivered. The remaining analyses are currently in the final steps of the review within the CMS collaboration. In the first months of 2015, other $\sim$10 publications in high impact scientific journals are expected.\\  [1em] 
%\years{Dec 2007 - today}Actively involved in the research activities of the exotic physics group (Exotica) of the CMS experiment, looking for evidence of new physics beyond the Standard Model of fundamental interactions [see ``Talks at Conferences'']. \\ [1em]
% DIJET SEARCHES
\years{09/2011 - today}Leading author of searches for resonances at
TeV mass scale decaying into a pair of jets (dijet) using the dijet mass spectrum 
%with 4.7 fb$^{-1}$ of data collected in 2011 
in proton-proton collisions at $\sqrt{s}=$7 TeV~\cite{CMS:2012yf,AN-12-012}, 8
TeV~\cite{Chatrchyan:2013qha,AN-12-229,Khachatryan:2015sja,AN-12-455},
and 13 TeV~\cite{Sirunyan:2016iap,AN-16-202,Khachatryan:2015dcf,AN-15-063,AN-15-175} with the CMS
detector. The search for new dijet resonances is among the most important ones at LHC because any hypothetical new
particle that might be produced originates from the colliding protons
and therefore it must couple to quarks and/or gluons. This search is
sensitive to the presence of a hypothetical, massive  mediator of the interaction between dark matter and
standard model quarks. These papers received in total more than 500 citations.\\ [1em] 
%% DIJET SCOUTING
\years{09/2011 - today}
Proponent of a novel trigger, data acquisition, and analysis
strategy to recover sensitivity to new dijet resonances at dijet
masses below 1 TeV~\cite{CMS-DP-2012-022} ({\it data scouting}).
Leading author of searches for dijet resonances using the data
scouting technique at $\sqrt{s}=7$~TeV~\cite{CMS-PAS-EXO-11-094} 
and $\sqrt{s}=8$~TeV~\cite{Khachatryan:2016ecr,AN-14-104}. Supervision
of PhD student at Sapienza working on calibration of online
reconstructed jets for the scouting analysis
at~$\sqrt{s}=13$~TeV~\cite{Sirunyan:2016iap,AN-16-202}. 
It is important to extend the dijet search in the mass region below 1 TeV 
in order to probe hypothetical hadronic resonances with small
couplings to quarks and gluons that similar searches performed at
previous colliders could not find yet. \\ [1em] 
%Update of the dijet analysis with the first 4 fb$^{-1}$ of data collected in 2012 at $\sqrt{s}=$8 TeV~\cite{Chatrchyan:2013qha,AN-12-229}, as well as with the full 2012 data sample of 19.6 fb$^{-1}$~\cite{CMS-PAS-EXO-12-059,AN-12-455}.
%\years{Sep 2011 - Jul 2012}Leading author of a novel trigger, data acquisition, and analysis strategy to recover sensitivity to new resonances decaying into a pair of dijets at dijet masses below 1 TeV~\cite{CMS-PAS-EXO-11-094} ({\it data scouting}). \\ [1em] 
%%
%\years{Mar 2012 - today}Supervising a PhD student from FNAL / Cukurova University (Turkey) for the update of the dijet analysis with the first 4 fb$^{-1}$ of data collected in 2012 at $\sqrt{s}=$8 TeV~\cite{Chatrchyan:2013qha,AN-12-229}, as well as with the full 2012 data sample of 19.6 fb$^{-1}$~\cite{CMS-PAS-EXO-12-059,AN-12-455}. \\ [1em] 
%The analysis contains improvements compared to 
%a previous published CMS dijet search~\cite{Chatrchyan:2011ns} and uses the entire 4.7 fb$^{-1}$  data sample collected in 2011, extending the exclusion on the resonance mass by 10\% to 30\% 
%depending on the resonance type.
%%
%% JETS
\years{01/2015 - today}Study of jet energy calibration using $\gamma$+jet events~\cite{AN-16-344} and
study jet substructure observables using energetic W bosons in events with
top quark pair production~\cite{AN-17-051}. The detailed understanding of both 
the energy scale and resolution of the jets is of crucial importance
for many physics analyses. Jet substructure observables are important 
in several searches for new physics to identify energetic W or Z bosons decaying to 
a pair of collimated quarks and reconstructed as single massive jets
in the detector. \\ [1em] 
%%
% WW/WZ/ZZ RESONANCES
\years{12/2011 - 09/2014}Primary author of searches for heavy
resonances decaying to WW / ZZ / WZ / qW / qZ in semi-leptonic $\ell\nu q\bar{q}'$ / $\ell\ell q\bar{q}$ \cite{Khachatryan:2014gha,AN-13-045,AN-13-040} and fully hadronic~\cite{Khachatryan:2014hpa,AN-12-393,Chatrchyan:2012yxa,AN-11-524} final states at CMS using jet substructure techniques to identify the hadronic decays of boosted vector bosons. The investigation of the di-boson 
production at high center-of-mass energy is a necessary ingredient for the understanding of the origin of the electroweak symmetry breaking and to disentangle the nature of the Higgs boson. \\ [1em] 
%The study of the diboson production at high center-of-mass energy is important to understand the mechanism of the electroweak symmetry breaking, being complementary to the direct Higgs boson searches. \\ [1em] 
%Supervising a PhD student from Peking University working on the $\ell\nu q\bar{q}'$  channel. 
%\years{Dec 2011 - today} Search for heavy qW/qZ/WW/WZ/ZZ resonances in the 
%W/Z-tagged dijet mass spectrum at CMS using jet substructure techniques 
%to identify the hadronic decays of boosted vector bosons~\cite{AN-11-524}.  \\ [1em] 
% LEPTOQUARKS 
%\years{Sep 2011 - today}Primary author of the $LQ$ analyses updates
%with 4.7 fb$^{-1}$ of data collected in 2011 
%at $\sqrt{s}=$7 TeV~\cite{Chatrchyan:2012vza,AN-11-492} and 8 TeV [analysis in progress]. \\ [1em] 
%as well as with the 19.6 fb$^{-1}$ of data collected in 2012 at $\sqrt{s}=$8 TeV [analysis in progress].\\ [1em] 
\years{12/2007 - 12/2014}Leading author of searches for pair production of first generation scalar Leptoquarks ($LQ$) in the decay channels $LQ \overline{LQ} \rightarrow ee qq$~\cite{Khachatryan:2010mp,EXO-08-010,AN-2010-230,AN-2008-070} and $LQ\overline{LQ} \rightarrow e\nu qq$ \cite{Chatrchyan:2011ar,AN-2010-361} with the CMS detector using the first 36 pb$^{-1}$ of LHC collisions at $\sqrt{s}=$7 TeV. Primary author of the $LQ$ analysis updates with full dataset at $\sqrt{s}=$7 TeV~\cite{Chatrchyan:2012vza,AN-11-492} and author of the 8 TeV analysis~\cite{Khachatryan:2015vaa,AN-2013-109}. These searches are sensitive to signals from Supersymmetry models with R-Parity violation that foresee stop$\rightarrow eq$ decays. \\ [1em] 
%ARC
%\years{2011 - today} Member of the internal {\it``Analysis Review
%  Committee''} for the scrutiny of public CMS results: top cross
%section measurements in all hadronic decay
%channel~\cite{CMS-PAS-TOP-11-007}; search for Randall-Sundrum
%gravitons decaying into a jet plus missing transverse 
%energy final state~\cite{Chatrchyan:2012rva,CMS-PAS-EXO-11-061} at $\sqrt{s}=$7 TeV. \\ [1em] 
%TOP publication: Chatrchyan:2013ual
%Prompt analysis during the very first LHC collisions at $\sqrt{s}=$7~TeV 
%[see ``Talks in Plenary Meetings of the CMS Collaboration`` 
%$\rightarrow$  presentation on behalf of the HCAL and Jet/MET groups] \\ [1em]
% MET
\years{11/2009 - 09/2010}Commissioning of missing transverse energy (MET) 
reconstructed with the first proton-proton ({\it pp}) collisions at
$\sqrt{s}=$0.9 and 2.36 TeV collected by the CMS experiment \cite{JME-10-002,AN-2010-219,AN-2010-029}. \\ [1em]
%HF PMT NOISE
\years{11/2009 - 09/2010}Development and implementation of algorithms for the identification 
of anomalous, beam-induced signals in the CMS hadronic forward calorimeter at $\sqrt{s}=$0.9, 2.36 and 7 TeV \cite{DN-2010-008,Chatrchyan:2009hy}. \\ [1em]
% TEST BEAM HCAL 2009
\years{06/2009 - 07/2009}Commissioning and calibration of the {\it ``delay wire chambers''} used for beam position measurements during the test beam of the hadronic calorimeter (HCAL) of the CMS experiment in 2009~\cite{Chatrchyan:2010zz}. \\ [1em]
% HCAL COMMISSIONING
\years{01/2008 - 07/2008}Commissioning of the hadronic calorimeter of the CMS experiment: expert ``on-call`` for trigger and data acquisition operations during early cosmic-ray data-taking.\\ [1em]
%GMSB (TESI DOTTORATO)
\years{12/2006 - 12/2007}Feasibility study of the search for Gauge Mediated Supersymmetry Breaking models in the prompt photon decay channel $pp \rightarrow \tilde{\chi}_1^0 \tilde{\chi}_1^0 + X \rightarrow \tilde{G} \tilde{G} \gamma \gamma + X$ 
\cite{Santanastasio:DOTTORATO}, with simulation of the CMS detector. \\ [1em]
%TEST BEAM ECAL+HCAL 2006
\years{07/2006 - 11/2006}Monitoring of the high voltage system of the CMS electromagnetic calorimeter (ECAL) and data-taking shifts in the combined ECAL+HCAL test beam in 2006~\cite{Abdullin:2009zz}.\\ [1em]
%ECAL HV
\years{03/2006 - 11/2006}Analysis and test of stability of ECAL high voltage system ~\cite{Bartoloni:2007hx}. \\ [1em]
%including development of software tools for data analysis~\cite{Bartoloni:2007hx}. \\ [1em]
%%
%pi0 CALIBRATION
\years{10/2005 - 10/2006}Feasibility study of the calibration of the CMS ECAL
using $\pi^0 \rightarrow \gamma\gamma$ decays~\cite{Adzic:2008zza,DN-2007-013,IN-2006-050}. 
\\[1em]
%LAUREA
%\years{Jan 2003 - May 2004}Study and implementation of the energy flow technique applied to the calibration 
%of the electromagnetic calorimeter of the L3 experiment at LEP (CERN) \cite{Santanastasio:LAUREA}.%[1em]
%\clearpage


%Exercises of classic mechanics for mathematics majors

%\section*{Physics Schools}
%\noindent
% FERMILAB 2008
%\years{12-22.08.2008}\textbf{2008 Joint CERN-Fermilab Hadron Collider Physics Summer School} \\ 
%Fermilab, Batavia, Illinois, US \\ [1em]
% LECCE 2005
%\years{09-14.06.2005}\textbf{Italo-Hellenic School of Physics 2005}  \\ 
%Martignano, Lecce, Italy \\
%{\it ``The Physics of LHC: theoretical tools and experimental challenges''}

%\section*{Languages}
%\begin{tabular}{l c l}
%\textit{Italian} (native speaker) & \makebox[4em]{} & \textit{English} (fluent)\\
%\textit{Italian} (native speaker) & \makebox[4em]{} & \textit{French} (fluent)\\
%\textit{English} (fluent) & &\textit{German} (basic)\\
%\end{tabular}

%\clearpage

%%%%%%%%%%%%%%%%%%%%%%%%%%%
%%% Service work
%%%%%%%%%%%%%%%%%%%%%%%%%%%

%\section*{Service work in Experiments and Collaborations}
%\subsection*{ATLAS Experiment}
%\noindent
%\textbf{Data Analysis: Supersymmetry Working Group} Working on data analysis, on exploring and implementing analysis strategies and on data files production\\
%\textbf{Development \& Upgrade} Working in the DAQ group, on the upgrade of the configuration DB system\\
%\textbf{Detector Operation} Shifter in the control room, at the Muon System, DAQ and Run Control desks\\
%\textbf{Software Framework} Taking part in code testing, and shifter for the build test system (RTT)\\
%\textbf{Documentation} Responsible person for a part of the documentation of the ATLAS data-format\\
%\textbf{Public Relations} Official ATLAS Guide, escorting VIP visits to the ATLAS cavern\\


%\clearpage

%%%%%%%%%%%%%%%%%%%%%%%%%%%
%%% Publications & Talks
%%%%%%%%%%%%%%%%%%%%%%%%%%%

\begin{thebibliography}{599}

\bibitem{Sirunyan:2016iap} 
  [CMS Collaboration],
  ``Search for dijet resonances in proton-proton collisions at sqrt(s) = 13 TeV and constraints on dark matter and other models,''\\
  Submitted to Phys.\ Lett.\ B on 14/11/2016, arXiv:1611.03568 [hep-ex].\\
  I'm co-coordinator of the analysis team that produced this CMS paper.
 
\bibitem{Khachatryan:2015dcf} 
  [CMS Collaboration].
  ``Search for narrow resonances decaying to  dijets in proton
    proton collisions at $\sqrt{s}=13$ TeV''\\
  Phys.\ Rev.\ Lett.\  116, 071801 (2016), arXiv:1512.01224 [hep-ex].\\
I'm co-coordinator of the analysis team that produced this CMS paper.

\bibitem{Khachatryan:2016ecr} 
[CMS Collaboration],
  ``Search for narrow resonances in dijet final states at $\sqrt{s}=8$
  TeV with the novel CMS technique of data scouting''\\
  Phys.\ Rev.\ Lett.\  117, 031802 (2016), arXiv:1604.08907 [hep-ex].
\\I'm one of the two main authors and the editor of this CMS paper based on
collision data.

\bibitem{Khachatryan:2015sja} 
  [CMS Collaboration],
  ``Search for resonances and quantum black holes using dijet mass
  spectra in proton-proton collisions at $\sqrt{s}=8$ TeV''\\
  Phys.\ Rev.\ D 91, no. 5, 052009 (2015), arXiv:1501.04198 [hep-ex].
  \\I'm the contact person and the main editor of this public CMS document. I supervised the main analyst working on this search (a PhD student from Cukurova University, Turkey).

\bibitem{Chatrchyan:2013qha} 
[CMS Collaboration],
 ``Search for narrow resonances using the dijet mass spectrum in pp collisions at $\sqrt{s}=8$ TeV''\\
Phys.\ Rev.\ D 87, no. 11, 114015 (2013), arXiv:1302.4794 [hep-ex].
 \\ I am the contact person of this CMS paper based on collision data. I supervised one of the two analysts working on this search (a PhD student from Cukurova University, Turkey).  

\bibitem{CMS:2012yf} 
[CMS Collaboration],
 ``Search for narrow resonances and quantum black holes in inclusive and b-tagged dijet mass spectra from pp collisions at $\sqrt{s}=7$ TeV''\\
 JHEP 1301, 013 (2013) arXiv:1210.2387 [hep-ex].
 \\ I am one of two analysts of the inclusive dijet search reported in this CMS paper based on collision data.

\bibitem{Khachatryan:2014gha} 
[CMS Collaboration],
 ``Search for massive resonances decaying into pairs of boosted bosons in semi-leptonic final states at $\sqrt{s} =$ 8 TeV''\\
  JHEP 1408, 174 (2014), arXiv:1405.3447 [hep-ex].
  \\I am one of the two contact people of this CMS analysis based on collision data, and co-editor of this CMS paper.

\bibitem{Khachatryan:2014hpa} 
[CMS Collaboration],
``Search for massive resonances in dijet systems containing jets
   tagged as W or Z boson decays in pp collisions at $ \sqrt{s} $ = 8 TeV''\\ 
JHEP 1408, 173 (2014), arXiv:1405.1994 [hep-ex].
 \\ I'm part of the analysis group involved in this CMS search which is constituted by almost 10 people from CERN, John Hopkins University, and \textit{L'Institut de Physique Nucleaire de Lyon} (IPNL).

\bibitem{Chatrchyan:2012yxa} 
[CMS Collaboration],
 ``Search for heavy resonances in the W/Z-tagged dijet mass spectrum in pp collisions at 7 TeV''\\
Phys.\ Lett.\ B 723, 280 (2013), arXiv:1212.1910 [hep-ex].
 \\ I'm part of the analysis group involved in this CMS search which
 is constituted by almost 10 people from CERN, John Hopkins
 University, and \textit{L'Institut de Physique Nucleaire de Lyon}
 (IPNL).

\bibitem{Khachatryan:2016yji} 
[CMS Collaboration],
 ``Search for massive WH resonances decaying into the $\ell \nu \mathrm{b} \overline{\mathrm{b}} $ final state at $\sqrt{s}=8$ TeV''\\
  Eur.\ Phys.\ J.\ C 76, 237 (2016), arXiv:1601.06431 [hep-ex].\\
The analysis was revieview within the ``CMS Exotica leptons+jets working group'' during
the period of my convenership.

\bibitem{Khachatryan:2015ywa} 
 [CMS Collaboration],
 ``Search for narrow high-mass resonances in proton–proton collisions at $\sqrt{s}=8$ TeV decaying to a Z and a Higgs boson''\\
  Phys.\ Lett.\ B 748, 255 (2015), arXiv:1502.04994 [hep-ex].\\
The analysis was revieview within the ``CMS Exotica leptons+jets working group'' during
the period of my convenership.

\bibitem{Khachatryan:2014ura} 
  [CMS Collaboration],
 ``Search for pair production of third-generation scalar leptoquarks
  and top squarks in proton–proton collisions at $\sqrt{s}=8$ TeV''\\
  Phys.\ Lett.\ B 739, 229 (2014), arXiv:1408.0806 [hep-ex].\\
The analysis was revieview within the ``CMS Exotica leptons+jets working group'' during
the period of my convenership.

\bibitem{Khachatryan:2014dka} 
  [CMS Collaboration],
 ``Search for heavy neutrinos and $\mathrm {W}$ bosons with
  right-handed couplings in proton-proton collisions at $\sqrt{s}=8$ TeV''\\
  Eur.\ Phys.\ J.\ C 74, no. 11, 3149 (2014), arXiv:1407.3683
  [hep-ex].\\
The analysis was revieview within the ``CMS Exotica leptons+jets working group'' during
the period of my convenership.

\bibitem{Khachatryan:2015bma} 
[CMS Collaboration],
``Search for a massive resonance decaying into a Higgs boson and a W or Z boson in hadronic final states in proton-proton collisions at $ \sqrt{s}=8 $ TeV''\\
  JHEP 1602, 145 (2016), arXiv:1506.01443 [hep-ex].
  \\ I was member of the {\it``Analysis Review Committee''} for the scrutiny of this analysis within the CMS collaboration. 

\bibitem{Chatrchyan:2012rva} 
[CMS Collaboration],
 ``Search for exotic resonances decaying into WZ/ZZ in pp collisions at $\sqrt{s}=7$ TeV''\\
 JHEP 1302, 036 (2013), arXiv:1211.5779 [hep-ex].
  \\ I was member of the {\it``Analysis Review Committee''} for the scrutiny of the ZZ search in jet plus missing transverse energy final state within the CMS collaboration. 
%  \\ This analysis with 7 TeV data is closely related with the searches for WW/ZZ resonances at 8 TeV which I was involved in.

\bibitem{Chatrchyan:2013ual} 
[CMS Collaboration],
  ``Measurement of the $t\bar{t}$ production cross section in the all-jet final state in pp collisions at $\sqrt{s}$ = 7 TeV''\\
  JHEP 1305, 065 (2013), arXiv:1302.0508 [hep-ex].
  \\ I was member of the {\it``Analysis Review Committee''} for the scrutiny of this analysis within the CMS collaboration. 

\bibitem{Khachatryan:2015vaa} 
[CMS Collaboration],
``Search for pair production of first and second generation
  leptoquarks in proton-proton collisions at $\sqrt{s}=8$ TeV''\\
  Phys.\ Rev.\ D  93, 032004 (2016), arXiv:1509.03744 [hep-ex].
\\I supervised the main analyst (a PhD student from Princeton University) of the first-generation search included in this CMS paper based on collision data. 

\bibitem{Chatrchyan:2012vza} 
[CMS Collaboration],
``Search for pair production of first- and second-generation scalar leptoquarks in pp collisions at $\sqrt{s}= 7$ TeV''\\
Phys.\ Rev.\ D 86, 052013 (2012), arXiv:1207.5406 [hep-ex]. 
\\I supervised the main analyst (a PhD student from Princeton University) of the first-generation search included in this CMS paper based on collision data. 

\bibitem{Chatrchyan:2011ar} 
[CMS Collaboration],
 ``Search for First Generation Scalar Leptoquarks in the evjj channel in pp collisions at sqrt(s) = 7 TeV''\\
  Phys.\ Lett.\ B 703, 246 (2011), arXiv:1105.5237 [hep-ex].
  \\ I am the contact person and one of the two analysts (from University of Maryland group) of this CMS paper based on collision data.

\bibitem{Khachatryan:2010mp}
[CMS Collaboration],
``Search for Pair Production of First-Generation Scalar Leptoquarks in pp Collisions at sqrt(s) = 7 TeV''\\
Phys.\ Rev.\ Lett.\  106, 201802 (2011), arXiv:1012.4031 [hep-ex].
\\I am one of the four analysts (from University of Maryland group) of this CMS paper based on collision data.

\bibitem{Chatrchyan:2010zz}
[CMS HCAL Collaboration],
``Study of various photomultiplier tubes with muon beams and Cherenkov light produced in electron showers''\\
 JINST 5, P06002 (2010).
  \\ The data used in this study were collected during the HCAL Test Beam 2009. I contributed to 
  commissioning and calibration of the {\it ``delay wire chambers''} installed along the H2 
  beam line (CERN, Prevessin site) for beam position measurements.

\bibitem{Chatrchyan:2009hy}
[CMS Collaboration],
 ``Identification and Filtering of Uncharacteristic Noise in the CMS Hadron Calorimeter''\\
  JINST 5, T03014 (2010), arXiv:0911.4881 [physics.ins-det].\\
I supervised a PhD student at University of Maryland working at the
the development and implementation of algorithms for the
identification of anomalous, beam-induced signals in the CMS hadronic 
forward calorimeter at $\sqrt{s}=$0.9, 2.36 and 7 TeV.

\bibitem{Abdullin:2009zz}
[USCMS Collaboration and ECAL/HCAL Collaboration],
``The CMS Barrel Calorimeter Response To Particle Beams From 2-Gev/C To 350-Gev/C''\\
Eur.\ Phys.\ J.\  C 60, 359 (2009), [Erratum-ibid.\  C 61, 353 (2009)].\\
I monitored the high voltage system of the CMS electromagnetic
calorimeter (ECAL) and performed data-taking shifts in the combined ECAL+HCAL test beam in 2006.

\bibitem{Adzic:2008zza}
[CMS Electromagnetic Calorimeter Group],
``Intercalibration of the barrel electromagnetic calorimeter of the CMS  experiment at start-up''\\
JINST 3, P10007 (2008).
\\ I performed a feasibility study of using $\pi^0 \rightarrow \gamma \gamma$ decays for the calibration of the ECAL crystals, with full detector simulation.

\bibitem{Bartoloni:2007hx}
 [F.~Santanastasio {\it et al.}],
``High voltage system for the CMS electromagnetic calorimeter''\\
 Nucl.\ Instrum.\ Meth.\  A 582, 462 (2007).
 \\ I performed part of the stability tests on the high voltage boards at CERN laboratory and most of the data analysis. 

\bibitem{Brianza:2015jia} 
[F.~Santanastasio {\it et al.}],
  ``Response of microchannel plates to single particles and to electromagnetic showers,''\\
Nucl.\ Instrum.\ Meth.\ A 797, 216 (2015).\\
I contributed to the preparation of the experimental setup for the
test-beam studies and to the data analysis.

%\clearpage

%------------------------------------------------------------------------------------------------------------------------------------------------------------
\vspace{0.1cm} \begin{center} \textsc{Preliminary results of the CMS
    Collaboration (quoted in this document)} \end{center} \vspace{0.05cm}
%------------------------------------------------------------------------------------------------------------------------------------------------------------

%\cite{CMS:2016pkl}
\bibitem{CMS:2016pkl} 
  [CMS Collaboration], 
  ``Search for low-mass
  pair-produced dijet resonances using jet substructure techniques in
  proton-proton collisions at a center-of-mass energy of $\sqrt{s}=13$
  TeV''\\
  CMS-PAS-EXO-16-029 (2016)
  \href{http://cds.cern.ch/record/2231062/files/EXO-16-029-pas.pdf}{link
    to pdf}.
  \\ I was chair of the {\it``Analysis Review Committee''} for the scrutiny of this analysis within the CMS collaboration. 

%%\cite{CMS-PAS-EXO-12-021}
%\bibitem{CMS-PAS-EXO-12-021} 
% {\bf ``Search for new resonances decaying to WW to l nu q qbar' in the final state with a lepton, missing transverse energy, and single reconstructed jet''}
%   \\{}[CMS Collaboration]
%  \\{}CMS PAS EXO-12-021 (2013), http://cds.cern.ch/record/1590301/files/EXO-12-021-pas.pdf
%  \\  I am the contact person of this CMS analysis based on collision data, one of the four main analysts, and the main editor of this public CMS document. I supervised one of the two main analysts working on this search (a PhD student from Peking University, China). The analysis has  been published in 2014 in combination with a complementary search for ZZ resonances (EXO-12-022).

%\cite{CMS-PAS-EXO-12-022}
%\bibitem{CMS-PAS-EXO-12-022} 
% {\bf ``Search for a narrow spin-2 resonance decaying to Z bosons in the semileptonic final state''}
%   \\{}[CMS Collaboration]
%  \\{}CMS PAS EXO-12-022 (2013), https://cds.cern.ch/record/1596494/files/EXO-12-022-pas.pdf
%  \\  I'm part of the analysis group involved in this CMS search which is constituted by about 10 people from CERN, KIT, Peking University, SPRACE, and University of Perugia. 
%The analysis has been published in 2014 in combination with a complementary search for WW resonances (EXO-12-021).

%\cite{CMS-PAS-EXO-12-024}
%\bibitem{CMS-PAS-EXO-12-024} 
% {\bf ``Search for heavy resonances in the W/Z-tagged dijet mass spectrum in pp collisions at 8 TeV''}
%   \\{}[CMS Collaboration]
%  \\{}CMS PAS EXO-12-024 (2013), http://cds.cern.ch/record/1563153/files/EXO-12-024-pas.pdf
%  \\ I'm part of the analysis group involved in this CMS search which is constituted by almost 10 people from CERN, John Hopkins University, and \textit{L'Institut de Physique Nucleaire de Lyon} (IPNL).
 
%%cite{CMS-PAS-EXO-12-059}
%\bibitem{CMS-PAS-EXO-12-059}
%{\bf ``Search for Narrow Resonances using the Dijet Mass Spectrum with 19.6 fb-1 of pp Collisions at sqrt(s)=8 TeV''}
%  \\{}[CMS Collaboration]
%  \\{}CMS PAS EXO-12-059 (2013), http://cds.cern.ch/record/1519066/files/EXO-12-059-pas.pdf
%  \\I'm the contact person and the main editor of this public CMS document. I supervised the main analyst working on this search (a PhD student from Cukurova University, Turkey). The analysis is aiming for publication in 2015 in combination with a search for high mass resonances decaying to pairs of b-quarks.

%\cite{CMS-DP-2012-022}
\bibitem{CMS-DP-2012-022}
[CMS Collaboration]
``Data Parking and Data Scouting at the CMS Experiment''
  CMS DP-2012/022 (2012)
  \href{http://cds.cern.ch/record/1480607/files/DP2012\_022.pdf}{link
    to pdf}.
  \\I'm the editor of this public CMS document.

%cite{CMS-PAS-EXO-11-094}
\bibitem{CMS-PAS-EXO-11-094}
[CMS Collaboration],
``Search for Narrow Resonances using the Dijet Mass Spectrum in pp Collisions at sqrt(s)=7 TeV''\\
  CMS PAS EXO-11-094 (2012)
  \href{http://cds.cern.ch/record/1461223/files/EXO-11-094-pas.pdf}{link
  to pdf}.
  \\ I am the main developer of the novel trigger, data acquisition, and analysis strategy employed in this search to recover sensitivity to new physics at dijet masses below 1 TeV.

%%\cite{CMS-PAS-EXO-12-041}
%\bibitem{CMS-PAS-EXO-12-041} 
% {\bf ``Search for Pair-production of First Generation Scalar Leptoquarks in pp Collisions at sqrt(s) = 8 TeV''}
%   \\{}[CMS Collaboration]
%  \\{}CMS PAS EXO-12-041 (2014), http://cds.cern.ch/record/1742179/files/EXO-12-041-pas.pdf
  
%%\cite{EXO-10-005}
%\bibitem{EXO-10-005}
%{\bf ``Search for Pair Production of First Generation Leptoquarks Using Events Containing Two Electrons and Two Jets Produced in pp Collisions at sqrt(s) = 7 TeV''}
%  \\{}[CMS Collaboration]
%  \\{}CMS PAS EXO-10-005 (2010), http://cdsweb.cern.ch/record/1289514/files/EXO-10-005-pas.pdf 
%  \\I am co-author and one of the four analysts (from University of Maryland group) of this public CMS Physics Analysis Summary based on collision data.

\bibitem{JME-10-002}
[CMS Collaboration],
``Performance of Missing Transverse Energy Reconstruction in sqrt(s)=900 and 2360 GeV pp Collision Data''\\
  CMS PAS JME-10-002 (2010)
  \href{http://cdsweb.cern.ch/record/1247385/files/JME-10-002-pas.pdf}{link
  to pdf}.
  \\ I worked on the section related to calorimeter MET cleaning algorithms and performances.

%\cite{EXO-08-010}
\bibitem{EXO-08-010}
[CMS Collaboration],
``Search for Pair Production of First Generation Scalar Leptoquarks at the CMS Experiment''\\
  CMS PAS EXO-08-010 (2009)
  \href{http://cdsweb.cern.ch/record/1196076/files/EXO-08-010-pas.pdf}{link
  to pdf}.
  \\I am co-author and one of the four analysts (from University of Maryland group) of this public CMS Physics Analysis Summary based on MC simulation.
%%\cite{CMS-PAS-TOP-11-007}
%\bibitem{CMS-PAS-TOP-11-007}
%{\bf ``Measurement of the ttbar production cross section in the fully hadronic decay channel in pp collisions at 7 TeV''}
%  \\{}[CMS Collaboration]
%  \\{}CMS PAS TOP-11-007 (2011), http://cdsweb.cern.ch/record/1371755/files/TOP-11-007-pas.pdf 
%  \\ I was member of the {\it``Analysis Review Committee''} for the scrutiny of this public CMS result within the collaboration.
  
%\cite{CMS-PAS-EXO-11-061}
%\bibitem{CMS-PAS-EXO-11-061}
%{\bf ``Search for Randall-Sundrum Gravitons Decaying into a Jet plus Missing ET at CMS''}
%  \\{}[CMS Collaboration]
%  \\{}CMS PAS EXO-11-061 (2012), http://cdsweb.cern.ch/record/1426654/files/EXO-11-061-pas.pdf 
%  \\ I was member of the {\it``Analysis Review Committee''} for the scrutiny of this public CMS result within the collaboration.

%\clearpage

%------------------------------------------------------------------------------------------------------------------------------------------------------------
\vspace{0.1cm} \begin{center} \textsc{Conference Proceedings} \end{center} \vspace{0.05cm}
%------------------------------------------------------------------------------------------------------------------------------------------------------------

\bibitem{Santanastasio:ICNFP2016} 
  ``Searches for BSM physics in final states with jets and leptons+jets
  at CMS''\\
 F.~Santanastasio\\
  Proceedings will be published in European Physical Journal Web of Conferences.\\
  {\it Prepared for the 5th International Conference on New Frontiers in Physics, Kolymbari, Crete, 6-14 July 2016} 

\bibitem{Santanastasio:ICHEP2014} 
  ``Search for heavy resonances decaying to bosons with the ATLAS and
  CMS detectors''\\
 F.Santanastasio\\
  Nucl.\ Part.\ Phys.\ Proc.\  273-275, 649 (2016).\\
  {\it Prepared for the XXXVII International Conference on High Energy Physics, Valencia, Spain, 2-9 July 2014}

\bibitem{Santanastasio:LHCPP2013} 
  ``Searches for Heavy Hadronic Resonances with the ATLAS and CMS
  detectors at the LHC''\\
F.~Santanastasio and C.~Doglioni\\
  PoS LHCPP2013, 015 (2013).\\
  {\it Prepared for the Workshop LHCpp 2013 - VI Workshop Italiano sulla Fisica p-p a LHC, INFN - Sezione di Genova, Genova, Italy, 8-10 May 2013}

\bibitem{Santanastasio:2013iz} 
  ``Exotic Phenomena Searches at Hadron Colliders''\\
 F.Santanastasio\\
  arXiv:1301.2521 [hep-ex] (2013).\\
  {\it Prepared for the XXXII Physics in Collision 2012 conference (PIC2012), Strbske Pleso, Slovakia, 12-15 September 2012}

\bibitem{MoriondEW2011}
``Exotica searches at the CMS experiment''\\
 F.Santanastasio\\
  Proceedings of the XLVIth Rencontres de Moriond 2011 Electroweak Interactions and Unified Theories, 125-132 (2011), edited by Etienne Auge, Jacques Dumarchez, and Jean Tran Thanh Van $\textcopyright$ The Gioi Publishers.\\
{\it Prepared for XLVIth Rencontres de Moriond 2011 Electroweak Interactions and Unified Theories, La Thuile, Aosta Valley, Italy, 13-20 March 2011}

\bibitem{Santanastasio:2010zz}
``Searches With Early Data At CMS''\\
 F.Santanastasio\\
  PoS DIS2010, 206 (2010).\\
  {\it Prepared for 18th International Workshop on Deep Inelastic Scattering and Related Subjects (DIS 2010), Florence, Italy, 19-23 Apr 2010}

\bibitem{Santanastasio:IFAE2009}
``Prospects for Exotica Searches at ATLAS and CMS Experiments''\\
 F.Santanastasio\\
Il Nuovo Cimento Vol.32 C, N.3-4 ncc9484 (2009).\\
{\it Prepared for Incontri di Fisica delle Alte Energie (IFAE 2009), Bari, Italy, Apr 2009}

%------------------------------------------------------------------------------------------------------------------------------------------------------------
\vspace{0.1cm} \begin{center} \textsc{Internal notes of the CMS
    Collaboration (quoted in this document)} \end{center} \vspace{0.05cm}
%------------------------------------------------------------------------------------------------------------------------------------------------------------

%\cite{AN-17-051}
\bibitem{AN-17-051}
``Measurement of W-tagging data/MC scale factors using ttbar semi-leptonic events using full 2016 dataset"
  \\{}F.~Santanastasio {\it et al.}
  \\{}CMS AN-2017/051 (2017)
  \\ I am one of the three authors of this internal note based on
  analysis of collision data. I supervised
  Simone Gelli (PhD student in Sapienza University of Rome) who is
  the main analyst.

%\cite{AN-16-344}
\bibitem{AN-16-344}
``Absolute residual jet energy corrections with $\gamma$+jet events at 13 TeV"
  \\{}F.~Santanastasio {\it et al.}
  \\{}CMS AN-2016/344 (2016)
  \\ I am one of the three authors of this internal note based on
  analysis of collision data. I supervised
  Federico Preiato (PhD student in Sapienza University of Rome) who is
  the main analyst.

%\cite{AN-16-202}
\bibitem{AN-16-202}
``Search for narrow resonances decaying to dijets in pp collisions at $\sqrt{s}=13$~TeV using 12.9 fb$^{-1}$"
  \\{}F.~Santanastasio {\it et al.}
  \\{}CMS AN-2016/202 (2016)
  \\ I am the editor of this internal note. I supervised Federico Preiato (PhD student in Sapienza University of Rome)
  working on the calibration of online reconstructed jets employed in the scouting analysis.

%\cite{AN-15-175}
\bibitem{AN-15-175}
``Search for narrow resonances using the dijet mass spectrum with 2.45 fb$^{-1}$ of proton-proton collisions at $\sqrt{s}=13$~TeV"
  \\{}F.~Santanastasio {\it et al.}
  \\{}CMS AN-2015/175 (2015)
  \\ I am the editor of this internal note presenting results on
  analysis of collision data. I supervised Giulia D'Imperio (PhD
  student in Sapienza University of Rome) who is the main analyst.

%\cite{AN-15-063}
\bibitem{AN-15-063}
``Search for narrow resonances using the dijet mass spectrum in proton-proton collisions at sqrt(s)=13 TeV (Phys14 MC analysis)"
  \\{}F.~Santanastasio {\it et al.}
  \\{}CMS AN-2015/063 (2015)
  \\ I am the editor of this internal note concerning a feasibility study of
  the dijet search. I supervised Giulia D'Imperio (PhD student in Sapienza University of Rome) who is
  the main analyst.

%\cite{AN-14-104}
\bibitem{AN-14-104}
``Search for dijet resonances at $\sqrt{s}=8$~TeV with data scouting"
  \\{}F.~Santanastasio {\it et al.}
  \\{}CMS AN-2014/104 (2014)
  \\ I am co-editor of this CMS analysis note based on collision data and one of the two main analysts. 

%\cite{AN-13-045}
\bibitem{AN-13-045}
``Search for a BSM resonance decaying to W vector bosons in the semileptonic final state"
  \\{}F.~Santanastasio {\it et al.}
  \\{}CMS AN-2013/045 (2013)
  \\ I am the contact person of this CMS analysis based on collision data and one of the four main analysts. I supervised one of the two main analysts working on this search (a PhD student from Peking University, China). The analysis is has been published in 2014 in combination with a complementary search for ZZ resonances (EXO-12-022).

%\cite{AN-13-040}
\bibitem{AN-13-040}
``Search for a BSM resonance decaying to Z vector bosons in the semileptonic final state"
  \\{}F.~Santanastasio {\it et al.}
  \\{}CMS AN-2013/040 (2013)
  \\ I'm part of the analysis group involved in this CMS search which is constituted by about 10 people from CERN, KIT, Peking University, SPRACE, and University of Perugia. 
The analysis has been published in 2014 in combination with a complementary search for WW resonances (EXO-12-021).

%\cite{AN-12-455}
\bibitem{AN-12-455}
``Search for Narrow Resonances using the Dijet Mass Spectrum in pp Collisions at sqrt(s)=8 TeV with full 2012 dataset"
  \\{}F.~Santanastasio {\it et al.}
  \\{}CMS AN-2012/455 (2012)
  \\ I am the contact person of this CMS analysis based on collision data. I supervised the main analyst working on this search (a PhD student from Cukurova University, Turkey). A public preliminary result has been released by the CMS collaboration on February 2013 in view of the Moriond/EW conference. The analysis is aiming for publication in 2015 in combination with a search for high mass resonances decaying to pairs of b-quarks.

%\cite{AN-12-229}
\bibitem{AN-12-229}
``Search for Narrow Resonances using the Dijet Mass Spectrum in pp Collisions at sqrt(s)=8 TeV"
  \\{}F.~Santanastasio {\it et al.}
  \\{}CMS AN-2012/229 (2012)
  \\ I am the contact person of this CMS analysis based on collision data. I supervised one of the two analysts working on this search (a PhD student from Cukurova University, Turkey). This analysis has been published in 2013.

%\cite{AN-12-012}
\bibitem{AN-12-012}
``Search for Dijet Resonances in the Dijet Mass Spectrum in pp Collisions at sqrt(s)=7 TeV"
  \\{}F.~Santanastasio {\it et al.}
  \\{}CMS AN-2012/012 (2012)
  \\ I am one of the two analysts (from a group of about 10 people from various 
  institutions including CERN) of this CMS analysis based on 4.7 fb$^{-1}$ of pp collision 
  data collected in 2011. I am the main developer of the novel trigger, data acquisition, and analysis strategy employed in this search to recover sensitivity to new physics at dijet masses below 1 TeV. This analysis has been published in 2012 in combination with a complementary search for heavy resonances decaying in pairs of b-quarks.

%\cite{AN-12-012}
\bibitem{AN-12-393}
``Search for heavy resonances in the W/Z-tagged dijet mass spectrum in pp collisions at 8 TeV''
  \\{}F.~Santanastasio {\it et al.}
  \\{}CMS AN-2012/393 (2013)
  \\ I'm part of the analysis group involved in this CMS search which is constituted by almost 10 people from CERN, John Hopkins University, and \textit{L'Institut de Physique Nucleaire de Lyon} (IPNL). This analysis has been published in 2014.

%\cite{AN-12-012}
\bibitem{AN-11-524}
``Search for qW/qZ/WW/WZ/ZZ Resonances in the W/Z-tagged Dijet Mass Spectrum from 7 TeV pp Collisions at CMS''
  \\{}F.~Santanastasio {\it et al.}
  \\{}CMS AN-2011/524 (2011)
  \\ I'm part of the analysis group involved in this CMS search which is constituted by almost 10 people from CERN, John Hopkins University, and \textit{L'Institut de Physique Nucleaire de Lyon} (IPNL). This analysis has been published in 2013.

%%\cite{AN-12-325}
%\bibitem{AN-12-325}
%``Significance estimation for the dijet resonance search using 8 TeV pp collision data''
%  \\{}F.~Santanastasio, C. Guichardant {\it et al.}
%  \\{}CMS AN-2012/325 (2012)
%  \\ I am the supervisor of this "2012 CERN summer student" project realized by an undergraduate student from \textit{L'Institut de Physique Nucleaire de Lyon} (INPL).

%\cite{AN-2013-109}
\bibitem{AN-2013-109}
``Search for Pair-production of First Generation Scalar Leptoquarks in pp Collisions at sqrt(s)=8 TeV''
  \\{}F.~Santanastasio {\it et al.}
  \\{}CMS AN-2013/109 (2013)

%\cite{AN-11-492}
\bibitem{AN-11-492}
``Search for First-Generation Scalar Leptoquarks in pp Collisions at sqrt(s)=7 TeV using the CMS Detector''
  \\{}F.~Santanastasio {\it et al.}
  \\{}CMS AN-2011/492 (2011)
  \\ I am one of the two analysts (supervising a PhD student from Princeton University) of this CMS analysis based on 4.7 fb$^{-1}$ of pp collision data collected in 2011. 
This analysis has been published in 2012 in combination with a complementary second-generation leptoquark search.

%\cite{AN-2010-361}
\bibitem{AN-2010-361}
``Search for Pair Production of First-Generation Scalar Leptoquarks Using Events Produced in pp Collisions at sqrt(s)=7 TeV Containing One Electron, Two Jets and Large Missing Transverse Energy''
  \\{}F.~Santanastasio {\it et al.}
  \\{}CMS AN-2010/361 (2010)

%\cite{AN-2010-230}
\bibitem{AN-2010-230}
``Search for Pair Production of First Generation Leptoquarks Using Events Containing Two Electrons and Two Jets Produced in pp Collisions at sqrt(s)=7 TeV''
  \\{}F.~Santanastasio {\it et al.}
  \\{}CMS AN-2010/230 (2010)

%\cite{AN-2008-070}
\bibitem{AN-2008-070}
``Search for Pair Production of First Generation Scalar Leptoquarks at the CMS Experiment''
  \\{}F.~Santanastasio {\it et al.}
  \\{}CMS AN-2008/070 (2009)

%\cite{AN-2010-219}
\bibitem{AN-2010-219}
``Results of a visual scan of high MET events in 7 TeV pp collision data''
  \\{}F.~Santanastasio {\it et al.}
  \\{}CMS AN-2010/219 (2010)
  
%\cite{AN-2010-029}
\bibitem{AN-2010-029}
``Commissioning of Uncorrected Missing Transverse Energy in Zero Bias and Minimum Bias Events at  sqrt(s)=900 GeV and  2360 GeV''
  \\{}F.~Santanastasio {\it et al.}
  \\{}CMS AN-2010/029 (2010)

%\cite{DN-2010-008}
\bibitem{DN-2010-008}
``Optimization and Performance of HF PMT Hit Cleaning Algorithms Developed Using pp Collision Data at sqrt(s)=0.9, 2.36 and 7 TeV''
  \\{}F.~Santanastasio {\it et al.}
  \\{}CMS DN-2010/008 (2010)

%\cite{DN-2007-013}
\bibitem{DN-2007-013}
``InterCalibration of the CMS Barrel Electromagnetic Calorimeter Using Neutral Pion Decays''
   \\{}F.~Santanastasio {\it et al.}
  \\{}CMS DN-2007/013 (2007)

%\cite{IN-2006-050}
\bibitem{IN-2006-050}
``Study of ECAL calibration with $\pi^0 \rightarrow \gamma \gamma$ decays''
\\{}F.~Santanastasio {\it et al.}
\\{}CMS IN-2006/050 (2006)

%\clearpage

%------------------------------------------------------------------------------------------------------------------------------------------------------------
\vspace{0.1cm} \begin{center} \textsc{Theses ( \textit{Laurea} and PhD)} \end{center} \vspace{0.05cm}
%------------------------------------------------------------------------------------------------------------------------------------------------------------

\bibitem{Santanastasio:DOTTORATO}
``Search for Supersymmetry with Gauge-Mediated Breaking using high energy photons at CMS experiment''
  \\{}F.~Santanastasio
  \\{}PhD thesis at \textit{Sapienza Universit\`a di Roma} (2007)
 \\{}\href{http://www.roma1.infn.it/cms/tesiPHD/santanastasio.pdf}{http://www.roma1.infn.it/cms/tesiPHD/santanastasio.pdf}

\bibitem{Santanastasio:LAUREA}
``Calibrazione di un calorimetro elettromagnetico tramite il flusso totale di energia''
  \\{}F.~Santanastasio
  \\{}\textit{Laurea} thesis at \textit{Sapienza Universit\`a di Roma} (2004)
  \\{}\href{http://www.roma1.infn.it/cms/tesi/santanastasio.pdf}{http://www.roma1.infn.it/cms/tesi/santanastasio.pdf}


\end{thebibliography}

%\vspace{1cm}
\vfill{}
\hrulefill

% FILL IN THE FULL URL TO YOUR CV
\begin{center}
%{\footnotesize \href{http://www.ias.edu/spfeatures/einstein}{http://www.ias.edu/spfeatures/einstein} — Last updated: \today}
{\footnotesize Last updated: \today}
\end{center}


\end{document}
