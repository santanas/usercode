%------------------------------------
% Dario Taraborelli
% Typesetting your academic CV in LaTeX
%
% URL: http://nitens.org/taraborelli/cvtex
% DISCLAIMER: This template is provided for free and without any guarantee 
% that it will correctly compile on your system if you have a non-standard  
% configuration.
%------------------------------------ 


% ! TEX TS-program = XeLaTeX -xdv2pdf
% ! TEX encoding = UTF-8 Unicode

\documentclass[10pt, a4paper]{article}
\usepackage{fontspec} 
\usepackage{xunicode} 
\usepackage{xltxtra}
% per le lettere accentate italiane sul Mac! :-)
%\usepackage[applemac]{inputenc} %with VIM
\usepackage[latin1]{inputenc} % with TeXShop


% DOCUMENT LAYOUT
\usepackage{geometry}
\geometry{a4paper, textwidth=5.5in, textheight=8.5in, marginparsep=7pt, marginparwidth=.6in}
\setlength\parindent{0in}

% ADDITIONAL SYMBOLS
%\usesymbols[mvs]

% FONTS
\defaultfontfeatures{Mapping=tex-text} % converts LaTeX specials (``quotes'' --- dashes etc.) to unicode
%\setromanfont [Ligatures={Common}, BoldFont={Fontin Bold}, ItalicFont={Fontin Italic}]{Fontin}
\setromanfont [Ligatures={Common}, BoldFont={Linux Libertine Bold}, ItalicFont={Linux Libertine Italic}]{Linux Libertine}
\setsansfont [Ligatures={Common}, BoldFont={Fontin Sans Bold}, ItalicFont={Fontin Sans Italic}]{Fontin Sans}
\setmonofont[Scale=0.8]{Monaco} 
% ---- CUSTOM AMPERSAND
\newcommand{\amper}{{\fontspec[Scale=.95]{Fontin}\selectfont\itshape\&}}
% ---- MARGIN YEARS
%\newcommand{\years}[1]{\marginpar{\scriptsize #1}}
\newcommand{\years}[1]{\marginpar{\footnotesize #1}}

% HEADINGS
\usepackage{sectsty} 
\usepackage[normalem]{ulem} 
\sectionfont{\rmfamily\mdseries\upshape\Large}
\subsectionfont{\rmfamily\bfseries\upshape\normalsize} 
\subsubsectionfont{\rmfamily\mdseries\upshape\normalsize} 
%modifying section numbering
\def\thesubsection{\arabic{subsection}.\ } 

% PDF SETUP
% ---- FILL IN HERE THE DOC TITLE AND AUTHOR
\usepackage[dvipdfm, bookmarks, colorlinks, breaklinks, pdftitle={Riccardo Maria Bianchi - Curriculum Vitae},pdfauthor={Riccardo Maria Bianchi}]{hyperref}
%\hypersetup{linkcolor=blue,citecolor=blue,filecolor=black,urlcolor=blue} 
\hypersetup{linkcolor=cyan,citecolor=blue,filecolor=black,urlcolor=cyan} 

% DOCUMENT
\begin{document}
\reversemarginpar
{\LARGE Riccardo Maria Bianchi}\\[1cm]
%Institute address
\begin{tabular}{ l c l }
 Physikalisches Institutes & & \\
 Albert Ludwig Universitaet & &\\
Hermann-Herder-Str. 3 & \makebox[3cm]{}& Tel: \texttt{+49 761 203 5879}\\
\texttt{79104} Freiburg & & Fax: \texttt{+49 761 203 5938}\\
Germany & & email: \href{mailto:rbianchi@cern.ch}{rbianchi@cern.ch}\\
\end{tabular}\\[1em]
% Home address
\begin{tabular}{ l c l }
\emph{Home Address}: & & \\
Seegasse 4a & \makebox[4.3cm]{} & Tel: \texttt{+49 7663 91 26 65}\\
\texttt{79268} Boetzingen & & Cel: \texttt{+39 05 65 089}\\
Germany & & Cel: \texttt{+41 76 70 16 266}\\ 
\end{tabular}\\[1em]
%\vfill
 Born:  4 October 1976---Roma, Italy\\
Nationality:  Italian
%\textsc{url}: \href{http://www.ias.edu/spfeatures/einstein/}{http://www.ias.edu/spfeatures/einstein/}\\ 

%%\hrule
\section*{Current position}
\emph{PhD Student}, Physics Institut, Prof. Gregor Herten Group, Freiburg University, Germany

%%\hrule
\section*{Areas of specialization}
 Particle physics, Data Analysis, Analysis Software Development, Supersymmetry
 
\section*{Areas of competence}
 Software Development, IT, Particle detector physics
 

%\hrule
\section*{Education}
\noindent
\years{Oct 2006 - present}\textbf{PhD in Particle Physics} \\
\textit{``Looking for signatures of Physics Beyond the Standard Model in ATLAS with an Automated Model-Independent General Search''}\\
\textsc{Advisors:} Prof. Gregor Herten, Dr. Sascha Caron, Dr. Renaud Brunelière (Freiburg)\\
\textit{Albert Ludwig Universitaet}, Freiburg, Germany\\
\textit{\small expected date: August 2010}\\[1em]
\years{2003-2006}\textbf{MSc (\textit{Laurea Magistrale}) in Nuclear and Subnuclear Physics} {\small (highest honours)}\\
\textit{``Study of the ATLAS MDT Muon Chambers calibration constants with data from a testbeam''}\\
\textsc{Advisors:} Prof. Toni Baroncelli (INFN), Prof. Filippo Ceradini (Roma Tre)\\
Mark: 110/110 \textit{``magna cum laude''}\\
\textit{Università Roma Tre}, Roma, Italy\\[1em]
\years{2000-2003}\textbf{BSc (\textit{Laurea}) in Physics} {\small (highest honours)}\\
\textit{`Multithreading for the ATLAS Data Acquisition System Data Flow`'}\\
\textsc{Advisors:} Prof.sa Fernanda Pastore (RomaTre), Dr. David Francis (CERN), Dr. Luis Tremblet (CERN) \\
Mark: 110/110 \textit{``magna cum laude''}\\
\textit{Università Roma Tre}, Roma, Italy\\[1em]



%%\hrule
%\section*{Appointments held}
\section*{Work Experience}
\noindent
\years{2010-now}\textbf{CERN, European Organization for Nuclear Research}, Geneva, Switzerland\\
\textit{Fellow Researcher}\\
Working on the upgrade and operation of the Data Aquisition system (DAQ) of the ATLAS experiment. And working on data analysis within the ATLAS Supersymmetry Working Group.\\[1em]
\years{2009-2010} \textbf{``Albert Ludwig" Freiburg University}, Freiburg, Germany\\
\textit{Software Developer}\\
Working in the Grid group, developing a web dashboard to monitor Freiburg BFG (\textit{Black Forest Grid}) machines\\[1em]
\years{2006-2008} \textbf{``Albert Ludwig" Freiburg University}, Freiburg, Germany\\
\textit{Teaching Assistant in Particle Physics}\\
Teaching at "Advanced Physics Lab Class" for students of Physics\\[1em]
\years{Jun-Jul 2003}\textbf{CERN - ATLAS TDAQ}, Geneva, Switzerland\\
\textit{Internship in Data Acquisition}\\
Summer Internship, where I worked in the Trigger \& Data Acquisition (TDAQ) Group of the ATLAS Experiment, on a Multithreading project for the Data Flow software.\\[1em]
\years{Jul-Aug 2002}\textbf{Fraunhofer-Institut für Photonische Mikrosysteme IPMS}, Dresden, Germany\\
\textit{Internship in Optics}\\
Summer Internship, where I conceived and realized a  microspectrometer for a fair, to show an innovative micro-electro-mechanical-optical device built at the Institute.\\[1em]
\years{Jun-Jul 2001}\textbf{CNRS-LTHE} , Grenoble, France\\
\textit{Internship in Geophysics}\\
Summer Internship (as "Stage de Licence"), where I worked on the field with the equipe from the Institute and other students, making measures of erosion and water infiltration on the marnes of Digne, France.\\[1em]
\years{1999-2000}\textbf{WWF  ``Macchiagrande" Protected Area}, Fregene, Italy\\
\textit{Civil Service (at the place of the Army Service)}\\
Working in the protected area taking care of the maintenance of the area, protecting the environment and the wild animals, and accompanying the visitors of the park.\\[1em]
\years{1995-1999}Rome, Italy. Waiter in several restaurants in Rome.\\


%%\hrule




%%%%%%%%%%%%%%%%%%%%%%%%%%%
%%% Grants, honors & awards
%%%%%%%%%%%%%%%%%%%%%%%%%%%

%\section*{Grants, honors \amper{} awards}
\section*{Grants}
\noindent
\years{2003} \textsc{INFN} Grant for Thesis abroad.\\[1em]
\years{2000-2001} \textsc{Erasmus} studies at \textit{``Université Joseph Fourier''}, Grenoble, France.
%\years{1921}Nobel Prize in Physics, Nobel Foundation
% ...se, magari!! ;-D



%%%%%%%%%%%%%%%%%%%%%%%%%%%
%%% Languages
%%%%%%%%%%%%%%%%%%%%%%%%%%%
  	 \section*{Languages}
\begin{tabular}{l c l}
\textit{Italian} (native speaker) & \makebox[4em]{} & \textit{French} (fluent)\\
\textit{English} (fluent) & &\textit{German} (basic)\\
\end{tabular}




%%%%%%%%%%%%%%%%%%%%%%%%%%%
%%% Skills
%%%%%%%%%%%%%%%%%%%%%%%%%%%



%\section*{Programming \amper\ IT}
\section*{Skills}

\setcounter{subsection}{0}

\subsection{Programming Languages}

\textbf{Python}\\
 Very good knowledge,\\
      Used in the day-by-day work. Multiparadigm programming. Very good knowledge not only of the language itself, but also of the standard library and external libraries, and experience with many third-part extensions (like distutils, vpython, beautifulsoup, ... ). Integration of Python with C++ custom code. Started building web applications and web services with the CherryPy framework, also together with the Qooxdoo Javascript framework as front-end.\\

\textbf{C++}\\ 
Very good knowledge,\\
      Used in the day-by-day work. Not only the language itself, but also the STD library and external libraries, like Boost.
      Intergation of C++ code with Python.\\

\textbf{Fortran}\\ 
Good knowledge,\\
      Used during the MSc Thesis work, to perform data analysis and toy-MonteCarlo programs.\\

\textbf{JavaScript}\\
 Basic knowledge,\\
      started building web application with the Qooxdoo Javascript framework as front-end, together with CherryPy as server-side back-end.\\

Markup languages: \textbf{XML}, \textbf{JSON}, \textbf{Wiki}, \textbf{HTML+CSS}. Good knowledge\\

Typesetting languages: \textbf{LaTeX}, \textbf{XeTeX}. Very good knowledge\\

\textbf{BASIC}. Good \textit{old} knowledge


	
\subsection{Physics Packages}

\textbf{ROOT Analysis Framework}\\ 
Very good Knowledge\\
     good knowledge of ROOT Classes\\
      day-by-day usage via CINT, PyROOT or compiled C++.\\

\textbf{ATLAS Athena Framework}\\
Very good knowledge, day-by-day use.\\

\textbf{LabView}, \textbf{TestPoint}, \textbf{Origin}\\ Basic knowledge\\
    
	
\subsection{Operating Systems}

\textbf{Linux}\\ Very good knowledge, also as \textit{Administrator} (especially SLC, RedHat, Debian)\\

\textbf{Windows}\\ Very good knowledge, also as \textit{Administrator}: Vista, XP, 9X, 3.X, MS-DOS\\

\textbf{Apple Os X}\\ Good knowledge: Snow Leopard\\


	
\subsection{Other IT-related skills}	

Good knwoledge of \textbf{GRID} usage, especially with DQ2, Ganga, Panda\\

Good knowledge of build tools like \textbf{make}, \textbf{autoconf}, \textbf{distutils} and basic knowledge of the \textbf{CMT}\\

Very good knowledge of \textbf{Subversion} version control system, also as \textit{Administrator} (in Freiburg I'm SVN Admin since 3 years)\\

Very good knowledge of the \textbf{Eclipse IDE} as multi-language development framework, also of many of its extra packages\\

Good Knowledge of \textbf{AFS file system} and basic knowledge of \textbf{Kerberos}\\

\textbf{CMS -- Content Management Systems}: Good Knowledge of \textbf{Joomla 1.5} administration and customization. Basic knowledge of \textbf{Wordpress2}\\

\textbf{OS and Computer Architecture}: Good Knowledge of computer architecture theory and Operating System architectures, and implications in software programming techniques (pipelines, buffers, caches, I/O, multithreading, ...)\\

Good knowledge of the \textbf{Hardware} Market, and good skills in assembling machines\\



%%%%%%%%%%%%%%%%%%%%%%%%%%%
%%% Service work
%%%%%%%%%%%%%%%%%%%%%%%%%%%

\section*{Service work in Experiments and Collaborations}
\subsection*{ATLAS Experiment}
\noindent
\textbf{Data Analysis: Supersymmetry Working Group} Working on data analysis, on exploring and implementing analysis strategies and on data files production\\
\textbf{Development \& Upgrade} Working in the DAQ group, on the upgrade of the configuration DB system\\
\textbf{Detector Operation} Shifter in the control room, at the Muon System, DAQ and Run Control desks\\
\textbf{Software Framework} Taking part in code testing, and shifter for the build test system (RTT)\\
\textbf{Documentation} Responsible person for a part of the documentation of the ATLAS data-format\\
\textbf{Public Relations} Official ATLAS Guide, escorting VIP visits to the ATLAS cavern\\

%%%%%%%%%%%%%%%%%%%%%%%%%%%
%%% Publications & Talks
%%%%%%%%%%%%%%%%%%%%%%%%%%%

\section*{Publications \amper{} talks}
\setcounter{subsection}{0}

\subsection{Peer-reviewed Journal papers}
\noindent

\textit{``WatchMan Project - A Python CASE framework for High Energy Physics data analysis in the LHC era"}\\
\textbf {R.M.Bianchi}, R.Brunelière\\
\emph{\small submitted to ``Journal of Computational Science'', Elsevier}\\


\textit{``WatchMan Project - Computer Aided Software Engineering applied to HEP Analysis Code Building for LHC"}\\
\textbf {R.M.Bianchi}, R.Brunelière, S.Caron\\
\emph{Proceedings Of Science}:   PoS(ACAT2010)061\\

\textit{``Discovery potential of Supersymmetry and Universal Extra Dimensions in the ATLAS experiment at the Large Hadron Collider at CERN"}\\
\textbf {R.M.Bianchi}\\
\emph{Proceedings Of Science}:   PoS(HCP2009)066\\

\textit{``Study of the ATLAS MDT Spectrometer using High Energy CERN combined Test beam Data"},\\
C. Adorisio , \textit{et al.}\\
\emph{Nuclear Instruments and Methods A}: A598:400-415,2009\\



\subsection{Peer-reviewed CERN public notes}

\textit{``Discovery Potential of SUSY and UED in ATLAS''}\\
\textbf{ R.M.Bianchi},  on behalf of the ATLAS Collaboration\\
Poster for Hadron Collider Physics Symposium (HCP) 2009 in Evian\\
CERN ATLAS Public:    ATL-PHYS-SLIDE-2009-361\\

\textit{``Prospects for SUSY and UED discovery based on inclusive searches at 10 TeV centre-of-mass energy with the ATLAS detector''}\\
\textbf{R.M.Bianchi}, R.Brunelière, S.Caron, J.Dietrich, M.Rammensee, Z.Rurikova\\
CERN ATLAS Internal Note:      ATL-PHYS-INT-2009-060, ATL-COM-PHYS-2009-302\\
CERN, Geneva, June 2009\\
This internal note became ATLAS PUBLIC:    ATL-PHYS-PUB-2009-084\\

\textit{``Prospects for Supersimmetry Discovery Based on Inclusive Searches with the ATLAS detector at the LHC (Long Version)''}\\
J.Abdallah, F.Ahles, S.Asai, J.Asal, A.J.Barr, \textbf{R.M.Bianchi},  \textit{et al.}\\
CERN ATLAS Communication: ATL-COM-PHYS-2009-261\\
CERN, Geneva, May 2009\\
published within the CSC Book "Expected Performance of the ATLAS Experiment, Detector, Trigger and Physics", CERN-OPEN-2008-020, Geneva, 2008.\\

\textit{``Study of MDT calibration constants using H8 testbeam data of year 2004''},\\
Baroncelli, T, \textbf{R.M.Bianchi}, S.Di Luise, A.Passeri, F. Petrucci, L.Spogli\\
CERN ATLAS Public Note: ATL-MUON-PUB-2007-004. Jul. 2006\\


\subsection{ATLAS internal notes}

\textit{``Usage of the Distributed Analysis Tools in The ATLAS Supersymmetry Working Group''},\\
Barr, A; \textbf{R.M.Bianchi},  M.Biglietti, O.Brandt, S.Caron, G.Carlino, A.Christov, \textit{et al.}\\
CERN ATLAS Internal Note:    ATL-COM-SOFT-2007-011\\
CERN, Geneva, Aug. 2007\\


%\subsection*{Books}
%\noindent
%\years{1954}Einstein, Albert (1954), \emph{Ideas and Opinions}, New York: Random House, ISBN 0-517-00393-7
%...magari!! :-)  :->

%\subsection*{Newspaper articles}
%\noindent
%\years{1940}Einstein, Albert, et al. (December 4, 1948), “To the editors", \emph{New York Times}\\
%\years{1949}Einstein, Albert (May 1949), “Why Socialism?", \emph{Monthly Review}.




\section*{Talks in international conferences}
\noindent
\years{8-11.07.2010} \textbf{EuroSciPy 2010} - 3rd International
Conference on Python in Science\\
\textit{``WatchMan Project - A Python CASE framework for High Energy
Physics data analysis in the LHC era''}\\ 
Talk about my own work\\ 
Paris, France\\[1em]
\years{22-27.02.2010} \textbf{ACAT 2010} - 13th International Workshop on Advanced Computing and Analysis Techniques for Physics,\\
\textit{``WatchMan Project: Applying Computer Aided Software Engineering to HEP Analysis Code Building for LHC''}\\
Talk about my own work\\
Jaipur, India\\[1em]
\years{16-20.11.2009} \textbf{Evian HCP 2009} - Hadron Collider Physics Symposium\\
\textit{"Discovery Potential of SUSY and UED in ATLAS"}\\
Poster on behalf of the ATLAS Collaboration\\
Evian, France\\



%\section*{Talks in national conferences}
% BFG conferences: Freiburg and Munich





\section*{Attended workshops \& conferences}
\noindent
\years{2-4.02.2010} \textit{Physics for Health in Europe Workshop},\\ CERN\\[1em]
\years{24.04-01.05.2009}\textit{1st International Workshop On Hadron Beam Therapy of Cancer},\\
 Erice, Sicily (Italy)
 



%%%%%%%%%%%%%%%%%%%%%%%%%%%%%%
%%% Teaching 
%%%%%%%%%%%%%%%%%%%%%%%%%%%%%%
\section*{Teaching}
\noindent
\years{2009}Supervisor of a CERN Summer Student\\
project: ATLAS data analysis\\
CERN, Geneva, Switzerland\\[1em] 
\years{2008}Assistant for the 4\textsuperscript{th} year Physics Laboratory
class: \textit{Subnuclear Physics}\\
\textit{Albert Ludwig Universitaet}, Freiburg, Germany\\[1em]
\years{2007}Assistant for the 1\textsuperscript{st} year Physics Laboratory
class: \textit{Mechanics and Electromagnetism}\\
textit{Albert Ludwig Universitaet}, Freiburg, Germany\\[1em]

%%%%%%%%%%%%%%%%%%%%%%%%%%%%%%%%%%%%%%%%%%%
%%% Activities for scientific dissemination
%%%%%%%%%%%%%%%%%%%%%%%%%%%%%%%%%%%%%%%%%%%
\section*{Scientific dissemination activities}
\noindent
\years{2010-2011}Official guide for VIP visits at the ATLAS experiment cavern at CERN.\\[1em]
\years{2007-2008}Guide at ``Physics Open Day'' for High School students at Freiburg University (Germany)\\[1em]
\years{2004-2005}Assistant at CERN/INFN ``Masterclasses'' for High School students, Rome (Italy)\\[1em]
 





%%%%%%%%%%%%%%%%%%%%%%%%%%%
%%% Training
%%%%%%%%%%%%%%%%%%%%%%%%%%%

\section*{Training}
\setcounter{subsection}{0}

\subsection{Academic training}
\noindent
\years{20-22.01.2010} \textit{Physics and Analysis at a Hadron Collider}\\
by Dr. Douglas Glenzinski (FNAL), CERN\\[1em]
\years{11-15.05.2009} \textit{Lectures on Multivariate Analysis Techniques},\\
by Helge Voss, \\
Freiburg University (Germany)\\[1em]
\years{9-10.02.2009 } \textit{Understanding Cross Sections at the LHC}\\
by  Dr. Stephen Mrenna (Fermi National Accelerator Laboratory, USA),\\
CERN\\[1em]
\years{2-5.02.2009 } \textit{Statistical Techniques for Particle Physics}\\
by  Dr. Kyle Cranmer (CERN-PH), \\
CERN\\[1em]
\years{ 21-23.01.2009} \textit{The Opposite Ends of Supersymmetry and their Implications for the LHC}\\
by Dr. Wells, James (CERN-TH), \\
CERN\\

\subsection{Technical training}
\noindent
\years{7.12.2009}\textit{"Developing Secure Software"} , CERN\\[1em]
\years{6-7.10.2009}CERN openlab / Intel Computer \textit{"Architecture and Performance Tuning Workshop"}\\


\section*{Summer schools}
\noindent
\years{12-22.08.2008} \textit{Fermilab/CERN Hadron Collider Physics Summer School 2008}\\
Fermilab, Chicago, USA\\[1em]
\years{28.08-08.09.2006} \textit{ 2\textsuperscript{nd} CASPUR Summer School on Advanced Computing}\\
Castel Gandolfo, Italy\\[1em]
\years{12-20.06.2005} \textit{NUFACT 05 Summer Institute on Neutrino Factories and Superbeams}\\
Capri, Italy\\[1em]
\years{17-21.05.2004} \textit{LNF Spring School "Bruno Tuscheck" in Nuclear, Subnuclear and Astroparticle Physics}\\
LNF, Frascati (Italy)\\





%\hrule
%\section*{Service to the profession}



%\vspace{1cm}
\vfill{}
\hrulefill

% FILL IN THE FULL URL TO YOUR CV
\begin{center}
%{\footnotesize \href{http://www.ias.edu/spfeatures/einstein}{http://www.ias.edu/spfeatures/einstein} — Last updated: \today}
{\footnotesize Last updated: \today}
\end{center}


\end{document}
