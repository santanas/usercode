%------------------------------------
% Dario Taraborelli
% Typesetting your academic CV in LaTeX
%
% URL: http://nitens.org/taraborelli/cvtex
% DISCLAIMER: This template is provided for free and without any guarantee 
% that it will correctly compile on your system if you have a non-standard  
% configuration.
%------------------------------------ 


% ! TEX TS-program = XeLaTeX -xdv2pdf
% ! TEX encoding = UTF-8 Unicode

\documentclass[10pt, a4paper]{article}
\usepackage{fontspec} 
\usepackage{xunicode} 
\usepackage{xltxtra}
% per le lettere accentate italiane sul Mac! :-)
%\usepackage[applemac]{inputenc} %with VIM
\usepackage[latin1]{inputenc} % with TeXShop


% DOCUMENT LAYOUT
\usepackage{geometry}
\geometry{a4paper, textwidth=5.5in, textheight=8.5in, marginparsep=7pt, marginparwidth=.6in}
\setlength\parindent{0in}

% ADDITIONAL SYMBOLS
%\usesymbols[mvs]

% FONTS
\defaultfontfeatures{Mapping=tex-text} % converts LaTeX specials (``quotes'' --- dashes etc.) to unicode
%\setromanfont [Ligatures={Common}, BoldFont={Fontin Bold}, ItalicFont={Fontin Italic}]{Fontin}
\setromanfont [Ligatures={Common}, BoldFont={Linux Libertine Bold}, ItalicFont={Linux Libertine Italic}]{Linux Libertine}
%\setsansfont [Ligatures={Common}, BoldFont={Fontin Sans Bold}, ItalicFont={Fontin Sans Italic}]{Fontin Sans}
\setmonofont[Scale=0.8]{Monaco} 
% ---- CUSTOM AMPERSAND
\newcommand{\amper}{{\fontspec[Scale=.95]{Linux Libertine Bold}\selectfont\itshape\&}}
% ---- MARGIN YEARS
%\newcommand{\years}[1]{\marginpar{\scriptsize #1}}
\newcommand{\years}[1]{\marginpar{\footnotesize #1}}

% HEADINGS
\usepackage{sectsty} 
\usepackage[normalem]{ulem} 
\sectionfont{\rmfamily\mdseries\upshape\Large}
\subsectionfont{\rmfamily\bfseries\upshape\normalsize} 
\subsubsectionfont{\rmfamily\mdseries\upshape\normalsize} 
%modifying section numbering
\def\thesubsection{\arabic{subsection}.\ } 

% PDF SETUP
% ---- FILL IN HERE THE DOC TITLE AND AUTHOR
\usepackage[dvipdfm, bookmarks, colorlinks, breaklinks, pdftitle={Francesco Santanastasio - Motivation for coming to CERN},pdfauthor={Francesco Santanastasio}]{hyperref}
%\hypersetup{linkcolor=blue,citecolor=blue,filecolor=black,urlcolor=blue} 
\hypersetup{linkcolor=cyan,citecolor=blue,filecolor=black,urlcolor=cyan} 

% Title of Bibliography
%\renewcommand\refname{References \\ \normalsize \begin{center} \quad \quad \textsc{Publications (relative to research activities)}\end{center} }

% DOCUMENT
\begin{document}
\reversemarginpar
{\LARGE Francesco Santanastasio}\\[1cm]
%Institute address
\begin{tabular}{ l c l }
\emph{Institute Address}: & & \\
University of Maryland & & \\
Department of Physics - John S. Toll Physics Building & &\\
College Park  & & \\
MD  \texttt{20742-4111} & \makebox[1.2cm]{} & Tel: \texttt{+1 301 405 3401} \\
United States of America & & Fax: \texttt{+1 301 314 9525} \\
\end{tabular}\\[1em]
% Work address
\begin{tabular}{ l c l }
\emph{Work Address}: & & \\
CERN (Conseil Europeen pour la Recherche Nucleaire) & \makebox[1.cm]{} & \\
\texttt{CH-1211} Geneve  \texttt{23} & & Tel.: \texttt{+41 22 76 75 765}\\
Building \texttt{8}, Room R-\texttt{019} & & Cel: \texttt{+41 76 22 86 127}\\ 
Switzerland &  & email: \href{mailto:francesco.santanastasio@cern.ch}{francesco.santanastasio@cern.ch} 
\end{tabular}\\[1em]
%\vfill
Born:  9 February 1980---Roma, Italy\\
Nationality:  Italian
%\textsc{url}: \href{http://www.ias.edu/spfeatures/einstein/}{http://www.ias.edu/spfeatures/einstein/}\\ 

%%\hrule
\section*{Current Position}
\emph{Post-Doctoral Research Assistant (Post-Doc) in Particle Physics} \\
Department of Physics, University of Maryland, College Park, US

%%\hrule
\section*{Motivation for coming to CERN}
In December 2007, I started an appointment as post-doctoral 
research assistant (post-doc) at the University of Maryland.
Since then I have been based at CERN, working in the CMS experiment on 
data analysis in the exotic physics group, commissioning of the hadronic calorimeter 
and MET performance studies. Thanks to my research activities, 
I had the opportunity of interacting with fellows and staff members 
working in the CERN group of CMS.
Based on this experience, I believe that a CERN fellowship can provide to 
young post-docs an excellent opportunity to form their scientific maturity in high energy 
physics in a challenging environment, which is essential to improve the personal skills
and the knowledge in this field of the physics. 

I am particularly interested to join the CMS CERN group, that I know already, for instance, 
playing a primary role in searches for physics beyond the standard model, 
implementation and commissioning of advanced event reconstruction 
algorithms (particle-flow), and performance studies on jets and missing transverse 
energy reconstruction. Those are in fact the research activities that I would like to 
focus on in the next months. 

In addition, I am enjoying during my post-doc the cross-disciplinary environment 
at CERN, that promotes, for instance, active cooperation between 
experimentalists and theorists, and regular seminaries on various research topics. 
I believe in fact that expanding its own knowledge, also in directions different from 
the primary research field, is important to build a solid physics background.

I would like to continue to work in the friendly and stimulating atmosphere at CERN, 
contributing, as CERN fellow, to the rich LHC physics program during the next years.

%For these reasons, I look forward to continuing, as CERN fellow, my research activities 
%in high energy physics during the next years that should 
%be particularly exciting for the Particle Physics.
 
%\section*{Areas of competence}
%Software Development, IT, Particle detector physics
 
%\vspace{1cm}
\vfill{}
\hrulefill

% FILL IN THE FULL URL TO YOUR CV
\begin{center}
%{\footnotesize \href{http://www.ias.edu/spfeatures/einstein}{http://www.ias.edu/spfeatures/einstein} — Last updated: \today}
{\footnotesize Last updated: \today}
\end{center}


\end{document}
