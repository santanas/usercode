%====================================================================%
%                  MORIOND.TEX                                       %
% This latex file rewritten from various sources for use in the      %
% preparation of the standard proceedings Volume, latest version     %
% for the Neutrino'96 Helsinki conference proceedings                %
% by Susan Hezlet with acknowledgments to Lukas Nellen.              %
% Some changes are due to David Cassel.                              %
%====================================================================%

%\documentstyle[11pt,moriond,epsfig]{article}
\documentclass[11pt]{article}
\usepackage{moriond,epsfig}
\usepackage{xspace}

\bibliographystyle{unsrt}    
% for BibTeX - sorted numerical labels by order of
% first citation.

% A useful Journal macro
\def\Journal#1#2#3#4{{#1} {\bf #2}, #3 (#4)}

% Some useful journal names
\def\NCA{\em Nuovo Cimento}
\def\NIM{\em Nucl. Instrum. Methods}
\def\NIMA{{\em Nucl. Instrum. Methods}~A}
\def\NPB{{\em Nucl. Phys.}~B}
\def\PLB{{\em Phys. Lett.}~B}
\def\PRL{\em Phys. Rev. Lett.}
\def\PRD{{\em Phys. Rev.}~D}
\def\ZPC{{\em Z. Phys.}~C}
\def\JINST{{\em JINST} }
\def\JHEP{{\em JHEP} }

% Some other macros used in the sample text
\def\st{\scriptstyle}
\def\sst{\scriptscriptstyle}
\def\mco{\multicolumn}
\def\epp{\epsilon^{\prime}}
\def\vep{\varepsilon}
\def\ra{\rightarrow}
\def\ppg{\pi^+\pi^-\gamma}
\def\vp{{\bf p}}
\def\ko{K^0}
\def\kb{\bar{K^0}}
\def\al{\alpha}
\def\ab{\bar{\alpha}}
\def\be{\begin{equation}}
\def\ee{\end{equation}}
\def\bea{\begin{eqnarray}}
\def\eea{\end{eqnarray}}
\def\CPbar{\hbox{{\rm CP}\hskip-1.80em{/}}}
%temp replacement due to no font

\def\GeVmass {GeV\xspace}
\def\GeVmom {GeV\xspace}
\def\sqrts {$\sqrt{s}=7$~TeV\xspace}
\def\etmiss {\ensuremath{E_{\mathrm{T}}\hspace{-1.1em}/\kern0.5em}\xspace}
\def\pt{\ensuremath{p_{\rm T}}\xspace}
\def\pp{proton-proton\xspace}
\def\Zprime{Z$^\prime$\xspace}
\def\Wprime{W$^\prime$\xspace}
\def\bprime{b$^\prime$\xspace}
\def\gravitonKK{$\rm{G}_{\rm{KK}}$\xspace}
\def\eejj{$\rm{eejj}$\xspace}
\def\mumujj{$\mu\mu \rm{jj}$\xspace}
\def\enujj{$\rm{e}\nu \rm{jj}$\xspace}
\def\munujj{$\mu\nu \rm{jj}$\xspace}
\def\nunujj{$\nu\nu \rm{jj}$\xspace}
\def\zjets{\ensuremath{\rm{Z}/\gamma}+jets\xspace}
\def\wjets{\ensuremath{\rm{W}}+jets\xspace}
\def\ttbar{\ensuremath{\rm{t} \bar{\rm{t}}}\xspace}
\def\ST{\ensuremath{S_{\rm T}}\xspace}
\def\pb{pb$^{-1}$\xspace}
\def\gluino{$\tilde{\rm{g}}$\xspace}
\def\neutralino{$\tilde{\chi}^{0}_{1}$\xspace}

%%%%%%%%%%%%%%%%%%%%%%%%%%%%%%%%%%%%%%%%%%%%%%%%%%
%                                                %
%    BEGINNING OF TEXT                           %
%                                                %
%%%%%%%%%%%%%%%%%%%%%%%%%%%%%%%%%%%%%%%%%%%%%%%%%%
\begin{document}
\vspace*{4cm}
\title{EXOTICA SEARCHES AT THE CMS EXPERIMENT}

\author{F. SANTANASTASIO \\(ON BEHALF OF THE CMS COLLABORATION)}

\address{University of Maryland, Department of Physics - John S. Toll Physics Building, \\ College Park, MD 20742-4111, United States of America}

\maketitle\abstracts{
This paper presents the results of searches for various new physics 
phenomena in proton-proton collisions at $\sqrt{s}=7$~TeV delivered 
by the LHC and collected with the CMS detector in 2010. 
While the sensitivity of these early searches varies, 
in many cases they set the most stringent limits on these 
new physics phenomena. These results demonstrate good understanding 
of the detector and backgrounds in a variety of channels, 
which is a fundamental component of successful searches in view 
of the much larger data sample expected to be delivered by LHC in 2011 
and beyond.
%% This is where the abstract should be placed. It should consist of 
%% one paragraph
%% and give a concise summary of the material in the article below.
%% Replace the title, authors, and addresses within the curly brackets
%% with your own title, authors, and addresses; please use
%% capital letters for the title and the authors. You may have 
%% as many authors and
%% addresses as you wish. It's preferable not to use footnotes in the abstract
%% or the title; the
%% acknowledgments for funding bodies etc. are placed in a separate section at
%% the end of the text.
}

\section{Introduction}
%% The standard model (SM) of particle physics has been extremely
%% successful in describing all phenomena at the highest 
%% attainable energies thus far. Yet, it is widely believed 
%% to be only an effective description of a more complete theory, 
%% which supersedes it at higher energy scales. Many theoretical 
%% extensions of the SM have been proposed in the past decades, 
%% which usually predict the existence of new particles. Examples 
%% of such conjectured particles are the \Zprime and \Wprime bosons, 
%% fourth generation fermions, supersymmetric particles, leptoquarks, 
%% excited quarks, gravitons, and many others.
%% Past experiments at the Fermilab Tevatron collider, 
%% and previously at the CERN SPS, HERA, and LEP colliders, 
%% have performed estensive searches for signs 
%% of such new physics. In absence of a positive signal, lower 
%% limits on the masses of such new particles have been set. 
%% With its higher centre-of-mass (CM) energy of 7 TeV, the \pp Large Hadron 
%% Collider (LHC) at CERN can produce particles with 
%% masses larger than the current limits, thus extending 
%% the search for new physics in an unexplored territory.
This paper presents the results of searches for various new physics 
phenomena beyond the standard model (SM)~\footnote{Searches for Supersymmetry at CMS are not 
discussed in this paper. These results can be found in other 
proceedings of this conference.} in \pp collisions 
at \sqrts delivered by the LHC and collected with the 
Compact Muon Solenoid (CMS)~\cite{CMSJINST} detector in 2010. 
For the majority of these searches the full dataset has been used, 
corresponding to an integrated luminosity of almost 40~\pb. 
%% The results are presented in different sections, 
%% depending on the phenomenology of the new physics scenario:
%% search for new heavy resonances are presented in 
%% Section~\ref{sec:resonances}; compositeness models are discussed 
%% in Section~\ref{sec:compositeness}; searches for signs of the existence 
%% of extra dimensions are described in 
%% Section~\ref{sec:extradimensions}; finally, search for long-lived 
%% particles and for other exotic final states are presented 
%% in Section~\ref{sec:longlivedplusothers}, 
%% followed by a brief summary in Section~\ref{sec:summary}.


\section{New Heavy Resonances}\label{sec:resonances}

\subsection{Dilepton and Diphoton Resonances} \label{sec:dilepdiphotresonances}

Many models of new physics and extensions of the SM 
predict the existence of narrow resonances, possibly at the TeV mass scale, 
that decay to a pair of charged leptons (such as \Zprime bosons) 
or to lepton and neutrino (such as \Wprime bosons).
Also the Randall-Sundrum (RS) model of extra dimensions foresees
the existence of Kaluza--Klein graviton excitations (\gravitonKK)
decaying to a pair of charged leptons or pair of photons.
The CMS Collaboration has searched for such narrow resonances in the 
invariant mass spectrum of dimuon/dielectron~\cite{Chatrchyan:2011wq}  
and diphoton~\cite{CMSPAS:EXO-10-019} final states, as well as in the 
transverse mass spectrum of electron+neutrino~\cite{Khachatryan201121} 
and muon+neutrino~\cite{Chatrchyan:2011dx} final states.
The spectra are consistent with standard model expectations
in both the bulk and the tails of the aforementioned distributions.
Figure~\ref{fig:resonances} shows the 95\% confidence level (CL) 
upper limits on the cross section 
of \Zprime/\gravitonKK (\Wprime) production, 
obtained combining the dielectron (electron+neutrino) 
and dimuon (muon+neutrino) channels.
A \Zprime (\Wprime) with SM-like coupling can 
be excluded below 1.14 (1.58) TeV. 
Model-independent lower limits on 
the \Zprime mass have also been reported 
in Ref.~\cite{Chatrchyan:2011wq} as a function of the couplings 
of the \Zprime to fermions in the annihilation 
of charge 2/3 and charge -1/3 quarks.
In the diphoton channel, limits are derived on the 
cross section for the production of RS gravitons, 
and hence on the parameters of the warped extra dimension model. 
For values of the coupling parameter ranging from 0.01 to 0.1, graviton masses 
below 371 to 945~\GeVmass are excluded at the 95\% CL.

\begin{figure}[htbp] 
%\vskip 2.5cm
  \begin{center}
    \begin{tabular}{cc}
      \psfig{figure=plots/zpr_ssm_ratio_mcmc_comb_40pb_c.ps,height=2in} &
      \psfig{figure=plots/CombLimit_Vers1f.ps,height=2.2in} \\
    \end{tabular}
    \caption{(Left) Upper limits as a function of resonance mass, on 
      the \Zprime cross section relative to standard model Z boson 
      production, obtained combining dielectron and
      dimuon final states. (Right) Upper limits as a function of 
      the resonance mass, on the \Wprime cross section for the 
      individual electron+neutrino and muon+neutrino channels, and their combination. 
      ADD COMMENT ON EXCESS AT 400 GeV.}
    \label{fig:resonances}    
  \end{center}
\end{figure}

\subsection{Leptoquarks}

The standard model has an intriguing but ad hoc symmetry between 
quarks and leptons. In some theories beyond the SM, such 
as SU(5) gran unification, Pati--Salam SU(4), and others, the existence of a new symmetry 
relates the quarks and leptons in a fundamental way. These models 
predict the existence of new bosons, called leptoquarks. 
The leptoquark (LQ) is coloured, has fractional electric charge, and 
decays to a charged lepton and a quark with unknown branching 
fraction $\beta$, or a neutrino and a quark with branching fraction 
$(1-\beta)$. Constraints from experiments sensitive to flavour-changing 
neutral currents, lepton-family-number violation, and other rare processes 
favour LQs that couple to quarks and leptons within the same SM generation, 
for LQ masses accessible to current colliders. 
Searches for pair-production of first and second 
generation scalar LQs have been performed in the \eejj~\cite{PhysRevLett.106.201802}, 
\enujj~\cite{Collaboration:2011ar}, and \mumujj~\cite{PhysRevLett.106.201803} channels. 
The dominant backgrounds for these searches arise from the SM production of 
\zjets, \wjets and \ttbar events. The reconstructed variable \ST, defined 
below~\footnote{In the \eejj and \mumujj channels, 
\ST is defined as the scalar sum of the transverse momenta of the two 
leading (in \pt) charged leptons and jets. In the \enujj channel, 
\ST is defined as the scalar sum of the transverse momentum of the electron, 
the missing transverse energy, and the two leading jets.} 
has a large signal-to-background discrimination power, 
and it is used to select LQ candidate events.
Figure~\ref{fig:leptoquarksAndDijets} (left) shows the exclusion limits at 95\% CL on 
the first generation leptoquark hypothesis in the $\beta$ versus LQ mass 
plane for the \eejj and \enujj channels, and their combination. 
First generation scalar LQ masses below 384 GeV (340 GeV) are excluded 
at 95\% CL for $\beta=1$ ($\beta=0.5$). In the \mumujj channel, a 95\% CL lower limit on 
the second generation scalar LQ mass is set at 394 GeV assuming $\beta=1$.

\begin{figure}[htbp] 
%\vskip 2.5cm
  \begin{center}
    \begin{tabular}{cc}
      \psfig{figure=plots/beta_vs_m_excl_comb.ps,height=2.5in} &
      \psfig{figure=plots/DijetLimit.eps,height=2.7in} \\
    \end{tabular}
    \caption{(Left) Exclusion limits at 95\% CL on the first generation LQ hypothesis 
    in the $\beta$ versus LQ mass plane. The shaded region is excluded by the 
    current D0~limits, which combine results of \eejj, \enujj, and \nunujj decay modes.
      (Right) 95\% CL upper limits on signal cross section for dijet resonances of type 
    gluon-gluon, quark-gluon, or quark-quark, versus dijet resonance mass, 
    compared to theoretical predictions for various new physics models.}
    \label{fig:leptoquarksAndDijets}
  \end{center}
\end{figure}

\subsection{Dijet Searches}
In the standard model, point like parton-parton scatterings in high energy 
\pp collisions can give rise to final states with energetic jets. 
At large momentum transfers, events with at least two energetic jets (dijets) 
may be used to confront the predictions of perturbative Quantum Chromodynamics 
(QCD) and to search for signatures of new physics. The new physics could manifest itself 
via the direct production of a new massive particle that then decays into a dijet final state
(quark-quark, quark-gluon, or gluon-gluon resonances), 
and/or the rate of dijet events could be enhanced through a new force that 
only manifests itself at very large CM energies (contact interactions).
Complementary search strategies have been pursued by the CMS experiment in the dijet channel: 
search for narrow resonances in the dijet mass spectrum~\cite{PhysRevLett.105.211801}, 
search for narrow resonances and contact interactions using the 
dijet centrality ratio variable~\cite{PhysRevLett.105.262001}, and 
search for contact interactions using dijet angular distributions~\cite{PhysRevLett.106.201804}. 
The first two analyses were performed with the early 3~\pb of \pp collisions 
at \sqrts, and they are now being updated with more data. Figure~\ref{fig:leptoquarksAndDijets}
(right) shows the 95\% CL upper limits on signal cross section versus 
dijet resonance mass, compared to theoretical predictions for various 
new physics models. String resonances, with mass less than 2.50 TeV, excited quarks, with 
mass less than 1.58 TeV, and axigluons, colorons, and $E_6$ diquarks, in specific mass 
intervals, have been excluded at 95\% CL. Using measurements of dijet angular 
distributions over a wide range of dijet invariant masses, a lower limit on the contact 
interaction scale for left-handed quarks of $\Lambda^{+}=5.6$~TeV ($\Lambda^{-}=5.6$~TeV)
for constructive (destructive) interference is obtained at the 95\% CL.

\subsection{Fourth Generation of Fermions and \ttbar Resonances}

Recently, there has been renewed interest in extensions of the SM 
predicting a fourth generation of massive fermions. 
Theoretical works have also shown that indirect bounds on the Higgs boson mass 
can be relaxed, and an additional generation of quarks may possess 
enough intrinsic matter and anti-matter asymmetry to be relevant for 
the baryon asymmetry of the Universe. Driven by this motivation, a search 
for pair production of heavy bottom-like quarks (\bprime) in trileptons 
and same-sign dilepton final states~\cite{Chatrchyan:2011em}, arising from the decay 
chain $\mbox{\bprime}\bar{\mbox{\bprime}} \rightarrow \mbox{tW}^{-}\bar{\mbox{t}}\mbox{W}^{+} \rightarrow \mbox{bW}^{+}\mbox{W}^{-}\bar{\mbox{b}}\mbox{W}^{-}\mbox{W}^+$, has been performed at CMS.
The total branching ratio for these channels is 7.3\% and the very small 
expected SM background comes mainly from \ttbar events. No events are found in
the signal region defined in the analysis, and the \bprime mass range from 
255 to 361 \GeVmass has been excluded at the 95\% CL.

The CMS experiment has also performed a model-independent search for 
new massive neutral bosons (such as \Zprime) decaying via a top-antitop 
quark pair~\cite{CMSPAS:TOP-10-007}. 
%, motivated by the numerous extensions of the SM that predict gauge 
%interaction couplings to third generation quarks to be enhanched. 
The event reconstruction and selection is optimized for the 
production of top quarks close to rest, with well separated decay products. 
The analysis focuses on decay channels of the \ttbar system that include a 
single isolated electron or muon. No significant deviation from SM expectations 
is found in the \ttbar mass spectra obtained from eight independent data samples, 
categorized by lepton type, multiplicity of jets and number of b-tagged jets. 
Upper limits on the production cross section times branching fraction, 
$\sigma_{\mbox{\Zprime}} \times \mbox{BR}(\mbox{\Zprime}\rightarrow\mbox{\ttbar})$, 
of the order of 25, 7, and 4~\pb for invariant masses in the 
region $m_{\mbox{\Zprime}}=0.5$, 1, and 1.5~TeV, respectively, are set. 
These results are competitive with the current limits from the Tevatron, 
particularly at high mass values.

\section{Compositeness Models}\label{sec:compositeness}

A fundamental question in the standard model (SM) of particle physics is the source of the 
mass hierarchy of the quarks and leptons. A commonly proposed explanation for the three 
generations is a compositeness model in which the known leptons and quarks are bound states 
of either three fermions, or a fermion-boson pair. The underlying substructure of these new 
bound states implies a large spectrum of excited states. Novel strong contact interactions (CI) 
couple excited fermions ($f^\ast$) to ordinary quarks and leptons ($f$) and can be described 
with the effective lagrangian $\mathcal{L}_{\rm{CI}} \propto (\jmath^\mu \jmath_\mu) / \Lambda^{2} $, 
where $\Lambda$ is the compositeness scale, and $\jmath_\mu$ is the fermion current. 

\subsection{Excited Leptons}
A search for the associated production of a lepton ($\ell$) and an oppositely 
charged excited lepton ($\ell^{\ast}$) is performed~\cite{CMSPAPER:EXO-10-016}. 
The final state contains two leptons and a photon, $\ell\ell\gamma$, arising 
from the decay $\ell^{\ast} \rightarrow \ell \gamma$, where $\ell$ is either an electron or a muon.
The SM backgrounds containing misidentified electrons or photons are estimated using 
data-driven methods. The maximum reconstructed invariant mass among the 
two possible lepton-photon combination, $\rm{M}^{max}_{\ell\gamma}$,
is used to discriminate between signal and SM backgrounds.
No excess of events is found in the $\rm{M}^{max}_{\ell\gamma}$ spectra 
above the SM expectation in the electron or muon channel. Figure~\ref{fig:compositeness} (left) 
shows the region excluded at 95\% CL in the $\Lambda - \rm{M}_{\ell^\ast}$ parameter space for 
the $\mu\mu\gamma$ channel, where $\rm{M}_{\ell^\ast}$ is the excited lepton mass.
A similar exclusion is obtained in the $\rm{ee}\gamma$ channel.

\subsection{Excited Quarks}
The CMS experiment has performed a search for anomalous production of highly boosted Z 
bosons in the dimuon decay channel arising from the decays of new heavy 
particles~\cite{CMSPAS:EXO-10-025}. The search is optimized for the detection of excited quark 
production and decay via $\rm{q}\ast \rightarrow \rm{qZ} \rightarrow \rm{q}\mu\mu$, 
with no explicit requirement on the jet recoiling against a high transverse momentum Z.
Figure~\ref{fig:compositeness} shows the dimuon \pt spectrum from data 
compared to the simulation of excited quark signals. 
The results are consistent with background-only expectations.
Limits are derived on excited quark production in the plane of compositeness 
scale $\Lambda$ versus mass for two scenarios of production and decay: 
one assuming excited quark transitions via SM gauge bosons only, 
and one including also novel contact interaction transitions from new strong dynamics. 
The $\rm{q}^\ast$ mass limits at 95\% CL with contact interactions are more sensitive 
than previous searches in scenarios where the coupling to gluons is suppressed 
relative to the electroweak gauge bosons, ruling out masses below 1.17 TeV in the 
extreme case when this coupling is zero. 

\begin{figure}[htbp] 
%\vskip 2.5cm
  \begin{center}
    \begin{tabular}{cc}
      \psfig{figure=plots/exclLim_mstar.ps,height=2.5in} &
      \psfig{figure=plots/dataVsTheory_lumi36000InvNb.ps,height=2.7in} \\
    \end{tabular}
    \caption{(Left) Exclusion at 95\% CL in the $\Lambda - \rm{M}_{\ell^\ast}$ 
      parameter space for the $\mu\mu\gamma$ channel. 
      (Right) The dimuon \pt spectrum distribution from data with a background 
      parametrization overlaid. Various excited quark signals are shown, corresponding 
      to different production mechanisms (gauge interaction and contact 
      interaction) and different $\rm{q}^\ast$ masses.}
    \label{fig:compositeness}
  \end{center}
\end{figure}


\section{Extra Dimensions}\label{sec:extradimensions}
Compact large extra dimensions (ED) are an intriguing proposed solution to the hierarchy 
problem of the SM, which refers to the puzzling fact that 
the fundamental scale of gravity $\rm{M}_{\rm{Pl}} \sim 10^{19}$~GeV is so much higher 
than the electroweak symmetry breaking scale~$\sim 10^3$~GeV. In the ADD 
model~\footnote{The original proposal to use ED to solve the hierarchy problem was presented 
by Arkani-Hamed, Dimopoulos, and Dvali (ADD).}, the SM is constrained to the common 
3+1 space-time dimensions, while gravity is free to propagate through the entire 
multidimensional space. The gravitational flux in 3+1 dimensions is effectively diluted by 
virtue of the multidimensional Gauss's Law. In this framework, the fundamental Planck scale 
can be lowered to the electroweak scale, thus making production of gravitons possible at the LHC.
Some of the experimental signatures of the existence of such extra dimensions are discussed below.

\subsection{Diphoton and Dimuon Channels}
Searches for virtual-graviton contributions in the 
diphoton~\cite{springerlink:10.1007/JHEP05(2011)085} and dimuon~\cite{CMSPAS:EXO-10-020} final 
states have been performed. Figure~\ref{fig:ADDdiphotdimuon} displays the diphoton (left) 
and dimuon (right) invariant mass distribution for the observed data, 
the backgrounds, and the ADD signal. 
%In the diphoton channel, the SM backgrounds arising from 
%misreconstruced photons are derived using data-driven methods.
The ADD signal, differently from the searches discussed in Section~\ref{sec:dilepdiphotresonances}, 
would not appear as a narrow peak but as an overall excess of events at high values of invariant mass. 
In both $\gamma\gamma$ and $\mu\mu$ channels, the data is found to be consistent with SM expectations.
Lower limits at the 95\% CL are set on the effective Planck scale in the approximate range of 1.4--2.3~TeV, 
depending on the final state considered, the number of extra dimensions, and the 
theoretical conventions used to describe the virtual-graviton production.

\begin{figure}[htbp] 
%\vskip 2.5cm
  \begin{center}
    \begin{tabular}{cc}
      \psfig{figure=plots/invMass_addADD_Nov_36.ps,height=2.5in} &
      \psfig{figure=plots/Data_Simulation_Dimuon_Mass_Dist_EXO_10_020.ps,height=2.5in} \\
    \end{tabular}
    \caption{ Diphoton (left) and dimuon (right) invariant mass spectra compared with the 
      SM prediction and some simulated ADD signals.
    }
    \label{fig:ADDdiphotdimuon}
  \end{center}
\end{figure}

\subsection{Mono-jet Final State}
A search for production of a real graviton G 
balanced by an energetic hadronic jet via the processes 
$\rm{q}\bar{\rm{q}} \rightarrow \rm{gG}$, $\rm{q}\rm{g} \rightarrow \rm{qG}$, 
and $\rm{g}\rm{g} \rightarrow \rm{gG}$ has been performed~\cite{CMSPAPER:EXO-11-003}. 
Since gravitons are free to propagate in the extra dimensions, they escape 
the detector and can only be inferred from the amount of missing transverse energy (\etmiss). 
The offline event selection requires large \etmiss, one high \pt jet, a veto on the 
presence of well-identified leptons and isolated tracks, and additional 
requirements to suppress the cosmics, beam halo, and instrumental backgrounds 
that can fake the mono-jet+\etmiss signature. 
Figure~\ref{fig:MonoJetAndBlackHole} (left) shows the \pt distribution of the leading jet after the full selection. 
A measurement of the electroweak background from $\rm{W} \rightarrow \mu\nu$ enriched 
data is used to derive a data-driven background estimate for the \zjets and \wjets remaining 
in the signal region. The number of observed events in data is in good agreement with the SM prediction, 
and significant improvements are made to the current limits on the fundamental parameters of 
the model describing real-graviton emission.
%~\footnote{Limits on the Unparticle model were not presented at the conference 
%but will be included in the final version of the paper.} 

\subsection{Microscopic Black Holes}
One of the exciting predictions of theoretical models with extra dimensions 
and low-scale quantum gravity is the possibility of copious production of microscopic black 
holes in particle collisions at the LHC.
Events with large total transverse energy are analyzed for the presence
of multiple high-energy jets, leptons, and photons, typical of a signal expected from a microscopic black 
hole~\cite{Khachatryan2011434}. 
Figure~\ref{fig:MonoJetAndBlackHole} (right) shows the distribution of the total transverse energy for data, 
background prediciton and various signal samples. Good agreement with the standard model backgrounds, 
dominated by QCD multijet production, is observed for various final-state multiplicities 
and model-independent limits on new physics in these final states are set. 
Using simple semi-classical approximation, limits on the minimum black hole 
mass are derived as well, in the range 3.5--4.5~TeV.

\begin{figure}[htbp] 
%\vskip 2.5cm
  \begin{center}
    \begin{tabular}{cc}
      \psfig{figure=plots/Jet1Pt_monojet.ps,height=2.5in} &
      \psfig{figure=plots/Results_Inclusive_Mul5.eps,height=2.4in} \\
    \end{tabular}
    \caption{ (Left) \pt distribution of the leading jet after the full mono-jet+\etmiss selection.
      The distribution for an ADD signal (shown in red) is overlaid.
      (Right) Total transverse energy (including the \etmiss in the sum) 
      for events with more than 5 objects (jets, leptons, and photons) for data, background prediction, and 
      black hole signals for three different parameter sets. 
    }
    \label{fig:MonoJetAndBlackHole}
  \end{center}
\end{figure}

\section{Long-Lived Particles and Other Exotic Signatures}\label{sec:longlivedplusothers}

\subsection{Massive Long-Lived Particles}

Heavy stable (or quasi-stable) charged particles appear in various extensions of
the SM. Heavy long-lived particles with hadronic nature, such as gluinos or stops, 
hadronize in flight, forming meta-stable bound states with quarks and gluons (so called R-Hadrons).
If the lifetime of R-Hadrons produced at LHC is longer than a few nanoseconds, the particle 
will travel over distances that are comparable or larger than the size of a typical particle detector, 
and hence might be detected. The CMS experiment uses two complementary strategies 
to look for such long-lived particles. 

A significant fraction of these massive particles 
(assuming masses greater than 100 GeV) will have a velocity $\beta=v/c$, smaller than 0.9. 
A Search has been performed to identify R-Hadrons through the distinctive signature of a 
high momentum (p) track with an anomalously large rate of energy loss through ionization (dE/dx)
in the silicon tracker~\cite{}. A lower limits at the 95\% CL on the mass of a stable gluino 
is set at 398~GeV, using a conventional model of nuclear interactions that allows charged 
hadrons containing this particle to reach the muon detectors. A lower limit is also set 
at 311~GeV, in a conservative scenario where any hadron containing this particle becomes 
neutral before reaching the muon detectors.

Searches have been also performed for very slow ($\beta \le 0.4$) R-hadrons containing a gluino, 
for which the electromagnetic and nuclear energy loss is sufficient to bring a significant fraction 
of the produced particles to rest inside the CMS detector volume 
(inparticular in the hadronic calorimeter HCAL)~\cite{}. These stopped R-hadrons will decay seconds, 
day, or weeks later (accordingly to their unknown lifetime), and out-of-time with respect to the LHC collisions. 
In a dataset with a peak instantaneous luminosity of $1\times 10^{32}\rm{cm}^{-2}\rm{s}^{-1}$, 
an integrated luminosity of 10~\pb, and a search interval corresponding to 62 hours of LHC operation, 
no significant excess above background (mainly instrumental noise) was observed.  
Limits at the 95\% confidence level on gluino pair production over 13 orders of magnitude of gluino lifetime 
are set. For a mass difference $\rm{m}_{\mbox{\gluino}} - \rm{m}_{\mbox{\neutralino}} > 100$~GeV, and assuming 
$\rm{BR}( \mbox{\gluino} \rightarrow \rm{g} \mbox{\neutralino} ) = 100$\%, $\rm{m}_{\mbox{\gluino}}<370$~GeV 
are excluded for lifetimes from 10 $\mu\rm{s}$ to 1000 s.

\subsection{Lepton jets}

\section{Summary}\label{sec:summary}

%\begin{table}[htbp]
%\caption{CAPTION}
%\vspace{0.4cm}
%\begin{center}
%\begin{tabular}{|c|c|c|}
%\hline
%& & \\
%& & \\ 
%\hline
%\end{tabular}
%\end{center}
%\end{table}

\section*{Acknowledgments}

\section*{References}
\begin{thebibliography}{99}

\bibitem{CMSJINST}The CMS Collaboration, \Journal{\JINST}{3}{S08004}{2008}.
\bibitem{Chatrchyan:2011wq}The CMS Collaboration, {\em arXiv:1103.0981} (2011). Accepted for publication in \JHEP.
\bibitem{CMSPAS:EXO-10-019}The CMS Collaboration, {\em CMS Physics Analysis Summary} {\bf EXO-10-019} (2011).
\bibitem{Khachatryan201121}The CMS Collaboration, \Journal{\PLB}{698}{21}{2011}.
\bibitem{Chatrchyan:2011dx}The CMS Collaboration, {\em arXiv:1103.0030} (2011). Accepted for publication in \PLB.
\bibitem{PhysRevLett.106.201802}The CMS Collaboration, \Journal{\PRL}{106}{201802}{2011}.
\bibitem{Collaboration:2011ar}The CMS Collaboration, {\em arXiv:1105.5237} (2011). Submitted for publication in \PLB.
\bibitem{PhysRevLett.106.201803}The CMS Collaboration, \Journal{\PRL}{106}{201803}{2011}.
\bibitem{PhysRevLett.105.211801}The CMS Collaboration, \Journal{\PRL}{105}{211801}{2010}.
\bibitem{PhysRevLett.105.262001}The CMS Collaboration, \Journal{\PRL}{105}{262001}{2010}.
\bibitem{PhysRevLett.106.201804}The CMS Collaboration, \Journal{\PRL}{106}{201804}{2011}.
\bibitem{Chatrchyan:2011em}The CMS Collaboration, {\em arXiv:1102.4746} (2011). Submitted for publication in \PLB.
\bibitem{CMSPAS:TOP-10-007}The CMS Collaboration, {\em CMS Physics Analysis Summary} {\bf TOP-10-007} (2011).
\bibitem{CMSPAPER:EXO-10-016}The CMS Collaboration, {\em CERN-PH-EP-2011-081} (2011). To be submitted for publication in \PLB.
\bibitem{CMSPAS:EXO-10-025}The CMS Collaboration, {\em CMS Physics Analysis Summary} {\bf EXO-10-025} (2011).
\bibitem{springerlink:10.1007/JHEP05(2011)085}The CMS Collaboration, \Journal{\JHEP}{05}{085}{2011}.
\bibitem{CMSPAS:EXO-10-020}The CMS Collaboration, {\em CMS Physics Analysis Summary} {\bf EXO-10-020} (2011).
\bibitem{CMSPAPER:EXO-11-003}The CMS Collaboration, {\em CERN-PH-EP-2011-070} (2011). To be submitted for publication in \PRL.
\bibitem{Khachatryan2011434}The CMS Collaboration, \Journal{\PLB}{697}{434}{2011}.


%\bibitem{ja}C Jarlskog in {\em CP Violation}, ed. C Jarlskog
%(World Scientific, Singapore, 1988).
%\bibitem{ma}L. Maiani, \Journal{\PLB}{62}{183}{1976}.
%\bibitem{bu}J.D. Bjorken and I. Dunietz, \Journal{\PRD}{36}{2109}{1987}.
%\bibitem{bd}C.D. Buchanan {\it et al}, \Journal{\PRD}{45}{4088}{1992}.

\end{thebibliography}

\end{document}

%%%%%%%%%%%%%%%%%%%%%%
% End of moriond.tex  %
%%%%%%%%%%%%%%%%%%%%%%

