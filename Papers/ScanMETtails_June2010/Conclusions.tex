\section{Conclusions}
We have performed a visual scan of high $\etmiss$ events 
(tc$\etmiss>60$~GeV OR pf$\etmiss>60$~GeV)
in a large inclusive sample of 7 TeV pp collision data (about 60M events), 
after applying the official noise clean-up available in CMSSW\_3\_7\_0\_patch2.

We observe that the main difference between calotower-based cleaning and PF cleaning, in 
terms of removal of noise events in the $\etmiss$ tails, consists in the ability to identify:
\begin{itemize}
\item in-time EB spikes occurring at the border between EB and EE, and EE spikes; 
\item in-time HF double-hits.
\end{itemize}

In particular, PF cleaning is able to reject all the ECAL spikes 
(25 events with tc$\etmiss>60$~GeV), and the majority of the HF double-hits (17 out of 22 events with 
tc$\etmiss>60$~GeV) identified by the visual scan of $\etmiss$ tails.
Work is ongoing to include similar identification criteria also in the calotower-based cleaning.

After the cleaning, some residual noise events are still present in the tails 
of the $\etmiss$ distribution at an approximate rate of $10^{-6}$-$10^{-7}$
(e.g., in the pf$\etmiss$ tails there are 19 HCAL noise events with pf$\etmiss>60$~GeV 
out of approximately 60M of Minimum Bias events).

It should be pointed out that the results of a visual scan 
are always subject to a personal judgment and cannot guarantee 
the consistency and reproducibility of a rigorous statistical analysis. 
Nevertheless, they should provide with good approximation a realistic 
picture of the events populating the $\etmiss$ tails after applying 
the current noise clean-up.

\section{Acknowledgements}
We thanks Warrren Andrews for providing information on ECAL spikes.
We thanks Patrick Janot for providing inputs on the events with large pf$\etmiss$ 
caused by a mis-reconstructed muon.

