\section{Introduction}

Commissioning studies performed with test beams, cosmic runs and 
early 0.9~TeV, 2.36~TeV and 7~TeV pp 
collision data have identified several sources of anomalous noise 
(i.e. noise not produce solely from expected fluctuations in the electronics)
in the calorimeters of the CMS experiment:
\begin{itemize}
\item {\it ECAL barrel spikes} - More details are available at XXX.
%Energy deposits in individual channels 
%affected by the noise are cleaned using both topological and 
%timing information of the reconstructed hits. This type of noise is correlated with collisions. 

\item {\it HF PMT hits} - More details are available at~\cite{Chatrchyan:1225105},XXX.
%Energy deposits in individual channels 
%affected by the noise are cleaned using both topological and 
%timing information of the reconstructed hits. PMT hit noise is correlated with collisions. 

\item {\it IonFeedback/HPD/RBX noise in HCAL barrel and endcaps} - More details are available at XXX.
%Events with identified 
%HPD/RBX noise are removed from the analysis using a filter based on both topological 
%and timing information of the reconstructed energy deposits. IonFeedback noise has also been observed 
%but typically affect a few channels and produces low energy signals. A cleaning for IonFeedback noise 
%is not yet available. HPD/RBX noise is not correlated with collisions. 
\end{itemize}
In addition, machine-induced background, in the form of 
beam halo [XXX] and beam scraping events [XXX], have been observed. 

The overlap of either anomalous noise or machine-induced background 
with a pp collision event produces an unbalance in 
the reconstructed missing transverse energy in the event, which can produce 
large tails in the $\etmiss$ distribution. 

We present the results of a visual scan of high $\etmiss$ events 
(tc$\etmiss>60$~GeV OR pf$\etmiss>60$~GeV)
in an inclusive sample of XX~nb$^{-1}$ of 7 TeV pp collision data, 
after applying the official noise clean-up available in CMSSW\_3\_7\_0\_patch2.
The full selection criteria are described in Section~\ref{sec:EventSelection}). 
The scan is performed separately for events with tc$\etmiss>60$~GeV and pf$\etmiss>60$~GeV
since the noise clean-up is implemented differently in the two $\etmiss$ algorithms.
The CMS software {\it Fireworks} [XXX] has been used to produce the event displays. 
The high $\etmiss$ events have been visually inspected and classified in different 
categories. The results of this scan can provide hints to further improve the noise 
cleaning and to identify possible problems and inconsistencies in the algorithms employed 
in CMS for the $\etmiss$ reconstruction.

