\section{Introduction}

Commissioning studies performed with test beams, cosmic runs and 
early 0.9~TeV, 2.36~TeV and 7~TeV pp 
collision data have identified several sources of anomalous noise 
(i.e. noise not produce solely from expected fluctuations in the electronics)
in the calorimeters of the CMS experiment:
\begin{itemize}
\item {\it ECAL barrel spikes} - Energy deposits in individual channels 
affected by the noise are cleaned using both topological and 
timing information of the reconstructed hits. Noise correlated with collisions. 
More details are available at XXX.
\item {\it HF PMT hits} - Energy deposits in individual channels 
affected by the noise are cleaned using both topological and 
timing information of the reconstructed hits. Noise correlated with collisions. 
More details are available at XXX.
\item {\it HPB/RBX noise in HCAL barrel and endcaps} - Events with identified 
HPD/RBX noise are removed from the analysis using a filter based on both topological 
and timing information of the reconstructed energy deposits. Noise not correlated 
with collision. More details are available at XXX.
\end{itemize}
In addition, machine-induced background, in the form of 
beam halo [XXX] and beam scraping events [XXX], have been observed. 

The overlap of either anomalous noise or machine-induced background 
with a pp collision event produces an unbalance in 
the reconstructed missing transverse energy in the event, which can produce 
large tails in the $\etmiss$ distribution. 

In this note, we present the results of a visual scan of high $\etmiss$ events 
($>45$~GeV) %(Calo$\etmiss>45$~GeV OR tc$\etmiss>45$~GeV OR pf$\etmiss>45$~GeV)
in a sample %of 12~nb$^{-1}$ 
of 7 TeV pp collision data, after applying the 
%official 
noise clean-up developed by joint effort of several groups in the CMS collaboration, 
and described in Section~\ref{sec:EventSelection}. 
The CMS software {\it Fireworks} [XXX] has been used to produce the event displays. 
The high $\etmiss$ events have been visually inspected and classified in different 
cathegories. The resuls of this scan can be used as a starting point to further improve 
the noise cleaning algorithms and to identify possible problems and inconsistencies 
in the three algorithms employed in CMS for the $\etmiss$ reconstruction.

